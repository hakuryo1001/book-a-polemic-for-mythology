
% *-----------------------------------------------------------------------*
% | Fonts and typography                                                  |
% *-----------------------------------------------------------------------*

% Set CJK main font (for Chinese/Japanese/Korean characters)

% \setmainfont{Times New Roman}
\setCJKmainfont{BabelStone Han}
% doesn't work
% \setCJKmainfont{JyutcitziWithSourceHanSerifTCRegular}[
% Renderer=Basic,
% UprightFont = * ,
% FallbackFonts={BabelStone Han}
% ]



% You can also use \newfontfamily for custom non-CJK fonts if needed
% \setCJKmainfont{JyutcitziWithPMingLiURegular}[Path = ./, Extension = .ttf]
% \setCJKmainfont{JyutcitziWithSourceHanSerifTCRegular}[Path = ./, Extension = .ttf]



\newfontfamily{\jczPMingLiU}{JyutcitziWithPMingLiURegular}[Path = ./fonts/, Extension = .ttf]
% This has the best rendition for latin characters 
\newfontfamily{\jcz}{JyutcitziWithSourceHanSerifTCRegular}[Path = ./fonts/, Extension = .ttf]
\newfontfamily{\batang}{batang}[Path = ./fonts/, Extension = .ttf]
\newCJKfontfamily\koreanfont{Batang}[Path = ./fonts/, Extension = .ttf]




% *-----------------------------------------------------------------------*
% | Formatting     |
% *-----------------------------------------------------------------------*
% Set global paragraph indentation and spacing
\setlength{\parindent}{2em} % Adjust this value for the desired indentation
\setlength{\parskip}{0pt}   % No space between paragraphs

% for quotes
\usepackage{epigraph} 


\makeatletter
\renewcommand{\@makefntext}[1]{\jcz{\@thefnmark.} #1}
\makeatother
% to control itemise spacing
\usepackage{enumitem}



% *-----------------------------------------------------------------------*
% | Math & Equations     |
% *-----------------------------------------------------------------------*
\usepackage{amsmath} % For advanced math formatting
\usepackage{amssymb} % For mathematical symbols
% \usepackage{tikz} % For drawing logic decision trees


% *-----------------------------------------------------------------------*
% | Table Management                                                      |
% *-----------------------------------------------------------------------*


\usepackage{graphicx}
\usepackage{array}
\usepackage{tabularx}
\usepackage{tabularray}
\usepackage{longtable}

\usepackage{float}      % Add the float package


\usepackage[table,xcdraw]{xcolor}
% 

% Load ruby package for furigana (Ruby text)
\usepackage{ruby}
% \renewcommand{\ruby}[2]{%
%   \ruby{\jcz{#1}}{\jcz{#2}}%
% }


% *-----------------------------------------------------------------------*
% | Chinese and Soochow Numerals               |
% *-----------------------------------------------------------------------*

% Define Chinese numerals for numbers 1-99
% 〇〡〢 〣 〤 〥 〦 〧 〨 〩 十 〹 〺 卅
% Include the numerals file
% numerals.tex
% Define Chinese and Soochow numerals for chapter management

\newcommand{\soochowNumeral}[1]{%
  \ifnum#1<10
    \ifcase#1 〇\or 〡\or 〢\or 〣\or 〤\or 〥\or 〦\or 〧\or 〨\or 〩\fi%
  \else
    \ifnum#1<20
      〸\soochowUnits{\numexpr#1-10\relax}%
    \else
      \ifnum#1<30
        〹\soochowUnits{\numexpr#1-20\relax}%
      \else
        \ifnum#1<40
          〺\soochowUnits{\numexpr#1-30\relax}%
        \else
          \ifnum#1<50
            卅\soochowUnits{\numexpr#1-40\relax}%
          \else
            \ifnum#1<60
              〥十\soochowUnits{\numexpr#1-50\relax}%
            \else
              \ifnum#1<70
                〦十\soochowUnits{\numexpr#1-60\relax}%
              \else
                \ifnum#1<80
                  〧十\soochowUnits{\numexpr#1-70\relax}%
                \else
                  \ifnum#1<90
                    〨十\soochowUnits{\numexpr#1-80\relax}%
                  \else
                    \ifnum#1<100
                      〩十\soochowUnits{\numexpr#1-90\relax}%
                    \fi
                  \fi
                \fi
              \fi
            \fi
          \fi
        \fi
      \fi
    \fi
  \fi
}

\newcommand{\soochowUnits}[1]{%
  \ifnum#1=0
  \else
    \ifnum#1<4
      \ifcase#1 \or 一\or 二\or 三\fi%
    \else
      \soochowNumeral{#1}
    \fi
  \fi
}

\newcommand{\chinesenumeral}[1]{%
  \ifnum#1<10
    \ifcase#1 〇\or 一\or 二\or 三\or 四\or 五\or 六\or 七\or 八\or 九\fi%
  \else
    \ifnum#1<20
      十\chinesenumeral{\numexpr#1-10\relax}%
    \else
      \ifnum#1<30
        二十\chinesenumeral{\numexpr#1-20\relax}%
      \else
        \ifnum#1<40
          三十\chinesenumeral{\numexpr#1-30\relax}%
        \else
          \ifnum#1<50
            四十\chinesenumeral{\numexpr#1-40\relax}%
          \else
            \ifnum#1<60
              五十\chinesenumeral{\numexpr#1-50\relax}%
            \else
              \ifnum#1<70
                六十\chinesenumeral{\numexpr#1-60\relax}%
              \else
                \ifnum#1<80
                  七十\chinesenumeral{\numexpr#1-70\relax}%
                \else
                  \ifnum#1<90
                    八十\chinesenumeral{\numexpr#1-80\relax}%
                  \else
                    \ifnum#1<100
                      九十\chinesenumeral{\numexpr#1-90\relax}%
                    \fi
                  \fi
                \fi
              \fi
            \fi
          \fi
        \fi
      \fi
    \fi
  \fi
}


% *-----------------------------------------------------------------------*
% | Table of contents & Chapter for chapter management |
% *-----------------------------------------------------------------------*

\renewcommand{\figurename}{圗} % So figures would be labeled with 圗 instead of "figure"

% * * * Now for the table of contents
\renewcommand{\contentsname}{目錄} % Traditional Chinese characters for "Contents"
\setcounter{secnumdepth}{0} % no numbering for sections 
% Increase chapter title size in TOC
\usepackage{tocloft} % For customizing table of contents
\renewcommand{\cftchapfont}{\Large\bfseries} % Large and bold chapter titles
\renewcommand{\cftchappagefont}{\Large\bfseries} % Large and bold page numbers
\renewcommand{\cftchapnumwidth}{3em} % Adjust the width for chapter numbers (increase the space)

% Custom chapter numbering in the TOC with Soochow numerals
\renewcommand{\cftchapleader}{\cftdotfill{\cftsecdotsep}} % Dotted line between chapter and page number
\renewcommand{\cftchapaftersnum}{\quad} % Space between Soochow numeral and chapter title

% Redefine how chapter numbers are displayed in the TOC using Soochow numerals
\makeatletter
\renewcommand{\cftchapleader}{\cftdotfill{\cftsecdotsep}} % Dotted line between chapter and page number
\renewcommand{\cftchapaftersnum}{\quad} % Space between Soochow numeral and chapter title
% comment this out if you don't want to use the soochow numerlas
\renewcommand{\numberline}[1]{\soochowNumeral{#1}\hspace{1em}} % Use Soochow numerals for TOC chapter numbers
\makeatother


% create index - run \makeindex in the document
\usepackage{makeidx}
\makeindex



% % Custom chapter title formatting with Chinese numeral chapter numbers
\usepackage{titlesec}

% Custom chapter title formatting with Chinese numeral chapter numbers
\titleformat{\chapter}[block] % 'block' means the title appears on a new line
  {\Large\bfseries} % Font size and bold formatting for the title
  {\soochowNumeral{\thechapter}} % Chinese character for chapter number
  {1em} % Space between the number and the title
  {\Large} % Custom style for the chapter title itself (can modify)

% Remove the default LaTeX behavior of forcing new chapters to start on a new page
\makeatletter
\renewcommand\chapter{\if@openright\cleardoublepage\else\clearpage\fi
  \thispagestyle{plain}%
  \global\@topnum\z@
  \@afterindentfalse
  \secdef\@chapter\@schapter}
\makeatother




\usepackage{listings}
\usepackage{xcolor} % Optional, for colors in code
\lstset{
    language=Python,
    basicstyle=\ttfamily\small,
    keywordstyle=\color{blue},
    commentstyle=\color{green},
    stringstyle=\color{red},
    breaklines=true,
    numbers=left,
    numberstyle=\tiny,
    stepnumber=1,
    frame=single,
    captionpos=b
}

\usepackage[all]{genealogytree}

% for vertical Chinese boxes
\usepackage{graphicx} % for \rotatebox

\newfontlanguage{Chinese}{CHN}

\setCJKfamilyfont{BabelStoneVert}[RawFeature={vertical;+vert},Script=CJK,Language=Chinese,Vertical=RotatedGlyphs]{BabelStone Han}

\newcommand*\CJKmovesymbol[1]{\raise.35em\hbox{#1}}
\newcommand*\CJKmove{\punctstyle{plain}% do not modify the spacing between punctuations
  \let\CJKsymbol\CJKmovesymbol
  \let\CJKpunctsymbol\CJKsymbol}

% Define a new environment for vertical text
\newcommand{\VertCell}[1]{\rotatebox{-90}{\CJKfamily{BabelStoneVert}\CJKmove #1}}


% IDC - ideographic description characters
% https://en.wikipedia.org/wiki/Chinese_character_description_languages#Ideographic_Description_Sequences

\newcommand{\superimpose}[2]{{%
  \ooalign{%
    \hfil$\m@th\text{#1}\@firstoftwo\text{#2}$\hfil\cr
    \hfil$\m@th\text{#1}\@secondoftwo\text{#2}$\hfil\cr
  }%
}}


% Define the \tb command

\newcommand{\tb}[2]{%
\scalebox{2}[1]{
\ooalign{%
    \hfil\raisebox{0.25em}{\text{\scalebox{0.33}{#1}}}\hfil\cr % Top text, squished and raised
    \hfil\raisebox{-0.25em}{\text{\scalebox{0.33}{#2}}}\hfil\cr % Bottom text, squished and lowered
  }%
  }
}

% The \lr command - can be combined with \tb
\newcommand{\lr}[2]{
  \scalebox{0.5}[1.0]{#1}\scalebox{0.5}[1.0]{#2}\!\!
}

% Define the \ul command for upper left positioning - for characters like 疒
\newcommand{\ul}[2]{%
  \ooalign{%
    \hfil#1\hfil\cr  % Top text (unscaled)
    \hfil\hspace{0.3em}\scalebox{0.8}{#2}\cr % Bottom text (scaled and raised)
    % \hfil\raisebox{0.2em}{\scalebox{0.5}{#2}}\hfil\cr % Bottom text (scaled and raised)
  }%
}

% Define the \tone command for upper right positioning of a diacritic
\newcommand{\tone}[2]{%
  \ooalign{%
    \hfil#1\hfil\cr  % Main text (unscaled)
    \hfil\hspace{0.9em}\raisebox{0.3em}{\scalebox{0.8}{#2}}\hfil\cr % Tone mark (scaled and raised)
  }%
}


\usepackage{stackengine}

% for laddering
\usepackage{calc} % Needed for arithmetic in lengths
% *-----------------------------------------------------------------------*
% | Separators and formatting                                              |
% *-----------------------------------------------------------------------*

% Command for centered asterisk separators
\newcommand{\separator}{\begin{center}* * *\end{center}}