\chapter{雜論}

\section{雜論1}
有好多人成日會揶揄粵字改革,話咩不如用國際標音仲好過啦。問題係文字唔係淨係標音系統。所有嘅拼音文字都係有表意性嘅,而表意性就係透過特定同有限嘅不規則拼音表示。同時間,絕對無誤嘅表音係會造成書寫同閱讀嘅嚴重障礙(香港人連 the 同 da 都分唔開,會分得開 p ph p’ 咩?)。但係最重要嘅係,文字唔係淨係攞黎表達口語,亦唔係淨係攞黎溝通。佢有好多好多嘅共用同職責,而其中一個就係賦予佢嘅用家群一個自己隸屬嘅群體,仲有與其相㨢嘅尊嚴。粵切字為粵語賦予尊嚴,務求可以好似諺文咁為韓語賦予尊嚴。呢個,亦係點解拉丁化乃為下策嘅原因。
話時話:如果有人真係蠢到用IPA黎寫自己嘅語言嘅話,好快就會用爛。因為群眾會頂唔順啲規則然後迅速簡化同大產特產例外用法,最後喪失最初嘅最大優勢:其百分百精準嘅表音能力。

\section{雜論2}
康德講過,上帝淨係畀咗兩個手段畀人類去建立自然嘅獨立群體,一個就係語言,另外一個就係信仰。而歷史話畀我哋聽,語言係通常都冇辦法喺異鄉度維持到超過三代。睇睇移民美國嘅愛爾蘭人,德國人,甚至啲ABC,就知。如果語言瓦解,個群體就會失去生命力,慢性死亡。語言唔夠掂,就要有宗教。香港人的確要有信仰,仲要有自己嘅(神)聖經(文)。猶太人嘅摩西五卷,就係佢哋文明嘅精髓。希伯來文曾幾何時已經變成咗死語言,以色列復國嗰陣就係靠摩西五經及其環繞嘅相關文獻重建、復活、同現代化佢哋嘅國語:希伯來文。我哋都要咁樣做。而係呢個浩瀚嘅工程裏面,粵切字一定要有地位。粵語配有自己嘅文字 ,而粵切字就係粵語應用嘅文字。

\section{雜論3}
官話地區基本上完全毫無髮損。粵語仲一枝公撐緊。吳語就已經死曬。粵切字,好大程度上只係文字改革嘅開始。官話必須有多個文字改革,全部砲彈用曬佢。拉丁化、阿拉伯化、改造諺文、自家發明拼音,全部都要派上場。吳語小字同粵切字已經照顧好吳語同粵語。女書必須改革同活化,參考日本假名繼而系統化為湘、贛提供拼音文字。至於客家話同閩南話,就睇班台灣人可唔可以疊埋心水推白話字或者佢地啲注音符號喇。

\section{雜論4}
粵切字係邊個發明,根本唔重要,亦唔應該去而家專研。假名係邊個發明,係空海定係啲平安時代嘅貴族女人,根本唔重要。諺文到底係世宗單人匹馬發明定係集賢寺眾人合力砌出黎,根本唔重要。漢字係倉頡定係伏羲氏整出黎,唔重要。到底係邊一條躝癱將$_{Phoenician}$借黎用寫做$_{Greek}$,邊一條友將希臘字母有借冇還變咗做拉丁字母,唔重要。我哋要嘅,淨係粵切字嘅流傳開去。因為我哋相信嘅就係,粵切字嘅勝利就係粵語嘅勝利。所以,所有睇緊呢個page嘅文青同文創家,請你哋狠狠咁強我哋嘅舖,用我哋嘅輸入法,去寫詩寫文寫小說做二次創作。你想做崔萬世鬧我哋嘅話都得。我哋最怕嘅,就係淪為$_{con lang}$,姐係畀人當係玩泥沙。所以,請大家大搶特搶。最好係當粵切字係石頭度爆出黎嘅野,係人人可以攞黎當係自己嘅野。我哋需要咁樣。我哋必須要咁樣。



\section{雜論5}
想了一下關於粵切字推廣和文學構建的策略。

將粵謳用粵切字(部分用家喜稱粵砌字)轉寫來得出非漢字的本土粵語文學,有一個很大的問題,就是這樣的轉寫文學是假的。如果粵切字文學代表著「真正」的粵性,那粵切改寫的粵謳所散發的「粵性」就是假的。

我們視萬葉集為日本文學的開端,但十九世紀的日本國族構建者卻覺得萬葉集的全漢字性還是讓他們的文化地位非常尷尬,反而由女人寫,全假名或幾乎全假名的《源氏物語》就沒有這一種的漢字尷尬了。

用粵切字改寫粵謳,目的無非在於生產粵語《源氏物語》。但粵謳的本質是《萬葉集》。以粵切字改寫粵謳以得出粵語《源氏物語》來提高粵切字或粵謳,我恐怕,會為粵切字惹來非常強烈的義責。粵切字可以為粵謳標音,但改寫恐怕是大逆不道,是製造偽書。

如果要有粵語(粵切字)的《源氏物語》,真的只能製造,不能作偽書。方法就是讓小學生和中學生用粵切字書寫。女中學生是最重要的瞄準群。

其實,所有在諸夏中嘗試從非政府層面推廣漢型拼音文字(即美感上與漢字相容/可以與漢字混用的拼音文字),瞄準年輕女性,都是策略上策。低下階層、海外僑民$_{ABC}$、本地外族($_{think}$ 香港的南亞裔,大陸人,鬼佬)都是推廣焦點。

女性在這一種的社會文字改革有巨大的策略價值。基本上搬西方的基本女權主義理論就可見端倪。而歷史上,我們也看到,女性所面對的壓迫或困難,是驅使他們遠離漢字的動力。假名是由日本女性貴族發明的;諺文反對者全部都是男性士大夫,而燕山君上台後血腥鎮壓諺文,而諺文就透過低下階層的女性保存;永江的女書就不用說了。
你可能會認為今天的女性沒有古代般受壓迫。的確是的,當時壓迫還是存在。這樣就夠了。你想想潮州家庭的那種重男輕女,還有傳統華人家庭的那種完全沒有私隱的變態,就可想像到香港的中學女孩子,會用粵切字來寫日記、短訊、愛情小說—粵切字的《源氏物語》就可面世了。
向外族人推廣粵切字是$_{capstone}$,不能是中間的過程。如果以為可以透過向南亞裔或鬼佬或大陸人推廣粵切字而讓粵切字變成為粵語的未來國書,是大錯特錯。這樣做,粵切字就會馬上被打成為外來入侵的文字,$_{blahblahblah}$ 你已經可以想像到連登和高登那些連梁天琦也要罵是支那人的傻瓜和依靠中文貴族秩序搵飯食的士大夫一定罵死—而他們會成功牽動到輿論—那粵切字就必定石沈大海永不超生了。向外族人推廣,是最後一步。
當然,粵切字的例外一個構建自己legitimacy 文化合義性的渠道,就是出面的抗爭。只要街上有一個塗鴉寫著「古王.夫玉.亾丈.丩王,厶子.大丐.丩百.文丁;亾丈·丩王·大玉·力甲,禾兮·央乜·此𥘅.力冇」,我就已經贏了。
以上的策略考慮,理應都適用於吳語小字。


\section{雜論6}
排斥係違反物理架。問題係粵語根本變緊中文,所以我哋嘅粵語細胞衍生咗可以畀普通話病毒依附感染嘅蛋白質。大陸詞彙湧入,係因為粵語出現真空,秩序較強嘅語言就湧入填補,唔可以就咁歸咎曬於政府。反而你地嘅中文老師,姐係所謂嘅士大夫,先至係罪魁禍首。你地要做嘅,係喺日常生活裏面毫不顧面子咁發明新詞。有人用士大夫嘅措辭黎質疑反感發狼戾你就繼續用,用到佢地吐血為止。「人地唔明」係一個白癡嘅辯駁。約定俗成先至係上帝。