\chapter{其他腍審野}


\section{吳語小字對簡體字嘅立場}
如果你去睇下啲用吳語小字去構建吳語書面語嘅文嘅話,會見到佢哋基本上係一律否定同排斥中共簡體字嘅。但係,佢哋會喺某啲位選擇使用異體字,譬如用「躰」唔用「體」咁。呢個選擇好明顯係完全唔係因為字義原因,而係因為美感原因。佢哋想喺呢度郁.大必.々.嗰度郁.大必·々,以達到一種同漢字官白文有難以言喻嘅距離—操作同效果就好似日文中嘅漢字咁。我哋好大機會都要做啲類似嘅野。



\section{女書嘅故仔}
呢個故仔我自己第一次睇到嗰陣真係喊出黎。話說喺1982年武漢大學嘅一位宮哲兵博士生喺湖南一個叫永江嘅小鎮度發現到呢隻文字,仲發覺成個鎮只有女人識呢隻字,故此叫佢做「女書」。點解淨係得女人識呢?因為男尊女卑,女人冇得讀書,只可以靠撞靠估偷偷摸摸攞男人用嘅漢字,以其漢字讀音書寫自己嘅野,通常攞黎寫日記、繡喺扇同手巾仔上,記載嘅通常都係佢哋盲婚啞嫁、老公打老婆、姊妹情嘅哀詩。有人認為,女書可能有至少150年歷史,因為有啲天平天國嘅錢幣上有女書。而喺文革嗰陣,大量大量數之不盡有女書嘅用品全部畀人破四舊燒晒。

女書嘅自然傳人,即係細細個就寫女書大,有自然女書觸覺嘅人,喺2009年辭世。女書,已經冇自然傳人。


東亞有好多嘅文字演變都係咁,男人用漢字,新嘅、地位低下嘅、低劣嘅新文字,就由女人發明、使用、傳承。日本嘅假名,係好大程度上由啲貴族冇得參政嘅女室,為咗證明自己都可以搭嗲講到文學而自己演變出黎;韓國嘅諺文喺成宗發明咗之後冇幾耐就被一眾儒士所拋棄咒罵,但因為已經流落咗民間,女性就攞佢黎用,咁就傳承咗600年—而諺文正式上台,係要等到20世紀日本帝國將韓國嘅王帝同儒家全部殺晒,先至開始抬頭。

我成日都話,我地,係喺呢個漫長嘅六百年旅程嘅開端。係,係六百年。係好長,好痛苦。但係,係會成功嘅。係一定會成功。我地,就係喺呢個漫長嘅六百年旅程嘅開端。



\section{「方言」呢個狗牌}
當我地叫一樣我地認之為存在嘅「事物」為乜乜,賦予佢名稱時,我地從中係會奠定咗我哋對佢嘅認知,並確立咗我哋對佢存在嘅信念。而好多時候,呢個名字同佢係社會嘅用法,會衍生會維持某種政治或權利架構。呢個名字亦可能未必能夠「真實」「確切」反映到被描述物嘅現實特點。簡單啲黎講,就係我地叫某樣東西為X,我地就不由自主地認定咗X嘅存在,以及其存在模式。

咁講咁多同「方言」一詞有咩關係?如上所述,我地稱得一啲嘢為「方言」(同一啲嘢為「非方言」嘅物體),我地已經認定咗「方言」嘅存在,並且喺潛意識上奠定咗「方言」同「非方言」嘅標準。但係到底「方言」同「非方言」嘅界定線係啲乜嘢?係咪肆意無由(arbitrary)?而且就算呢個分類係人為嘅,係構建出黎嘅,佢有冇用?好唔好用?係可以擴大人類嘅知識,定係其實會導致人類概念模糊?用呢個詞黎剖釋世界,有冇咩政治同權力組態嘅衍生副作用?

先(試)處理一下「方言」呢個概念。我喺呢度我唔再以問題引渡,直接慳啲時間講小弟點睇算。

首先要講嘅「方言」一詞,無論喺大眾嘅觀點之中,抑或喺大中華學術界中,呢個詞語所表達嘅概念都同「dialect」有啲出入。雖然中西學術交流已經使到兩者逐漸趨同,但從其者在學術中所產生的蛛絲馬跡,我們仍然可以見到兩者有異。

「方言」同dialect嘅共通點就係佢地都係描述緊某種同「語言」相對嘅語碼(code)現象。呢個係釐定「方言」同「語言」嘅出發點,亦係普羅大眾對呢兩個詞嘅普通理解。

但係好明顯呢個理解係冇辦法自立,根本就自相矛盾,而且只要一直堅持「方言」同「語言」係相斥嘅關係,就必定無辦法成為一個內部邏輯通順嘅語言詮釋範式(interpretative paradigm). 原因好簡單,因為「語言」language 一詞,普遍概念上包括咗「方言」dialect嘅概念。你可以諗諗,方言如果唔係語言嘅一種存在形式,咁佢係啲乜野?如果「語言」嘅定義係「一個以人類口頭發聲按著某種邏輯規律以傳達信息資訊嘅系統」,咁「方言」又豈能不是一種語言嘅一種存在形式?

之所以會出現以上嘅類悖論情況,係因為我地冇釐清呢個同「方言」相對嘅「語言」嘅概念係啲乜東東。事實上,茲「語言」不同彼「語言」。

「方言」始終喺我哋日常嘅理解當中係相對於「語言」。我地好多時候講X係「語言」而唔係「方言」,言下之意可能係指X按某種標準而言比較「正式」;反之,當我地話X係「方言」而唔係「語言」,言下之意就係X「(唔夠)正式」。

用以上對「語言」同「方言」嘅理解,如果講得抽象同哲學一啲,就係當咗「語言」同「方言」係「一位謂詞」(1 place predicate).

有時候我地又會以以下嘅方式去演繹。我地可能會話X係Y嘅方言,而喺呢種講法係Y係「語言」,X係「方言」,而「語言—方言」係一種階級關係,Y(語言)支配住X (方言)。例子包括今日講到滿城風雨嘅教育局偉論:「粵語係漢語方言」。政治上冇咁具爭議嘅例子就可能有:African American English 係英文嘅方言/ Canadian French 係法文方言 / 北京話係官話方言 / 四邑話係粵語嘅方言 / Bavarian 係德語嘅方言。

但仲未完。有時候我地又會以以下嘅方式去演繹。我地可能會話X係C嘅方言,而Y係C嘅語言。例子:Spanish 係西班牙語言而 Catalan 係西班牙方言 / 普通話係中國語言而粵語係中國方言。

以上兩者都係將「方言」當為係「兩位謂詞」(2 place predicate):前者係「語言—方言」嘅兩位謂詞 Rxy where x 係方言 y 係語言;後者係「語言/方言—地方」嘅兩位謂詞 Rxc where x 係某種語碼,c係地方,而x係方言定係語言就視乎x係乜同c係乜。

到呢度「方言」呢個概念有咩問題應該已經可以略見端倪。好明顯,以上三個對「方言」嘅詮釋,係冇可能同時並立嘅。(呵呵我相信用以上嘅定義可以用數學證明出黎,但而家就唔搞呢範野啦)。

第二個問題就係以上三個嘅詮釋,都係只可以釐清「語言」同「方言」嘅關係,但係冇畀任何指示我地去決定咩時候咩係語言咩係方言同咩係咩嘅方言。簡單而言就係我地仲係冇釐定語言同方言嘅實質標準。

姑勿論以上三個詮釋明顯會要求有三個不同嘅標準呢個問題,就算我地暫且只專注求祈一個,我地都會發覺,個標準好難定。點解?因為有好多例外,同好難得到「普遍性」同「泛可用性」。呢個問題其實唔係好複雜嘅姐,但係要講就真係好煩水蛇春咁長,一句既之曰其理則為「語言只不過係有軍隊嘅方言咁解」。而正因茲原因,語言學家普遍都唔會嚴格定義「方言」係啲乜東東,只會當「方言」一詞為rule of thumb 速語,唔會胡亂定性乜乜語言為「方言」或「語言」 。

好啦,到戲玉啦。到底點解「方言」一詞有問題?

請循其本—我地開頭就講咗,我地社會用語中嘅詞彙同用法,係可以塑造我地對現實嘅理解。標籤某一種語碼為「方言」,係可以(亦幾乎必定會)產生龐大嘅政治作用,而呢個作用往往係具壓迫性,打壓性,同賤貶性。使用「方言」一詞黎稱呼同標籤某種語碼,就係會使被標籤者馬上蒙受語言權威同威望(prestige)嘅損失,而往往他者嘅損失,就係某者嘅(政治、經濟、文化)得益。例子實在太多,中國內對粵、吳、客、閩諸華語、法國對 Occitan, basque, 西班牙對 Catalan, 等等。

以上嘅原因適用於dialect同埋「方言」,但「方言」一詞就更加有問題。「方言」一詞除了包括晒dialect嘅絕大部分潛意思之外,佢係仲額外承載住濃郁但難以言喻嘅大中華思想同中華中心主義。呢個好複雜,難以詳述,但小弟盡下力。第一,「方言」一詞歷史上係相對於「雅言」,而雅言就係天朝上國嘅語言(口語)。所以按呢個邏輯,家下天朝上國嘅雅言係普通話,所以粵語、吳語、客家語等等就通通都係「方言」。呢個邏輯體系以前仲比較強,而家就已經弱,但係始終係普羅華人大眾中仍然有無色無聲嘅影響。要知道,喺清末民初搞翻譯書院嗰陣,藏語、蒙古語、粵語、甚至英語、法語、日語,都列為「方言」。梁啟超當初讀西方邏輯嗰陣,唔知係乜,都挾硬將「邏輯」一學列為「方言」。

二、西方嘅dialect同(standard) language嘅普羅釐定界線,唔係「相互可通性」mutual intelligibility, 就係國界。中國嘅「方言」除了一向兩者,仲包括文字。如果個語言用得漢字,就係「方言」。所以按呢個邏輯,即使北京人去上海去香港去潮州去台南聽唔明上海話廣東話潮州話閩南話,佢地都係會話佢哋冚把爛係方言。在極端啲嘅連日文,甚至歷史上用過漢字嘅韓語同越南語都唔放過話係方言。

以上兩點加埋「方言」一詞所可以帶來嘅政治作用,應該足夠說明點解「方言」茲詞理應慎用。




\section{囻之語音}

\ruby{囻}{󱼒}之語音
今日,十月九號,係諺文日。

諺文日,又稱之爲「韓字日」({\koreanfont 한글날}),係南韓爲咗記念喺1446年,世宗大王公佈《訓民正音》({\koreanfont 훈먼정음})而定嘅國定假期。

北韓都有自己對應假期,定喺一月十五號。

之所以要咁大陣象舉國慶祝,係因為「諺文」嘅發明,解決咗朝鮮呢個國家嘅文字問題,事關喺諺文之前,韓國係冇自己嘅文字。諺文嘅發明,唔單止為韓語嘅「有音無字」「言文分離」提供咗解決方案,奠立咗「韓文」嘅基礎,仲為韓國奠基咗佢哋自己嘅獨立文化嘅基礎。如果倉頡係華夏文明嘅開端,話諺文係朝鮮文明嘅開端都唔係冇得拗。

\subsection*{諺文嘅發明背景}

「諺」嘅原意係「俗語」,顧名思義「諺文」一詞就係「表記俗語(朝鮮語)之文字」嘅意思。呢度其實都已經見到諺文嘅發明開端。

嗰陣時,韓語只不過係平民百姓嘅口語,貴族同士大夫雖然口講韓語,但係寫嘅就係漢字。想寫野,就必須用漢字。用漢字,你可以寫文言文。如果想將篇文言文用韓語讀返出黎,就要用一啲非常複雜、無咩系統嘅規矩,去將文言文句子轉化成為符合韓語語法嘅句子,先至得。如果你想寫韓語,姐係想「我手寫我口」,你淨係有漢字可以用。你就只可以用漢字,攞住漢字嘅音去寫韓語,亦即係通篇用「假借」嘅手法去寫,好聽你就係《萬葉集》,難聽啲你真係同「港女文」冇乜分別。而呢啲嘅查實係見招拆招嘅書寫手法,就衍生咗所謂嘅「鄉札」、「口訣」、「吏讀」。漢字本身要學就已經成本高,噉樣嘅文字秩序就令韓文嘅讀寫變得更加係難上加難。

朝鮮國嘅第四代國王世宗大王,好想喺佢嘅國家推行儒家禮教,但係佢寫用黎教育大眾嘅書,淨係可以用漢字,庶民根本睇唔明。佢嘅臣民,有野想講,都冇辦法喺書面上同佢表達。為咗解決呢個極度嚴重嘅「言文分離」問題,佢就下令要發明一隻新文字。喺《世宗原詔》度,佢話(原文係漢文):
\begin{quotation}
  國之語音,異乎中國,與文字不相流通,故愚民有所欲言,而終不得伸其情者多矣。予爲此憫然,新制二十八字,欲使人人易習便於日用耳。

\end{quotation}
用粵語白話文講嘅話就係:

\begin{quotation}
  我哋國家嘅語音,同中國嘅唔同,所以同中國嘅文字唔啱牙。正因為係咁,愚民(唔識字嘅平民)就算有野想講(有冤情),最終好多都冇辦法同國家申訴。我為呢個問題深感悲哀,對佢哋面對嘅情況深感憐憫。所以,我而家就整咗二十八個新字出黎,希望人人都可以好容易學識,畀喺佢哋日常生活度攞黎用。
\end{quotation}

世宗大王發明嘅諺文,都真係相當之驚為天人。呢隻文字,表音奇準,使用規則奇簡,字符同字符之間喺字形設計上有耐人尋味嘅隱性邏輯,仲將中國「天地人」同陰陽學說暗合咗入去,仲借鑒咗漢字嘅六書原則並加以闡發托展——基本上前無古人,後無來者。同樣係漢字衍生出黎姊妹書寫系統,比如西夏文、女真文、喃字、方塊壯字、日本假名等等,都冇諺文咁科學——諺文稱之為東亞文字最為犀利者實在當之無愧。

呢隻文字,真係「智者不終朝而會,愚者可夾旬而學」。意思姐係「聰明人唔使一個朝頭早就可以學識,就算係白癡都可以十日內搞掂」。

\subsection*{反對諺文同戀慕中華嘅士大夫}
世宗嘅諺文推出咗冇幾耐,就有人出黎大大聲聲反對。最有代表性嘅,就係一個叫崔萬理嘅儒家士大夫。佢寫咗一篇題為《反對創建韓文》(又名《上疏反對世宗推行諺文》)畀世宗睇嘅上疏。裏面開宗明義就話:
\begin{quotation}
  我朝自祖宗以來,至誠事大,一遵華制,今當同文同軌之時,創作諺文,有駭觀聽。儻曰諺文皆本古字,非新字也,則字形雖倣古之篆文,用音合字,盡反於古,實無所據。若流中國,或有非議之者,豈不有愧於事大慕華?
\end{quotation}

我哋唔使翻譯晒出來都睇到佢嘅反對重點,就係「採用諺文,就係脫離中國,咁樣係違反『一遵華制』、『至誠事大』、『事大慕華』嘅國策。」乜嘢係「事大」呢?「事大」,姐係「事大主義」。「事大」一詞來源於《孟子》嘅「以小事大」。。「事大」裏面嘅「事」,可以理解為「服侍」。「事大」就係「為大行事」或者「服侍個『大』嘅野」。咁乜野(或者邊個)係「大」呢?就係中華,亦姐係中原皇朝,亦姐係中國。而「事大主義」注意,就係外交政策上視中原皇朝為中華(故此為「大」),以自己為「小」,而呢個外交政策上嘅體現,其中一部分就係文化上靠攏中國,將自己變成為「小中華」。崔萬理嘅意思就係,如果推諺文,就係違反朝鮮作為小中華嘅身分——言下之意就係,做得中華,就一定要用漢字。

佢之後又話:

\begin{quotation}
  自古九州之內,風土雖異,未有因方言而別爲文字者,唯蒙古、西夏、女眞、日本、西蕃之類,各有其字,是皆夷狄事耳,無足道者。《傳》曰:「用夏變夷,未聞變於夷者也。」歷代中國皆以我國有箕子遺風,文物禮樂,比擬中華。今別作諺文,捨中國而自同於夷狄,是所謂棄蘇合之香,而取螗螂之丸也,豈非文明之大累哉?
\end{quotation}

簡單黎講,就係話自古以來,就算方言唔同,都唔會改用其他文字(換句話就係佢認定韓語只不過係方言)——而唔用漢字嘅,就係自作夷狄。由華夏文明去開發夷狄,令佢哋變成為中華,就係古而有之嘅大道理姐;調返轉頭本來已經喺華夏文明裏面,跳返出黎做野蠻人,邊有咁嘅道理架?之後佢又話朝鮮好耐之前(春秋戰國時期)就已經有中原嘅「箕子」蒞臨,朝鮮仲繼承咗佢所帶黎華夏文化添
——我哋嘅文化,根本就可以同中華有得揮。家下你另起爐灶發明諺文,係放棄中國而將自己化為冇文化嘅野蠻人,係放棄蘇合呢種香草嘅香,而換黎曱甴卵嘅行為,噉樣仲唔係文明大倒退?

\subsection*{諺文之後嘅發展}

崔萬理$_{\text{fi li fe le}}$$_{\text{bi li baa laa}}$嘅理性批鬥,都唔係好阻止到世宗大王推行諺文。諺文好快就喺民間度流出去散播。但係唔係個個都接受同肯採用。開初大部分嘅名門望族同士大夫多對諺文嗤之以鼻,繼續用佢哋嘅漢字,要到二十世紀上流社會先至出現啲毫無系統嘅漢諺混寫。諺文就反而喺低下階層,尤其是係女人同奴婢之間度廣泛採用。唔少本來唔識字,亦唔會有機會識字同寫野嘅人,就攞住諺文黎寫日記。
但冇幾耐,諺文就畀人$_{\text{ban}}$咗。

朝鮮國嘅第十任國王燕山君,係一個大暴君,專搞埋晒啲寸斬、炮烙、拆胸、碎骨飄風等等嘅酷刑。而且佢又淫蕩非常,後宮膨脹,又時不時將啲佛寺改建成為妓院。百姓民怨沸騰,淨係得把口就用諺文寫野侮辱同詛咒佢。燕山君就下令取締諺文。而自此之後,諺文都主要淨係喺婦女同僧侶之間流傳使用,故諺文亦稱為「女書」或者「僧字」。

呢個情形一直到十九世紀末、二十世紀初朝鮮半島民族意識強烈提升先致開始有改變。大韓帝國高宗國王喺1894年至1896年間推動甲午改革,其中一部分頒布命令規定「法律條文與公文基本上應採用諺文;但全漢字或漢字與諺文混用的版本於必要時可以增加」。之後冇幾耐日本帝國夾硬吞併韓國嗰陣,日本人頒發咗《朝鮮教育令,規定埋一個星期中韓語同諺文嘅教育時數》(但韓語同諺文係冇官方地位),亦編製咗「詞根用漢字,虛詞用諺文」嘅韓文教科書。但係到第二次世界大戰開戰,諺文就被視為係朝鮮國族主義嘅象徵,又事被禁止使用。

諺文嘅下一個歷程碑,係1948年政府提出嘅《諺文專用法》。自此,韓漢混寫就真係抬頭。到咗六七十年代,極力主張使用諺文嘅總統朴正熙,喺1970年發表漢字廢止宣言,小學冚把爛廢除漢字教育。到咗1980年代中期,韓國嘅報紙、雜誌等,開始逐漸降低漢字使用頻率。噉係因為幾乎冇接受漢字教育嘅世代(諺文世代)開始佔多數,搞到漢字嘅出版物賣唔出——漢字續漸安樂死。

與此同時,北韓做咗自己嘅諺文拼寫改革,亦一刀切完全廢除晒所有漢字;中國改用簡體字,二簡字唔成功要打倒褪;星加坡曾經試過自己簡化漢字,但用用下就直接採用中國嘅簡體字方案快靚正;台灣、香港、澳門就繼續用繁體字;越南完全廢除漢字,採用以法文為參考基礎嘅拉丁拼音文字發明咗「𡨸國語」(chữ
Quốc ngữ);而日本就繼續漢字假名混用。

而到咗近年,漢字教育喺南韓仍然係教育政策嘅一大爭議。有人認為廢除漢字造成嚴重文化斷裂,搞到韓國人連《世宗原詔》同自己嘅憲法都睇唔明。廢除漢字亦導致韓語無法攞自己嘅漢語詞根發展新詞彙,社會同經濟發展所需要嘅新詞彙只可以全部透過音譯英文呢種嘅烏呢媽叉手段黎解決。唔識漢字,亦令韓國同中國、日本、台灣等地文化交流上出現隔膜,甚至因為同音字問題而搞出「賤出名將事件」。但亦有人認為,漢字「三多五難」,而且而家諺文已經完全成熟,學漢字根本就多餘同嘥時間。況且,諺文咁鬼犀利,咁鬼精準,點解要學人地國家(中國)嘅野?最大力反對漢字教育嘅「韓字學會」甚至話「韓字係可以喺全世界面炫耀嘅科學文字」、「漢字係特權階層嘅反民主文字」、「根本就冇南韓國民認為韓字專用唔方便」。呢種嘅心態同朝鮮民族主義結合,都滋生咗唔少語出驚人嘅說話,譬如咩「韓國之所以可以科技發展一飛沖天係因為諺文奠定咗韓國嘅數學同邏輯基礎」。到咗而家,漢字復用喺南韓仲係處於一個唔嗲唔吊嘅狀態。

\subsection*{諺文畀粵語嘅啟示}

我哋而家嘅粵語,仍然處於一個未能夠全面「我手寫我口」嘅落後境況。「本字考」仍然係處理粵語「有音冇字」嘅主流方案,以拉丁字母為基礎文字嘅粵拼亦有其推廣——但呢啲嘅手法其實都係自己嘅問題同盲點。「本字考」嘅基礎理論同方法論其實非常可疑,生產解決方案龜速,而且毫無系統,民眾參與唔到。而且本子考完全係事後解決主義,係社會度有咁上下數量嘅人用一個詞,我哋先至會搵個本字出黎,故此追唔上民間粵語嘅高速發展,甚至係排斥自然發展。其實呢度已經見到本字考嘅方法論謬誤——如果個詞係新嘅,《康熙字典》裏面又點會有本字呢?咁樣唔係刻舟求劍係乜野?所以話呢,本字考其實係延續住粵語言文分離嘅劣況,雖然有陣時佢哋都生產到啲雅味不俗嘅方案,譬如「齮齕」(gee
gut)、「䒐䒏」(忟憎),同「𪘲牙聳䚗」(依牙鬆鋼)噉,都咪話唔話有聯綿詞嘅feel。


至於拉丁拼音方案,字型美感上同漢字相斥,即使全民識用,都冇人會當作為正式文字。除非我哋用極其粗暴極權嘅手段全面廢除漢字,將漢文粵語完全消滅打殘,將粵文構建成果推倒重來,喺呢個一片荒蕪嘅曠野度畀粵語全面採用粵拼,否則粵拼就只會係類似普通話拼音嘅輔佐子系統。外國人或者香港嘅少數族裔學咗粵拼,其實都會依舊係文盲,因為根本冇野係用粵拼寫。

韓國廢除咗漢字,其實可以話係個陰差陽錯嘅偶然,而唔係佢演化嘅必然結果。漢字好似已經喺韓國徹底死亡,但其實不然,佢仍然有復活嘅機會——只要政治環境風向改變,基本上係必定會復活,因為好似崔萬理嘅士大夫喺韓國依然存在。漢韓混用,反而先至係最自然同效益最大化嘅方案。所以,我哋應該從諺文歷史度專注嘅,唔係漢字嘅死亡,而係諺文點樣完成咗「韓語有韓文」個工程。更重要嘅,廢除漢字然後諺文專用所損失嘅,係比唔上唔用諺文夾硬用漢字寫韓文嘅荒謬。

「{\koreanfont \ruby{감사합니다}{󱢮󱍖󱪪󰻦󰣖}}」當然唔夠「感謝{\koreanfont 합니다}」咁多資訊啦;「{\koreanfont 안녕하세요}」(「安寧{\koreanfont  하세요}」)、「{\koreanfont  죄송합니다}」(「罪悚{\koreanfont 합니다}」)、「{\koreanfont 미안합니다}」(「未安{\koreanfont 합니다}」)、「{\koreanfont  실례합니다}」(「失禮{\koreanfont 합니다}」)等等嘅例子都睇到漢字嘅表意同跨語言溝通功效啦。但係我哋要對比嘅唔係「諺文專用」同「漢諺混用」,而係「諺文專用」同「漢字專用」。如果呢乍野全部用漢字寫,噉啲「habnida」點算呢?用「合尼大」假借黎寫?如果用漢字假借黎寫韓文會覺得係篤眼篤鼻嘅,噉點解「多謝曬」裏面嘅「曬」我哋又唔覺得係問題?呢個「曬」,無論你係用「曬」又好,「晒」又好,個詞都係同個漢字冇意思。你用漢字,反而係隱藏同模糊咗粵語嘅語法部件。同樣道理,「做咗」「做緊」「做埋」「做過」「做住」「做親」「做做下」嘅「咗、緊、埋、過、住、親、下」其實全部都係攞咗漢字黎做啲漢字唔應該做嘅野。

更重要嘅係,我哋香港人因為我哋嘅政治成見同我哋引以為傲嘅文化背景,好容易無視咗漢字教育,真係需要巨大成本。要民眾學漢字,你係要投入大量資源同事件架。雖然我哋會覺得喺依家呢個世代,呢啲錢同事件根本唔係啲乜野。但係諺文發明咗之後幾百年黎都冇政府支撐,都可以喺低下階層繼續傳承發展,反而漢字就無法擴張佢自己嘅領域,就已經暗示住邊一個嘅成本效益比較好。我哋因為愛戀漢字,所以抗拒所有漢字以外可能可以解決到我哋語言寫唔出嘅方案。某程度上,我哋係寧願保住漢字,粵語唔可以原汁原味我手寫我口都冇所謂。再簡單啲黎講,我哋個個都係崔萬理。

廣東話,配有自己嘅文字。粵語,配有自己嘅文字。我哋,配有自己嘅文字。
所以,要粵字改革。

央乙·止子·丩丐·丩百·亾乇·禾会·
亾丐·夫丂·〡〧〩·乃千·〡〇·央乙·〩·央乜·


