\chapter{擸雜改革}

\section{我地要嘅}
我地要嘅唔係一大乍「識聽唔識講」「識講唔識寫」嘅茄喱啡。我哋要嘅係廣東話狂熱份子,廣東話教條主義,廣東話塔利班。
\section{廣東話喺一眾嘗試逆勢建立自己語言共同體嘅諸漢語言中,有類似周天子嘅崇高地位。}

廣東話喺一眾嘗試逆勢建立自己語言共同體嘅諸漢語言中,有類似周天子嘅崇高地位。如果唔係香港展示過,展示到,同繼續展示緊一種另外嘅語文可能性,顯示住「中文」嘅荒謬,吳語根本就冇可能會有咁樣嘅自我意識甦醒,畀承德話殺死咗淪為《國語》同《詩經》裏面嘅方言之後,仲要畀萬世嘅中文教授言之鑿鑿亂噏廿四,入棺之後墓碑都冇得你正名。



\section{}
基本上所有講得出「廣東話好博大精深㗎,你睇下我地嘅粗口幾咁精湛?」嘅人全部都係幫倒忙、坨衰家嘅戇鳩仔。等語心諗。以粗口宣揚粵語嘅策略完全係白痴策略。係市井僂儸嘅維園亞伯言忱。

之不過咁,我哋似乎有好大嘅需要去將「撚」字 $_{\text{normalise}}$,因為佢喺「好撚」一詞裏面扮演住廣東話$_{\text{very}}$嘅角色,而且似乎個節奏感已經深入廣東話人嘅語言肌肉記憶。

粵切字可以幫佢擺脫到佢嘅粗口出身,俾佢上檯。

好撚 —> 󱭱󰹱

喺呢到脫除漢字嘅表意,解鎖演化。

成唔成功就取決於我哋幾快可以忘記到「」嘅粗口出身。當然,讀歪音就一定有幫助,就好似「々揈」󱝚出身根本就係「條屌揈」咁俾人遺忘。喺呢一方面,假若「好很」個「很」字本身唔係煲東瓜嘅常用詞彙嘅話,就會易搞啲,冇咁驚會唔覺意人入咗啲藍青化門路,搞到遲早整班躝癱學者話「很」本身就係廣東話方言嘅證據。



\section{訓讀嘅最大好處}
\begin{itemize}
  \item 訓讀嘅最大好處就係可以脫嚟「字本位」嘅思維,可以一次過弱化大量士大夫依賴嘅漢字基建。
  \item 訓讀可以比我哋廣東話完全接受白言文,亦保護到同呵護到呢個我哋此時此刻喺度演變緊,臊孕緊嘅文化。只不過當然,有人控訴話呢家野就正正係會導致廣東話藍青化嘅原動力,佢哋係完全講得冇錯嘅。
\end{itemize}


\section{香港人 愛有 自己 嘅名}
香港人有兩個名,一個係所謂嘅「中文名」,另外一個就係所謂嘅「英文名」。

呢個咁嘅係唔可以持之以恆,亦唔方便我哋嘅國族構建。講中文名,試問你同中國人有咩分別?一樣都係三個字,連啲姓都係一樣。唔好同我講韓國人同越南人都係漢字名。韓國人有諺文󱼙,越南人有拉丁字母,日本人有訓讀。我哋依樣同所有畀中國殖民嘅諸夏人民一樣,用同一個姓氏體系,都係三個字。

我哋堅要開始打造我哋自己姓名體系,畀人一睇就知我哋係香港人。係呢一方便,我哋嘅羅馬音譯名,按照住廣東話音而唔係支那話音嚟拼寫,已經唔錯有一定嘅效果。但係呢啲嘅恐怕往往靚係畀外國人睇到,我哋自己仲係畀我哋自己嘅漢名所禁錮同勞役。我哋愛嘅係,要睇到字嘅個名嗰陣時,聽到自己個名嗰陣時,好不猶疑自己同中國人完全冇挐冇掕,就好似日本人見到自己個漢字名個陣唔會諗同中國有咩關係。

大改特改嘅方法好簡單,學日本係台灣嘅廢漢姓改日本姓名的運動嗰種拆字法都得,但好明顯呢個方法比較敏感。

一個比較正兼且照顧到我哋已有習慣習俗嘅方式,就係將我哋所謂嘅英文名,用粵切字寫,然之後同漢字孖埋起嚟,黐埋一齊,變成一個完整嘅個體。咁樣,我哋就可以係唔大幅動搖到啲人嘅名嘅情況下,產生新嘅姓名系統,同時間兼顧到同特顯到我哋同枝那有所區別嘅歷史同文化傳統。

譬如「陳大文」,佢英文名係 Tommy Chan,英文全寫可能係 Tommy Chan Tai Man。我哋而家按照上述方案改佢個漢名嘅話,就會係「陳大文」。「陳」就啱啱好對應返英文名系統嘅 Tommy Chan,「陳大文」就對應到舊制漢名嘅寫法。但係好處同優益就係在於漢字上呈現到英文,而唔係強行將英文所表現同沉澱左嘅文化背景抹煞同消滅。漢粵運用嘅瀄緊出嚟。

但係呢個仲係有未盡完善嘅位。姓依然以漢姓為主導。如果洋姓或者印姓香港人嘅話,佢哋又要被迫改漢姓,接受「史思明」「彭定康」同「何鴻燊」變做中國人嘅命運。我哋要有辦法吸納呢啲洋姓、印度姓、和姓,同啲雜巴冷嘅非漢姓,去溝淡我哋嘅漢䋛基因。

最簡單嘅方法就係𢬿剩間呢三條規矩嚟打造新姓:(1)非漢姓一律同粵切字寫左先,(2)當非漢姓同漢姓嘅人結婚,漢姓同非漢姓結合做新複姓,(3)複姓同漢姓結婚,照樣做新複姓——新姓取依舊取複姓非漢部分,同漢姓結合,成為新姓。

\begin{lstlisting}[caption={Python Surname Class}, label={lst:surname}]
  class Surname: 
    def __init__(self, hon, joeng, sex):
        self.hon = hon
        self.joeng = joeng
        self.sex = sex

    def isMale(self, p1, p2):
        return (p1, p2) if p1.sex == "male" else (p2, p1)

		def haveChildren(self, mSurname, fSurname, child_sex):
        
        new_surname = Surname(None, None, child_sex)

        if mSurname.hon is not None:
            if mSurname.joeng is not None: 
	             new_surname = Surname(mSurname.hon, mSurname.joeng, child_sex)
	           else if mSurname.joeng is None:
		           if fSurname.joeng is not None:
			           new_surname = Surname(mSurname.hon, fSurname.joeng, child_sex)
			         else:
									# fSurname.joeng is None: 
									new_surname = Surname(mSurname.hon, None, child_sex)	        
        else:
          # mSurname.hon is None - so mSurname.joeng must exist 
          if fSurname.hon is not None: 
	          new_surname = Surname(fSurname.hon, mSurname.joeng, child_sex)
	        else:
		       # fSurname.hon is None
		        new_surname = Surname(None, mSurname.joeng, child_sex)
            
        return new_surname
        
        
        new_surname = Surname(None, None, child_sex)

        if mSurname.hon is None:
            if fSurname.hon is None:
                # Both mSurname.hon and fSurname.hon are None
                new_surname = Surname(None, mSurname.joeng, child_sex)
            else:
                # mSurname.hon is None, fSurname.hon is not None
                new_surname = Surname(fSurname.hon, mSurname.joeng, child_sex)
        else:
            if mSurname.joeng is None:
                if fSurname.joeng is None:
                    # mSurname.hon is not None, both joeng are None
                    new_surname = Surname(mSurname.hon, None, child_sex)
                else:
                    # mSurname.hon is not None, mSurname.joeng is None, fSurname.joeng is not None
                    new_surname = Surname(mSurname.hon, fSurname.joeng, child_sex)
            else:
                # mSurname.hon and mSurname.joeng are not None
                new_surname = Surname(mSurname.hon, mSurname.joeng, child_sex)

        return new_surname     
   

\end{lstlisting}



% 睇吓例子:

% 1. **John Smith** - `hon`: None, `joeng`: Smith (), M
% 2. **Alice Windsor** - `hon`: None, `joeng`: Windsor (), F
% 3. **Satoshi Nakamoto** - `hon`: None, `joeng`: Nakamoto (), M
% 4. **Clementina Ângela Leitão** - `hon`: None, `joeng`: Leitão (), F
% 5. **Vivek Mahbubani** - `hon`: None, `joeng`: Mahbubani (), M
% 6. **Aisha Singh** - `hon`: None, `joeng`: Singh (), F
% 7. **何鴻燊** - `hon`: 何, `joeng`: None, M
% 8. **陳一美** - `hon`: 陳, `joeng`: None, F
% 9. **李二** - `hon`: 李, `joeng`: None, M
% 10. **董英美** - `hon`: 董, `joeng`: None, F
% 11. **張三** - `hon`: 張, `joeng`: None, M
% 12. **鄧漣洳** - `hon`: 鄧, `joeng`: None, F
% 13. **彭國** - `hon`: 彭, `joeng`: None, M
% 14. **文家寶** - `hon`: 文, `joeng`: None, F
% 15. **侯泰公** - `hon`: 侯, `joeng`: None, M
% 16. **廖鹿** - `hon`: 廖, `joeng`: None, F



\resizebox{\textwidth}{!}{
  \begin{tikzpicture}
    \genealogytree[template=signpost, id suffix=@p]
    {
      child{
          g[male]{paternal grandfather}
          p[female]{paternal grandmother}
          child{
              g[male]{paternal uncle}
              c[male]{cousin}
              child{
                  g[female]{cousin}
                }
            }
          child{
              g[female]{paternal aunt}
              c[male]{cousin}
              child{
                  g[female]{cousin}
                }
            }
          %OLD WAY
          %child[phantom*]{
          %g[male,id=father]{father}
          %p[female]{mother}
          %c[male]{brother}
          %c{\textsc{ego}}
          %c[female]{sister}
          %}
          %MIRRORED FROM MATERNAL TREE (SEE FIRST IMAGE)
          %child[phantom*]{
          %p[male,id=father]{father}
          %g[female]{mother}
          %child{
          %g[male]{brother}
          %c[male]{nephew}
          %child{
          %g[female]{niece}
          %}
          %}
          %child{
          %g{\textsc{ego}}
          %c[male]{son}
          %child{
          %g[female]{daughter}
          %}
          %}
          %child{
          %g[female]{sister}
          %c[male]{nephew}
          %child{
          %g[female]{niece}
          %}
          %}
          %}
          %}
          %MIRRORED FROM MATERNAL TREE WITH THE TWEAK (SEE SECOND IMAGE)
          child[phantom*]{
              g[male,id=father]{father}
              p[female]{mother}
              child{
                  g[male]{brother}
                  c[male]{nephew}
                  child{
                      g[female]{niece}
                    }
                }
              child{
                  g{\textsc{ego}}
                  c[male]{son}
                  child{
                      g[female]{daughter}
                    }
                }
              child{
                  g[female]{sister}
                  c[male]{nephew}
                  child{
                      g[female]{niece}
                    }
                }
            }
        }
    }
    \genealogytree[template=signpost, id suffix=@m, set position=father@m at father@p]
    {
      child{
          g[male]{maternal grandfather}
          p[female]{maternal grandmother}
          child{
              p[male,id=father]{father}
              g[female]{mother}
              child{
                  g[male]{brother}
                  c[male]{nephew}
                  child{
                      g[female]{niece}
                    }
                }
              child{
                  g{\textsc{ego}}
                  c[male]{son}
                  child{
                      g[female]{daughter}
                    }
                }
              child{
                  g[female]{sister}
                  c[male]{nephew}
                  child{
                      g[female]{niece}
                    }
                }
            }
          child{
              g[male]{maternal uncle}
              c[male]{cousin}
              c[female]{cousin}
            }
          child{
              g[female]{maternal aunt}
              c[male]{cousin}
              c[female]{cousin}
            }
        }
    }

  \end{tikzpicture}
} % end of the resize box


\begin{tikzpicture}
  % Start with the top of the family tree
  \genealogytree[template=signpost, id suffix=@1]
  {
    child{
        g[male]{John Smith} % First generation
        p[female]{Alice Windsor}
        child{
            g[male]{Satoshi Nakamoto} % Second generation
            p[female]{Clementina Ângela Leitão}
            child{
                g[male]{Vivek Mahbubani} % Third generation
                p[female]{Aisha Singh}
                child{
                    g[male]{何鴻燊} % Fourth generation
                    p[female]{陳一美}
                    child{
                        g[male]{李二} % Fifth generation
                        child{
                            g[female]{董英美}
                            child{
                                g[male]{張三} % Sixth generation
                                child{
                                    g[female]{鄧漣洳}
                                    child{
                                        g[male]{彭國} % Seventh generation
                                        child{
                                            g[female]{文家寶}
                                            child{
                                                g[male]{侯泰公} % Eighth generation
                                                child{
                                                    g[female]{廖鹿}
                                                  }
                                              }
                                          }
                                      }
                                  }
                              }
                          }
                      }
                  }
              }
          }
      }
  }
\end{tikzpicture}







% | John Smith |  |  |  |  |
% | --- | --- | --- | --- | --- |
% | Alice Windsor | Smith |  |  |  |
% | Satoshi Nakamoto |  | Smith |  |  |
% | Clementina Ângela Leitão | Nakamoto |  |  |  |
% | Vivek Mahbubani |  |  | Smith 何 |  |
% | Aisha Singh | Mahbubani |  |  |  |
% | 何鴻燊 |  | Mahbubani 何 |  |  |
% | 陳一美 | 何 |  |  | Smith 李 |
% | 李二 |  |  |  |  |
% | 董英美 | 李 |  |  |  |
% | 張三 |  | 李 |  |  |
% | 鄧漣洳 | 張 |  |  |  |
% | 彭國 |  |  | 李 |  |
% | 文家寶 | 彭 |  |  |  |
% | 侯泰公 |  | 彭 |  |  |
% | 廖鹿 | 侯 |  |  |  |

% For this order:

% 1. **John Smith** - `hon`: None, `joeng`: Smith (), M
% 2. **廖鹿** - `hon`: 廖, `joeng`: None, F
% 3. **侯泰公** - `hon`: 侯, `joeng`: None, M
% 4. **Alice Windsor** - `hon`: None, `joeng`: Windsor (), F
% 5. **Satoshi Nakamoto** - `hon`: None, `joeng`: Nakamoto (), M
% 6. **文家寶** - `hon`: 文, `joeng`: None, F
% 7. **彭國** - `hon`: 彭, `joeng`: None, M
% 8. **Clementina Ângela Leitão** - `hon`: None, `joeng`: Leitão (), F
% 9. **Vivek Mahbubani** - `hon`: None, `joeng`: Mahbubani (), M
% 10. **鄧漣洳** - `hon`: 鄧, `joeng`: None, F
% 11. **何鴻燊** - `hon`: 何, `joeng`: None, M
% 12. **Aisha Singh** - `hon`: None, `joeng`: Singh (), F
% 13. **李二** - `hon`: 李, `joeng`: None, M
% 14. **陳一美** - `hon`: 陳, `joeng`: None, F
% 15. **張三** - `hon`: 張, `joeng`: None, M
% 16. **董英美** - `hon`: 董, `joeng`: None, F

% | John Smith |  |  |  |  |
% | --- | --- | --- | --- | --- |
% | 廖鹿 | Smith 廖 |  |  |  |
% | 侯泰公 |  | Smith 廖 |  |  |
% | Alice Windsor | Windsor 侯 |  |  |  |
% | Satoshi Nakamoto |  |  | Smith 廖 |  |
% | 文家寶 | Nakamoto 文 |  |  |  |
% | 彭國 |  | Nakamoto 文 |  |  |
% | Clementina Ângela Leitão | Leitão 彭 |  |  |  |
% | Vivek Mahbubani |  |  |  | Smith 廖 |
% | 鄧漣洳 | Mahbubani 鄧 |  |  |  |
% | 何鴻燊 |  | Mahbubani 鄧 |  |  |
% | Aisha Singh | Singh 何 |  |  |  |
% | 李二 |  |  | Mahbubani 鄧 |  |
% | 陳一美 | 李 |  |  |  |
% | 張三 |  | 李 |  |  |
% | 董英美 | 張 |  |  |  |

% In jyutcitzi this would look like

% | John Smith  |  |  |  |  |
% | --- | --- | --- | --- | --- |
% | 廖鹿 | 廖 |  |  |  |
% | 侯泰公 |  | 廖 |  |  |
% | Alice Windsor 愛麗絲 | 侯 |  |  |  |
% | Satoshi Nakamoto   |  |  | 廖 |  |
% | 文家寶 | 文 |  |  |  |
% | 彭國 |  | 文 |  |  |
% | Clementina Ângela Leitão | 彭 |  |  |  |
% | Vivek Mahbubani |  |  |  | 廖 |
% | 鄧漣洳 | 鄧 |  |  |  |
% | 何鴻燊 |  | 鄧 |  |  |
% | Aisha Singh | 何 |  |  |  |
% | 李二 |  |  | 鄧 |  |
% | 陳一美 | 李 |  |  |  |
% | 張三 |  | 李 |  |  |
% | 董英美 | 張 |  |  |  |

% But suppose the surname Nakamoto 中本, given it can be written as Chinese characters, is taken to be a hon surname, then:

% | John Smith |  |  |  |  |
% | --- | --- | --- | --- | --- |
% | 廖鹿 | 廖 |  |  |  |
% | 侯泰公 |  | 廖 |  |  |
% | Alice Windsor | 侯 |  |  |  |
% | Satoshi Nakamoto 中本 |  |  | 廖 |  |
% | 文家寶 | 中本 |  |  |  |
% | 彭國 |  | 中本 |  |  |
% | Clementina Ângela Leitão | 彭 |  |  |  |
% | Vivek Mahbubani |  |  |  | 廖 |
% | 鄧漣洳 | 鄧 |  |  |  |
% | 何鴻燊 |  | 鄧 |  |  |
% | Aisha Singh | 何 |  |  |  |
% | 李二 |  |  | 鄧 |  |
% | 陳一美 | 李 |  |  |  |
% | 張三 |  | 李 |  |  |
% | 董英美 | 張 |  |  |  |



\section{我哋全部要晒}
我哋唔好再諗「啦咋」嘅「正字」係啲乜嘢。我哋唔好再詏到底係「嗱喳」󱔖、「拿渣」󱔖、「揦苴」󱔖、「揦鮓」󱔖、「藞䕢」󱔖、「\lr{巾}{賴}\lr{巾}{殺}」󱔖、定係「藞苴」。我哋唔好再討論呢啲嘥時間嘅問題。之所以咁麻煩,搞咁耐,同咁難形成共識,個根本問題就係喺個方法嗰度。討論得呢個問題,其實就係問緊「正字」,係字本位主義,係字大晒義忱。被踢出門嘅,唔俾侵埋一齊玩嘅,係詞本位主義,係時文本位義忱。姐係話,問得呢個問題,就係仲係囿於一個「一粒字一粒字」嘅諗法,而唔係「一舊詞一舊詞」、「一舊時文一舊時文」嘅諗法度。

我哋應該脫離「一個時文,一個唐字寫法」嘅教條。「一個時文,一個唐字寫法」本身就係「字本位」思維,跟得多就會字大晒,口講嘅詞彙變成書面上啲字嘅組合。思維嘅會畀字坐咗,而唔係語素。
% 正正就係因為攞字嚟做出發點,所以先至會係「」


我冇興趣同佢哋班哎吔士大夫詏餐懵。佢哋鍾意文人,我哋就\tone{由}{´}得佢哋佢哋思哲\tone{癮}{´},\tone{由}{´}得佢哋爽佢哋嗰鋪易服癖。因為我哋志在嘅,係廣東話榮登世界思哲舞台嘅一日。我哋要嘅,話大事揸𢝵嘅權揸晒喺我哋手,佢哋由´得佢哋󱐂班八婆嘈到天黑啦。

我󱝚目標係愛促進一個發展粵語󱝚新正字法,又同時間保持󱅽同現有粵語文字󱝚尊重連續性。 為󱃡󱜩,我會採用粵切字同粵拼。 點用同幾時用󱝚普遍原則󰳞,就係實詞繼續𢬿漢字黎寫,而虛詞就儘可能用粵切字黎寫。 冇漢字共識󱝚實詞󰳞(名詞、動詞、形容詞、副詞)都會攞粵切字黎寫。 我諗法係,解決󱟡一詞多異寫󱝚單詞󱝚最好符𢝵,𠄡係𢬿過字典󱝚權威或者用本字考黎一錘定音標準化,而係通過類似日文熟字訓所帶黎度󱝚開放胸襟黎畀佢全部同時存在。 就好似「ほととぎす」󱪙日文󰧵可以根據文本的上下文同語域按照作者需要寫做「子規」、「不如帰」、「杜鵑」、「蜀魂」、「郭公」咁,我諗,應該畀作者󱪙「閉翳」共「贔屭」之間去揀,又或者󱪙「揦鮓」、「嗱喳」、「藞䕢」共「\lr{巾}{賴}\lr{巾}{殺}」之間任君選擇。


如果唐字嘅優勢,係在於佢可以捨表聲嚟取表義,咁點解一個時文唔可以有幾隻唔同嘅唐字寫法?點樣寫,就取決於個寫野嘅人佢篇文章啊,佢自己想表達啲乜嘢啊,佢啲naam n
% 用得漢字,就當然有「正字」嘅概念,呢個嘅概念唔係應


\section{蒙古人點做,我哋要比佢哋做得更狠}
你唔好同我講話咩「其實,普通話、廣東話,兩者都識晒,冇咩壞姐」。你噏得出呢句,你唔係腦殘就係奸細。淨係識普通話,就已經經係對我哋廣東話人係危險。一個識聽普通話嘅靚仔,就係一個可以接收到普通話思維同宣傳嘅接毒體;一個識講普通話嘅人,就係一個會講普通話,喺要喺方便同


蒙古人點做,我哋要比佢哋做得更狠。
為抵抗漢人的殖民同化,蒙古人都做出過哪些努力?67年前的南蒙古人提出了以下對策:
一挖:當代蒙古語中不常用的詞彙,優先從古典老蒙文書籍中挖掘
二創:對於實在挖掘不到的新事物,用本民族的語言思維創造新詞
三借:當前兩者都不盡如人意時,直接借用英語或俄語單詞(以英語為主)

廣東話亦當如此
所以我地其實獵巫得係完全唔夠,所有害怕獵漢詞巫嘅都係冇膽匪類同petty叛語者。我地廣東話必須比蒙古語同韓語做得更狠。



\section{消滅詞彙,你擔當得起?}
反對用「智械」
明明已有「人工智能」可以用

長遠應該完全開放新詞發明

個問題係,係咪咩詞彙都可以做新名詞呢?字面意思都未能理解,新明詞≠難理解⋯⋯ 有邊個會去製造一啲難以理解嘅詞彙做新名詞?時代進步幾時都係新+貼地⋯

除此之外,我認為好需要參考使用率,譬如定一個使用下限,當某一個詞輸入達到某個數量,先納入新詞條考慮


係。難理解係非常個人問題,而且好大程度上係領域同個人教育背景嘅問題所致。喺宏觀嘅提高整體語言表達能力嘅目的黎講視野根本不值得理會。你係一個數學家,你發明嘅詞彙文科人睇唔明,so what?又譬如你係一個詩人,發明咗「仙氣」「榮光」「Eyeball」等詞,普通人睇唔明,so what? 況且,根本就冇所謂「睇唔明」嘅現象,亦無可能出現。如果一個詞彙,係有所指嘅,係有實用規則嘅,係可同其他詞彙配搭嘅,咁用用下就會有一定數量嘅人識用,唔識用嘅人都可以學得識。任何話「已經有同義詞」然後話應該對發明新詞採取保守態度嘅人,基本上都係冇參透過喺語言嘅尖端度建模嘅掙扎。只有完全開放詞彙發明嘅語言,先至可以得天下。任何因為文字基建或者語言群守舊嘅語言,都必定滅亡。

你呢段話係完全將我表達嘅嘢,解讀成長你想要嘅意思。其實我個point好簡單:喺add新詞彙嘅時候嚴謹啲去考慮係咪有呢個必要性,定抑或已有其他嘅詞彙或者有更好嘅詞彙替代⋯⋯

而你係解讀成咗似乎我係完全反對開放詞彙㗎喎⋯⋯真係完整咁樣詮釋左為反駁而反駁😅


我冇解讀錯。你係話要考慮有冇必要性。我就係話任何考慮所謂必要性嘅提倡都實際係篩選,姐係反對納某啲詞入紀錄。你自己唔清楚自己想要嘅有啲咩意味姐。

「必要性」根本就係個人情緒選擇。只有全知者先至可能知道喺漫長同浩瀚嘅語言時空裡面一樣野係咪「必要」。人根本就做唔到呢樣嘢。

又好嚴格按照「必要」嘅定義黎討論,如果有一個詞係不必要嘅,佢根本就唔可能出現。佢之所以出現,係因為佢嗰時嗰刻嘅語言時空構成佢出現嘅必要。話其之冇必要,係嚴重缺乏想像力同對語言神力敬畏嘅人先至會講嘅嘢。你話佢冇必要,目的就係要剔除佢—係消滅嘅詞彙啊大佬。消滅詞彙,你擔當得起?

\section{香港人必須放棄漢形名}
香港人必須放棄漢形名,要有自己獨特容易辨認嘅名。

最簡單嘅方法就係 漢字寫English first name + 漢姓 + 漢名

漢姓可以再參考日本喺台灣同韓國做過嘅皇民改名易姓手段,達至去漢化嘅效果。

再顛啲,我哋可以加個「源自地方」好似 “von Hayek”, “van der waal”, “d’anethan”。

介詞可以係英源,用粵切字寫,如「from (夫今)九龍」「of (个夫)蘇豪」

William Chan of Wong Tai Sin Tai Man
威廉陳个夫黃大仙大文
禾子力子央今陳个夫黃大仙大文

\section{嗰啲自然演化出粵切字嘅平行宇宙}
嗰啲自然演化出粵切字嘅平行宇宙,同我地嘅世界其實唔係距離好遠。

粵切字坐正咗,大量好難寫嘅擬聲詞就會即刻雨後春筍咁·氵比么·出黎,之後發展多一輪,就會好似日文裡面嘅擬聲詞,變成為上至首相下至地痞僂儸嘅語言裏面不可或缺嘅詞彙


\section{堅定流?}
我地係特意唔用「流」而用「留」,取音避義。
點解「力久·料」嘅「力久」係「流」?冇任何字典叫我地咁樣寫,但係自自然然我地會咁樣寫。可能係因為我地心裏嘅漢字兆物觀話「力久,同「流」嘅嗰種「不定」係同一個本質,被個水字旁所表達,用「流」就特別符合同有詩意」。但係咁樣其實可以話係污染同干擾,令我地本來要語義分析嘅「力久」多左一層本來唔關事嘅意思要兼顧。

\section{廣東話再上唔到車}
人類嘅科技火車越開越快。廣東話再上唔到車,就會永遠消失。我地其實只需要一本宏大作品,就可以流傳萬世。耶穌講 Aramaic, Aramaic 就得以存活。我地廣東話,有啲咩人講過?有啲咩人係可以二千年之後都有人記得?



\section{我地必須虛心懺悔反省}
我地必須虛心懺悔反省,點解講粵語嘅人,咁多都係思哲不全,言無黹語不法,氣如蠻夷,思緒污穢。

\section{「粵語入文」係一個極度自我鄙視嘅思維模式}
其實,「粵語入文」係一個極度自我鄙視嘅思維模式。佢其實就係文字「官話作主粵作客」秩序嘅呈現,亦係白言文對廣東話嘅根本性歧視同排斥嘅運作機理。
《迴響》裏面有人講過,有「粵語入文」,咁係咪都有「普通話入文」同「英文入文」?
我哋要嘅,唔係咩粵語入文。我哋要嘅,唔係做二房。我哋要嘅粵文,係粵語白話文運動,係文化獨立,係世宗路線。呢個,就係我哋嘅願景,亦係我哋嘅責任。

乜你忍受到「啲」呢個喺廣東話裏面有舉足輕重語法地位嘅詞以普通話嘅「的」加個口字旁嘅方式存在落去咩?乜你忍受到廣東話嘅書面語視覺上呈現住「我哋係普通話嘅變體」係信息咩?乜你忍受到漢字對廣東話喺書面上宣判為方言嘅呢種對待咩?粵語配有自己嘅文字,因為粵語係隻有尊嚴嘅語言。



\section{當一個人同你講「你寫返好啲中文先啦」}

當一個人同你講「你寫返好啲中文先啦」,佢所意味嘅係佢想思哲上殲滅你,但係佢冇料,所以要用埋晒啲󰲎󱂧$_{\text{grammar nazi}}$式嘅旁門左道黎擾敵。此外,佢仲可以藉此打下飛機,覺得自己好勁好好野。「正音」啊「寫錯字」啊「用錯成語」甚至「哦你寫殘體字」(「我話俾老師聽」)都係同樣嘅行為。

可知道,如果你喺英文講同樣嘅野,你係等同犯晒󰠲󰒦$_{\text{faux pas}}$。一個二個會\scalebox{0.5}[1.0]{目}\scalebox{0.5}[1.0]{},因為你好似喺飯檯度瀨屎咁。捉人字蝨係冇󱐡󱝚行為,係對自己身分有自信嘅人絕對唔會做嘅野。

我地嘅語言要昇華,我地就必須杜絕呢種嘅自瀆言辭行為,將呢種嘅污染放逐,令呢種嘅行為變成為言辭上嘅不可以。專注力轉移至語法、詞彙嘅精準、言辭嘅內容複雜性。


如果我哋成功咗,我哋當然可以肯定,我哋而家99\%嘅砌好粵切字喺一千年後全部都會被淘汰,除咗啲考古學家同文學教授識之外普羅大眾冇人識睇—就好似先秦嘅諸夏方塊字,日本平安時代嘅遣遺假名同變態假名,成宗時代嘅古諺文...未來嘅人睇我哋而家嘅粵切字就會好似我哋去睇隸定出黎嘅六國文字—似懂非懂,似曾相識,有一種口噏噏唔出嘅陌生親切感。




\section{論重符「々」}
粵字改革其中一個提議,就係每當用字重複時,唔好咁戇居真係兩隻寫晒佢。我哋可以將後面嘅一隻字用重符「々」代替。呢個用法非常古老,可以追索至春秋戰國啲竹簡。當時通常用「二」,所以亦有人話「仁」字其實係「人人」(以「人」的方式對待「人」)嘅重符縮寫:人人$\rightarrow$人二$\rightarrow$仁。而「々」呢個重符嘅寫法,大興於唐宋。之後畀日本借咗之後就係日本落地生根,變成咗佢地語文嘅一部分。





\section{粵語動漫}
我哋香港,係絕對可以同經濟同文化上需要,將粵語動漫變成為一樣野。我哋必須要通過動漫,將我哋嘅語言唔單止散播同宣揚去呢個地球嘅每一個角落,仲要將粵語殖民到每一個可能宇宙度,等我哋萬世不滅。


\section{粵語動漫}
點解Netflix嘅《末世列車》要有呢一段嘅廣東話呢?原因同點解啲美國大牌子𦧲飯應話BLM一樣:錢。香港人有錢,有國際地位。我哋喺世界萬國秩序度已經有身分。之所以點解模式列車上面有香港人,係因為佢地已經認為香港人,已經係 the 1\%,所以先有錢上到模式列車。

但係只不過咁,香港人呢個嘅身份嘅存在,係危危乎嘅,無時無刻都承受住四方八面嘅威脅。中國想消滅我哋不在話下,其實西方就好似呢個女車長咁,笑騎騎同你講廣東話,但係其實佢維持嘅秩序就係一個你冇得唔用英文應佢嘅秩序。喺佢哋嘅世界秩序,英文先至係王道,粵語嘅地位只不過係畀你地班香港人可以講嗰句「你啲廣東話好咗好多喎—唔駛急慢慢學」畀你自韰一下咁大把。

我哋粵語一定要走日本路線同法文路線—我哋唔需要建立一個類似英語帝國嘅世界流通人口,我哋反而應該將粵語標榜成為高尚嘅象徵,貴族嘅語言,優雅同質素嘅標誌,使人將粵語就聯想起氣派、原則、精闢嘅精神同意象。英文,個個都識,故此係臭西。日文、法文,要擺錢投資先至識,係上流社會嘅標誌。粵語,必須咁樣走。

\section{不切實際論、冇可能架喇論 全部 其實 都係 求其反對論}
「不切實際」
「冇可能架喇」
「香港係華夏正統」
「唐詩點算」
「廣東話好好地用漢字有咩問題」


以粵切字改革粵文,脫離漢字文字政權,奔向諸夏,畀香港獨立





1. 講啲特別哽耳嘅說話,另佢哋太難受,另佢地講「唔講廣東話」同埋「講普通話」同「唔係香港人」概念上掛鉤。要另佢哋喺香港感覺「異化」,被排擠,不受歡迎。譬如:
- 「你邊度人黎架?講普通話。」
- 「香港人唔講普通話㗎咯~」
- 「唔講廣東話邊忽係香港人?」
- 「香港唔歡迎普通話」
1. 有權用嘅就要特別對待用普通話嘅人,刁難佢地,服務差啲。濫權都要。譬如:
-
1. 見到有人喺公眾場合講普通話,要另佢哋唔舒服,可以:
- 鬧佢、
- 騷擾佢
- 錄影佢
- 直言用「支那」、「殖民者」、「蝗蟲」開口鬧
- 肢體驚嚇佢:如果佢食緊M記,上前推冧晒佢啲嘢。





新造字

風申/丰申 phone
甲申app
扁糸 print
卜手 book
昷定 run
勞申 load
黎忄like
吞辵 ten3 (本字理應為「退」)
Link 寧手/寧糸
Prefer
Preference

Form
highTo 舀
The 羍
In 兗
So 蘇
That 躂
With 业乎
Of 咢、咢乎
for 沎
From 虖壬/冘
En- 恩-
-ment 璺/忞/門/亹/悶 (加於動詞後,將其動詞名詞化虛化,以表達高一層的概念):
-tion 純 (使用於部分並列結構的動詞,以及後,表示虛化以表達高一層的概念): representation 代表純
-ary/-ery
-ness 弥斯 瀰
-Chy 亓
-er 儿、儞
-ly
-y 漪
-ium 烎
-al 奡
-ality 奡之忑 (唔要)
-ity 茌,之茌
-ence 艮,艮斯
-ism 依忱,之忱,忱。依意. 意忱
義忱、意忱  (理論、論述)
依忱  (主張)
現忱  (現象)
-ist 依忱者,士,依忱士。依意者,
-ive 枼
-able
-ize 厓斯、淣斯 施、斯、貰
En- 摁
-able 也圃
-tic 忒 克
Sub-
-pseudo
-faux
-quasi
-dom 丼
-hood 乎特
Under 奀打
In
On
At
Or
And
Anti-
Un-
In-
Non-
Meta-
Under-
-fy
-ing 營








\section{英文可以做得我哋嘅,廣東話就係。}






\section{convinced}
I was beginning to be convinced but now I am utterly convinced. That Cantonese must have spaces, like Korean. The calligraphic issue must give way. For the space itself is a grammatical marker that marks the beginning and the end of a word. This tool of demarcation will allow poet and playwright to invent new words by putting words together within the confinements delineated by the spaces between words. Written Cantonese needs all the tools imaginable for it to revitalise and resurrect its lost vocabulary. A Hebrew -esque recycling off ancient words for purposes anew is the way to go. But we can’t do that if we can’t tell if this is a new word because we can’t tell if these  characters familiar so and so sequenced are merely a fanciful poetic playful arrangement or other mark of the invention of a new word, where a familiar noun is turned into a verb or verb is turned into an adjective or an adjective is now henceforth interpreted as a noun in this particular context.

我啱啱開始被說服,但而家完全信服。粵語書寫必須要有空格,正如韓文咁。書法嘅問題必須讓步,因為空格本身就係一個語法標記,標示一個詞語嘅開始同結尾。呢個劃分嘅工具可以畀詩人同劇作家透過將詞語拼埋一齊,喺詞語之間嘅空格劃定嘅界線內,去發明新詞。書面粵語需要所有可以想像嘅工具,去振興同復活失落嘅詞彙。好似希伯來文咁,將古詞回收再利用,用於全新嘅目的。但如果我哋唔能夠知道呢啲字係咪新詞,我哋無法分辨呢啲熟悉嘅字符排列係純粹玩味嘅詩意遊戲,定係創造新詞嘅標記,喺呢個特定嘅語境入面,名詞變成動詞,或者動詞變成形容詞,又或者形容詞從此變成名詞。






\section{
    粵文要有空格}
    飲茶 : 飲 茶
食飯 : 食 飯
喫屎: 喫 屎
講嘢:講 嘢

大家見唔見到空格嘅定詞(之後喺文人筆下衍生做詞)嘅能力?


「飲茶」由空格定左出黎獨立存在嘅呢一個詞,所指嘅係 yumcha, 係一種行為,一種進膳嘅模式,同「high tea」「打邊爐」, or more accurately, 同 “let’s have takeout for lunch” 嘅 takeaway 係 詞性類似 (“let’s have yumcha for lunch” as expats might say in Hong Kong) 

「講野」 doesn’t mean speaking. It means “bullshitting”. To give it a space is it to highlight its independent existence and it’s specific meaning in a context: 

黑社會 大佬: 你 講野 啊?✅
黑社會 大佬: 你 講 野 啊?❌


Now we see spaces can delineate and frame where words start and end, we can then see how it actually gives us a subtle power to create new words more efficiently.  Like “to day -> today”, “none the less -> nonetheless”, “co operation -> cooperation”, “take away -> takeaway”. For example: 

有 咩 野 食 : 有 咩 嘢食 

Here, we created a word 野食, and we presumably we can use it elsewhere like in 呢啲 野食 都 冇 咩 特別 姐。
Note that we have actually made a noun of the form 野+verb in this change where we treat 野食 as one single word. 
And given this form of 野+verb we might be tempted to now create new words of the same form. And some work. Some do not. 

野飲
嘢睇 
—— maybe work? —-
野做
野睇
野聽
—- doesn’t work—-
嘢讀
野喫
野訓
野屌
野愛
野考
野賺
野養

But the only reason why it doesn’t work right now is because the patten has not yet become a construction custom, and the custom has not yet become a rule that writers can employ. This then, is just a matter of time.  




\section{that}
我想同你講 就係話 ——「就係話」呢句野 有小小似 「that」
