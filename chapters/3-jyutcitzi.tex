\chapter{粵切字}

\section{粵切字可「有邊讀邊」,毋需教育局,就可無師自通}

「粵切字」,又稱「粵砌字」,最犀利嘅·夫子·此居,就係佢可以啟動激活到香港人「有邊讀邊」嘅閱讀漢字能力,加上透過上文下理推理同我哋對既定詞彙格式嘅把握,基本上係可以一目瞭然讀得明。

呢個點係點解粵切$_{easily}$比所有其他拼音方案優勝嘅原因,包括諺文、假名,甚至任何一個拉丁拼音。佢哋全部都要教,但粵切字基本上唔太使:要識讀基本上唔識教,要識寫就當然要背字符——但係,你唔鍾意一個字符,都可以換。只要「反切」嘅拼音原則喺度,就唔使太擔心字符點樣變。「變態假名」幾百個,底層嘅原則係萬變不離其宗。

時粵切字,可以繞過官方教育機構,唔使去攞呢個所有人都認為必須攞到先至可以展開語文構建工程嘅體制嘅特質,係一樣就連「粵拼」都未必做到嘅嘢。所以,所有話必須攞到政權先好搞語文改革嘅說話,都係錯誤。粵切字,就係一個唔使去奪權就可以架空現時語文政權嘅方案。

而粵切字做到有邊讀邊,意味住佢可以話係我哋現時「粵製漢字」嘅進化品。與現時「粵製漢字」,或者「粵方言字」唔同嘅係,粵砌字係有系統,可以有規則地預測、組裝、延伸。唔似而家雜亂無章、不成系統,只有局內人識睇識用,局外人就一頭霧水。

粵切字,係我哋應該攞黎用嘅自然文字系統。粵切字,就係未來嘅粵字。

\section{粵切字非常之靚}
粵切字並唔樣衰。呢個係客觀事實。乎$_{if}$ 粵切字 $_{is}$ 樣衰,   $_{it is as}$ 樣衰 $_{as}$ 諺文 喃字 新字體。只有腦袋仍然中國中心主義嘅人先至會覺得樣衰。呢點之前講過好多次。基本上任何設計上嘅批評,都係萬民論驢,口爽爽矣。
其餘嘅批評基本上係閱讀理解錯誤嘅結果。「粵語一日用口字旁,就一日係方言」呢句說話嘅意思係,只要粵語係以口字旁黎書寫,文字上同觀感上就必定強烈產生粵語係「次」「附屬」「變體」嘅直覺,繼而有助於嗰啲別有用心去構建「粵語係方言」論述嘅人。鬼唔知粵語唔係方言,問題係個文字系統令人容易話佢係方言。

此外,呢種嘅文字體系的確會令粵語變成為方言 - 語法同詞彙上向主宰住漢字嘅「中文」靠攏 - 亦姐係「藍青化」。呢個現象喺吳語已經非常清楚。我地粵語嘅語法亦已經藍青化非常嚴重。



\section{點解要粵字改革?點解改革要用粵切字?}



粵字改革,就係改革粵語所用嘅文字,去完成廣東話嘅書 面語構建工程。粵切字,就係粵字改革嘅最好方案。

漢字專用係死路一條。家陣嘅粵文基本上就係漢字專用, 要用漢字呢種阻手阻腳嘅文字黎書寫口語,一係就索性唔 理你口語辭彙,由得佢自生自滅,一係就係靠埋曬啲不成 系統嘅形聲同假借手段。而我地知道,唔寫唔寫,啲 粵詞就會靜々喺黑夜中瓜咗老襯都冇人知。所以,就算 我地而家好似咩都已經可以用漢字寫到,我地冇任何理由 去為此安逸。

假借形聲呢種落後當有趣,當醒目嘅並非方法, 係會殺死粵語。粵語漢字專用,面對住呢個又係漢 字專用嘅語言,個博弈格局已經框死咗你,數學上毫無取 勝策略。漢字專用,就好似紮腳咁,箝制住廣東話嘅詞彙 同口頭邏輯嘅自然演化同約定俗成,仲要潛移默化將我哋 嘅語法向普通話靠攏,將我地便成為名符其實嘅方言。最 攞命嘅喺,漢字專用嘅粵文,運作原理同《越人歌》同《萬 葉集》根本就係一樣落後未開發嘅語文,同港女文一樣擔 任唔到文字嘅任務。咁樣嘅語文秩序,喺漢文嘅視覺係唔可能有大義嘅核突化身,喺嘗試學你隻語言嘅外國人黎講 根本就係精神虐待。

漢字專用唔得,咁拉丁化呢?至於拉丁化呢家野,係會將 我地嘅文化資本全部一鋪清袋抌曬落鹹水海。更加陰功嘅 係,拉丁化之後,學習粵語就基本上唔會有啲咩價值。你 拉丁化咗,你自己嘅人同英文嘅距離近咗,咁我咁辛苦繼 續講呢種冇咩經濟價值嘅語言做咩?點解唔講英文?你未 答到呢個問題,粵語共同體就會喺你子女嘅嗰一代解 體—— 就好似咗自己子女去國際學校嘅老豆老母,一代 仲係醒目中產香港人,下一代就已經喺 。同樣道理, 啲鬼佬啊盛,見到你隻語言,全部都係拉丁化,唔見得有 啲咩獨特文化。要學嘅,我學你一兩個詞然後喺英文度咪得囉,點解要咁辛苦去學?拉丁化嘅後果,唔係人 人過黎學你。拉丁化咗,廣東話就會變成為英文嘅 大 池,一個用自己嘅血為英文提供源源不絕嘅 嘅世界 二流語言。如果真係拉丁化,就真係囉。

粵切字,就係一個考慮曬廣東話書面化路途上所會面對嘅 問題,以最低成本嘗試去奪取最大利益嘅粵文改革方案。

粵切字,因為個社會本身巧妙地 同到粵語人嘅 「有邊讀邊」直覺,人人都可以某程度上「無師自通」。

亦姐係話,粵切字係可以繞過教育局,喺群眾度散播出去, 成為一種有機文字。加上粵切字可以同漢字混用,粵語文 化自立嘅運動可以完全去中心化,唔使政權都可以食到 糊。

粵切字運作簡單,系統易明,所有「有音無字」嘅問題唔 使經啲咩學究過程就可以一次過處理曬。粵切字仲可以畀 廣東話嘅書面語表現到已經同化咗外來詞,同片假名賦予 咗日語吞噬英詞異曲同工。粵切字會核平所有無聊同冇用 嘅「本字考」,高速完成粵文構建工程,將語文發展嘅重 任由士大夫嘅手上轉移至大眾,畀語言嘅主人發揮口語大 民主嘅智慧。長遠而言,以玩文字 為樂嘅士大夫同 文字貴族,粵切字將會慢慢蠶食同凌遲佢地嘅喺文化上來 之不義,用之不利嘅話語權。但唔使驚,佢地墨水所承載 嘅文化唔會消失。佢地所為嘅博大精深文化,會繼續連綿 不斷,但好處係,社會唔需要再為佢地嘅文化自瀆埋單。

粵字改革,就係為廣東話放腳。粵切字嘅系統性同可預測 性,將會畀一眾粵語人一個自由、民主、同開放嘅工具去 探索宇宙大義玄黹。粵切字係一種 文字,萬變不 離其宗。粵語終於可以追上泰西文明,唔使下下要討論高 階思哲概念個陣都要本字考到天昏地暗嘈餐懵。更加重要 嘅係,粵語嘅語法特徵唔會再因為漢字專用而被模糊了之。 係「粵漢混用」嘅文字秩序之下,粵語嘅語法特徵唔單止 見得到,我地嘅語言思維更加會因此而變得清晰。漢字嘅 核心,正如老生常談,係象形會意性為主。攞漢字黎寫冇 可能由象形會意性表達嘅語法部件,姐係所為嘅虛詞,根本就唔妥當。粵切字所提供咗嘅一個畀粵語好似日文漢字 假名混用嘅出路,唔單止廢除咗萬惡嘅「口字旁」,仲特 獻埋粵語語法出黎,畀粵語喺視覺上堂堂正正無可否認地 做一直獨立嘅語言。

咁樣,戳破語言偽術同濫用語言會易得多。靠思哲博大霧上位嘅可恥之徒,同埋啲句句歪理喺度污染環境嘅孬種, 將會辭窮而無所遁形。公道唔會因為講噏到嘴唇邊唔出而 要咩「自在人心」自己呃自己,雄辯唔會再喺真理嘅潛反 義詞。思哲不誠實將會成為傾計嘅死罪。更重要嘅係,粵切字嘅民主化形聲系統,將會畀我地重新用翻我地嘅$_{\text{reason}}$,畀我地睇到 ,畀我地喺言談之間\scalebox{0.5}[1.0]{言}\scalebox{0.5}[1.0]{}\scalebox{0.5}[1.0]{言}\scalebox{0.5}[1.0]{}。

粵切字所應承嘅,就係將粵文化提升至日韓$_{\text{level}}$嘅升$_{\text{le}}$。喺 維持到自己本有文化嘅前提下,粵切字將會畀粵語地位躍 升至日韓地位。除非陰差陽錯搞到好似韓國咁唔少心廢除咗漢字,粵切字所$_{\text{offer}}$嘅,係一個畀現時大部分以漢字黎 寫嘅粵文作品繼續延續落去,但又可以擺脫漢字專用遺毒嘅道路。最緊要嘅喺,粵切字係一個得好好嘅文字。 就算遞時有人要粵切字專用,廢除漢字,粵切字就好似諺 文咁,係一個隨時隨地都可以畀漢字復活嘅文字系統——唔似羅馬化。

廣東話嘅書面語運動,唔淨止係五四運動嘅「我手寫我口」。 我地要做到嘅,係為廣東話賦予尊嚴。我地要做到嘅,係講廣東話嘅\scalebox{0.5}[1.0]{礻}\scalebox{0.5}[1.0]{}\scalebox{0.5}[1.0]{礻}\scalebox{0.5}[1.0]{} 同\scalebox{0.5}[1.0]{忄}\scalebox{0.5}[1.0]{}彰顯於世。粵切字要做嘅,就做係 廣東話所配有,所值得擁有嘅文字。粵字改革,就係要為 粵語富裕尊嚴。係因為尊嚴,我地先只要改革,要咁樣改革。




\section{我們暫時建議大家用粵切字處理有音無字的粵詞}

我們暫時建議大家用粵切字處理有音無字的粵詞,以及已經完成來粵音化的外來詞。但這不是唯一的用途,也不是最好的粵語語文改革方案。最好的粵語語文改革方案,是要進行語法分析,把所有具有粵語語法意義的詞彙抽出來,和「漢詞」分辨出來——語法詞彙用粵切字,「漢詞(包括粵語實詞)」以漢字書寫。如果這樣的話,「漢詞」中的粵詞,也要發明漢字。(當然,這樣無限發明漢字的主張,是不可能的——日文也不會這樣,所以,極其量按照這條思路,最多也是像日文一樣)。

這是我們建議的使用方法,但實際上你用粵切字,暫時上用就只有那幾種。不用我們教,普羅大眾也會發現。而我們更加希望,粵語人會在這個我又發明用法,你又發明用法的過程當中,產生約定俗成效應,衍生出一些使用常規。這樣,粵切字的語文體系就會越趨成熟。

合上面兩段,某程度上我們可以看到,我們建議的粵切字使用方式,是不用推廣的。只要大家使用粵切字,粵切字就會流開出去,無師自通。

反而,全面代替漢字,卻是一個我們不可能做得到的事情。我們不主張全面替代漢字(我覺得有很多人以為我們這句話只不過是無賴騙人的定心丸大話,係喺度呃細路,一直都對我猜疑lol),但我們不主張不是因為不可能。粵切字是有能力全面代替漢字的——我設計的時候,幾乎確保了粵切字是有能力全面替代漢字的,你放心。我是不主張我們這一世代全面替代漢字。

粵切字的確有能力在未來完全代替漢字,使粵文完全變成類似諺文的無漢字語文。但這個不是必然,而是取決於粵語人的。粵語人或者會選擇使粵文變成一無漢字,但也可能漢字粵切字混用。

如果有朝他日,不知道為什麼歷史陰差陽錯出現了粵語人要全面廢除漢字,使用粵切字,站在今天世代的我,不會向未來假象歷史的人反對。但在這一世代,我們不會支持全面廢除漢字的人。

最主要的是,在這個世代就提出使用粵切字廢除漢字的人,是會退粵切字入深淵的,然後使粵語變成為像喃字越南文的全漢字語文。這樣的粵語,是會死的。



其實,我成日都覺得,你地跟本唔係唔知。你地問嘅問題,咩同音詞blah blah blah 嘅問題嘅答案,你地全部都知答案係乜嘢。你地只不過唔知你地知道。而當用起上黎,你地就會知道原來一直以來自己係乜都知。就好似你地見到粵切字一見如故。你地都唔係文盲,你地都識漢字,你地根本就知道係點運作。你地唔需要答案。你地只需要去做。





\section{粵切字嘅下一步推廣策略係乜嘢?}
粵切字成功嘅唯一可行策略,必定係一個行行下就會自己消滅「推行」嘅需要。姐係要去中心化,落之群眾,輕「公佈」重「流佈」。因為咁樣粵切字先至會有有機性。
要激活同啟動到呢種嘅散播模式,首先要粵切字嘅文字基建搞掂先。(注意:係文字基建,唔係語文基建。我哋淨係需要\lr{忄}{}.你寫唔寫到出黎,我哋唔\lr{忄}{}.你點寫)。
而家已經有輸入法同字體,已經成功咗多過一半。但係問題係已組裝嘅粵切字淨係可以喺$_{word doc}$度出現。網上面冇辦法讀取到呢啲字,至多只可以用啲未組裝嘅粵切字黎寫。咁樣就大大侷限咗我哋流通出去嘅可能性。解決呢個問題係我哋嘅首要任務。
點搞?最簡單最快嘅方法就係整兩個$_{Google chrome extension}$,(1)一個咗之後就可以讀取到曬啲組裝粵切字,(2)另外一個裝咗之後就可以直接喺個$_{Google chrome}$ 瀏覽器度打粵切字。(1)重要性遠超於(2)。
呢個我哋已經做緊。但係極度唔夠人。如果你想幫手,敬請兼且歡迎毛遂自薦。
呢樣野搞掂咗之後相信使用人口就會大增。咁之後嘅策略,就係要鼓勵產生粵切字嘅《源氏物語》《萬葉集》《奧德賽》《理想國》同《形上學》。粵語唔可以再·疒厶乜.疒夫乍$_{\text{\ul{疒}{}\ul{疒}{}}}$.無詩無文無哲無數嘅呢種落後局面。仲要重譯《聖經》。



\section{呢兩日有位好朋友不停咁同我講,粵切字嘅推行唔可以再等,一定要儘快出輸入法。}
呢兩日有位好朋友不停咁同我講,粵切字嘅推行唔可以再等,一定要儘快出輸入法。

我本來嘅諗法係,由於粵切字本身係一個闡發漢字內裡理則嘅發明,佢本身有啲野係處理唔到。所以某程度上,與其話粵切字係發明,粵切字係漢字嘅演化理則嘅蘊涵。點解粵切字會有野處理唔到?因為我哋香港嘅粵語已經大量引入咗英語,而呢啲嘅英文詞彙粵化嘅依靠音系,唔係我哋「原生粵語」嘅音系。而粵切字係建立係粵拼之上,而呢個拼音方案嘅音系,係比較接近「原生粵語」。所以,粵切字本身係處理唔到我哋而家部分嘅外語詞。
如果要處理,就要跳出漢字嘅理則系統,真正無中生有咁發明。但係呢個係一個大嘅問題。粵切字之所以可以產生到無師自通嘅效果,正正就係因為佢係建基於漢字理則。跳出呢個框框,就係進入無人疆域。咁樣做冇咩原則性嘅問題,問題係如果淨係我咁樣做,就會太過獨裁。語言改革上嘅獨裁唔係問題,佢只係一個手段,而獨裁能否兌現到發展,係視乎獨裁是否建基於一定嘅認受性之上。而家嘅粵切字獨裁有成功嘅可能,係因為佢有一定嘅認受性。佢之所以有一定嘅認受性,係因為粵語人認受漢字,咁既然粵切字係漢字嘅理則蘊涵,漢字嘅認受性就自自然然黐咗啲落粵切字嗰度。但係跳出框框就冇晒認受性,反而可能會對粵切字系統嘅核心造成認受性危機。之所以咁樣,我先至話,要手寫先,要有一少撮人試用先,等佢地喺粵切字嘅系統度定約成俗,產生漢字之外嘅約定俗成理則,咁我哋就可以整輸入法喇。

但係佢唔同意。佢認為我哋已經冇時間。首先佢好猛烈抨擊話手寫住先根本就唔係辦法,係一個冇人會做嘅野。手寫亦因乎其本質限制咗粵切字用家之間嘅交流,令到約定俗成根本起飛唔到。另外,佢話,如果我唔盡快推,粵切字喺我哋呢個嘅抗爭度產生唔到作用,冇辦到手嘅話,咁粵切字就唔單止喪失咗一個推廣流傳嘅大好時機,就更加甚至淪為一個得意嘅文字玩意咁大把。佢話,而家粵切字只係一個·扌臼干·言力円.(con lang),再難聽啲話就係一個非常複雜嘅語文打飛機。

咁我都認同嘅。粵切字而家的確只不過係一個·扌臼干·言力円\textsubscript{\lr{扌}{}\tb{}{言}}。你睇下啲人唔係鬧我破壞中華文化,反而係笑鳩我玩泥沙就知道—啲人唔係覺得粵切字會構成威脅中國嘅野,反而係覺得一個搞笑戇鳩嘅產品。如果佢哋鬧,反而仲好,因為咁樣反而係意味以粵切字寫嘅粵文地位真係可以同「中文」有得揮。

其實,要推粵切字,有好多方法。可以出野黎賣,整T裇、海報、毛巾乜乜柒柒,人地用住免費為粵切字宣傳。又可以整個輸入法,畀大家有一個可以大規模但低成本搞亂對家收集情報嘅溝通語文。

問題係,我冇錢,我唔夠人,我自己做唔到咁多野。

我遲下會搞發報會,同大家交流一下我四本野裏面嘅思想,講解一下粵切字同漢字兆物觀,同埋我哋當下面對嘅政治嘅關係。最重要嘅係,希望可以搵到啲志同道合嘅朋友,一齊同心協力推廣粵切字,改革粵字粵文。

睇下點啦。



\section{粵切字能否用來書寫其他漢系語言}
粵切字能否用來書寫其他漢系語言,係可以的。但這個「可以」係有謂述限定的。這裏展示的潮州話寫法,是以粵音透過切字法來書寫潮州話,而不是用潮音配切字法來書寫潮州話。也就是說,這個書寫辦法,是運作上等同「普字滬語」(上海寧、一剛一剛一剛)「普字粵語」(猴塞雷)、「粵音普語」(禾都鳴豬你講咩禾,厚多士)。咁唔使多講,潮州話以咁樣嘅書寫方式黎寫自己嘅語言,係毫無尊嚴嘅。但係,我地咁樣寫,粵語就可以引入潮州話,喺字面上表示我地粵語中有嘅潮州話詞彙,等同日文同英文搶人地詞彙變成自己一樣。咁樣,粵語嘅詞彙就會變得豐富同多姿多彩。


A language has rights as long as its people think it matters and find it worthwhile to preserve. If its people do not think so, then it all comes down to cultural utilities. If a language is almost dead, or already dead, then it only comes down to cultural utilities. In other words, the more successfully eradicated a language, the more it does not matter. It sounds paradoxical, but the alternative is that all languages matter equally and independently, which is something that appeals to people’s sense of equality but it’s just that and nothing else. In fact it is deeply counterintuitive because it means the rights of the living are to be weighed against the rights of the dead. Do you not see? “Rights of the dead”. How could the dead have any rights? The dead do not have any rights except ceremonialiously through the living. We do not mutiliate the dead because we the dead have the rights not to be mutiliated, but because in life their living connections have the right to not be offended. The dead are merely lumps of matter. They themsleves do not have rights.
