\chapter{粵語}

\section{唔識唐字都係照咁學}

粵語其實唔難學。每年都有一大咋工人姐姐$_{\text{literally}}$盲字都唔識多隻去學廣東話。喺我屋企湊咗我成十年嘅泰國工人姐姐真係完全唔識唐字,都係照咁學。我哋唔好畀「廣東話好難學」嘅呢個論述去阻嚇想學粵語同需要學粵語嘅人,更加千祈唔好有意無意畀呢個論述變咗做一種非正式嘅隔離政策嘅合理化論述基礎。


\section{淺論粵字發展方向}

邇來對藏漢語系產生了興趣,加上頻頻在大學跟不少陸生交流有關粵語的種種見解,有不時為學習粵語的陸生覆閱功課,洞悉了一些新的見解,亦點破了一些有關粵語、粵文和粵字的盲點。當中最引起我關注的是粵語正字的問題。

一、正字:究竟為何物?

當我們要討論粵語正字時,我們必須要承認的是粵語不如所謂的現代漢語普通話一般規範統一。雖然粵語廣東話基本上是普通話之外,世界上唯一於大學本科教授的語言,但其根基遠不如普通話穩厚,具有排山倒海的大學研究來鞏固其文法和詞彙根本。粵字非常不規則和非常不統一,幸然情況遠不及其他藏漢語系下的語言本混亂。面對這樣的文化現實,無論對於粵語用家、研究粵語的發燒友,抑或粵語學者,簡單直接視粵語「正字」為前人或古書所收納到的研究結果,是一向的普遍做法。即時有時可能不同的資源下會引申出不同的結論,可能會因而產生要辨清正字的問題,但宏觀而言,整個過程完全是被動和消極的,徹底缺乏任何正面、積極的建設。回答「某字的正字是什麼」這個問題往往離不開尋找和考究古書古文古字,類同考古,欠缺創造的元素。

把粵語正字的理解完全捆綁與過去,然後以所謂的「正統」之名煞停辯論,對粵語、粵文和粵字的發展是百害而無一利的。這樣的文字傳承亦容易滋生語言與文字出現矛盾的問題。由於文字往往趕不上語言發展的速度,若然確立正字的工作一概以傳統和古人智慧為根本,以解決現今社會文字的問題,輕則會釀成文字和語言脫節的情況,重則會造成有字讀不出,等同有字如無字的荒謬。當然,筆者並非否定先人文化資源於整頓文字的重要性,亦非徹底否定參考古人智慧以解決今人問題的優點,但盲然把傳統和古字提升至不可挑戰,不用觸碰的神壇地位,並任由其主宰當今語言、文字的發展,顯現不妥。

二、「古寫」與「俗寫」

屬於藏漢語系的語言以及其方言不乏「有言無字」和「字無人曉」的情況,故此訂立和推廣正字的工作毫不瑣碎。決定正字向來是傳統古寫(下稱「古寫」)和民間俗寫(下稱「俗寫」)的一場角力戰。但很多時候,正寫和俗寫孰贏孰輸,驟眼而言是沒有緊密的邏輯可言的。依照傳統而立的正字多數跟古文古字一脈相承,因此寫出來較具雅氣。同時間,因為文化根本比較紮實豐厚,要把該字用於定立其他詞彙,或是要在其他漢語方言中傳播,都比較容易。但所謂的「正寫」亦可能非常生僻,跟社會已經脫節,出現各種字不符音、音不符字的現象。還有,隨著社會進步發展,語言演化導致出現一字多義的情況。最麻煩的是可能所謂的正寫的那個字本身不能充分擔當表義標音的責任,但同時間民間又沒有其他更好的俗寫以供參考,那社會就被逼面對要用一個不達標的字來記錄語言無陵兩可的情況。但完全依賴俗寫來決定正字亦有其弊端。雖然俗寫基本上是民眾集思廣益的文字產物,但俗寫往往非常不統一。一字多寫的弊況泛濫,而且可能不是所有俗寫都沿用漢字,當中可能夾雜了外文,如拉丁字母。俗字的內部字體結構亦可能什麼明顯的表義邏輯可言,或是其內部邏輯非常粗疏,或跟整體漢字體系沒有共同的聯繫邏輯,因此未能完全彰顯漢字之美。由此可見,訂立正字時,要同時考慮古寫和俗寫的優劣,並嘗試取其優者。雖然這樣的說法很籠統,但對於社會上要整理的粵字,已經可以得出不少令人滿意的結果。當然,如上所言,筆者不認為透過所謂的正寫和民間俗寫打一場正字爭奪戰必定可以得出一個最好的候選正字,但這過方法依然是可以產生不少令人稱心滿意的結果的。我們可以看一看幾個例子。

我們先為討論「啲」字之亂。「啲」字於粵語廣東話中地位舉足輕重,在文法中站以給非常重要的地位。 「啲」字的俗寫繁多,特別是在網絡世界上,有「啲」,「尐 」、「D 」和「o的」。而「尐」亦有多種異寫,如下者:


「啲」字之亂源於坊間有有人將「尐」這個生僻字,講成「啲」的本字或正字,原因「尐」解作少,同「啲」(解作一點、一些)有相似地方,更有專欄作家同報章評論員誤傳。加上此誤非近年興起,而乃前人之誤未經撥亂反正的後果。「尐」字是「啲」字「正寫」的說法,最早見於孔仲南在1933年撰寫的《廣東俗語考》〈釋情狀〉一節之中,他在書中的詳解,引用《說文解字》,轉述《說文》的反切,並舉出了例子,可見有其說服力。另外有人認為「啲」乃香港人自創新字,認為是錯字,正字或本字是「尐」。

幸好,「啲」字之亂未算嚴重,其解亦算直接易明。我們可以從歷史、讀音和意思三方面剖析孰者為正,孰者為非。首先,「啲」早於明末清初《第八才子書花箋記》已經出現,曰:「奴系綠窗紅粉女,日長針指度芳年。婚姻自有高堂在,啲該兒女亂開言?」因此,「啲」字乃港人自創的錯字茲命題不攻自破。其二,「啲」讀音為「di 1」,但「尐」讀音係為「zit 3」,因「尐」反切為「子列切」。既然「尐」理應讀音不符現有讀音,即使疑中留情,以變音之由解釋音調之別,依然論帶牽強,「尐」字為正字的可能性因而相對較低。其三,查閱《說文解字》,曰:「尐,少也。」清.鮑廷博.重校方言.卷十二亦曰:「尐,杪,小也。」由此可見,若然「尐」的確為正字,把「多啲」(義:多一點)寫為「多尐」(義:多少)就會語帶含糊。綜合而論,以「啲」為正字,較為合理。如有興趣,不妨延伸閱讀此篇題為《「尐」字之亂(增修版)》的博客。

「啲」字之亂是少數透過參考韻書和古文就較為輕易解決的正字爭議。其他有待學者考究的粵字難題往往非常困難費神,一個好例子是另一粵語常用詞,「傾計」。同「啲」,「傾計」俗寫花多眼亂。在香港,「傾計」俗寫就有「傾計」、「傾偈」,大陸網絡則有「king 給」。翻查古典則有「傾偈」、「傾蓋」和「謦欬」,其中後二者用次較多,如:

《史記 •魯仲連鄒陽列傳》:諺曰:白頭如新,傾蓋如故。何則?知與不知也。

司馬貞索隱引《志林》曰:傾蓋者,道行相遇,軿車對語,兩蓋相切,小欹之,故曰傾。

《孔子家語 •致思》:孔子之郯(音談),遭程子於塗,傾蓋而語終日,甚相親。

有關「傾蓋」例子當然不止於此,但數目繁多,不能盡列,決下省留待讀者自掘。而「謦欬」於古籍出現的例子則包括:

唐朝黃滔《啟侯博士》: 「蜀璧端居, 管牀兀坐,既佩茲謦欬,益勵彼顓(音專)愚。」

宋蘇軾《黃州還回太守畢仲遠啟》:「路轉湖陰,益聽風謡之美;神馳鈴下,如聞謦咳之音。」

南宋曾幾:「又得清新句,如聞謦欬音。」

明朝王世貞西城宫詞:「寒霜不敢蒙頭坐,暖閣時聆謦欬聲。」

清朝舒白香《遊山日記》:「文人之事,所以差勝於百工技藝,豈有他哉?以其有我真性情,稱心而談,絕無嬌飾,後世才子可以想見陳死人生前面目,如聆謦欬,如握手膝,燕笑一堂,不能不愛,則稱之,稱則傳,傳斯不朽。」

《清史稿》:「隔顏色而可親謦欬。」

顯然,「傾蓋」和「謦欬」之爭兩者分庭抗禮,不相伯仲。若然要尋根究底,相信會是一個相當具挑戰性的學術難題,筆者就當然不會在此以些不文不類的谷歌資料蒐集班門弄斧。但是,筆者依然希望就此略抒己見。速查中大網上粵音字典,就會發現「謦」讀音為 「hing3」或「king2」,而「欬」異讀則有「 haai 1」、「 kat 1」、「 hoi 4」和「 koi 4」-- 顯然,「欬」字完全不符廣東話中「傾計」的讀音。雖然如此,但「謦」解言笑,有高談闊論意,歷朝用例眾多,貿然捨棄似乎有點可惜。加上「謦」以「殸」為聲符,「言」為意符,字形設計高雅優美。再者以「殸」為聲符的漢字甚多,如「聲」,「馨」和粵語中的「㷫」(「㷫烚烚」,解「熱」或形容人生氣),「謦」字符合整體漢字體系內的邏輯。故此,筆者認為以「謦欬」為正字為佳。

三、開發新字

如上所見,現時學術界確立正字的工作如同考古,其過程往往囿於古典書籍,薄於取材於今。確立或訂立正字的工作幾乎完全止於述而不作的階段,缺乏任何積極和主動的新穎創作--具體而言,當漢語學者要為一些長期因為政治或歷史原因而缺乏統一文字體系確立文字體系時,永遠只沉醉於古典文化中尋章摘句,幾乎從來未想過研發新字。當然,要尋找近代中國學者為中國內的語言和方言由頭開始建立文字體系,或翻新以及現代化其既有文字體系,亦有其例,但就藏漢語系下的十大語系而言,官、晉、徽、湘、閩、贛、吳、客、粵、平,訂立語言的政策,從來未曾涉及任何創造或開發新字的文字工程。

我們可以以臺灣客家語的保育、傳承和推廣的文字工作參考,以比較訂立粵語廣東話正字面對的困難。客家語跟粵語一樣,長期缺乏權威知識分子的呵護和整頓修理,又長期被政府忽視,甚至打壓,造成嚴重的文字混亂,和社會地位被貶、被矮化的慘況。台灣戒嚴結束,開始踏上自由民主化的道路後,台灣各地原先被打壓的種種語言得到解放,得以重新立足於社會,並可以自由弘揚開去 --客家語乃其中受益語言之一。可惜,由於受到長期抑壓,台灣很多本土語言元氣大傷,加上極權語言政策經已造成了不可磨滅的破壞,解放語言需要政府支援,才能重整旗鼓於社會跟其他語言博弈。有見及此,為加強台灣本土語言客家語研究、保存與推廣,中華民國教育部於2008年11月成立了由客家語南北四縣腔、海陸腔、大埔腔、饒平腔、詔安腔之專家學者組成的「客家語書寫推薦用字小組」(下稱「客家語小組」),研訂客家語書寫推薦用字(下稱《推薦用字》。)

客家語小組採納於《推薦用字》的字是根據嚴格的選用原則,並由專家加以反覆討論和研究後才採納的。遺憾筆者不懂客家語,無能以自己智慧審視其推薦用字是否適宜。但就其選用原則驟眼而言,似乎並無什麼大不妥。《推薦用字》的選用準則相當精密,形、音、義三者皆為審批準則。客家語小組又為用家著想,盡量避免僻字和罕用字,又考慮採納社會約定俗成的俗字,甚至酌量客家字於電腦大行其道的天下會有什麼互動和影響。整體而言,切磋琢磨後而成的《推薦用字》考慮周詳。 雖然如此,客家語小組依然沒有創造任何新字,其工作依舊圍繞考究的方向進行。

筆者認為這樣的文字整理策略略為狹隘,未盡求履行為方言訂立適宜漢字的責任。事實上,「古寫」和「俗寫」有不少無法提供有效整理文字的資源,以致出現所謂從以上兩者制定而成的「正字」依然有各式各樣的缺陷。為證茲論,筆者下列四個例子不等,而闡述其由,諸者為「虢礫𡃈嘞 」、「蝦蝦霸霸」中的「蝦」、「瀡滑梯」的「瀡」和「麻甩佬」。

粵語廣東話中有一非常生動盞鬼的詞語,但因為香港語言演化速度一日千里,詞彙日新月異,此詞有失傳之危。若然筆者記憶無誤,理應讀為「kik1 lik1 kaak1 laak1」,義解「林林總總」,跟粵語中另一詞彙「雜不lung1」義同(部分網上資源指正寫為「雜不剌」,但「剌」字讀音為「laat6」,可能有音變或錯字之嫌,暫不就此詳議。)坊間有些業餘研究粵字的意見認為正字是「虢礫𡃈嘞」。據香港中文大學網上的《粵語審音配詞字庫》,「虢礫𡃈嘞」四字中「𡃈」、「嘞」兩字並無收錄,讀音不詳。取其音符,我們可以推論「虢礫𡃈嘞」讀音為「gwik1 lik1 kaak1 laak6」。顯然「虢」字的讀音與筆者所憶有所出入,可能是因為音變或次方言音。這裡有三點需要詳細闡述:一、既然《粵語審音配詞字庫》並無收錄「𡃈」、「嘞」兩字,我們可以有信心地斷定這兩字很可能是(非常)近代的人把「緙」、「勒」加上「口」字部而新創的文字;二、我們由此可以再而推論「緙」、「勒」只不過為標音字而已,完全沒有獨立的個體意思,(暫時)不能用於構造新詞;三、一個更大膽的推論是「虢礫𡃈嘞」不是自古已有的粵語廣東話詞彙,很可能是近代從外語或其他中國語言或方言傳入的新詞。就第三點而言,筆者信心不大。雖然機緣巧合下聽聞到「虢礫𡃈嘞」一詞源自上海話。可惜,小弟家族的上海話傳承欠佳,上海話一竅不通,遺憾未能斷定茲說之真偽。

從「虢礫𡃈嘞」一詞身上我們可以清楚洞悉到只從古書經典和坊間智慧採納正字的缺陷,限制和弊端。首先,除了「礫」有「碎石」的意思可以聯繫到同本詞詞義,其餘四字跟本詞完全沒有任何意思上的關聯。「𡃈」、「嘞」兩字以上經已提到,而「緙」、「勒」兩字亦沒有任何跟「林林總總」顯然易明的意思關係。「虢」字沒有什麼特別的字義,只不過是周代其中一個諸侯國的名稱。翻查《說文解字》,內曰:「虢字本義久廢,罕有用者。」亦即是「虢」字在許慎的年代經已是僻字,何況是距離東漢幾乎一千九百年的今天!毫無疑問,「虢礫𡃈嘞」四字只不過是標音符號,完全沒有什麼字義可言。這個幾乎肯定是民間所創的寫法,沒有什麼參考價值。另外一方面,「虢礫𡃈嘞」一詞於古書記錄不詳,無法斷定茲詞來源,更無所謂的「古寫」以供參考。面對這個模棱兩可的棘手局面,解決辦法只得二者:我們一就把現在用來寫「虢礫𡃈嘞」一詞的四個字納為正寫,將錯就錯,無視其字標音不表意的根本性缺陷;要不就當這個詞語從來都沒有存在過,把它完全驅逐於粵文,任由它在茫茫的語言大海中自生自滅--這就是不容創立新字所帶來的矛盾。

同樣的問題於「蝦蝦霸霸」中的「蝦」字上重演。這個「蝦」有「欺負」之義。既然如此,這個「蝦」字就明顯除了標音之外,就毫無表意功能,選字顯然不妥。據筆者粗疏的資料收集結果,這個「蝦」字並非中原古漢語的傳承物,而很大機會是秦代以前居於現今兩廣、越北一帶的百越族的語言,與古漢語混雜而成的產物。因此這個「蝦」字與現今居住於廣西的壯族所說的壯語可能很有關連。雖然亦有意見認為「蝦」與「蟹」概念相通,而「蝦蝦霸霸」即用於形容某人狂妄自大,猶如螃蟹般橫行霸道。但筆者找不到任何稍具權威的意見證明茲論,因此以此為俗解,決不再議。不過,如果「蝦」的確與壯語有親屬關係,那要從漢族經典中尋找訂立其粵字的字選,幾乎不可能。雖然壯族自古就有根據漢字結構已成立的方塊壯字,但壯字的情況遠比粵字混亂,要查找其正字的工作將會非常艱巨。愚見以為,與其徒勞精力於凌亂不堪的壯字中大海撈針,倒不如多快好省另立新字,盡求取得形、聲、義,三者之平衡。

我們再看「麻甩佬」一例。「麻甩」可以獨立成詞,於「麻甩佬」一詞中則用於形容「佬」。「佬」通常指中年男性。所以,「麻甩佬」一詞通常用來稱呼樣子粗魯、好色鹹濕的男人,具貶義。但這個形態生動有趣的詞語其實並非粵語本有,而是近代中國與西方交流而成的。清末時法國與比利是於廣東省相當活躍,法國人甚至從清政府奪得總面積達1300平方公里的廣州灣為法屬殖民地,面積比連同新界的香港還要大。因此法語和法國文化當時於廣東省影響非同小可。當時廣州有很多法國和比利時來的傳教士醫生。每當有病人上門求醫的時候,就會用法語講「malade」(意為病人)。廣州人對法語一竅不通,以為講的是「男人」的意思,就用了在粵語裏和「malade」發音相近的「麻甩」兩個字代替。因為當時廣州人對外國人甚無好感,於是「麻甩」便有了代指粗魯男人的蔑視意義。久而久之,「麻甩」一詞就融入於粵語廣東話中,其身世就被粵語人遺忘,甚至誤會為廣東話既有之詞彙。其實,「麻甩」一詞並非廣東話獨有。畢竟清末民初時,法國人於中國影響甚廣,於上海更建立了長達94年的法租界,因此其在華的語言影響不容忽視。在上海話、無錫話裏,亦稱呼那些遊手好閒的人為「馬郞黨」或者「馬浪蕩」。其中的「馬郎」、「馬浪」都是「馬流」的音變異寫而來的。

既然「麻甩」一詞由西方傳入,繼而在華基因變異而融入漢系語言,那漢字就必定無法提供任何古典文字資源以供參考。雖然「麻甩」一詞的寫法於社會經已可謂根深蒂固,甚至已為粵語人約定俗成所公認的正寫,但「麻」、「甩」兩字標音不表意,其選字幾乎完全肆意無由,既無法表達「好色」之義,亦無法彰顯其詞之源自法語的前世今生。即管「麻」、「痲」兩字某程度上相通,以「痲」代「麻」可以跟有效地表達「麻甩」一詞的「病態」次義,「甩」字依然無法表意。再且,於此用「甩」不符粵文中「甩」字「脫」、「脫掉」的其他意思,無助完善粵字內部邏輯。翻查香港中文大學網上的《粵語審音配詞字庫》,指「甩」字於粵語中竟然無同音字!那若然我們不造新字,那就要硬吞「甩」字,放棄自我完善粵字粵文的機會。

有時候,即時有所謂的「正字」,其字可能依然千瘡百孔,急需糾正,如「 瀡」一字。「瀡」粵音為「seoi5」,普遍讀為「sir4」,見於「瀡滑梯」(義:於滑梯流下玩耍)和「烏瀡瀡」(義:矇查查,指對事情不了解,不明白,或因些兒作出就作出判斷)。「瀡」並非現代所創,其字出現於「滫(sau2)瀡」一詞,解「淘米水」,《禮·內則》:「滫瀡以滑之,脂膏以膏之。」「滫」亦可用於「豬滫」中,解「豬的糧食」。以「瀡」所做的詞橫跨數個朝代,未能盡列。「瀡」字的主要問題不在於它的用法,而是個字的字形結構。

漢字的總數量是非常驚人的,有「總匯漢字之大成」評價的《康熙字典》,在書後附有《補遺》,「盡收冷僻字,再附《備考》,又有音無義或音義全無之字」,收錄的漢字是4萬多個,而1994年出版的《中華字海》收入了87019個漢字。但普遍通行於民間書信文章和官文的漢字來來去去不外乎三千多個,而這三千多個漢字的結構比其他異體字較為工整有規律。「瀡」字是形聲字,「氵」(水)為形,「隨」為聲,但兩者結合後形成了一個又「水」部, 又「邑」部,「廴」 又部, 一女嫁三夫的怪物。我們寫字時要辨清其字的結構,工整地在一個方塊空間內安排不同筆劃,寫出來的字才會美麗。同時間,列印讀物的字體亦要符合茲等原則,讀者閱讀才會舒服暢快。要寫「瀡」字以符合漢字書寫的空間原則,一就拆為「氵」、「隨」兩部分,其面積比例一比二;二則拆為三個面積相若的部分,如「游」、「街」、「鐵」等。這是手寫抑或電腦定字都必然要面對的問題。很明顯,兩種拆法都不湊效,因為「瀡」字結構上一分為四,而普遍通行的漢字根本沒有這樣的結構,亦不能容納這樣的結構。「瀡」字結構部件過多,根本無法於四方盒內安排。與其為「瀡」這個僻字在字形上不停改組,倒不如另立新字,推倒重來。

筆者希望以上例子能夠闡明只靠古籍和俗寫以訂立漢字的局限。語言的發展不是線性的。語言與語言的發展路線亦非平行,而是互相交錯,互相影響的。字母與方塊漢字比較,字母體系能較容易和較快地適應和容納外來詞語,而身為語素文字的漢字則可能要經過深思熟慮才能把其字收歸並據為己有。所以,古籍和古人智慧顯現不能為現代社會提供所有文字問題的答案。同時間,人民緊貼社會發展,會隨著需要而創立新詞新字。但坊間往往由於急於應付眼前需要,在沒有統一或有規律的學術主導下,亂造新字。譌變,假借,拼音,自創形聲字等情況氾濫,導致文字體系非常混亂。

筆者認為整理文字的工作不能缺少創造新字這個手段。當然,造字要深謀遠慮,不能不加思索,天馬行空。但天下間除了中國大陸簡化漢字的文字工程之外,所有藏漢語系語言的文字整理工作都沒有造字的成分。愚見以為茲等現象與華人把造字這個行為神化了有關。華人學習漢字,往往第一個聽聞的故事必定是倉頡造字的傳說。有時候故事更會夾雜了什麼倉頡造字的一刻,驚天地,泣鬼神,轟動了天地之間的萬物鬼神等云云。這樣的論述有意無意把造字的行為神化和迷思化,創造新字就因而變成了忌諱。眾人對漢字的理解固化僵化,奪去了後世以創意解決文字問題的權利。這樣守舊的形式主義對文字發展和整理似乎弊大於利。

四、從上海話現況洞悉正字的重要

藏漢語系下的吳語下的上海話,當地人寫為「上海言話」或「上海閑話」,又稱「滬語」。上海話現時流行於上海一帶。上海話於清末民初曾經輝煌一時。由於上海乃華洋混集,中西通商的重要海港,有「東方巴黎」的美譽,於中國的社經地位地位顯赫。上海話亦一時成為了長江下游的經商共同語,甚至成為了上流社會的象徵,地位舉足輕重。二戰後上海元氣大傷,勢力大不如前。1949年後國民政府敗走台灣,中共上台,大量母語為上海話的資本家紛紛逃亡香港或海外。改革開放後,上海再次成為中國大陸最重要的資本中心,但因多年的推普和引入大量外來人口的政策,上海話現時面對前所未見的巨大壓力。推普政策於幼兒園,小學和中學遏止了幼童於關鍵性學習階段吸收上海話知識;上海的高等學府又一概獨用普通話;大量的外省人湧入上海,溝淡了上海資本階級的上海話人口。諸如此類的社經因素導致上海話無法發展其學術和經濟地位,結果現時淪為市井語言,甚至面臨完全徹底滅絕的危機。


吳語上海話式微歸咎於種種不利的社會和歷史因素和政府政策,當然絕無不妥。但從其文字體系解釋吳語上海話傳承遜於粵語廣東話,其由主要為一、簡體字表音偏頗普通話,二、「滬語普寫」,三、蘇白失傳,俗寫混亂。

先簡單討論簡體字對上海話的影響。漢字的標音系統並非完美無瑕,其邏輯亦間中有所例外,如粵語廣東話中「寺」(ji6)身為「時」(si4)「侍」(si 6)「持」(chi 4)、「恃」(chi 4) 的音符, 「待」字卻讀為「doi 6」。但整體而言,經過數百年,甚至千餘年的演化,漢字與中華各地語言經已產生了一種相輔相成的微妙關係,每套語言經已有自己的內部邏輯。清末民初漢字簡化運動開始。民國首次簡化漢字嘗試失敗,中共上台後接棒,大力進行漢字簡化,幾經波折後,如二簡字之亂,於改革開放前後幾年終於穩定下來。現時全中國大陸統一使用簡體字。

先不論繁簡之爭,亦暫且不論繁簡孰優孰劣。我們要清楚明白的是,絕大部分簡體字,是以形聲字代替原本的非形聲漢字(「華」變「华」),或者以筆劃較簡較少的音符取代原有音符(「運」變「运」)或索性刪去原本的音符,以其他符號代替(「鷄」變「鸡」。)簡體字透過把繁體漢字形聲化以達簡化的效果,很多時候所選擇的新音符取自普通話音--換句話說就是簡化前和簡化後讀音是否一致以普通話音為標準。這樣做成了很多一個簡體字普音對但其他漢語發音不對的情況。譬如粵語中,從非形聲字改為形聲字出現字不符音的例子有「华」(華)、「宪」(憲);因改掉音符後造成普音對粵音錯的例子有「宾」(賓)和「识」(識);而刪掉音符後無音符的例子則有「广」(廣)、「团」(團)兩者。雖然簡體字中也有源自各地方言既成的俗體字,而有部份的音符於普通話中亦出現字不符音的現象,如「 药」(藥),但宏觀簡體字,說對非官話語系的漢語造成文字制度不公,實無誇張。很可惜,如上提到,筆者不懂上海話,未能收錄簡體字於上海話字不符音的例子。但是,其論之邏輯正確無誤,理應不乏例子。相反,粵語廣東話有香港這個沿用繁體字的地區支撐,避開了簡體字對粵音傳承造成的壓力,可以較無拘無束地發展。

另一個嚴重傷害上海話傳承,發揚光大和增加外來用家的原因,是「滬語普寫」的現象。「滬語普寫」通常出現於互聯網上海人居多的聊天室,指滬語人寫的是上海話,但要用普通話的諧音把段落讀出來,才能披露隱藏的上海話句子。同一道理以粵語為例,如「猴赛雷」普通話讀出來會出現「好犀利」的廣東話詞語。回到上海話,「滬語普寫」的例子有:「上海宁」(上海人)和「神民广场」(人民廣場)等。較為複雜的例子則有「册佬棒搓桑挡相挡」和「一刚一刚一刚」。「册佬棒搓桑挡相挡」正字應該為「赤佬搭畜生打相打」,意思為「那些小屁孩在那裏很不成體統地打來打去」。至於「一刚一刚一刚」,由於同音字繁多的問題,解法很多(故此此句於上海人的網上論壇相當流行),其中一個正字解法為「 伊剛伊戇伊講」,意解「 他 / 她竟然說他 / 她笨」。「一刚一刚一刚」的例子當中,我們要把其句子先以普通話讀出,再以領會,才能騰出上海話的意思。

「滬語普寫」的弊處甚多。首先,「滬語普寫」並無統一,任由用家自由發揮,天馬行空;二、其文幾乎完全標音,毫不表意,對不曉上海話的人,或正在嘗試學習上海話的人來說,若然要做課本外吸收上海話,要過五關斬六將才能明白「滬語普寫」的內容。同時間,以「册佬棒搓桑挡相挡」/「赤佬搭畜生打相打」為例,前者為「滬語普寫」,不懂上海話的人根本摸不著頭腦,看起來莫名其妙,但後者即使是完全不懂上海話的人,也能透過字義猜測句子整體意思,幾乎略懂一二。顯然;「滬語普寫」增加了方言與方言之間的隔閡;三、「滬語普寫」的現象很容易造成上海話沒有正字,亦不可能有正字的誤解,助長「這種那種非普通話漢語是土話」的社會論調,繼而構成各種非普漢語要面對的傳承壓力。

第三個吳語上海話遜於粵語廣東話的原因是蘇白失傳和民間俗寫混亂。中國語言繁多,其實很早就經已有「我手寫我口」。歷朝歷代的白話文包括官話白話文,粵語白話文,吳語白話文和中州韻白話文,又分別稱為京白,廣白,蘇白和韻白。今天的白話文源自京白,是自五四運動開始,中國各地白話文競爭博弈後,最具勢力的白話文。後來透過民國和中共政府屢次推行,正式成為「標準漢語」。至於吳語白話文,是明清時期開始使用的一種吳語書面文體,其著名文學作品包括清初的《豆棚閒話》,清中期的彈詞腳本「沈氏四種」:《報恩緣》、《才人福》、《文星榜》和《伏虎韜》,清末民初的《何典》、《海上花列傳》、《海天鴻雪記》、《九尾龜》、《吳歌甲集》,近代朱瘦菊的《歇浦潮》、張恨水的《啼笑姻緣》、秦瘦鵑的《秋海棠》,和當代王小鷹的長篇小說《長街行》等。可見,蘇白不乏文學根基。

然而,蘇白於歷史長河沒有得到好的流傳,今時只活於少數吳語人。現時大部分的吳語人書寫白話時,隨意用字,導致俗字氾濫成災,令人非常容易混淆,甚至連所謂的上海話教科書亦慘受影響。坊間的上海話教材中「我」竟然有「我」、「吾」兩寫,「ve」 有「勿」、「伐」、「不」、「弗」,「ge」又有「搿」、「額」等。如此未經整理亂七八糟的文字實在令人望而生畏。而且大量俗寫吳字標音不表意,若然要以這樣的系統來學習即等同要死記硬背,當中的邏輯思維和歷史文化被也通通閹掉,明顯無助彰顯吳語之優雅。

五、正字的重要

雖然,訂立和推廣正字並不一定代表語言的傳承可以一勞永逸,但顯然沒有適當、公認和權威的正字,捍衛和弘揚任何語言的道路必然會荊棘滿途。回歸粵語,若然粵語希望千世萬世傳揚下去,就必需做好粵字的整理工作。這固然對粵語的發展是百利而無一害的,但對其它藏漢語系的語言其實也有有好處。

就如於本文屢次提到,研究和訂立正字的最大好處是過程和結果能夠完善語言內部邏輯,消除矛盾。這有助粵語人對自己的語言有更深的理解。本文支持的適度有為地創立新字亦有助填補古人和坊間智慧的漏洞,提升粵字的整體自我完整性。同時間,正字運動令社會重新發現到經已於書面語失傳的古老詞彙原來於口語中得以保存,而正字就恰恰扮演著表意和維繫歷史文化的角色。再論,制定正字的工作不多不少都牽涉古文學的研究。於古文中為粵語尋根有助充分彰顯粵語的古風雅氣,繼而提供後世粵語人發展粵文和廣大詞彙。一種語言的文字體系有了邏輯和文化根本,要惡意詆毀,用讒言把它打成土話,也會因而變得艱難。一種語言總不能永遠只靠內部語言用家傳宗接代,或多或少都要外人接棒弘揚,否則很容易自我滅亡。正字減少了和消除了外來人學習粵語時面對的障礙和挫折,有助粵語發揚光大。最後,粵語訂立正字,粵語並非唯一的受益者,其他藏漢語言同樣得益。粵語的詞彙既然有了較為表意的文字,要流通於其他漢系語言也較為容易。其他語言的詞彙就因此得以增加,表達和溝通複雜的抽象概念就更為容易和多變。 這間接為粵語和粵語人帶來無可量度的文化軟實力,其好處筆者就不於此冗論贅述。

六、結語

粵語是一種非常美麗的語言,筆者希望他可以萬世不滅,長傳於世。同時希望有朝他日,藏漢十大語系,官、晉、徽、湘、閩、贛、吳、客、粵、平,都有自己完善的文字系統,以便促進各民族、文化平等相處交流。最後,冀望中華文化由此可以得以鞏固,讓她可以遊走於世界萬國之中,分享其優美和智慧,豐富人類文明。






\section{「冇」字}
「冇」字這個字發明得實在太天才了,一讀就可以明白,名符其實「視而可識,察而見意」 — 「有」字少了兩劃為「冇」不就是「沒有」的意思嗎?

但更加有趣的,是這個字能夠充分暴露六書說的限制。

「冇」字就一定不是形聲字,所以只可能是象形、指事、會意三者其一。它是不是象形呢?這難說,有點似是而非。它沒有了兩劃來表示「沒有」的概念這個表達機制可以說為象形的表現,但即時我們不按照許氏象形「畫成其物,隨體詰詘」,或者就算包括段玉裁的合體象形說法再以參詳,「冇」始終跟「日」、「月」、「女」、「人」、「魚」、「牛」、「龜」、「木」、「水」、「火」、「雨」、「萬」等等有某種表意直接性上的分別。

「冇」是指事字嗎?這又難說。「冇」不像「本」、「末」、「刃」、「亦」、「天」等明顯取意自一個在象形字上增加指示符號以標示事情的表意機制,那如果他是指事字,就必定是「上」、「下」、「囗」、「八」、「厶」等全抽象指事字。但這又不像。這類的全抽象指事字,具有原子性,不能再細分,也不能追溯其結構到另外一個結構上較為原始的漢字。但顯然,「冇」源自「有」。指事似乎又說不通。

那「冇」是會意字嗎?沒了兩劃,的確能「以見指撝」,但「冇」是個獨體字,無法分割拆開,完全沒有「比類合誼」的特徵,根本毫無會意字的合併性這個根本特徵。

「冇」屬六書何者呢?

\section{速議「廣東話是否方言」}
當年在微信上有一位初蒞報到嘅大陸小伙子,拋咗兩句「粵語方言論」嘅說話,我予以對質,其對話部分如下。

好吧這其實是個語言分類學的問題,說廣府話是粵語的一個方言,或者說北京話是官話的一個方言你肯定都能接受,但當你說到「漢語」這個更加大的概念的時候,討論吳閩客粵諸語是方言還是語言就沒有意義了,因為漢語顯然是一個更大的分類,而吳閩客粵這些方言與官話方言之間的親緣關係又顯然更近於漢藏語係中不同語族間的關係。所以講粵語是漢語的方言generally不能說是無知的。

回:

首先,「漢語」存在嗎?如果「漢語」只是「漢系語言」(sinitic language family)的簡稱那漢語就絕對不是以一種語言的形式存在,就像印歐語言(Indo-European language family)一樣不是以語言的方式存在。二、為什麼討論漢語(語系或以「漢語」代稱的普通話)討論吳粵客就沒有意義?如果你這個藏漢語系中藏系和漢系中有更大的差別故此討論漢系內部分別沒有意義的話那只要放到再大那討論藏漢分別也沒有意思了。

不明白為什麼要討論到藏漢語系中藏系和漢系中有更大的差別的語言故此討論漢系內部分別就會變得沒有意義。

其實重點根本不在討論什麼有意思沒有意思。問題徹底在於一個人認不認為粵語是方言,情況 context 是什麼,還有理由是什麼。但很明顯,絕大部分的人當他們說粵吳閩客是方言時,他們的理由不是資料錯誤或理解混亂,就是有政治動機和既定立場。

當說「吳閩客粵這些方言與官話方言之間的親緣關係又顯然更近於漢藏語係中不同語族間的關係」來理直「說粵語是方言不是無知」時,我完全看不到理由。要說這句話是理直和 justify「粵語是方言」的理由時,我得問你「方言」的概念是什麼(你的這個「方言」的定義是一個描述一個語言的本質的概念還是一個形容語言與語言之間的概念)?為什麼這個概念合適用於嚴謹的語言學分類?這個概念跟我們日常用於中的「方言」一詞有什麼關係或不同?為什麼按照這個概念和「吳閩客粵與官話之間的親緣關係又顯然更近於漢藏語係中不同語族間的關係」這個事實可以得出「粵語是方言」的結論?

如果一個人堅持粵語是方言但又無法回答以上的問題的話,除非他用「方言」一詞是一種完全 formalistic 形式性沒有內涵,甚至自相矛盾的用法,否則他就必定是無知,或有政治動機。


\section{你每日都冇留意嘅粵語漢化}

「Cantonese...may derive from a language similar to proto-Viet-Muong, although a Tai ancestor has also been suggested. In any event, there has been such heavy sinicization that its origins are almost entirely obscured」—— William Mecham


\subsection{\jcz{粵語 󱝚 漢化 係 樣 發生󱜱?}}
\begin{itemize}
  \item[] 1. 中華主義者隱瞞粵語嘅壯侗語、南亞語源頭,當粵語係漢語族群嘅一種。

  \item[] 2. 虛構「粵語係古漢語/雅言/文言/唐宋口語」嘅傳說,滿足咗粵人嘅虛榮感,粵人就忘記自己真正祖源。

  \item[] 3. 忽視越源詞、百越底層,引導粵人以為所有粵語詞都有古華夏源頭。

  \item[] 4. 正字運動令人以為所有粵語詞都可以用漢字寫,唔識寫就係你唔夠文化。
\end{itemize}




\subsection*{粵語 漢字化 點解 弊家火}
\begin{itemize}
  \item[] 1. 大批非漢源字詞寫唔出,逐漸被人遺忘。例如liu lun,kal lal,kik lik kak lak。粵文詞語趨向單一死板。

  \item[] 2. 新造字吸納唔到入規範書寫系統,例如hea。造字能力受制。

  \item[] 3. 非漢源詞假借漢字書寫,同漢字本身意義不吻合,粵文亂晒籠。例如「十蚊雞」,同「蚊」、「雞」根本一啲啦𠹌都冇。
\end{itemize}




\section{所謂的「中文」}

香港講求所謂的「中文」,其霸權性的心態,是一個食古不化,故步自封,戀棧塵憶,沒有文學和語言破格創新膽量的蔬菜瓜果般的存在。保育、維持、堅持固有的傳統,沒有問題。但以此為由,抗拒新事物,不但是文化化石化,更是對自己和子孫的文化和權利閹割。


\section{廣東話從來冇「中英夾雜」呢回事}
廣東話從來冇「中英夾雜」呢回事。廣東話唔係中文。亦冇英文詞。有嘅只係粵源詞,同英源粵詞。所有反對引入同正式接納同normalise 英源粵詞嘅人,都係固步自封,食古不化,不知天高地厚,不知羅馬人何等語言淫蕩而語言先進嘅白癡。英文,係天下間最大嘅語言蕩婦。而正正因為咁,所以佢先至可以雄霸天下,要講咩就有咩詞。德法俄拉義西阿日韓全部吞曬落肚完全冇問題。我地都一定要咁樣,我地嘅廣東話先至可以「講到野」。


「拗撬」係廣東話但係「乍.丩么.文云」(argument)就唔係,呢啲係咩道理?


\section{Cantonese graphisation}
In the past years or so I have been wrestling with the problem of Cantonese graphisation. One of the first major conclusions I've drawn is that unless Cantonese rejects the use of the sinoglyphs at least up to the level like Japanese, Cantonese will never be able to develop properly and healthily, because the world of the Chinese is so bedevilled and enchanted and bespelled by the characters that there can simply be no alternative paradigm of linguistics and culture using sinoglyphs available to us in a short period of time. The imperialistic celestial empire mentality is bonded with the sinoglyphs - tho their marriage is not a necessary one, the divorce is probably way too difficult to execute and will likely take an awfully long time. The conclusion is therefore Cantonese graphisation must abandon the sinoglyphs. I came to this conclusion last year in a 茶樓 with my family.

And so with this conclusion I started to investigate what are the likely writing systems that Cantonese can adopt. I was immediate in my rejection of the Latin alphabet. I find Vietnamese writing utterly asethetically repulsive and the idea of using the Latin alphabet to escape the cultural clutches of one empire only to fall under the dominion of another thoroughly preposterous. I also relegated the use of Hangul to the very back of the list - despite the fact that I knew Hangul holds most promise in graphising cantonese, even tho back then my command of linguistics and writing systems was very primitive. Japanese kana was simply out of the question - simply because it didn't fit and it couldn't fit. If Japanese kana were used to graphisize Cantonese it would change Cantonese - it would japanize Cantonese. I turned to an unlikely candidate - the republican Zhuyin Fuhao - the ones still in use in Taiwan. I discovered that the Taiwanese government has invented additional extensions for Minnan and Hakka on top of the traditional ones for Mandarin, and I also discovered that there was also a primitive set invented for Suzhouese and Shanghainese, and also one invented very recently by online enthusiasts for Cantonese.

But then it soon became clear to me that there are multiple problems. First with the Zhuyin fuhao - it is quite unsuitable for graphisimg any sinitic language. Zhuyin is an alphabet, and in writing, the script does not graphisize morphemes as single units. The script graphisizes phonemes, and not morphemes - it breaks apart even the morpheme. So in this regard, it is worse than Hangul. Hangul preserves the monosyllablicity of sinitic tongues on paper - but Zhuyin breaks it apart. This makes reading very difficult. But sure, one can stomach that, because all romanisation schemes suffer the same problem. Pinyin and Jyutping all suffer the same problem. What's the deal then? The issue, is that even if we were to put aside the more tired and cliched arguments about why sinitic tongues cannot adopt a fully phonetic system (I'm looking at you 施氏食獅史)-  a huge deal of already present literature in Classical Chinese would simply be rendered incomprehensible - because the vocabulary is dead. Cantonese would suffer massively because a great deal of its wealth, preserved in very elegant songs of the 80s, would also be rendered very difficult to comprehend. Future vocabulary development would also be severely hampered because distinct morphemes are now undifferentiable on paper and differentiation can only be obtained via that natural language instinct living in the general populace - and we don't have that. This suggests that my earlier conclusion must be revised or moderated. A complete abandonment of the sinoglyphs would be strategically disastrerous and must be the unthinkable final option. Given that, and given that Zhuyin and the characters are on paper structurally oppositionary - you really can't read them together using a same system of mentality - at least for me I can't go a very smooth degree - a Zhuyin and characters mixed script is impossible.

Some people say I'm stubborn and I don't listen to critics - but that can't be further away from the truth, for I myself have critiqued myself tirelessly and many times I have abandoned previously held positions because of new analysis. Indeed, one such example was that one of the prime motivations of a systematised Cantonese script is the sad sight of overflowing 形聲字 and 擬聲字 in Cantonese (啊喺哋嘅嘢唔咩㗎啦喇喎 etc ) these characters under the traditional scheme of asthetics are ruled to be lowly - and they therefore are guilty of making Cantonese seem lowly in the traditional sinitic conception of aesthetics. The solution is to therefore expel them all. And phoneticisatipn can do just that - heck, there are not even sinoglyphs left! But then I read in a book about written Taiwanese how a particular Taiwanese linguist admires the Cantonese for their ingenuity for arbitrarily creating phono-semantic characters for the spoken vernacular that is somehow magically understood by other Cantonese speakers. He says this is extremely rare and absolutely extraordinary - and he was absolutely jealous of Cantonese in possession of this magical community characteristic, and very upset that Taiwanese does not. This gave me the idea of the system of democratised phono-semantic characters. The 大民主粵語形聲字符系統。this system will run parallel with characters. As you can see, this is a twofold rejection of my previous positions.


How would it work? Notice that in the past, oftentimes when the Cantonese graphisize a spoken word, they take a 聲符 that either corresponds exactly or corresponds roughly to the actual spoken sound, in combination with a 意符 that roughly indicates the meaning or nature of the morpheme to be graphisized - they combine the two, and viola! They have a character.          This is nothing extraordinary of course, for this is just how phono-semantic characters 形聲字 work. What is extraordinary is that the Cantonese manages to read ad hoc and arbitrary creations without much external help or assistance. This is very peculiar and is NOT necessarily a given - as evidenced by the aforementioned Taiwanese scholar - and also indeed the mainland Chinese I have encountered - for some reason, when the mainland Chinese encounter a character they don't know how to read, oftentimes they can't even venture a guess. But hong Kongers always can. I've seen this multiple times when I present the characters 冚家鏟/剷 to them. Not knowing how to read 冚 in Mandarin is normal - but they can't even guess how to pronunce 鏟/剷 - sometimes even in simplified. So, I have confidence that this phono-semantic decomposition literacy, is somewhat tied to the Cantonese language. It's a big jump in logic, but one that helps, and one's whose contingent falsity doesn't jeopardise my project.

So, the ingenious part of this new system, is that instead of having a character that is set into stone once created , and have ad hoc creations using arbitrary 聲符意符 - we systematise it all, do that there is a fixed and finite set of 聲符意符, and that the resultant combined characters are not combined to form characters in the traditional sense per se, but symbols coming from a secondary system to complement and run parallel with the traditional sinoglyphs - just like how Japanese has two additional systems running on top of the Kanji.

I've selected a set of 36 意符, which are:

金木水火土糸
人言骨肉疒示
衣食行辵車力
手足心口耳目
魚鳥牛豸蟲艸
申天玄義不々

Most of them should be familiar , except for the last line. I've chosen these very much reflecting my own philosophical position and analysis of the problems and weakness of Chinese philosophy - and the ills therein that perpetuate all kinds of 思維問題。the absence of the 女 radical should already speak volumes. But I've also include 申 which is 电 in the oracle bone script to cater to the modern technological vocabulary. I've included 天、玄、義 to forcibly inject metaphysical considerations into the character system so to remedy the problem that every time someone speaks about justice and metaphysics and lofty ideals it sounds or reads like nonsense. I've placed in the negation 不 after being inspired by the 合音字 such as 甭、孬、歪、覔 in Beijingese and 覅、朆、朆 in shanghainese. The 々 is an empty 意符.

申褱申夬


I won't give out the  answer here but it was clear to me that there is much promise. And so thorough the summer I worked on this system. I got some more interesting results. But eventually it seems that this system is going to be way more complicated than I thought, and will suffer from readability and differentiation issues. The problems are too numerous to be listed here. Although I sincerely believed it was the right direction to go, part of me knew it's very likely going to be a dead end. There are over 4000 聲母韻母 combinations in Cantonese - and that's already ignored the tones. There are also bizarre exceptional situations where there are no adequate corresponding 聲符. I was ready to freeze this.

Then I had an epiphany. It came today when I was playing mahjong with my super mother, my 姨媽, and my 三舅父。I got a particularly difficult hand so I took my time, and not before I long I was scolded by my 姨媽 for being 娿哿. This is a very uncommon term now - rarely seen in use by this generation. It means 慢吞吞 慢條斯理 with a dash of 三姑六婆ness. It's a rather sexist term, which you can already tell from its 女 radical. It was nice to hear such a classical Cantonese vocabulary term, and I did the usual analysis. Clearly it was very likely an ancient term, possibly dating back to pre-Qin times, for it is clearly a 聯綿詞, as it is a disallyabic morpheme, indecomposable, and the 韻母 Rhymes with each other. The first is clearly a phono-semantic character, but the second... something is funny.

哿 go4 = 加 gaa1 可 ho2

This is some kind of 反切合成字!an ingenious system that can massively reduce the number of constituent characters! It will be easy to learn, easy to write, and easy to read! This can either be combined with the previous phonosantic system, or run as a third parallel system. I haven't decided yet. But this is the greatest discovery since I learnt about Tangut characters! There is hope!




\section{所謂󱝚「粵語書面語」}

「粵語書面語」呢樣野,其實係殖民主義嘅concept。

任何一種語言,喺解殖之後,都係用返自己發音嘅文字。當年二戰結束之後,亞洲國家逐漸開始解殖殖,韓國人開始用Hangul,越南人開始用Chữ Nôm,宜家台灣人都用返Pe̍h-ōe-jī 。

「我用繁體字」呢種slogan,其實係一種自我矮化嘅行為。

現時中文呢種語言嘅標準書寫系統,就係簡化之後嘅文字。而廣東話做為一種同中文/華語完全唔同嘅語言,書寫系統係粵文,唔係繁體字。

呢種「繁體字」嘅觀念,亦都導致香港嘅教育係兩文三語,唔係三文三語,中國亦都殖民得名正言順。

類似「普通話係滿語拼漢字發音」呢種嘅講法,唔單止係對蒙古、滿族嘅種族歧視,仲好無知。滿語係一種獨立嘅語言,有獨立嘅文字,同宜家北京用緊嘅中文根本係兩樣嘢。

爭論中文真係好無聊,無論中文發音點變都好,可能有好多borrowing words,但佢哋中國本土嘅先至係正宗嘅中文。

都係果句,粵文唔係中文,繁體中文係外語。


\section{反問}
我地講粵語嘅,離不開漢字、漢系語、同漢經典嘅魔咒。其中一個一直抑壓住我地思維嘅最可惡魔咒,就係「反問」。

我地成日用反問,因為我地嘅語言驅使我地去用反問。我地幾乎語言上冇法不用反問黎釐清或說明我地嘅觀點或道理,因為(1)我地嘅詞彙缺乏抽象理想概念嘅詞,或者嗰啲詞口噏出黎硬係有啲古怪,好似個語言唔畀我哋講嗰啲關乎玄義價值嘅野—講野取易不取難,所以個個就口噏噏都係用反問你帶出自己嘅觀點同道理;(2)我地嘅語言習慣(陋習)已經形成咗,冇特別嘅意識去作出改變;(3)所謂嘅經典同先賢都用反問,一直缺乏理則嘅運用,具體嘅邏輯,截卒嘅論證,我哋想拾人牙慧都冇,而且如果我地嘗試直接論證,某程度上就會係違反已經成立咗同根深蒂固已成嘅論證文化,係唔埋堆嘅表現。

我地一定要有意識地抗衡呢種嘅語言陋習。我地唔好再反問,要直接說明。

反問嘅運作原理,就係要從問題引申出一個情感或理則演繹,而呢個情感或理則演繹最後會衍生出一個邏輯結論,而呢個邏輯結論係要係自悖,或不符一般普遍不予質疑嘅理念,繼而逼使思題者接受嗰個自悖嘅邏輯嘅否定。

由此可見,反問係一個非常之迂迴嘅論證方式。但係佢唔單只係迂迴其實佢亦都好低效率,成功率亦都唔係非常之差同低保證,論證質素亦非常之唔得掂。

首先,反問係一個問題, 人面對問題第一個嘅反應唔係去進行嗰個理則演繹,而係直接答咗個問題佢,咁我哋想要嗰個效果就冇咗啦。第二,你要得到嗰個自悖嘅邏輯,係要通過一段嘅邏輯演繹,但係可能人哋自己本身有其他嘅先設命題,而呢啲命題會影響理則演算嘅吞吐品,導致佢得唔出你想要嘅邏輯自悖結論。

說服力方面,反問作為修辭嘅小手法,其實冇乜野,但係問題就係在於反問(至少喺漢系語言裏面)有一種自韰嘅效果,容易導致一用反問就一發不可收拾。試問如果你喺度聽一個人演講,佢一輪嘴咁不停咁同你提出問題,仲邊有時間消化同進行以上嘅理則演繹?所以反問嘅重複使用,甚至濫用,係會導致說明嘅道理越來越膚淺、越黎越忽視細節:因為只有咁樣嘅命題或道理先至可以被反問所拉倒出黎畀人睇,深入啲高深啲嘅論證就係咁先。而正正因為咁,所以最後尾都係有理說不清,稍微唔思哲上完全死蛇爛鱔嘅人就會唔擺呢個邏輯,不歡而散。

我地要直接說出真理,唔好兜圈,唔好反問,要直接洗對家腦,否定同排斥反問主義,養成好嘅語言習慣。
