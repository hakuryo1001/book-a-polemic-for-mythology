\chapter{粵語拉丁化}

\section{拉丁化係粵切字嘗試避免嘅命運}

簡單講講粵語文字秩序發展所面對嘅棋局。按照現時嘅發展,最大嘅兩個玩家就係純漢字粵文同粵語拉丁文。而粵切字基本上係不成氣候,競爭地位同粵語諺文同粵語假名一樣不入流。而因為基本上粵語諺文同粵語假名基本上冇可能發展出去,所以發展潛能粵切字稍微比諺文同假名好少少。

純漢字粵文基本上同本字考嘅關繫係曖昧不清。本字考勝出嘅機會係0,但係佢可以喺純漢字粵文呢度嗰度\lr{氵}{}.啲落去咁苟且偷生。呢點冇咩好爭議,因為而家就已經係咁樣發展緊。但可惜嘅係,純漢字粵文係慢性自殺。過去一年嘅事已經將自殺速度大大加速,整個粵語生態圈會喺20年內完全崩潰,香港會上海化。

喺呢種文化被清洗咗嘅情況下,就好似真空必須會由空氣填滿一樣,拉丁粵文就會立刻暴發,就好似啲菇菌喺朽木度滋生咁。

注意,拉丁粵文迅速自然填補真空嘅前提,係有一大乍識講廣東話但係零漢字能力嘅人。呢個前提係唔確定嘅。我哋面對更加有可能嘅情況係粵語滅亡,普通話大行其道嘅情況。

基本上,我哋而家係唔會有粵拼文化出現。我哋太過漢字本位喇。要喺而家呢個情況度推粵拼,仲要有受整個群體認可嘅粵拼文學產品出台,根本就係冇可能。八間大學嘅士大夫全部群起而攻之啊陰功。

姐係話,拉丁粵文唯一可以出台稱霸嘅時機就係光復之後。如果光復之後你繼承嘅語言生態圈係仍然有粵語群同有文化資本生產力嘅粵語漢字使用家,你以粵拼作為粵語嘅文字系統呢個舉措就係打爛自己飯碗。如果你唔廢除漢字,羅漢並用,歷史話比我哋聽(1)呢條路係唔穩定,幾乎一定變成為拉丁專用,(2)你透過你自己文化所賺嘅錢一定會相對於現實下降。拉丁化嘅$_{best case scenario}$ 係越南(注意:係 $_{best case}$。呢個就係天花板),差啲嘅就係台語、客家語、壯語、彝語呢啲拉丁化咗嘅語文。你死又死唔去,但係又冇文化資本,毫無吸引力,你自己嘅舊野睇唔明,遺臣士大夫想同普通民眾講自己嘅文化遺產講都講唔到,跟住自己嘅細路全部以普通話同英文去出面搵食,成個共同體形同虛設。
如果粵切字失敗,香港人就一係繼續純漢字粵文而慢性自殺,一係就等拉丁化。而按照現實香港人嘅態度,會繼續純漢字,而光復之後會拉丁化。
個重點係,     $_{\text{all natural paths ahead lead to latinisation}}$. 如果我哋唔有意識地去扭軚,拉丁化就係我哋嘅命運。用個數學比喻,拉丁化係一個  $_{\text{local minimum}}$. 你嘅路徑係趨向嗰度。你入咗去就會好穩定咁喺嗰度發展。

有啲人覺得拉丁化好好。我哋粵字改革學會係唔會駁,亦冇得駁,因為佢哋覺得拉丁化好嘅原因就係繼續用純漢字唔得掂嘅原因。

但係,我哋要嘅,係留畀下一代香港人比我哋呢代好嘅一手牌。拉丁化,係由頭黎過。係要白手興家架。

粵語拉丁化嘅方案好多。大家可以睇睇覺得點,然後諗下,咁樣嘅文字秩序會畀到我哋點樣嘅資產去發展。



\section{On the whole, Jyutcitzi is preferable to Jyutping}


Using Jyutping to teach Cantonese would indeed be extremely helpful for the education and proliferation of Cantonese. However, using Jyutping to accompany the current writing system for Cantonese, is still a very suboptimal solution. Since Cantonese would still be written entirely and only with Chinese characters, which are not phonetic, Jyutping could only play an annotating role, like Hanyu Pinyin for Mandarin. Jyutping, would not be *a* writing system for Cantonese. Jyutping would be used to teach Cantonese, and might be used to annotate Cantonese readings of Chinese characters, but it would not be used as the script in which Cantonese is written. This is entirely like how pinyin is used to annotate Mandarin texts as a reading aid, but the system itself would not be used to write any text. This effectively means that even with Jyutping, as long as Cantonese is written with and only with Chinese characters (sinoglyphs), fluency in Jyutping would not imply any literacy whatsoever. You can know your jyutping very well, but you would still be illiterate if presented a vernacular Cantonese newspaper article written entirely in Sinoglyphs.

This is why some advocate Cantonese to completely abandon Chinese characters as the script to write Cantonese. Some believe that it is far better to completely romanise Cantonese—— i.e. write Cantonese entirely and only in jyutping. This would be akin to what the Vietnamese and the Zhuang have done.

This might appear to be the most straightforward and the simplest solution. After all, the logic seems undeniable. The Latin script has time and time again demonstrated its advantages, its flexibility, its impeccable infrastructure in terms of how every single computer on the planet is able to process it without any problems whatsoever. However, the cost of latinisation would be complete decimation and severance of one’s cultural heritage and cultural assets. This is not something to be glossed over. It would mean the decimation of access to old cultural products—— which are used to generate new cultural products, project soft power, and allow the reaping of economic benefits.

The Cantonese Script Reform Society believes the best way forward, is to adopt a script that is compatible and mixed with the Sinoglyphs—— just like how the Japanese’s writing system allows for the mixing of Kana and Kanji, how the Korean’s allows for the mixed use of Hangul and Hanja. We believe, Jyutcitzi, a phonetic script that is roughly based on the phonetic principle of *faancit*, offers the best way forward.

Jyutcitzi takes two Chinese characters, and combine them to form one single sound. For example the 廣 gwong2 in 廣東話 is composed of the initial (聲母) gw- and the final (音母) -ong. By combining 古[gw]u and 王w[ong] and composing them, we get one single glyph 古王. Tones would be optionally indicated by means of dakuten-like tone marks, (which are also like the tone marks in the bopomofo system that the Taiwanese use). In essence, we have created a phonetic system, in which Chinese characters would serve as phonetic letters. In particular, we have carefully selected the list of letters such that their spatial combination would yield the best aesthetics.

Jyutcitzi could also be combined with semantic components—— just like 90\% of all Chinese characters are phono-semantic characters, i.e. they are composed of a phonetic component (which roughly suggests the sound) and a semantic component (which roughly suggests the meaning). This means that Jyutcitzi is entirely in line with the evolutionary pathway of Chinese characters, and indeed with how Cantonese speakers have long been inventing new characters. In fact, this principle is not new at all. This principle of cleverly using Chinese characters to write one’s language has long been used by Hakka characters, Chu Nom (the Vietnamese characters), Zhuang characters, and even in certain Korean Gukja and Japanese Kanji as well. Certain script reform proposals from Japan in the late 19th century are exactly like this.

Most important of all, because of the 有邊讀邊 intuition (“pronounce by reading the phonetic component if there is one”) Cantonese speakers, especially Hong Kong Cantonese speakers have, this system, if adopted as *the* Cantonese script, could very well proliferate naturally and organically, without need of centralised education authority. Furthermore, given how this script can be used along Chinese characters, this script can seamlessly integrate into current Cantonese writing, thereby maximising cultural continuity and minimising cultural destruction. Most important of all, this script is phonetic, so it carries all the benefits of Jyutping, and that users of this script would be fully functionally literate.

And there is much left to be said about this script’s cultural potential and its capability in resolving the problems of 有音無字 (incidences in which there is no Chinese character for some spoken Cantonese word) once and for all, with a completely predictable, scalable, and logical system.

It is our vision to make Cantonese as dignified and as culturally and economically productive as Korean and Japanese. If our vision interests you, we would be very happy to answer your questions and tell you more about our project. We have already created an input system for this script, usable offline on microsoft docs. We are also working hard to bring this online so our reform can take off as speedily as we can. Cantonese deserves a script—— because it deserves dignity.

PS: in the picture included in the link, the tone marks are represented through soochow numerals. This is an old version. The most updated version now uses bopomofo / dakuten tone marks to indicate tones.

≈% https://www.facebook.com/permalink.php?story_fbid=142237560779560&id=100970604906256




\section{Surely jyutping is a subpar romanisation system}
Surely jyutping is a subpar romanisation system as it does not cohere well with English phonetic intuition. What’s the point of a romanisation scheme if it doesn’t allow the English speaker to enunciate your words and allow your culture to ascend into the Roman Republic like the Japanese?


Hmmm word segmentation. I have wondered why should 嘅 in many cases be considered a separate particle but not a suffix that modifies or marks an adjective. Harkens me back to the republican era where 的 (possessive) and 底 (adjectival) are differentiated. 紅色 的 蘋果 under this regime would mean “the apple owned by red” (which makes no sense) and 紅色底 蘋果  would mean “the red apple”. Surely these choices of word segmentation in Cantonese are nothing but the effects of pollution from the questionable and chaotic confusions of sinitic linguistics that arise from idiocy and confusion from the intellectual bondage of the  sinoglyphs.
Furthermore just because meanings are discretely identifiable does not mean it wise to separate them into different words. Why should we write “cannot” and not “can not”, “forever” and not “for ever”, “tomorrow” and not “to morrow”?
Surely there is an argument to be made that in fusing these words seemingly discrete units of meaning, morphemes if you like, into words, you give the writer more resources fo play with. For whence then can we have our Shakespeare, to lift this wretched pathetic and literatureless language that is Cantonese my mother tongue to rival French as did English? Where is our Shakespeare? Where is our King James Bible? Where is our manyoshu? God, give us our own Ju Sigyeong, and let him infiltrate into the bowels of Google and Apple so our language may ride the Roman Republic and multiply across the stars.





\section{書面化羅馬字不是完全的「我手寫我口」}

漢文可分成文言文(孔子字)、現代漢文(唐人字)以及口語文(這邊個人姑且稱之臺語字)這三個層次。其中前兩者為書面形式,不是我手寫我口,以求讀者通曉。

如果將口語文寫成羅馬字,是否真能達成「我手寫我口」之目的?實則不盡然也。

漢字由於不是全音素文字,所以沒有這方面底問題;即使有形聲偏旁,那也不是構成音位底部件。在某種程度上,漢字成功地在書面上消弭了每個人底口音差異問題,可缺點就是不能完全「有邊讀邊」。

臺羅也好、白話字也好,這些羅馬字轉寫系統皆為全音素文字,而每個人底口音會受當地腔調或俗讀所影響,明明指的是同一類的東西,若完全照個人口音寫出來,拼灋就會很「豐富」了,顯然不符合「讀者通曉」這些期待。好比說「啥物」都有 siánn-mi̍h、siánn-mih、siánn-mí、sánn-mi̍h、sánn-mih、sánn-mí、siám-mih、siám-mi̍h、siám-mí、sím-mi̍h、sím-mih、sím-mí 這麼多種講灋,在這些例子中如果讀者能照着羅馬字讀出發音,應該還是能知道都是指同一個意思;但不是每個案例底口音差異都是為人通曉的,好比說「枇杷」就有 pî-pê、gî-pê、kî-pê、khî-pê、tî-pê、pî-pêe 等,至於「萵仔菜」則諸如 ue-á-tshài、e-á-tshài、er-á-tshài、meh-á-tshài、gor-á-tshài、o-á-tshài。

這讓我也想到了漢字也有通假字,但漢字幾乎不會因為個人口音或是俗讀而改變形狀,頂多就是筆劃或形狀被約化(灋→法、㪅→更)、增加了偏旁(㐭→稟)、部件被替換(艷→豔)、偏旁被移走(羣→群),或是另外造個形聲字(个→個)給它。而且字形也因為官方或民間底相關規範,相對而言也穩定許多。這同時反映出所謂標準正字,其實都是被某些結構權力所「建構」出來的,甚至可異於學界所考證出來的本字。

矛盾的是,某些白話字使用者主張英文不會因為口音而改變拼灋,而白話字作為書面文字亦然(例如打馬字白話字不會有 or、ir、ee、er 這種韻母,故以為臺羅只不過是音標方案罷了);但面對臺灣話底腔調和俗讀問題,卻說這些都是被容許的,如此一來便忽視了作為書面文字所需的拼灋穩定性。倘若白話字是經過規範的書面文字,這是不是意味着羅馬字底標準,是不是同樣被某些人(包括那些規範底制定者)所「建構」出來的?書面化的羅馬字其實並無灋達成完全的「我手寫我口」?

至少從自己接觸了臺灣話文社群以來,從未見過全羅派會聲稱「我手寫我口」,他們更常標榜在書寫形式上能兼容外來語。此外,他們還會以文化對立底觀點來批判符碼,並認為使用漢字的過程,是為了將大中華階級觀引進臺灣且鞏固之。個人對此不以為然,不過這不是這篇要說的重點,就不對此多作評論,待下次再談這部分。

為什麼全羅派自己都不太會聲稱「我手寫我口」?因為理想中底書面化羅馬字,其調值底內部差異就必須被「音位化」,這樣才能確保口音各異的人們拼灋相同而統一,不致於造成理解困難,誠如英文那樣(不考慮美式、英式拼灋差異的話);甚至也得面臨制定「標準音」這個階段,不過這樣就得面臨一個問題,就是孩子們在學習的過程中,會很容易丟失應有的特色腔調,或是對特色腔調產生了自卑感乃至「不標準」的偏見。咱們別忘了,要是由現行「國家」這個體制去訂出標準音,那對特色腔調底傷害真的很難彌補得回來。

很現實的是,「書同文」是臺語人團結重要的階段,光是臺羅派和白話字派雙方都僵持不下了,更何況還是犧牲腔調差異這點。「書同文」縱然效益大,或許能讓臺灣話文活得再久一點,不過「團結」亦意味着犧牲內部差異有其必要,以此來換取族群內底穩定。

在此想問諸位,是不是該為了讓大家都讀得懂,而必須在書面形式這方面向其他口音妥協?到底什麼樣的口音才是「大家都讀得懂」的拼灋?如果你要使用全羅臺文,你想根據自己底個人口音來寫,還是以優勢口音來寫?≈

\section{Romanization is an ugly, undigified consignment to third world backwardness}

To romanize is to become Malaysia, the Philippines, Indonesia, and Vietnam. It is to forever consign and surrender oneself to a dominated and decimated civilization. It is to forever mark your own civilization as a lesser civilization, a late civilization. 

It is to forgo and abandon hundreds if not thousands of years of already accumulated success and effort of naturalization of the Sinoglyph. 

It is to forever give up calligraphy. It is to forever trivialize and make laughuable conlang calligraphy out of your funny attempt of naturalizing the latin script.

If you have had no writing system prior, perhaps it is still admissable and salvageable. You were learning from scratch. Humanity can forgive that. Cherokee is the example. 

But to switch writing systems, as a seasoned and established civilization, is to announce to the whole world that you believe all the generations before you were wrong - and you've decided to smash every porcelain and carved enamel, burn every mahogany table and bamboo chair, and tear up every painting and calligraphic work that you've inherited. 

And worst of all, these are all done by people who have always enjoyed culture. These are bourgeois revolutionaries, who have never experienced hunger or famine. They have lived their entire lives bathed in culture, however thin or precariously maintained they were - they have lived in culture. They have never imagined a single day being bankrupt, destitute, salvaging for anything that resembles capital. Having read a book or two, but having never designed or ran any major project or enterprise, they fashion and liked themselves as great revolutionaries, operating under a naive delusion that destruction will be naturally followed by innovation and prosperity. Lathspell they are.

% But what about Turkey? Do we not recognize that Islam is the most putrid, vile, and detestable ideologies ever known to mankind? If so, how could one fault Turkey for abandoning the Arabic alphabet in favour of the Latin? And if so, how can what's good and wise for Turkish be bad and stupid for Cantonese?

% Isn't the immediate reason a matter of proximity? And historical narratives? Turkey is neighbour to Europe, whereas Cantonese is 

% But how could 