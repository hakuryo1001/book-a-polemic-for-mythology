\chapter{普通話}

\section{講普通話係殖民嘅行為}
講普通話係殖民嘅行為。大義凜然仲要大條道理叫囂畀人歧視根本就係腦入水。唔係冇事就手足有事就蝗蟲,係發現原來我地信錯人,根本就唔係真係同我地企喺同一陣線,亦唔係真係同我地同心同德。佢只不過係透過香港黎凸顯自己係超級中國人。佢覺得有權喺香港講普通話,冇可能唔知個殖民性。姐係話,佢根本就只不過係兜個圈但之後都係殖民咁解。

唔好講到佢好hurt. 佢畀人窒唔畀講普通話,香港人日日如是,呢樹要同你講英文嗰樹同人講普通話,一黎個自價不菲嘅中國人就要即刻遷就轉台香港人都未嗲,邊度輪到佢嘈?仲有,死咁多人,佢呢啲雞毛蒜皮嘅野搞到滿城風雨,佢仲好意思?

我自己係非常非常想世界各地嘅人同我地行埋一齊,但係咪覺得你同我地行埋就好巴閉好大支野要乜都包容。粵語係香港嘅語言,而且應該係至尊語言。你入黎香港人無任歡迎,但唔好妹仔大過主人婆客家當地主。好煩。好討厭。


我地必須日本化,做到日本已經成功做咗嘅之餘,我地仲要超越佢哋。我地要成功將我地嘅文化發展到班鬼佬同大陸人仆崩鼻過黎崇拜我地同加入我地,要佢哋為自己可以講到粵語而自豪,要令佢哋自願同興高采烈咁摒棄自己嘅語言,就好似班入日嘅Gaijin淨係講日文咁。我地仲要令到佢哋自己嘅圈子都內部講粵語,就好似好多honkies abc 同喺英國讀書得耐嘅香港人自己圍位都會講英文。



因為絕對冇好處。唔玩呢個遊戲,只會俾人覺得你冇了,你廢,你玩唔掂先至喺度發爛渣。輸者無抗議之直。呢個現象一直都冇徹底打破: 抨擊姨得最犀利嘅,不外乎胡適嘅「八不主義」、 五四嘅新文化同新文學運動嗰陣嘅文風新倡議,同埋毛澤東主張嘅「大眾文學」。雖然股民同文言已經係所有一個需要用語文推進發展嘅範疇中被白話文所取代,但係呢種嘅引經義忱沒而不歿,陰魂不散,揮之不去,仲係死纏爛打。之所以係咁,有所以由官話演化出嚟嘅「白話文」再次有文言分離嘅跡象,漸漸演變成為「白言文」。

引經義忱最揦脷之處,就係佢驅使同鼓勵,趨逳啲引經言忱越嚟越難名,越嚟越難拆,唔搞到你𢱑晒頭都唔放棄,引嘅經典越刁鑽就越顯得你學識淵博,用嘅詞越難讀越睇唔明月冇辦法望文生義就越顯得你思考深邃。引經言忱,咁樣催生左一種秘語言忱:引經義忱秘語義忱。

秘語義忱天下就係一個用舊語主宰今事嘅義忱。喺呢個義忱之下人會變得思哲上不誠實,唔老實,爽韰為上,真想義理遺下。人只求得到嗰一下嘅韰,同埋自己身邊群組嘅認同:到底有冇道理,有冇玄理,有冇義理,話叉知佢。而正正因為咁樣,冇晒動力去以個人,獨立,新穎,批判性嘅視覺去剖析事情。冇新嘅言忱,冇新嘅意忱,冇——嘅義忱,一切都係舊酒新樽。賦予墨水靈魂嘅唔係真、實、啱、確,嘅玄理同玄義,而係死念。墨水都變得污穢。

懶醒,懶而不醒,就係一個清醒同誠實有勇氣凝視真理嘅人睇嗰個上海妹妹引用《鄭公克段於鄢》個鋪嘅唯一結論。


我地講粵語嘅,離不開漢字、漢系語、同漢經典嘅魔咒。其中一個一直抑壓住我地思維嘅最可惡魔咒,就係「反問」。

我地成日用反問,因為我地嘅語言驅使我地去用反問。我地幾乎語言上冇法不用反問黎釐清或說明我地嘅觀點或道理,因為(1)我地嘅詞彙缺乏抽象理想概念嘅詞,或者嗰啲詞口噏出黎硬係有啲古怪,好似個語言唔畀我哋講嗰啲關乎玄義價值嘅野—講野取易不取難,所以個個就口噏噏都係用反問你帶出自己嘅觀點同道理;(2)我地嘅語言習慣(陋習)已經形成咗,冇特別嘅意識去作出改變;(3)所謂嘅經典同先賢都用反問,一直缺乏理則嘅運用,具體嘅邏輯,截卒嘅論證,我哋想拾人牙慧都冇,而且如果我地嘗試直接論證,某程度上就會係違反已經成立咗同根深蒂固已成嘅論證文化,係唔埋堆嘅表現。

我地一定要有意識地抗衡呢種嘅語言陋習。我地唔好再反問,要直接說明。

反問嘅運作原理,就係要從問題引申出一個情感或理則演繹,而呢個情感或理則演繹最後會衍生出一個邏輯結論,而呢個邏輯結論係要係自悖,或不符一般普遍不予質疑嘅理念,繼而逼使思題者接受嗰個自悖嘅邏輯嘅否定。

由此可見,反問係一個非常之迂迴嘅論證方式。但係佢唔單只係迂迴其實佢亦都好低效率,成功率亦都唔係非常之差同低保證,論證質素亦非常之唔得掂。

首先,反問係一個問題, 人面對問題第一個嘅反應唔係去進行嗰個理則演繹,而係直接答咗個問題佢,咁我哋想要嗰個效果就冇咗啦。第二,你要得到嗰個自悖嘅邏輯,係要通過一段嘅邏輯演繹,但係可能人哋自己本身有其他嘅先設命題,而呢啲命題會影響理則演算嘅吞吐品,導致佢得唔出你想要嘅邏輯自悖結論。

說服力方面,反問作為修辭嘅小手法,其實冇乜野,但係問題就係在於反問(至少喺漢系語言裏面)有一種自韰嘅效果,容易導致一用反問就一發不可收拾。試問如果你喺度聽一個人演講,佢一輪嘴咁不停咁同你提出問題,仲邊有時間消化同進行以上嘅理則演繹?所以反問嘅重複使用,甚至濫用,係會導致說明嘅道理越來越膚淺、越黎越忽視細節:因為只有咁樣嘅命題或道理先至可以被反問所拉倒出黎畀人睇,深入啲高深啲嘅論證就係咁先。而正正因為咁,所以最後尾都係有理說不清,稍微唔思哲上完全死蛇爛鱔嘅人就會唔擺呢個邏輯,不歡而散。

我地要直接說出真理,唔好兜圈,唔好反問,要直接洗對家腦,否定同排斥反問主義,養成好嘅語言習慣。
