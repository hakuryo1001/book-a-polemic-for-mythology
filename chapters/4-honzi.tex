\chapter{漢字}


\section{叫唐字}
叫唐字,唔好叫漢字。

Oxzach, [29/8/2021 9:14 AM]
It highlights the history of Sinificiation of the canton region which only seriously began in the Tang Dynasty.

早於2007年、2009年,便有粵文維基百科用戶提出應把「唐字」改名為「漢字」、「唐文」改名為「中文」,但皆被中國大陸用戶聲稱以「沒有共識」、「維護傳統粵語」為由提出反對。但隨著越來越多維基人質疑條目的命名不是普遍用法,只是小圈子徹頭徹尾的自相呼應,這些討論在2012年、2019年再次掀起。而到了2021年,「唐字」條目終於改名為「漢字」,[3]而「唐文」條目終於改名為「中文」。

Oxzach, [29/8/2021 9:15 AM]
Also calling it Tongzi gives us a unique and differentiating quality in the sinosphere


- 我哋點樣叫漢字,係彰顯都我哋嘅對漢字嘅歷史認知,同埋蘊含咗我哋同支拿帝國嘅關係
- 用「唐字」為漢字嘅叫法,可以好明顯區分度我哋同其他諸夏語言
- 恢復返舊有嘅叫法,係彰顯返同重建返我哋自己而家嘅文化同嗰陣時ge文化嘅啦能、連續性、一脈相承性。
- 我哋目標唔係唐字同漢字都通用,而係「漢字」呢隻時文俾人打成過街老鼠,好似「西紅柿」「魚丸子」咁難聽,咁obviously係支語。

- 所有撐用漢字而唔係唐字嘅人都係輔助緊語言殖民
- 提日韓嘅人都係冇錢扮闊佬,太監扮馬撚

話「『唐字』呢個用法,已經斷咗攬,瓜咗,好應該退位讓賢」嘅人,係中文先生,係叛語者,係會反對我哋恢復返自己本有詞彙,淨係畀藍青化唔俾恢粵復古嘅哎吔叛徒。

其實廣大粵語語言學界已經笑咗呢種特登為唔同而唔同嘅「粵維式粵文」好耐 《-- why do we care? They have no vision. They don't know what they want. They don't have an architectural plan.
粵維就梗係「用粵辭」。「漢字」都係粵辭。只有喺「就算罕見都好,總之特登揀個同其他漢字文化圈語文唔同嘅詞語」先會俾「唐字」贏。其實廣大粵語語言學界已經笑咗呢種特登為唔同而唔同嘅「粵維式粵文」好耐[1],而「架撐」等等唔係最常用但都普及嘅用詞已經係粵維以外嘅粵文社群接受到嘅極限。驚俾人同化嘅話,唔會在於粵文用多幾個內容語共通詞,反而因為粵維用詞生僻搞到未睇過粵維嘅粵語母語者唔夠膽貢獻粵維,到時就真係拱手相讓喇。<-- on9



\section{本字考的根本哲學同政治取捨}

本字考的根本哲學同政治取捨固然與粵語是敵對關係,但最好的處理方法不是陷入與本字考範式中討論哪個正字對對對或本字考的方法論哪個好好好,而是不要跟他們討論,直接照單全收,全部出現過北被推舉為「正字」的漢字全部接受,然後放進字典,全部都納為可接受寫法。一方面這樣可以使這些冥頑不靈的士大夫永遠跟他們一樣白痴無聊的士大夫狗咬狗骨至死,一方面我們可以大大增加我們粵語詞彙的書面文化多樣性。基本上就是採取日語熟字訓的作法。反正粵切字在這樣的文字秩序中是有絕對優勢的,士大夫要做文章就由他們吧。

\section{點解漢字專用粵文對廣東話黎講係慢性自殺}

% \begin{center}
% \rotatebox{-90}{\fbox{\begin{minipage}{10em}
% \CJKfamily{BabelStoneVert}\CJKmove
% 『朝发轫于苍梧兮,\\
% 夕余至乎县圃。\\
% 欲少留此灵琐兮,\\
% 日忽忽其将暮。\\
% 吾令羲和弭节兮,\\
% 望崦嵫而勿迫。』
% \end{minipage}}}
% \end{center}

% \begin{center}
%   \begin{tabular}{|c|c|c|}
%   \hline
%   \VertCell{第一列} & 
%   \VertCell{第二列} & 
%   \VertCell{第三列} \\ 
%   \hline
%   \VertCell{内容 1} & 
%   \VertCell{内容 2} & 
%   \VertCell{内容 3} \\ 
%   \hline
%   \fbox{\VertCell{更多内容 1}} & 
%   \fbox{\VertCell{更多内容 2}} & 
%   \fbox{\VertCell{更多内容 3}} \\ 
%   \hline
%   \end{tabular}
% \end{center}

\begin{longtable}{|l|p{8cm}|}
    \hline
  
    \textbf{語言} & \textbf{}                                                                                                                                                       \\
    \hline
    \jcz{}
    \batang{}
    「中文」        & 人人生而自由,在尊嚴和權利上一律平等。他們賦有理性和良心,並應以兄弟關係的精神相對待。                                                                                                                     \\
    \hline
    粤語          & 人人生而自由,喺尊嚴同埋權利上一律平等。佢哋有理性同埋良心,而且應當以兄弟關係嘅精神相對待。                                                                                                                  \\
    \hline
    上海話         & 人人生而自由,拉尊嚴脫仔權利上一律平等。伊拉有理性脫仔良心,並應以兄弟關係個精神相對待。                                                                                                                    \\
    \hline
    客家話         & 人人生而自由,在尊嚴同權利上一律平等。佢丁人賦有理性同好心田,並應以兄弟關係個精神相對待。                                                                                                                   \\
    \hline
    福建話         & 人人生而自由,在尊嚴合權利上一律平等。因賦有脾胃合道行,並著以兄弟關係的精神相對待。                                                                                                                      \\
    \hline
    日語          & 寸部天乃人間波、生礼奈加良仁之天自由天安利、加川、尊厳止権利止仁川以天平等天安呂。人間波、理性止良心止遠授計良礼天於利、互以仁同胞乃精神遠毛川天行動之奈計礼波奈良奈以。                                                                            \\
    \hline
    贛語          & 人人生而自由,在志向跟權利上一律平等。渠們賦有理性跟良心,並理當以弟兄義氣相對待。                                                                                                                       \\
    \hline
    天津話         & 人個頂個生而自由,在尊嚴和權利上般兒般兒大。他們趁理性和良心,並應以兄弟關係的精神相對待。                                                                                                                   \\
    \hline
    韓語          & 毛木人間𠃍\tb{他}{厂}仒\tb{那}{乙}\tb{\lr{夕}{夕}}{厂}部\tb{他}{仒}自由又于厼㐓尊嚴果權利厂\tb{㇏}{叱}仒同等下夕。人間𠃍天賦的\tb{乙}{〇}又 理性果良心乙賦與\tb{馬}{加}\tb{牙}{叱}\tb{乙}{〇}厼覀又兄弟愛厶精神〇乙又行動丷余𠃌\tb{丷}{万}夕。  \\
    \hline
    越南語         & 畢哿每𠊚生\lr{罖}{出}調特自由吧平等𧗱人品吧權利。每𡥵𠊛調特造化班朱理致吧良心吧勤沛対処𠇍饒冲情英\lr{女}{奄}。                                                                                                  \\
    \hline
    壯語          & \scalebox{0.5}[1.0]{扌}\scalebox{0.5}[1.0]{󰖖}佈佈𮜃\lr{丁}{刂}𨑜\tb{云}{天}就𠷯自由,尊嚴𪝈權利佈佈平等。伝𠷯理性𪝈良心,應當待㑣\lr{彳}{比}\lr{彳}{農}一樣。 \scalebox{0.5}[1.0]{扌}\scalebox{0.5}[1.0]{󰖖}
    \\
    \hline
    \caption{  個々都係「漢字專用」,都係「中文」,咁點解係要學你嗰隻?}
  \end{longtable}
  
  
  
  % \begin{center}
  %   \begin{tabular}{|c|c|c|c|c|c|c|c|c|c|c|}
  %   \hline
  
  %   \VertCell{壯語} &
  %   \VertCell{韓語} &
  %   \VertCell{越南語} &
  %   \VertCell{日語} &
  %   \VertCell{天津話} &
  %   \VertCell{客家話} &
  %   \VertCell{福建話} &
  %   \VertCell{贛語} &
  %   \VertCell{上海話} &
  %   \VertCell{粤語} &
  %   \VertCell{「中文」} \\
  
  
  %   \VertCell{佈佈𮜃\scalebox{0.5}[1.0]{丁}\scalebox{0.5}[1.0]{刀}𨑜\scalebox{0.5}[1.0]{云}\scalebox{0.5}[1.0]{天}就𠷯自由,尊嚴𪝈權利佈佈平等。伝𠷯理性𪝈良心,應當待㑣 \rotatebox{0}{\scalebox{0.5}[1.0]{彳}\scalebox{0.5}[1.0]{比}}
  %    彳比彳農一樣。}
  %    &
  %   \VertCell{毛木人間𠃍他厂仒那乙夕夕厂部他仒自由又于厼可乙尊嚴果權利厂㇏叱仒同等下夕。人間𠃍天賦的乙〇又	理性果良心乙賦與馬加牙叱乙〇厼覀又兄弟愛厶精神〇乙又行動丷余𠃌丷万夕。} &
  %   \VertCell{畢哿每𠊚生罖出調特自由吧平等𧗱人品吧權利。每𡥵𠊛調特造化班朱理致吧良心吧勤沛対処𠇍饒冲情英㛪。} &
  %   \VertCell{寸部天乃人間波、生礼奈加良仁之天自由天安利、加川、尊厳止権利止仁川以天平等天安呂。人間波、理性止良心止遠授計良礼天於利、互以仁同胞乃精神遠毛川天行動之奈計礼波奈良奈以。} &
  %   \VertCell{人個頂個生而自由,在尊嚴和權利上般兒般兒大。他 們趁理性和良心,並應以兄弟關係的精神相對待。} &
  %   \VertCell{人人生而自由,在尊嚴同權利上一律平等。佢丁人賦有理性同好心田,並應以兄弟關係個精神相對待。} &
  %   \VertCell{人人生而自由,在尊嚴合權利上一律平等。因賦有脾胃合道行,並著以兄弟關係的精神相對待。} &
  %   \VertCell{人人生而自由,在志向跟權利上一律平等。渠們賦有理性跟良心,並理當以弟兄義氣相對待。} &
  %   \VertCell{人人生而自由,拉尊嚴脫仔權利上一律平等。伊拉有理性脫仔良心,並應以兄弟關係個精神相對待。} &
  %   \VertCell{人人生而自由,喺尊嚴同埋權利上一律平等。佢哋有理性同埋良心,而且應當以兄弟關係嘅精神相對待。} &
  %   \VertCell{人人生而自由,在尊嚴和權利上一律平等。 他們賦有理性和良心,並應以兄弟 關係的精神相對待。} \\
  
  
  
  
  %   \hline
  %   \end{tabular}
  % \end{center}
  
  
乜野係「漢字專用」?「漢字專用」就係淨係用漢字黎寫野。乜野係「粵文」?「粵文」就係將粵語口語完全透過文字表現於書面嘅文。咁,「漢字專用粵文」,就係淨係用漢字黎將廣東話口語完全表現於書面嘅文。

我哋而家大部份嘅粵文都係漢字專用,譬如《迴響》、蘋果,同啲$_{youtube}$字幕嘅粵文,都係漢字專用。

\subsection*{多語漢字專用嘅集體博弈不穩定}
如果廣東話以漢字專用嘅模式去書寫粵文嘅話,廣東話呢種嘅語言就會慢慢死亡。點解?因為廣東話呢者口講嘅語言,就會因為佢嘅書面語,係淨係用漢字寫。而喺一個時空之內,世間上最到只可以有一隻語言嘅書面語係漢字專用。如果世界上有多個一隻語言係漢字專用緊,就會構成唔穩定。而呢一個唔穩定嘅情況,會好快陷入競爭動態,到得返一個留低嗰陣先至會停止,回歸穩定。

\subsection*{漢字學習成本極高,係要學嘅話就一定係學利益最高嘅}
人要考慮邊隻語言去學或者用嗰陣,我哋係無法避免利益行頭嘅成本效益衡量分析。大部分人係唔會脫離到市儈嘅效益運算。人學語言,基本上籠統而言,就梗係盡求以最少嘅成本同負擔,就可及換取到最大嘅利益。

而好多時候,一個人去學習同使用某一隻語言嘅成本,係同嗰隻語言嘅文字掛鉤。如果兩者語言都係用同一隻文字,咁佢哋就會有一樣嘅文字學習成本。而對於用開其他文字嘅人黎講,兩隻都係用漢字嘅語言,學習成本會係差唔多——都係圍繞住漢字嘅學習成本黎計數。

\subsection*{漢字專用,客源重疊}

漢字嘅學習成本,係非常高。而個使用漢字到能夠存取利益嘅門檻,亦係非常高。基本上所有漢字專用嘅語言,都係面對住同樣嘅情況:學習成本高,啟鎖取利路途長、所需累積語言能力高。既然係咁,唔去學預計效益最高嘅語言嘅機會成本效益比就會大到難以承受—— 而大部分以利行頭嘅人,睇到呢一點,選擇已定。佢哋就必定係會去學效益最大嘅漢字專用語言,其他嘅都置之不理。而而家效益最大嘅漢字專用語言,係普通話,而唔係廣東話。所以,如果廣東話同普通話都一樣係漢字專用,廣東話嘅客源就必定輸畀普通話。

\subsection*{漢字學習成本極高,學完一隻漢字專用語言,難會再砌另外一隻}
雖然學咗一隻再學另外一隻,譬如學完普通話再學廣東話咁,因為你已經把握咗漢字,所以學第二隻漢字專用語言嗰陣可以慳返唔少成本。但係所慳返嘅同所賺到嘅比例唔係吸引得拒無可拒,加上唔係話個個都咁有魄力去不停咁學,好大機會出現嘅情況就係一百個人學咗一隻漢字專用語言嘅人裏面,之後再去學另外一隻嘅只有極度之少。姐係話,一旦一個人去咗學普通話之後,你想佢再兜返轉頭學廣東話?想創你個心。最緊要嘅係,所有漢字專用嘅語言都純粹因為佢哋係漢字專用而係死敵。佢哋因為自己係漢字專用,而逼使咗大家互相嘅客源為同一班人。而廣東話,係因為佢嘅政治同經濟現實,唔會鬥得贏普通話架。

\subsection*{中產:見利遷語,語言不忠}
仲有,以上嘅論點同分析,唔係淨係適用於外國人,而係粵語母語人士都適用。粵語為母語嘅父母,只要語言忠誠度稍微略低,比較容易見利忘義,就會表現曬上述嘅「棄粵追普」現象。加上嗰種完全無知嘅「咪都係用中文字!咪都係中文!」嘅思維,粵語為母語嘅父母就係將自己嘅子女送曬去學普通話,廣東話就係「無所謂啦」處理作罷。你見到而家幾多中產家庭由細路出世開始就淨係講英文,就已經可見端倪。粵語同普通話直接喺同一個文字體系度競爭,情況就更加差。廣東話,就會咁樣喺漢字專用嘅語文秩序度,監生被棋局餓死。一山不能藏二虎嘅規則顯然易見。漢字專用嘅語言,只可以有一個。

\subsection*{要漢字專用,就要坐穩「中文」嘅帝位}
呢個亦意味,如果你想廣東話係漢字專用,你就必須使所有其他都係漢字專用語言,一係就停止漢字專用改為「漢甲混用」,一係就吞噬佢哋。要漢字專用,你就要變成「中文」。要變成「中文」,你就要將所有其他嘅變成你嘅方言。

廣東話要漢字專用,就要殺死所有其他漢字專用嘅語言。

\subsection*{「阿乃椅子羽個乃教師二有馬下」—— 如果日文係漢字專用,你學得黎?}
曾幾何時,日本、朝鮮、越南,都係以漢字專用嘅語文秩序黎書寫佢哋嘅語言。日文嘅《萬葉集》就係咁樣寫「阿乃椅子羽個乃教師二有馬下」(昨日、あの椅子はこの教室にありました。)寫日文,論盡程度有而家粵文嘅漢字專用過之而無不及。

到咗而家,日本已經係假名漢字混用,韓國就諺文專用,而越南就係已經全盤拉丁化。因為漢字專用而語言消亡畀中國嘅中文吞噬嘅可能性因此而減。試問,如果佢哋仲係用漢字專用緊,論論盡盡咁去寫佢哋嘅語言(如圖),擺喺普通話隔離,你仲會唔會咁容易有心力去學佢哋?但係中國內嘅諸夏語言幾乎個個都仲係用漢字。上海嘅上海話、潮州嘅潮州話、惠州嘅客家話,安徽嘅安徽話,南昌嘅贛語,仲有香港嘅廣東話,通通都係用漢字專用。喺咁嘅情況下,你有咩誘因去犧牲學習普通話嘅機會,轉移去學其他嘅漢字專用語言?何來划算?答案就係:一啲都唔划算。而唔划算嘅後果就係你漢字專用嘅語言慢慢被淘汰,直至你迷失於歷史長河之中。

上述嘅演化機理已經喺香港展開咗好幾年。黎香港讀書嘅外國學生冇幾個係會學廣東話,甚至更加會因為已經學咗普通話,加上漢字同中文學術界嘅不誠實宣稱廣東話係中文方言(試問「話」點可能係「文」嘅「方言」?),個個都不屑學廣東話,不屑融入香港本地文化,甚至厭惡我哋此類嘅要求,倒返過黎發狼戾。我哋自己嘅中產,就更加係表現咗語言不忠嘅極致。佢哋攞住英文喺世界各地橫羅語利,講廣東話嘅普通階層就繼續捱廣東話嘅貧窮。廣東話,變緊一隻窮人同$_{non-stakeholder}$嘅語言。

\subsection*{漢字專用係自殺,粵漢混用可奔日韓}
漢字專用,係慢性自殺。如果廣東話要有希望可以逃出生天,要成為好似日文、韓文、越南文、甚至係好似英文、法文、德文咁偉大嘅語言,我哋就必須放棄漢字專用,放棄對漢字嘅迷戀。粵語配有繼續生存落去發展嘅天直—— 我哋配有自己嘅文字。粵切字,就係一個可以畀粵語直奔日韓嘅文字系統。粵切字,故此,應該成為粵字。粵切字,就係等住成為我哋未來嘅粵字。





\section{簡體字係我哋嘅戰略隊友}

係時候又再重提一下我哋之前講過,相信令到好多人又嬲又㷫又o嘴嘅關於簡體字說話:簡體字,係香港人嘅朋友。一旦中國恢復繁體字,香港依靠「我哋係華夏正統」呢個構建出黎嘅論述就會立刻破產,依靠呢個論述借返黎嘅野就會全部落入中國嘅袋裏面。而我哋可以肯定,只要中國一日係使用漢字而唔採用啲咩拼音文字,繁體字係100\%會復活—到時,就會有一大咋大學教授啊,明報記者啊,啲作家啊,中學老師啊,就會出晒黎大大聲講「皇上英明」之類嘅說話,之前所有嘅深仇大恨全部一下間唔記得晒。可能會連提議恢復繁體字嘅官同下令出兵到香港係同一批人都會拋諸腦後。大家千祈唔好畀麻木嘅憎恨蒙蔽咗自己所身處嘅博弈生態圈,講故仔唔好唔小心呃埋自己,要小心仔細睇清楚自己信嘅野內裏有冇矛盾:以「香港係華夏正統」自居,同「香港同中國唔同」,係有矛盾架。粵切字,某程度上就係觀察到呢個矛盾位,嘗試以最最最低成本、最唔挑起大家神經嘅手段,喺理論上(但係唔喺美感上)否定前者,肯定後者。我哋已經冇乜時間。好快香港人就會流亡四海。如果我哋仲係唔睇清呢個矛盾,仍然依戀前者而唔全面瞓身肯定後者,就算你去到外國唔同中國人行埋一齊,你嘅子女都會,就好似 canto-mando 嗰個 YouTube channel 裏面咁。當然明白呢個講法難 wrap your head around, 亦要時間接受,你甚至可能覺得接受唔到。有問題嘅,歡迎過黎討論~~


問題有兩個:第一個就係(似乎)呢度suggest如果我哋要引入或者自我衍生類似 tion, ment, ize 等等嘅詞綴,粵切字可能因為佢自己容許加意符嘅呢個操作,搞到最後尾都係失敗-自己冇意思嘅語素,只有語法標記作用嘅詞綴變咗做實詞語素。this is highly unsatisfactory.

(不必同唔好覺得咁樣引入係糟蹋自己,據說(我冇深究),日文裏嘅「的」呢個漢字最初發明係用黎模仿英文「tic」呢個詞綴)

第二個問題就係按乎同樣機理,粵切字最終都係冇辦法畀粵語容易存取或者衍生到存取玄上/超越(transcendental)或者形上(metaphysical)嘅義值(value)。

第二點當然係極具爭議性,而且連點解係或者唔係嘅立論都極之複雜。

點解我關心呢樣嘢?因為我認為似乎粵語,乃至普通話同所有用漢字嘅漢系語言,都係冇辦法存取到呢啲野。個後果就係個語言群體對真理呢種嘅概念把握同理解嘅共識變得非常不穩定。如果「真理」同「正義」呢啲嘅概念唔穩定,所帶來嘅災難性後果不言而喻。

我第一次留意到似乎我哋對真理嘅理解有啲古怪,係我開始學 liar paradox嗰陣。似乎粵語或者普通話要生產到英文嗰種一睇”this sentence is false” 嘅矛盾感,係難得多。之後再research一下,就會發現原來幾千年黎中國都係冇邏輯學。到民國時期連true 同 false嘅翻譯都十幾個—你可能會話「真」咪係對「true」囉。咁「false」呢?「假」?如果你細心諗諗又會好似唔係咁簡單。當香港嘅高官講 falsehoods嘅時候,你話佢哋講假話,佢哋講嘅野係「假」,你嘅意思好似同話「what they say is false」好似有啲唔一樣。

如果你再dig deeper,更會有西方學者話「真」同「true」呢個嘅對應,其實係魏晉時期由佛教引入先至有,先秦時期嘅意思唔係咁。(好似趙元任都係咁睇)

呢個似乎意味用漢字嘅語言,要存取到呢啲抽象嘅概念,係有困難。

當然你可能會指話:餵,「義」唔係抽象/玄上/形上概念?「道」呢?「仁」呢?「德」呢?「玄」呢?

呢個就係難搞嘅部分。

但我嘅 working hypothesis 就係漢字本身要存取呢啲概念係有困難。之所以咁樣有一個 working hypothesis, 係因為咁樣係 err on the side of safety.

粵切字要避開呢啲問題,最直接嘅方法就係完全字母化,或者將佢嘅意符子體系變成為一個平行系統,等你有一個類似諺文或者假名咁完全冇表意性嘅拼音系統。咁 tion ment ify 等等嘅詞綴就可以引入/衍生。假以時日,粵語亦會比較容易衍生到形上嘅義值詞彙。到其時,語言偽術就會唔需要辯駁都會自爆其醜。literally 講唔符合邏輯或者不符玄義嘅人,會聽起上黎自打嘴巴。我哋係100毛上面見到嘅嗰啲廢佬,以後講嘢都唔會咁要自信,亦唔會咁容易惹人落搭。this is exactly how the English language works. This, is part of what I sometimes call, 「English rationality」.

粵字改革,某程度上就係想做到呢樣野。所以,粵字改革,唔係淨係文字改革。佢仲係語言改革。



我哋可以肯定,大量出走英國嘅香港人,不出兩代,佢哋嘅粵語量詞就會退化到淨係得返「個」。呢個發展當然會令到粵語嘅色彩大大淡化啦,「生舊叉燒」同「生個叉燒」分別非常大,「一篤屎」同「一舊屎」同「一塊屎」同「一個屎」完全唔係同一回事。之不過但係,咁樣發展會有一大好處:就係完全打通咗量詞變冠詞(a,the) 嘅路徑,「個」就可以完全變成為百搭冠詞,咁科學性嘅討論就會方便同自由得多,說話嘅精準性亦會大大提升。Russell 嘅 “the current king of France is bald”嘅句子當中嘅思哲趣味同理則,喺粵語度呈現亦會更加形式上明顯同易把握。如果你諗到有辦法將粵語嘅量詞生態系統保存,但又可以將粵語嘅冠詞系統成熟化,請話畀我哋知。我哋會非常想聽。



有趣。當佢賦予-ER呢個後綴個人字旁「亻」,佢就將呢語素擺上咗一個獨立化嘅道路,亦實義化咗佢—佢唔再係就咁一個需要依附其他詞先至可以有意思嘅野,而係一個似乎有獨立性嘅野。呢個現象,有啲似古人用形聲字佢砌個音出黎,整咗啲聯綿詞出黎,譬如「琵琶、蜘蛛、駱駝、尷尬、葡萄」咁,但之後嘅人就(因為個意符?)以為「琵」同「琶」係獨立有意思嘅字。呢種嘅諗法好明顯係漢字讀壞腦同漢字打飛機打壞腦嘅結果—唔好笑,呢種咁白痴嘅諗法竟然到清朝先至有人推翻,到而家仲有人用呢種嘅思路去分析「LAA ZAA」兩個嘅本字應該係乜野。同時間,呢度呢個「亻⺍乍」,因為配上咗意符,很容易會變成一個有意思獨立成詞嘅語素,就好似「尷尬」而家喺大陸同台灣甚至已經傳到香港嘅「尬聊、尬舞、老尷」咁等等。呢個現象有好有唔好—最令我擔憂嘅就係可能意味住 OUR WORST FEARS CANNOT BE AVOIDED - 羍(THAT) 只要一日有意符,我哋就跳唔出漢字兆物觀嘅限制,存取唔到形上玄義。


\section{shit}

呢度我哋㨢一㨢返去粵切字嗰度。由上面嘅分析,我哋大概可以睇到,令到簡體字有「殘」忄夫么.嘅核心原因,係「草書楷化」引入咗啲繁體字唔存在嘅筆畫,而呢啲筆畫所散發出嘅美感,同繁體字筆畫體系嘅美感相斥,造成違和感。粵切字所選用嘅所謂「簡體」字符,全部都唔係草書楷化嘅簡體字字符,而係古代異體,民間異體,係避開咗「草書楷化」而生嘅美感問題。咁當然,呢個意味粵切字符號同符號嘅組合全部都符合繁體字美感。有啲美感組態係跳出咗,或者係繁體字裏面極低少見—譬如「爻」打橫放;「禾」有「八」「介」個頂又係「八」,重疊起上黎就會兩個「八」,呢個情況喺「谷」度出現就幾乎係犯規,而之所以咁「釜」先至會「父金合併」。


注意:我哋唔係問你點解反感簡體但唔反感繁體同日本新字體嘅理由,我哋係問緊點呢個反感嘅現象會係以咁樣嘅形式呈現。前者假定咗反對簡體字係一個選擇,所以反對係源自理由;後者係假設咗反感係現象,只有原因,而沒有理由。我哋唔以且唔可以以前者黎問,係因為前者會意味簡體字嘅反感至係口味問題,但「簡體字醜樣」係一個非常穩定嘅共識,唔似係口味問題。

咁,呢啲導致簡體字樣衰嘅原因係咩呢?有好多人畀好多唔同嘅理由。譬如咩:
1. 簡體字破壞漢字象形性(華->华,車->车,馬->马,門->门)
2. 簡體字求其用符號代替部件(鄧->邓,趙->赵,雞->鸡)
3. 簡化唔跟規律(觀權歡-> 观权欢,但係 灌罐-> 灌罐)
4. 胡亂減省更改部件(攝->摄,業->业,與->与,學->学,榮->荣,興->兴,這->这,龍->龙,龜->龟)
5. 多字合一(發髮->发,余餘->余,復覆複->复,幹乾干->干)
6. 無視演化歷史,強行恢復古字(電->电,雲->云,麗->丽)
7. 亂改聲符,或改用同音字,聲符向普通話靠攏(賓->宾,鬱->郁,認->认,憲->宪,溝->沟)
8. 改動消滅或減少了象形象意性嘅美感(國->国,塵->尘)
9. 草書楷化,違反漢字本有嘅筆畫體系(樂->乐,車->车,語->语,飯->饭,專->专,為->为,東->东,書->书,舊->旧,義->义)


今日同大家講一下「簡體字」。

唔駛擔心。我哋係絕對唔會討論啲乜「愛無心,親不見」咁老生常談講到爛而且其實毫無營養嘅半桶水學問。

點解要講?唔係因為有好多人又問又鬧我哋同共產黨有咩分別,而係因為大家對簡體字嘅歧視,已經非常嚴重不理性。問題唔係歧視,亦唔係嚴重,而係在於其不理性。而呢種嘅不理性,係會矇蔽香港人,係會導致香港人睇唔清自己個棋局,搞到落錯棋。

香港人畀咗「簡體字」一個蔑稱,「殘體字」。相信全世界冇一個蔑稱比呢一個更耐人尋味。香港人討厭憎恨簡體字,政治原因係次要,由心而發嘅真心美學反感,先至係真正原因。簡體字令人反感,係因為佢核突;核突在於其「殘」。簡體字就好似跌爛咗然後嗚哩孖叉咁重新組裝出黎嘅漢字。

佢令人反感同不安,就好似受輻射導致基因突變嘅人,無端端少咁咗隻眼,多咗隻手,三頭六臂,然之後我哋望落去所由心底萌生嘅嗰種不安。(呢個比喻唔係我哋講,而係一個大陸人同我哋講嘅)

簡體字核突,基本上根本就係一個純漢字語文世界嘅共識。而且係一個非常強烈同顯著嘅共識。呢樣野係無需爭辯,亦冇得爭辯。事實就係所有人都認為佢核突。

但係呢樣野係非常奇怪。普天之下皆以簡體為醜嘅事實呢一點,暗示住似乎美感判斷有客觀性,或者至少喺一個體系之內可以有客觀性。呢一點本身就已經非常驚人。姐係話,簡體字醜樣,係有原因嘅,如果唔係冇會可能有咁強烈嘅共識。亦即係話,我哋覺得簡體字核突,係有原因嘅,係因為美感邏輯有其宣判。我哋可能未必可以即刻精準地講得出呢啲原因係乜嘢,但係我哋可以肯定,我哋覺得簡體字核突,係因為我哋感知到呢啲原因。

咁,呢啲原因係咩呢?

有好多人畀好多唔同嘅理由。譬如咩:

簡體字破壞漢字象形性(華->华,車->车,馬->马,門->门)

簡體字求其用符號代替部件(鄧->邓,趙->赵,雞->鸡)

簡化唔跟規律(觀權歡-> 观权欢,但係 灌罐-> 灌罐)

胡亂減省更改部件(攝->摄,業->业,與->与,學->学,榮->荣,興->兴,這->这,龍->龙,龜->龟)

多字合一(發髮->发,余餘->余,復覆複->复,幹乾干->干)

無視演化歷史,強行恢復古字(電->电,雲->云,麗->丽)

亂改聲符,或改用同音字,聲符向普通話靠攏(賓->宾,鬱->郁,認->认,憲->宪,溝->沟)

草書楷化,違反漢字本有嘅筆畫體系(樂->乐,車->车,語->语,飯->饭,專->专,為->为,東->东,書->书,舊->旧,義->义)

改動消滅或減少了象形象意性嘅美感(國->国,塵->尘)

諸如此類,諸如此類。以上提出嘅理由,絕對不成互斥共耗嘅理由集合,但唔重要。重要嘅係以下兩個問題:

1。繁體字喺演化過程中,無論係「小篆->楷體」定係「古楷體->今楷體」,以上嘅簡化改動都曾經出現過,點解我哋又唔反感呢?

2。日本漢字嘅新字體亦有好多係好似簡體字咁簡化,點解我哋又唔反感呢?

注意:我哋唔係問你點解反感簡體但唔反感繁體同日本新字體嘅理由,我哋係問緊點寫個反感嘅現象會係以咁樣嘅形式呈現。前者假定咗反對簡體字係一個選擇,所以反對係源自理由;後者係假設咗反感係現象,只有原因,而沒有理由。我哋唔以且唔可以以前者黎問,係因為前者會意味簡體字嘅反感至係口味問題,但「簡體字醜樣」係一個非常穩定嘅共識,唔似係口味問題。

如果你認真、誠實、嚴謹地嘗試回答上面嗰兩個問題,就會發現係非常之難答。

我哋會點樣答呢?我哋會咁樣嘗試答:
(1)我哋之所以會覺得簡體字核突,最核心嘅因由來自「草書楷化」。草書楷化為簡體字引入咗違反繁體字嘅書寫筆畫,而又因為簡體字繼承住絕大部分嘅繁體字筆畫,所以簡體字內裏嘅美感理則係同繁體字嘅一樣。繁體字嘅美感理則判處草書楷化有罪,簡體字同樣使用同一個美感理則,所以簡體字都判處簡體字美感有罪。咁樣就解釋到點解簡體字會自己打自己呢個畸型現象。

但係注意,呢個答案係解決唔到點解我哋因為草書楷化而覺得簡體核突,但係日文同樣都有草書楷化,但又唔覺得特別明顯核突嘅問題。日本漢字類似草書楷化嘅有:實->実(实),劍->剣(剑),圖->図(图)

我哋可以回答話:日本漢字嘅簡化冇簡體字咁大動作,而且好多都大致保留咗繁體字嘅型態。加上假名本身就有另外一種美感理則,日文所呈現出黎嘅美感就別樹一幟,變相漢字美感嘅走動空間就大咗。

講到呢度,我哋探討一下以上討論嘅政治意味。好多人見到大陸十四億人用晒簡體字,就抌晒心口喊住話「哎呀嗚咦呼冇陰功啊,繁體字玩完啦」,然後就順便搬埋晒啲「崖山之後無中國」嘅論調,繼而論證「香港乃華夏遺民」嘅立論。

佢哋咁樣講嘅政治意味我哋遲下先講。先講佢哋


\section{漢字專用導致嘅語法頹敗}
漢字語文嘅美感趨使人感有減省書寫虛詞嘅壓力,結果導致語法結構殘省化,表達複雜關係嘅句子結構發展不良,導致思哲複雜性和精準性長期鬰廢不發,理則不習故頹。

\section{漢字專用導致嘅語法頹敗}
因為字典嘅選字形狀好多拼砌起上黎,完全唔·禾勺。美感唔得就即刻收皮。而且呢種延續啲韻書嘅方案,根本完全脫節離地,難學到死—但係按照現時讀音加上字型拼合考慮就可以大大提高無師自通嘅機率。呢個係一個壓到性嘅優勢。跟隨韻書只係一種除咗滿足士大夫階級嘅文化自瀆之外就完全毫無利益嘅極度無聊設計操作。呢樣野已經係《創會宣言》非常清楚道出咗。我哋唔需要玩文字遊戲,我哋要嘅文字系統。而基本上任何一個玩跟傳統韻書砌出黎嘅準文字系統,都必定非常複雜,規則繁多,例外處處,而且士大夫必定呢樣嘈嗰樣嘈,最後淨係搞到啲類似老國音同通字方案嘅無用玩意。之後佢地嘅發展歷史係點我哋下省萬五字。

耶魯同粵拼做唔到文字,同埋基本上一日有漢字就唔駛諗佢哋可以出人頭地。喺我哋已經全民識字嘅情況,唯一可以推到拉丁拼音嘅方法,就係極權廢止漢字。而唔使講,全面拉丁拼音化嘅代價就係全部文化嘅資產一鋪清袋。如果係咁嘅話,粵切字專用都好過全面拉丁化—粵切字專用至少可以假假地有個又唔係日本書法又唔係韓國又唔係漢字嘅書法傳統。唔通你去學越南人玩毛筆字寫拉丁字母?一定冇市場。故此,我哋可以見到,拉丁字母專用,成本高效益低,低過粵切專用。莫搞。

羅漢混用係死路。臺語同客家話都係用拉丁字母必然長期人口因漢字美感同經濟勢力而流失嘅例子。羅漢混用係必定會輸畀漢字專用嘅粵語白話文,而漢字專用嘅粵語白話文係死路。羅漢混用,就係全體語文風格港女化。這個問題就連專制硬推也解決不了,因為反感是源自兩種文字本身的美感理則。

沒有文化呢個唔係理由,因為根本粵語而家係零文化,否認係自欺欺人。策略性地構建論述,就更加應該咁樣自己對自己講。你同外人講粵語有咩文化係另外一個問題,可能仲可以策略上合義化到要吹大啲。但係對自己就必須承認,我哋係零文化,係乜都冇—否則就會出現「文化自信」所帶來嘅慢性自殺。而如果我哋睇睇邊一種嘅文字係最能夠畀我哋輕易大量同高質地生產自洽同身分絕不模糊嘅效益嘅話,純漢字同任何一個嘅拉丁化方案都必定入選唔到。沒有文化,可以做出黎。但係邊一種文字最方便去做呢樣文化構建工程,可以將成本同文化獲利嘅比例整到最大先?粵切字就係企喺度,同緊你講,佢係一個非常有潛能嘅選擇。

當然,好明顯我呢句「粵語係零文化」同「全面拉丁拼音化就會粵語文化一鋪清袋」係完全矛盾—兩者不可兼得。要解釋一下,我哋之所以必須同自己講「粵語係零文化」,係因為我哋嘅文化嘅內容係身分極低尷尬。我哋所以要咁樣同自己講,免得自己忘記或無視咗呢啲尷尬嘅存在。拉丁化嘅效果,係整到你冇嘢可以尷尬。

不見得花碼有幾難學。而且推花碼係由其他輕微嘅語文風格構建用處。花碼係我哋極度值得推廣嘅野—推廣成本極低,但回報可觀。試諗下,以後餐廳餐牌價格都係花碼,寫文件目錄係用花碼,咁樣係有龐大嘅文化標誌效益。而喺唔同地方都使用花碼,都可以提升回報。咁當然,呢啲雞毛蒜皮嘅嘢,可以改,但唔係而家,標音用咩系統呢樣嘢根本唔係好重要。當下嘅獨裁同選擇,卻又既定路徑依靠嘅美妙效果。我而家獨裁用花碼,會提升花碼喺群眾嘅認知,喺其他地方推廣花碼就會容易啲。有咗一定成效之後,你想用四角標音定係好似台灣嘅注音標調符號,睇點。

其實用花碼係咁多個選擇中最為策略上可取。你用啲注音標調符號一定畀人感覺係普通話化(?),四角標調學死人,阿拉伯數字美感上唔.力乍.丩生。

希望解答到你嘅提問。





\section{依戀漢字者}
依戀漢字者,十居其九文字手淫以為業
全面羅拼者,上七及八亂棋橫飛以為功



我再回應一下你第一段,但基本上要講嘅嘢已經講咗。讀錯音係一個非常細·爻兮·嘅問題,只需要喺需要嘅情況下標調已經搞掂。

我可以完全唔反對你話要展示廣東話特色呢個理由,但係呢個理由係唔會改變到一但強制標調就一大乍問題蜂擁而至。亦即係話,強制標調嘅代價,可能就係咩都搞唔成。

至於選字嘅問題,字音符唔符合之外非常重要嘅考慮就係字型美感。選擇咗嘅都係考慮晒。當然會有人反對,認為根本美感上未臻完善。呢啲.言臼今.言并力丁.都係可理嘅,但問題係喺呢個階段基本上所有嘅解決方案都係差過已經被選擇咗嘅嗰啲。故此,如果要改,就要等一陣—等到粵切字嘅邏輯已經通行咗喇,大家冇咁敏感喇,粵切字有咗hegemony之後,到時可以為民眾接受嘅選擇就會多好多,到時慢慢揀。

至於音譯嘅問題,我嘅建議係要訓練自己接受呢啲野,因為呢個係你存取同霸佔文學multiverse嘅最便宜同效益最高嘅道路,甚至可能係唯一道路。如果你要全部原創意譯,唔係唔得,但係有幾個問題,而呢啲問題係會導致意譯體系難以推廣。我呢度唔長列—個重點係,如果你而家咁牙煙嘅情況下唔包容外來詞,單靠意譯,你好容易仆街。

我絕對支持大力發展意譯或透過漢字或粵語詞根去構建新詞彙,以豐富我哋嘅語言。詩哲文數理科呢啲範疇更加係應該咁樣做。但係唔應該係專用。好危險。

此外,我相信大部分人都為我哋能夠消化外來詞深感自豪。「麻甩」一詞就係。問題係咁樣嘅過程喺漢字體系下幾乎次次都要係偷偷雞先至會畀。粵切字唔單止容許我哋光明正大去咁樣做,而且意符嘅支系統更加容許外來詞融入我哋嘅文化範式之中。





\section{正字本字的迷思}

香港民間近期興起了一系列的「粵語正字」活動。有的是民間自發的,有的是些社會中的知識和思哲份子所領導建設的。「粵語正字」活動在沒有任何政府機構或權威語言機構的領導扮演一錘頂音的大環境下,百花齊放、割據天下、家家爭鳴。

我是反對繼續維持這個沒有及缺乏統一權威語言機構的文字安娜奇(無政府狀態)的,但這個在這暫且不論。這篇文章在意分析貫穿所有這些正字運動派系的共同論述和價值前設,並希望點出對該等價值前設的堅持,不但有各種潛藏的迷思性和迷信性,其堅持更會持續維持著某種對粵語建立文字系統(語文化 graphization) 可以採用的手段有消極性和打壓性的負面影響。

所謂的「粵語正字」活動,其實是一個雜亂無章、沒有中央協調、各自為政的大民主語言運動。這樣的文字不是要抹黑或醜化,只是要明確不含糊地點出現實。可知道,正正因「粵語正字」運動沒有中央領導,故此當中的各派系的要求、主張、以至使用的詞彙和秉持的觀點都有所不同。茲等現象的其一具代表性的彰象,是「本字」和「正字」的混淆和對等化。

很粗略地說,一般人如果要對「正字」和「本字」做出某種按照字義的意思分析的話,可能會有以下的推論過程:所謂的「正字」,就是正確的字。正確的字就是原本當初創造出來來代表該口頭語音的文字。所以,要處理正字問題,解決方法就是要尋找本字。

這路邏輯是一個貫穿漢系語言中所有文字化工程的共同共有邏輯,但在其他的語文中卻較為少見:沒有人會說某某英語或法語詞應該怎樣怎樣串因為那個詞最初的串法是怎樣怎樣。這不是說詞源及詞演變史不是他們訂立和考究正詞串法的考慮因素,但可以肯定的是,這一定不是最為重要,也絕對不是決定性的因素。




\section{意借形借:與假借}
意借:與假借一樣,其運作原理都是把已經存在有自己字義的字借來表達新的意思。但與假借不同,假借是按照被借字的字音衍生新的字義(也就是借音不借義),意借是透過漢字的結構做出新穎的分析,並賦予或衍生新的字義。意借字數量很少,例子包括:
厹:九(狗,homonyn)+厶(公)=狗公
囧:代表某種無奈發飆發爛渣的表情,取其「囧」象😩之形。
耄:大陸網路用語。耄,从老从毛,亦毛澤東也。
冊:用於香港粵語詞彙「出冊」中,解「放監」,「冊」本解古代的竹冊,也就是古人的書,但在此「冊」因像監獄鐵籠之形就被重新釋形。
蟈:香港網絡用語。蟈,从虫从國,虫指蝗蟲,大陸人貶稱,固蟈,中國貶稱也。這個也許不能完全視為意借字,畢竟乃同音字,有假借成分。

So why does this matter? 因為 fundamentally speaking, 假借、另造新字、形聲字系統化等措施都有其限制,整體而言都不足以完全賦予粵語一套文字系統,不能百分百將粵語文字化 (ferguson 講嘅 graphisization)。活用、訂立、統一意借字,可以給予粵文發展另一條路。

當然,意借字有一定的地區和文化局部性,只有有限的跨地域時空流通性。所以某程度上可以提升粵文同曼文(written man[曼]darin)嘅文字體系距離,為粵語增加一層系統性和文字結構性的城門河。


\section{親,簡體字是個好東西}

我這一個在香港街頭見到簡體字會破口大罵的人,為什麼會說這樣如此不協調、大逆不道的話呢?

簡體字不美,或至少簡體字不夠繁體字美,是一個華夏文明的共識。當然,有多人會否定這個無可爭議的事實,說簡體字也很美——這樣論調的《人民日報》、《環球時報》文章多得是。但對於所有思哲上誠實,思維真誠,願意直視自己心中思緒的人,都得承認,簡體字的美感,真的不行。口裏說不,身體卻很誠實。香港、台灣、澳門的餐牌、單張、文件等出現簡體字,是因為反骨的經濟考慮。但大陸是沒有任何這樣的經濟考慮的:寫繁體字是不會吸引到任何有意思的新客的。既然如此,為什麼那麼多的餐廳和各式招牌都會自發使用繁體字呢?星馬泰的中文報紙,標題很多時候都是繁體字,正文才用簡體字。為甚麼?很簡單,因為靚。

這個「美感」不是單純指視覺上漢字的型態美,而是包括他的存在美和運作美。這個被破壞了的「美感」不只是那個我們看著「言」變成了「讠」,「金」變成了「钅」,「食」變成了「饣」而驅使我們不由自主地打冷震的「美感」,而是主宰漢字運作的理則系統。這個理則系統,非常美麗,而漢字的字形美,只不過是這個理則系統的一部分。而簡體字,就是破壞了這個理則系統。所以,其實香港人就簡體字強予的污名「殘體字」,是很有意思的——這個污名就正正道出了簡體字的理則崩壞。

這個理則系統以及其散發著的兆物觀,就是為什麼用簡體字的人會誠心誠意去學習繁體字,但用開繁體字的人最多只會用 簡體字,而不會學。

那當然,簡體字的理則崩壞不是全面的,否則根本就不可能出現學簡體字的人仍然對漢字產生敬畏和瘋愛的心。那些崩壞的實質例子,可見其崩壞的原因,也不是不存在於繁體字之中的。換言之,很多簡體字的弊病,那些與其餘理則系統不符的情況,繁體字中也有。但是,不同之處有二:一、繁體字自漢朝以來,有一千五百年的時間去適應、平撫、內化,和收服這些矛盾,並使他成為理則系統裏面的合法、合理、合義例外。譬如,有些字,明顯是部件過多,根本塞不進一個漢字方塊格裏面的,如「爨」、「釁」、「鬱」等等。「釁」是非常明顯的例子——你嘗試一下,可否成功地把「釁」不寫成長方形,而寫成規規矩矩的正方形?這個是不可能的——連你現在讀著的電腦字體也無法做到這一點!但是我們都接受了他,而無視他的矛盾。這是因為時間的磨合。但簡體字年輕,很多字要經過這個磨合的過程。還未磨合,故此出現違和感。這個很明顯待世代過去就會解決,但有些簡體字對漢字理則的衝擊,是不可能被漢字理則所融合的——除非漢字的理則本身產生改變。這些字包括「发」(發、髮)、「干」(乾、幹、干)等一簡多繁的合併,以及因草書楷化而新增的筆畫部件,如以下字中出現的部件:书、专、门、发、马等。這些都是漢字理則無法在不改變自己的情況下收容的。第二中的理則破壞,是永久的——但是!漢字理則不是脫然存在的,而是透過漢字而存在。也就是說,漢字如何,漢字的理則則如何。這些有另外一種理則主宰運作的漢字,其實是改造了漢字的運作理則。用「破壞」這個詞,是有價值取向的,而這個價值取態是先於漢字的——漢字可能可以給予你理據去取購這個價值取向,但這個價值取向不是漢字的理則蘊涵。

為什麼簡體字是個好東西?因為他散發著的理則,是違反傳統中國秩序的。這個亦意味,一個用簡體字的中國(斯指自持為中國、華夏正統、華夏文明的合道守護者的政權),按照其本身的政治論述,是理虧的。是不完全合義的。這也就是盧安迪、陳雲、陶傑等依戀中國之派的腔調:他們對中國的批評,不是出於中國對人的摧殘,而是出於中華人民共和國的道統理虧。如果,中共改用繁體字,他們不但會拍掌叫好,也會立即投共——也就是投中,回到祖國的懷抱裏。

簡體字的存在,促成了中國秩序的不穩定和自我矛盾,讓所謂的「中華正統」脫離了河南,脫離中原,轉移到香港台灣等化外之地,並在西方的薰陶下孕育出新的思維。同時間,正因中國秩序在中國核心無分完全理直,依然依戀中國秩序的人被迫否定中共,並接觸西方的價值觀,甚至否定「中國」的整個價值體系。如果,中共改用繁體字,那中共就與中國再無有意思的分別——中共就可以堂堂正正打著中國的旗幟來為非作歹,真心支持的人也會更多,怒號反對的人要找原因也會難得多。現在搞你香港,還會有人覺得這是不正統的匪寇在消滅華夏的正統。大陸變成正統了後,學寫繁體字,想做個真正中國人的陳秋實還會出來嗎?

但最重要的是,繁體字和簡體字劃分而治的局面,揭示了多個中國的可能性,去中國的可能性,和中國的本質。要釐清這一點,這裏恐怕三言兩語做不到——因為要說明漢字的兆物觀,其所衍生的神聖性和趙汀陽所道的精神性,以及其所趨生的政治觀。這裏做不了——但我會說,如果你認為香港是華夏文明的一顆明珠,一個實驗性的恩賜,而且如果你認為百年恥辱的教訓,不是單純的「我們中國要富強起來」,而是要我們反思「要不要中國」——那我就告訴你,簡體字是個好東西。我們要更多、更癲的簡體字,要把漢字完全殘化,使華夏大地有很多不同的漢字體系。讓河南用二簡字、三簡字、甚至拼音字吧。這是必須的——否則,你所危乎於的,是中國,和他靚麗的野蠻和恐怖。