% https://zh-yue.wikipedia.org/wiki/Talk:%E6%BC%A2%E5%AD%97

% \usepackage{calc} % Needed for arithmetic in lengths

\newcommand{\ladder}[2]{%
  \par\noindent
  \begingroup
    \leftskip=\dimexpr #1em * 2\relax
    #2\par
  \endgroup
}

\chapter{命名問題 (2007年)}

\ladder{0}{HenryLi 2007年3月5日 (一) 12:25 (UTC):
漢字呢個名係日本呢啲地方叫,表示係外來字。響中國地方,叫字就夠囉。中華民國就叫國字,香港就叫佢中文字。好似係近年先至流行叫漢字。}

\ladder{1}{WikiCantona 2007年3月5日 (一) 14:51 (UTC):
改中文字都好呀。}

\ladder{2}{我哋越人 2007年3月8日 (四) 08:09 (UTC):
反對改。中國內地本身就係叫漢字。中國大陸、韓國、日本都叫漢字。「字」同「漢字」唔同,字嘅範圍大啲。而「中文」係一個模糊唔科學嘅世俗叫法,好似「廣東話」「中國話」呢啲叫法噉。}

\section{對2007年命名討論嘅討論}

\ladder{0}{根據Special:diff/657162,本身呢篇文喺2012年5月18號之前係叫「漢字」嘅,係喺2012年5月19號先畀人搬版搬咗去「唐字」,所以上面講嘅「唔好改」係保留「漢字」唔改。同後來嘅討論唔同,𠝹番開先。}

\ladder{0}{仲有,明明2012年之前嘅討論得咁少,當中根本就冇討論出搬去「唐字」嘅共識,無啦啦就搬咗版。咁嘅操作真係……Well,苦笑。--Cangjie6 (傾偈) 2021年8月22號 (日) 16:52 (UTC)}

\ladder{1}{@Cangjie6、H78c67c、Z423x5c6、Detective Akai、SC96、Deryck Chan、Pokman817:你地睇返上面嗰次討論,2012年搬文嗰次討論有3個人參與,有1個人係明顯反對(User:我哋越人),有2個係支持(Henry、WikiCantona)。再睇返近日嘅Talk:黑膠唱碟\#原創研究成分,有1個人——User:Kowlooner係明顯反對,有2個人係明顯支持(特克斯特、Akai)。喺參與討論人數嘅情況根本同嗰次(2007年嗰次)一模一樣,呢2次(黑膠唱片+2007年漢字討論)嘅討論大家都係畀咗理由 而唔係盲目投票玩票數,再加埋User:WikiCantona喺呢到講嘅:Special:diff/1671405(唔係大多數支持就ok),已經充分證明嗮2012年嗰次搬文夠係未符合「大多數」情況,同Talk:黑膠唱碟嘅根本一模一樣(都係2對1嘅情況)。2個人支持用「唐字」就可以搬文(1人反對),2個人支持用「黑膠唱片」就唔可以搬文(1人反對)??}

\ladder{2}{如果W君視咁多粵維常客嘅討論同揀用字嘅理由方面係操控票數、投票認為「漢字」係更好選擇就係同化同破壞粵維嘅行為(即係User:Cangjie6話嘅「污名化」大家粵維常客),咁我呢到都可以當你喺黑膠唱碟文章自把自為、濫用權力,喺唐字文章唔跟返討論意見堅持搬文係雙重標準。}

\ladder{3}{喺下面章節嘅討論,大家都睇到W君喺被主流意見反對下講咗粵維一直唔用「投票」嚟決定,但係呢次參與基數唔係4、5個人同非活躍用戶咁簡單,再加埋「Special:diff/1671405(唔係大多數支持就ok)」,佢個句「大多數」呢三隻字嘅意思又一次自打嘴巴(自己言下之意都證明咗大家去決定用邊隻字係冇問題,唔使畫公仔畫出腸 寫明一定唔可以用「投票形式」 但係最尾討論結果其實係基於「大多數用戶意見同留言」去決定,咁樣佢只係喺到偷換概念),有導向(符合WP:NOT\#DEMOCRACY嘅原則)、有理由、唔係盲目投支持同反對,呢啲意見理應聆聽,去睇下下面討論「大多數」用戶覺得邊隻字用字更準。}

\ladder{3}{憑乜嘢Henry搬返返去?Henry都有責任出嚟解釋,一係WikiCanona喺黑膠唱碟亂嚟(包括禁止非管理員搬文又係,吳君如音樂作品畀人搬咗2次 次數同黑膠唱碟一模一樣,又係有人反對,又唔見你鎖文?又喺到玩雙重標準?自把自為,唔怪得喺Wikipedia:城市論壇\_(政策)\#個別案例嘅處理問題一路遊花園 講埋啲偏離話題嘅嘢 User:Cangjie6有興趣去食花生都可以望下)。}

\ladder{4}{我就建議User:Cangjie6喺粵維活躍少少,唔係長遠落去會趕走嗮啲人,對呢到發展唔好,亦都學Akai喺Talk:拉爾夫史坦曼呢到咁講:你好意思同我講「羅馬都唔係一日起好」(嗰到佢哋2個討論緊點解粵維開張15年呢到都係咁少人肯入嚟寫文,User:Cangjie6都知道點解呢到一直都咁少人肯入嚟啦)。— 特克斯特 (傾偈) 2021年8月23號 (一) 16:12 (UTC)}




\ladder{5}{@特克斯特、Cangjie6:有理由懷疑WikiCantona同Longway22係以為自己係呢到嘅地頭蛇,加上十幾年嚟從未有人質疑過佢地既做法、或者做埋啲同佢地背道而馳嘅野,令到佢地自把自為,隨隨便便就打擊異見者。另外,最先話要投票既人係Dr.}
\ladder{5}{Greywolf(「我不嬲都嗌開「漢字」。投票囉。}
\ladder{5}{Dr. Greywolf (傾偈) 2021年8月21號 (六) 14:58 (UTC)」),唔係我,Dr.Greywolf絕對有責任出嚟講返野,但佢居然叫人地唔好再ping佢,明顯係怕事同埋唔負責任。Akai 博士 (傾偈) 2021年8月24號 (二) 03:50 (UTC)}

\ladder{6}{關鍵在於中段時最後akai閣下嘅(總結提出voting表決壓制討論)做法、即正式總結將成個未有確切成投票採樣案嘅議程,包裝成一個投票案甚至係一個絕對壓制少數意見嘅決議案,先至係最為打壓少數異見嘅做法,請幾位認真考慮幾位依家嘅做法係咪繼續製造不公同其他問題,唔好兜著人多勢眾系度模糊幾位經已明顯違反一般討論同辯論嘅群狼做法 Longway22 (傾偈) 2021年8月24號 (二) 04:02 (UTC)}

\ladder{7}{@Longway22:由2007年嘈到依家,嘈左接近十幾年,十幾年都未有結果,你係咪要諗下呢個問題呢?跟據Wikipedia:請求移動響英維嘅解釋,有一種情況係唔畀搬文(「A title may be disputed, and discussion may be necessary to reach consensus」),明顯呢篇文響2012年同埋2021年嘅討論已經唔適用呢個條例,單係Henry響2012年嘅搬文已經係唔啱。Akai 博士 (傾偈) 2021年8月24號 (二) 04:40 (UTC)}

\ladder{7}{@Longway22:其實我最初唔係好想搞投票,只不過見有絕大部份既維基友都支持用漢字,但反對既目前只有幾個,而呢篇文已經由2007年爭議到依家,不如直接投票嚟個一刀兩斷。我好希望可以停止一個長達十幾年嘅爭議。Akai 博士 (傾偈) 2021年8月24號 (二) 04:46 (UTC)}

\ladder{7}{ 呢點並冇改變閣下經已造成嘅後果,即使閣下提出表面合理嘅理解,同時亦係未有解決當下爭議同擴大咗程序問題等衍生風險,請閣下一併考慮埋同適時中止風險嘅繼續 Longway22 (傾偈) 2021年8月24號 (二) 05:18 (UTC)}

\ladder{8}{如果要唔好擴大程序問題等衍生風險,噉係咪應該首先改正番錯誤嘅程序先講得通?即係將2012年呢一次錯誤嘅搬版更正番,還原番去「漢字」先。-Cangjie6 (傾偈) 2021年8月24號 (二) 06:53 (UTC)}
                
    
        
\ladder{1}{@Longway22、Cangjie6:我發現Longway22淨係識得針住我既意見,但未有反思自己亂咁指控人嘅錯誤,又唔去諗下點解咁多人支持用漢字而唔係唐字;當Cangjie6、特克斯特提出要搬文,Longway22亦未有回應兩位嘅意見,反而一直玩針對、恃住自己係少數人就要求大部份人聽佢既意見,以為自己受到不公平既侍遇,但事實係Longway22一直堅持己見、未有企響其他人嘅角度去諗;粵維非常多文章內容都好少,但我見到Longway22成日都響同人嘈,好少見佢有認真寫文,如果佢將同人嘈嘅精力放響寫文到,相信可以對粵維貢獻更多;而我認為我既做法即使唔係響維基常用,亦唔覺得會引起好似Longway22所講嘅咁大風險,投票就算唔啱響討論入面用,但面對一篇嘈左接近十幾年都未完既文,可以睇到討論已經解決唔到問題,因此響呢個時候,投票係唯一選擇。Akai 博士 (傾偈) 2021年8月24號 (二) 07:26 (UTC)}

    \ladder{2}{所以本案經已係嚴重違反多個經已提出嘅問題同規程、但係以上由akai帶起嘅幾位根本唔當一回事,十分遺憾,本編保留依家由於akai帶起大陣仗嘅相對少數異見位置。唔再重複。 Longway22 (傾偈) 2021年8月24號 (二) 08:25 (UTC)}
        \ladder{3}{ 我唔覺得會引起大陣仗;另外,如果跟規程,咁呢篇文章響2012年嘅搬文咪一樣唔符合規程。好心Longway22嘈少啲野,執多啲文啦。Akai 博士 (傾偈) 2021年8月24號 (二) 08:46 (UTC)}
\ladder{1}{ 呢輪興用「屈」。User:特克斯特閣下,唔係話!我2007年明明話用「中文字」,嘩,你都幾叻個喎,覺得我能夠預知未來,2007年根本都冇人提過「唐字」。咁又入我數啊,仲有,2007年 User:我哋越人係反對「字」或「中文字」。五年之後(2012年)先有人搬去「唐字」(可能係第三個出路),當時冇有人反對?再過三年後,2015年又有人提再搬,討論直到而家。就算你講黑膠碟嘅點人數情況係正確(又只係講咗啲唔講啲 - 不盡不實,麻煩你收返無理指控),根本完全就係兩回事。User:特克斯特閣下對之前嘅事仲係耿耿於懷,咁麻煩你返去處理緊嘅地方,繼續講你嘅觀點啦。--WikiCantona (傾偈) 2021年8月24號 (二) 11:11 (UTC)}
    \ladder{2}{@WikiCantona:哈佬你終於又響度出現啦;咁你認為2012年,搬去「唐字」是否符合規程?發表你既高見Akai 博士 (傾偈) 2021年8月24號 (二) 12:18 (UTC)}
        \ladder{3}{@Detective Akai: 喂,哈佬,你都好忙下,又寫嘢,有傾過唔停。冇法啦,又有人「捩橫折曲」,唔出嚟唔得。我呢啲地頭蟲,高見就唔敢啦,不過,「符合規程」你可唔可以澄清下先?2012年嘅英文規程?2012嘅呢道嘅慣常做法?因為近年嘅做法?--WikiCantona (傾偈) 2021年8月24號 (二) 12:30 (UTC)}
            \ladder{4}{跟據Wikipedia:請求移動響英維嘅解釋,有一種情況係唔畀搬文(句子「A title may be disputed, and discussion may be necessary to reach consensus」),即係如果篇文(響搬文前)係對命名有爭議嘅話,咁就要加{{Move}}或者經討論後先至搬文,明顯2012年嘅搬文係未得到反對者嘅同意,即係響冇共識既情況下搬文。唔知WikiCantona、Cangjie6、特克斯特同唔同意Akai 博士 (傾偈) 2021年8月27號 (五) 08:55 (UTC)}
                \ladder{5}{首先要多謝 User:Detective Akai 閣下,「同意」同「唔反對」唔係一樣。「共識」係咪要得到每一個人嘅「同意」?定係每一個人都「唔反對」、「唔出聲」就得呢?可以深入討論,不過,不過操作上,「唔反對」會啱呢度些少。吓吓要啲討論人,返去表態「同意」,唔係人人有興趣/想噉做,共識就好難。例如「黑膠碟」嘅討論,閣下你唔出聲,User:特克斯特閣下都覺得有大致嘅共識。2012年嘅情況,要 2007年 User:我哋越人反對搬版,你知啦呢度啲用家神龍見首不見尾,流動人口多,要搵返佢問佢同意,可能有啲難度。直接搬咗先,睇吓有冇人反對,都唔係一個唔可行嘅做法。最重要嘅係,2012年搬版呢味嘢,真係各有各做,其實到而家都係(好少少)。所以你有興趣賞面都去 Wikipedia:城市論壇\_(政策)\#搬版,改名嘅本地框架再商議傾下。--WikiCantona (傾偈) 2021年8月27號 (五) 09:50 (UTC)}
                    \ladder{6}{What,之前WikiCantona唔係一直話要有共識既咩。Akai 博士 (傾偈) 2021年8月27號 (五) 09:54 (UTC)}
                        \ladder{7}{係,冇錯。近年我眞係覺得共識緊要。現實係 2012 年嘅嘅情況唔同,我就係嗰陣時嘅情況作出一啲評論?!--WikiCantona (傾偈) 2021年8月27號 (五) 10:38 (UTC)}
        \ladder{2}{唔明WikiCantona講呢個說話既理由,冇記錯我多次見到WikiCantona響討論頁提出要搵共識。Akai 博士 (傾偈) 2021年8月27號 (五) 11:47 (UTC)}
            \ladder{3}{@Detective Akai:閣下,當有兩個有你冇我嘅意見時,而兩方都企得好硬,(除非一方讓),共識嘅形成係冇可能。所以好多時會退而求其次,喺兩個冇辦法妥協嘅方案之間,搵第三條出路,第三個選擇。黑膠碟嘅討論就係呢個情況,黑膠碟係黑膠唱片同黑膠唱碟嘅第三選擇。當然我哋可以要求每一個有份討論嘅人都要「同意」呢個選擇,實際上係幾難。不過,只要討論嘅人唔再提出反對(唔出聲),會 easy 啲。咁大致可以睇成有共識。希望噉解釋,可以清楚少少。2012年「唐字」嘅搬版,亦用咗呢個唔出聲就即係唔反對嘅原則。因為噉,2012年 User:HenryLi 閣下搬版之後冇人反對。所以喺呢個細維基嘅運作,水靜河飛嘅時候,都唔係一個唔實際嘅做法。 --WikiCantona (傾偈) 2021年8月27號 (五) 12:07 (UTC)}
        \ladder{2}{@Detective Akai:Akai醒少少先啦,喺黑膠唱碟嗰陣都係同一個月份內討論,同2012年User:Henry Li喺冇人留意嘅情況下搬文類比唔到(User:Cangjie6都認為要返返去初版;W君喺呢到自己都識得講)(2012年「唔出聲就即係唔反對」亦都係唔係你喺呢到回應大家搬文冇問題原因、打圓場嘅講法,要留意依家所有規則之類全部係佢自己up 龍門任佢搬)。回歸呢轉討論,仲有兩方都堅持,咁就要睇下多數參與討論者意見係用邊個,其他2位嘅,頂多符合User:SC96講嘅「各自表述」(Special:diff/996178),PQ77wd當年搬文手法同佢哋2個今次嘅手法一模一樣。特克斯特 (傾偈) 2021年8月27號 (五) 19:17 (UTC)}
            \ladder{3}{@Detective Akai: 哈哈,「Henry Li喺冇人留意嘅情況下搬文」,嘩,特記(親切啲唔再稱呼做閣下),好嘢,你係全世界啲人心裏面啲蟲?定係 super AI?又知道2012-2015年之間冇人留意到?2015年有講嘢啦,三年之後。「打圓場嘅講法」,都眞係希望化解矛盾。話啲 rules 由我噏出嚟?噉 Akai 閣下睇吓我講得有冇道理啦?最後,特記有講User:PQ77wd當年搬文,我都想知多啲,不如去Wikipedia:城市論壇\_(政策)\#搬版,改名嘅本地框架再商議講下(唔知鍾意搵嘢嚟拗嘅特記一定唔會去?)--WikiCantona (傾偈) 2021年8月27號 (五) 22:45 (UTC)}
        \ladder{2}{@Detective Akai、Z423x5c6、Cangjie6:同上情況,又係佢唔認數就齋講「呢輪興用「屈」」。至於,「咁麻煩你返去處理緊嘅地方,繼續講你嘅觀點啦」,喺黑膠唱碟呢篇文已經明顯有咗共識係用「黑膠唱片/黑膠碟」呢兩個寫法,點都唔會係「黑膠唱碟」,係閣下仲喺道扮揾唔到共識唔去搬返去「黑膠唱片/黑膠碟」,我唔似閣下喺「Talk:唐字」咁樣「繼續講你嘅觀點啦」,咁樣同死撐唔尊重共識冇分別。至於Akai講嘅「地頭蟲」我會返Wikipedia:申請做管理員/Z423x5c6嗰度講。特克斯特 (傾偈) 2021年8月25號 (三) 22:28 (UTC)}
            \ladder{3}{有冇係我新嘅資料 update 呢?亦都講咗,揀一個題目名,有多個因素要考慮。如果將共識純粹睇成多數 vs 少數,閣下根本就冇興趣傾,唔想搵妥協嘅方案,「死撐」呢啲字對個討論冇乜幫助啵。--WikiCantona (傾偈) 2021年8月27號 (五) 09:53 (UTC)}
                \ladder{4}{「「死撐」呢啲字對個討論冇乜幫助啵」,因為要學User:Z423x5c6咁講,話要做有意思嘅討論,你先前攞嗰啲資料已經畀佢全部反駁嗮,亦未符合User:Cangjie6嘅意見(睇清楚佢提倡嘅嘢先好繼續拎資料出嚟)。唔話「死撐」,你會繼續喺度留言對住空氣講野,自己以為自己仲係同Z423x5c6等用戶有效回應個問題同參考資料,咁先最冇幫助(人哋明明嘅意見都係用「漢字」說服唔到其他人用「唐字」)。仲有如果共識唔包「多數 vs 少數」(喺充分討論下 唔係齋投票 呢次凝聚共識當然唔係齋投票了事),如果繼續唔認數,呢啲亦都係直接違反共識嘅明顯證據,亦都未讀透Wikipedia:共識。特克斯特 (傾偈) 2021年8月27號 (五) 19:17 (UTC)}
                \ladder{4}{今次我已經盡量避開無謂拗撬。哈,你都唔係第一次,你講嘢聽落有汶有路,最叻就係講啲唔講啲 - 不盡不實 - 避重就輕。對住你真係唔知好嬲定好笑。我話共識唔係純粹睇多數對少數,你就話「如果共識唔包「多數 vs 少數」」,真係豎手指公,啲邏輯都冇。「喺充分討論下」家下未有,「唔話「死撐」,你會繼續喺度留言對住空氣講野」,嘩,好嘢,原來有人唔回應我嘅觀點,又係我錯!喺 WP共識「編者都應該作出善意努力」,我就做緊啦,幾鐘頭之前先始 upload 咗張相,討論「漢字」嘅可能性,我會繼續提出,有如果真係有誠意嘅話,去傾吓。--WikiCantona (傾偈) 2021年8月27號 (五) 22:45 (UTC)}
        
        \ladder{2}{@特克斯特:嘩,我瞓個覺姐,討論已經多左咁多...,我想講,WikiCantona並未提供一個合理解釋畀我。如果跟足維基規舉做野,共識係好重要,元維基都係咁講。另外,「唔出聲就即係唔反對」,佢冇出聲你點知佢唔反對(笑)?Henry響2012年既搬文,明顯係違反左Wikipedia:請求移動,係未有共識,唔應該搬文;我唔理有冇人留意。總之我目前仲見到WikiCantona係未承認多數人支持「漢字」既結果。我想多口問一句:當初Henry響兩人支持、一人反對下都可以搬文,咁點解依家係九人支持、兩人中立、三人反對既情況下,就唔可以搬文呢?Akai 博士 (傾偈) 2021年8月28號 (六) 00:57 (UTC)}
            \ladder{3}{我都好想知道點解。而且,而家大家發覺當年嘅搬文咁有問題,對比桑切斯(暫名)呢篇文,哪怕呢個名唔符合粵音,但係產生改名爭議後,WikiCantona都搬返去呢個名先,嚟繼續討論。係噉,點解而家呢篇文又唔係用同一種做法,搬番去「漢字」先?-Cangjie6 (傾偈) 2021年9月3號 (五) 11:48 (UTC)}
                \ladder{4}{要答答 @Cangjie6:先,其他嘅仲寫緊,畀啲時間我。桑切斯係畀 User:Detective Akai 加點同加長個名,我搬返轉頭,拎走個點,令一位 User:特克斯特唔同意,再反轉,我用管理員嘅身份,改返做原名,為嘅係唔好有編輯戰,轉頭討論,唔關「符合粵音」嘅事。之後,為咗大家舒服啲,亦交畀其他管理員搞。呢篇嘅情況,原則係「討論開始咗就唔好搬版」,英文維基百科嘅規矩,近期喺 Talk:墨索里尼度,因為我以用街坊用家嘅身份,玩佢搬兩次我搬兩次,後來 User:Detective Akai 閣下話,論開始咗就唔應該搬版,講得有道理,聽佢話,我自己搬返去 Detective Akai 閣下搬完之後,有人提出反對,嘅名「貝尼托·墨索里尼」。同呢度比較相似。}
                \ladder{4}{User:Detective Akai 閣下講:「當初Henry響兩人支持、一人反對下都可以搬文」,User:Detective Akai 閣下可能係嚟自未來世界(笑),所謂嘅「支持」係 2017,2015年。2012年 HenryLi 搬版,你最多話當時冇人反對,唔通 HenryLi 有時光機,知道後來會有人「支持」佢?!嚴格咁講, 佢當時冇人支持,亦冇人反對。2007年嗰「一人」係反對搬去「字」同「中文字」,唔通佢預知未來 ,知五年後會搬去唐字,「反對」定未來。所以你句說話超乎事實,結論亦唔合理。}
                \ladder{4}{所以,希望搵到個雙方面都可以接受嘅辦法,先郁都未遲。--WikiCantona (傾偈) 2021年9月4號 (六) 04:10 (UTC)}
        \ladder{5}{@WikiCantona:2007年既討論明顯係未有共識;Henry亦響搬文個陣冇提出任何理據。依家大多數人都係支持用「漢字」,好似係得以WikiCantona為首既人唔肯承認,響度死賴。Akai 博士 (傾偈) 2021年9月4號 (六) 10:22 (UTC)}
        % 咁根本就係雙重標準——有啲文即使未有定名共識,但係發現搬版違規,就首先搬番去舊名(哪怕舊名有問題)先繼續討論。而,而家呢篇,就聲稱要「搵到個雙方面都可以接受嘅辦法,先郁都未遲」。噉樣雙重標準法,叫人點接受?點服人?
        % 所謂「雙方面」,就只不過係閣下憑偏見一直極力要用假古董,而好多人認同要用「漢字」呢個共識係好明顯嘅。係咪只要唔啱閣下心水,閣下就有權用「玩規程」而夾硬黎講,最終粵維就係聽閣下你支笛?
        % 仲有,而家呢篇文係HenryLi喺2012年違反規則擅自搬版嘅,而HenryLi違反規則搬版唔係第一次。睇到島嘅編輯歷史,喺2016年,佢又係冇討論結果就擅自搬去洲,畀殘陽孤俠鬧佢:「身為管理員更加唔應該知法犯法未經討論就改名。」然後到咗2018年,HenryLi無啦啦又再擅自搬版,而殘陽孤俠再還原、再鬧佢:「請唔好再用一百幾十年前嘅標準來規範宜家嘅文章。」點解島呢篇文畀HenryLi違規破壞後,可以噉樣即刻還原番,但係漢字唔只唔還原,仲要有明顯共識都仲繼續唐字落去?而家粵維係咪興玩雙重標準?-Cangjie6 (傾偈) 2021年9月4號 (六) 04:21 (UTC)
        
        % \section{命名問題 (2012年畀人搬咗去「唐字」之後)}
        % 最新留言:3 年前
        % 16則留言
        % 5個人參與討論
        % 我呢個土生土長嘅香港人,聽過「中文字」,聽過「漢字」,就係冇聽過「唐字」。我唔認為粵文維基百科應該故意標奇立異,違反大眾嘅約定俗成叫法。無論改做「中文字」定「漢字」,我都贊成,就係反對叫「唐字」。Cangjie6 (傾偈) 2019年10月12號 (六) 21:34 (UTC)
        
        % 反對改。最新論述可以睇下底。——Longway22 (傾偈) 2019年10月13號 (日) 03:48 (UTC)
        % 反對改,睇下底嘅參攷。--WikiCantona (傾偈) 2021年8月21號 (六) 06:46 (UTC)
        % 反對反對改,見下低。--Cangjie6 (傾偈) 2021年8月21號 (六) 14:18 (UTC)
        % 反對。依家廣東話用字,甚至文法、字音,都一直畀強勢語言蠶食緊。戰後大批外省人走難到香港,佢哋已經以民國教育嗰套,代替咗原來廣東人用字。學校都日漸畀依啲人所控制,無聽過有乜咁出奇。學校迫人用「書面語」。無學校教育,而大家從來唔睇唔學廣東話嘅文學遺產,又點會知?以前教書嘅人叫做「先生」,後尾嗰啲國語為中心嘅,逐漸改晒「老師」。咁樣,細路長到大人,可以一世都未聽「先生」咁叫法。長此下去,無「糖水」、只有「甜湯」。無「生菓」,只有「水果」。無「片」睇,只有「視頻」。只「吃」無「食」,只「喝」無「飲」。廣東話用字並無靠山,若果只係自己閱歷淺未聽過,從來唔回顧傳統,廣東話真係危危乎。只要清洗落去,廣東話用字,只係國語、普通話翻版,頂多「的」改為「嘅」,「是」改為「係」,咁依度開來又有乜意思?依度開來就承傳廣東話文化用意,而唔只翻版國語、普通話。HenryLi (傾偈) 2021年8月23號 (一) 01:27 (UTC)
        % 反對樓上嘅反對。但凡活生生嘅語言,一定會keep住有變化,只有已經死咗、放入博物館嘅語言(例如哥德語)先至會唔變,一隻唔變嘅語言絕對唔係咩好事。唔係凡變化都關畀人控制乜乜乜、蠶食乜乜乜嘅事。如果明明係喺自然語境度,大家都冇人講嗰個詞語,甚至摷歴史文獻文本,都淨係得幾多僻例,完全冇普及、廣泛過嘅痕跡,噉根本就係本身成個語言群體度大家都唔用,夾硬將個大家都唔用嘅叫法定做標準,噉根本唔係「傳統」,而係「製造假傳統」,違反嗰隻語言本身嘅面貌。面對語言呢,水清無魚,水濁亦無魚,講出某個叫法有問題,唔等於可以上綱上線推到極端,話邊個邊個日常叫法又會消失,喺自然語境叫「生果」、「糖水」、「食」、「飲」等等周街都係,同周街都冇人叫「唐字」呢個case完全唔同,唔可以打橫嚟強行類比。有時唔同叫法,亦都唔一定係非黑即白的排斥狀態,例如平時我哋叫「食嘢」、「飲嘢」、「飲飲食食」,但係都會講「吃喝玩樂」,唔可以夾硬改成「食飲玩樂」。如實咁尊重自然語境,用番「吃喝玩樂」,係尊重現實,絕對唔係咩「翻版國語、普通話」,麻煩樓上唔好咁樣離地老屈。但係而家「葡萄糖」改做「提子糖」,「漢字」改做「唐字」,偏偏就係呢種情況。「承傳廣東話文化」,都要承傳堅嘅嘢,而唔係自己老作啲假標本假古董迫人跟,「承傳」埋啲離地離到上太空嘅假嘢——例如偽正字,例如標奇立異嘅叫法。最後,廣東話用字絕非無靠山,事實上粵語粵文嘅歷史文獻文本,已經多過好多漢語語言,問題只係大家客觀噉面對佢哋吖?定係用已經先入為主嘅態度,先有結論而後砌推論,唔顧客觀吖?-Cangjie6 (傾偈) 2021年8月23號 (一) 09:52 (UTC)
        % 粵語經歷咁多年嘅洗禮,依家嘅粵語早就同上古時代唔一樣;粵語好多發音,其實都已經變左。相信例如英文、國語都會變。時代變,人會變,粵語都會變。Akai 博士 (傾偈) 2021年8月24號 (二) 04:03 (UTC)
        % 呢個討論已經完咗,處理結果:離題。請唔好對呢個歸檔做任何改動。
        % \section{篇文寫到好似文言咁款}
        % 最新留言:12 年前
        % 2則留言
        % 2個人參與討論
        % 如題 --Victor-boy (傾偈) 2012年10月16號 (二) 15:27 (UTC)
        
        % 邊度似?不過廣東話本來就近文言。HenryLi (傾偈) 2012年10月17號 (三) 00:34 (UTC)
        % \section{命名問題:「漢字」/「中文字」/「唐字」}
        % 最新留言:3 年前
        % 9則留言
        % 7個人參與討論
        % 「唐字」呢個名,我幾乎從未聽過,身邊都唔多聽見。與其叫呢個文章做「唐字」,不如改返去叫「漢字」算罷,咁樣起碼更多人用更多人明。仲有,「唐字」呢個叫法,其實唔係幾咁符合維基百科嘅命名規則,少用嘅名最好都係避免用。而「漢字」呢個名,就喺兩岸三峽、日韓、甚至海外都流行通行,適宜採之。--N6EpBa7Q (傾偈) 2015年1月15號 (四) 23:46 (UTC)
        
        % 贊成移動,Unicode都係叫「漢字」(Unihan)。UU (傾偈) 2017年11月7號 (二) 13:25 (UTC)
        % 反對搬,你可以話廣東話嘅字係中文字,之前亦反對過。所以唔改好過。--WikiCantona (傾偈) 2017年11月7號 (二) 22:15 (UTC)
        % 就系咯 点解个个都叫汉字 呢度创造一个咁古怪嘅名?我同意“汉字”。--⼥⼉ (傾偈) 2019年10月13號 (日) 00:00 (UTC)
        
        % 反對搬,粵維嘅命名,需要合乎返粵語本身嘅表述傳統,仲有更多考慮返對歷史文化嘅充分傳承同保育,呢啲係粵維度比較可以做到嘅。——Longway22 (傾偈) 2019年10月13號 (日) 03:44 (UTC)
        % 反對反對搬。我想問,所謂「粵語本身嘅表述傳統」係咪就係作個周圍都冇人咁叫嘅罕見偏僻名,違反日常粵語嘅約定俗成?由出世到大我都講粵語,唔係上粵維都唔會發現有「唐字」呢個叫法;我父母都講粵語,我祖輩都講粵語,佢哋都話冇人講「唐字」,仲質疑係咪少數華僑叫噉嗌定係新造出嚟(質疑原因係有少數華僑會講「唐話」呢個詞語,不過「唐字」就從來冇聽過,只係同「唐話」構詞格式類似)。粵維嘅命名,應該要符合粵語嘅約定俗成,唔係自己作啲偏僻罕用怪名。而且粵語本身嘅約定俗成先至係真正嘅粵語本身嘅表述傳統。(仲有,明明係同一個討論,唔好特登拆件,營造出處處係你執尾刀扮結論嘅假象,唔該。)Cangjie6 (傾偈) 2019年10月13號 (日) 11:24 (UTC)
        % 睇返咗記錄,傑作係另位,宜家搬返個話題到最初位置先。仲有補充返,「漢」實唔係粵語嘅習慣傳統。仲有係希望朋友講清楚想搬去邊個,唔好一大段駁咗又畀唔到有紋路嘅道理,粵維嘅朋友好難喺度繼續好好探討。——Longway22 (傾偈) 2019年10月13號 (日) 15:08 (UTC)
        % 無論搬去「漢字」定係「中文字」都贊成,起碼現實入面有人噉講,而「唐字」就聽都未聽過。Cangjie6 (傾偈) 2019年10月21號 (一) 18:11 (UTC)
        % 「中文字」?138.229.19.202(討論) 2021年9月28號 (二) 20:25 (UTC)
        
        
        % \section{2021年重開集中討論}
        % 最新留言:3 年前
        % 136則留言
        % 14個人參與討論
        % 外部來源
        % 再返對,寫返啲參攷,費事拗:唐字,唐字,龍行天下☞粤港澳海内外捍衛粤語大聯盟,唐話,我有翻譯過唐字,唐字之下,你唔通係寫唐字啞。,English Made Easy 《唐字調音英語》1905...--WikiCantona (傾偈) 2021年8月21號 (六) 06:15 (UTC)
        
        % 本人支持用「漢字」,「唐字」真係比較罕用;退一步可以用「唐話字」Akai 博士 (傾偈) 2021年8月21號 (六) 13:18 (UTC)
        % 話唐字幾好嘅,叫唐話字係咪有啲畫蛇添足咗:P Longway22 (傾偈) 2021年8月21號 (六) 13:25 (UTC)
        % @Deryck Chan、Z423x5c6、SC96:呢個同User:Cangjie6一樣意見,揀寫法都係揀「漢字」(雖然會畀人話同中維撞咗)。用User:Detective Akai意見嘅「唐話字」,我諗佢呢個諗法係基於避免同中維一樣,而跟返呢到用「唐」開頭嘅寫法,雖然寫法又啲怪。
        % 講起參考資料,夠有好多資料用緊「漢字」:1 2 3 4,不過學User:Cangjie6話齋呢點粵維有人一直以嚟做到走火入魔係冇錯,應該用返日常粵語嘅約定俗成,即係建議嘅「漢字」。特克斯特 (傾偈) 2021年8月21號 (六) 13:54 (UTC)
        % 咦。全部參考都係「國文」嘅,閣下係可能忘記咗呢度粵文維基百科。--WikiCantona (傾偈) 2021年8月21號 (六) 14:57 (UTC)
        % 粵維有規定參考完全禁止書面語咩?特克斯特 (傾偈) 2021年8月21號 (六) 15:09 (UTC)
        % 當然唔係啦,只係有粵文嘅參攷應該睇下。--WikiCantona (傾偈) 2021年8月21號 (六) 15:11 (UTC)
        % 語言係變化嘅,而且語言係約定俗成嘅,而家喺各個粵語重點地區,都好肯定用「漢字」或者「中文字」多個「唐字」同「唐話字」多多聲。可唔可以尊重下事實?強行為咗唔同而唔同,標奇立異,只會嚇走人,令人覺得粵維古靈精怪咋。Cangjie6 (傾偈) 2021年8月21號 (六) 14:18 (UTC)
        % 唔同意係「標奇立異」,只係尊重傳統。咁閣下可能想改氼水做潛水、黃䘆做蚯蚓... --WikiCantona (傾偈) 2021年8月21號 (六) 14:43 (UTC)
        % 講真我真係喺呢到先第一次見到「唐字」呢個詞--Z423x5c6 (傾偈) 2021年8月21號 (六) 14:55 (UTC)
        % 我不嬲都嗌開「漢字」。投票囉。Dr. Greywolf (傾偈) 2021年8月21號 (六) 14:58 (UTC)
        % 我投「漢字」。Dr. Greywolf (傾偈) 2021年8月21號 (六) 14:59 (UTC)
        % 我都投「漢字」。特克斯特 (傾偈) 2021年8月21號 (六) 15:09 (UTC)
        % 「漢字」+1。Z423x5c6 (傾偈) 2021年8月21號 (六) 15:31 (UTC)
        % 我都投「漢字」+1。仲有,標唔標奇立異,唔係幾個偏執狂嗌覺得唔覺得就可以作準,唔信嘅咪做吓隨機街坊、抽樣訪問或者廣泛問卷囉。Cangjie6 (傾偈) 2021年8月21號 (六) 16:11 (UTC)
        % 我投「漢字」,我都未聽過「唐字」,以前1993年至2006年香港中學會考中文科都有一課書叫做《漢字的結構》。Pokman817 (傾偈) 2021年8月21號 (六) 16:17 (UTC)
        % 我支持用「漢字」Akai 博士 (傾偈) 2021年8月22號 (日) 03:47 (UTC)
        % 唔明白,點解係「偏執狂嗌」?理性咁樣搵咗咁多參考資料,偏偏冇人理。唔明白。--WikiCantona (傾偈) 2021年8月21號 (六) 16:22 (UTC)
        % 你啲所謂「理性咁樣搵咗咁多」嘅資料,有幾普及?花1年又10個月先摷到「咁多」(8項)偏門資料,仲要撻落嚟咁嘅語氣(你嗰句「費事拗」),偏偏特登無視約定俗成、日常講嘢。堅離地離到上太空。噉都仲明白點解係「偏執狂」?-Cangjie6 (傾偈) 2021年8月21號 (六) 16:30 (UTC)
        % 仲有,「有資料」唔等於大晒。譬如某個字點讀,要認眞面對眞相而唔係求其嘅話,韻書、字典都有大把資料啦,但冇可能排斥田野調查,而且田野調査嘅重要性高過唔少資料。如果淨係靠幾本韻書反切就唔理調查結果,人哋調查出嗰個結果,你就𢲡住本韻書話個結果一定係錯嘅話,好多人都會批評係偏執狂啦。同樣道理,噉仲唔明?Cangjie6 (傾偈) 2021年8月21號 (六) 16:37 (UTC)
        % 如果閣下覺得我語氣有問題,道個歉先,「費事拗」改「我都唔想再拗落去」。偏門資料,唔係喎,有喺學術文章出現㗎;「靠幾本韻書反切」有唔係喎,上面都冇韻書,睇嚟閣下根本就冇睇過啲資料,點解咁快全盤否定?約定俗成真係睇幾時,抽樣訪問眞係睇問邊啲人。--WikiCantona (傾偈) 2021年8月21號 (六) 16:41 (UTC)
        % 其實呢個題目都丟埋一邊幾年啦,真係冇乜興趣講落去。希望搵啲資料(其實用咗15分鐘左右),唔係你話乜我話乜,咁唧。其實我亦都冇批評你本人,只係想對事,不過睇嚟理性討論嘅空間就唔係好存在。--WikiCantona (傾偈) 2021年8月21號 (六) 16:47 (UTC)
        % 「咦。全部參考都係「國文」嘅,閣下係可能忘記咗呢度粵文維基百科。--WikiCantona (傾偈) 2021年8月21號 (六) 14:57 (UTC)」就睇下你15分鐘搵咗啲咩參考資料返嚟:
        % 唐字:國文嚟,用你嘅標準應該唔算係參考資料。
        % 唐字:例句入面無「唐字」,得「漢字」,不過佢句英文係指日文嗰啲漢字。
        % 龍行天下☞粤港澳海内外捍衛粤語大聯盟:hashtag有用「唐字」,不過內文全部都係用「漢字」。
        % 唐話:呢條link好似直程連中文字都無隻。
        % 我有翻譯過唐字:本書係…1874年出版嘅?
        % 唐字之下:1922年出版,國文。
        % 你唔通係寫唐字啞。:1922年。
        % English Made Easy 《唐字調音英語》1905:1905年。
        % 唔好意思,真係唔係好睇到你啲參考資料點樣去支持你嘅論點。理性討論,對事不對人,希望我噉樣講唔好令你覺得我針對你。--Z423x5c6 (傾偈) 2021年8月21號 (六) 17:25 (UTC)
        % 又一次多謝你,太感動啦,最少肯睇睇。其實我同大家幾似,未睇到呢篇野嘅時候,唐話、唐人街,唐山大兄就,乜嘢「唐字」呀!?不過check 過吓,有好似有啵。我之前嘅觀點係用「中文字」比「漢字」好。因為呢排 User:Cangjie6 成日畀人提起,又心血來潮,search 下啦,先發現,有唔少嘢,只係用咗 15 minutes。--WikiCantona (傾偈) 2021年8月21號 (六) 19:07 (UTC)
        % 呢啲書(1905,1922,1874年)就表明咗,呢個字唔係死字,呢兩本Cantonese Conversation--grammar, 第 2 篇 1967;First Year Cantonese, 第 2 卷 1966新少少。有用家嘅個觀點係「傳統」,喺英文書記錄粵語,正正反映呢個用詞係粵語之中用嘅。唔好唔記得係我哋寫維基百科之前,有冇人眞正用過地道廣府話去寫作品,亦可能因為咁,有傳統字嘅時候,即是家陣唔係再咁普及,都應該用。再駁多你少少,唔係針對你。用國文/文言文寫嘅,亦可以作為參考,參考價值係至少呢個字唔係生安白做出嚟(最低門檻)。亦即係,揀「傳統」同埋「之後外來辭」嘅分別,諗到嘅例子係手提電話同手機,手提電話係 cellular phone 出現時已經有㗎,「手機」就遲啲引入,依家叫做手機嘅多好多。P.S. 多謝指出有一兩個 Source 唔合格。--WikiCantona (傾偈) 2021年8月21號 (六) 19:40 (UTC)
        % 討論唔應該由單一情緒化傾向嘅投票取代。WikiCantona閣下係搵返到史料係支持咗呢個命名,認為同時係再次反映到呢個命名會係同粵圍/唐人有深刻嘅聯繫,引用返收錄指引嘅密切粵文明啲條款絕對應該保留依家嘅空間名。 Longway22 (傾偈) 2021年8月22號 (日) 00:11 (UTC)
        % 唔同意話個投票係「單一情緒化傾向」。搵到個淨係喺幾本咁多年前歲月文獻度先有嘅叫法,唔代表呢個叫法一直有傳承落嚟,上面亦有好多唔同論者都講咗係喺呢度先第一次聽到「唐字」呢個嗌法。就算呢個叫法真係喺某個歷史時空出現過,但根本唔普及,亦晨早斷咗纜,又何來有「聯繫」仲要話係「深刻」呢?成個條目、成個討論,畀人睇到嘅就只係有好少數偏執狂用盡一切理由或藉口,將佢哋嘅超堅離地主張無限放大,為咗佢哋先入為主嘅主張架空現實。Cangjie6 (傾偈) 2021年8月22號 (日) 09:03 (UTC)
        % @WikiCantona:唔好意思,唔係一兩個唔合格。:-) Z423x5c6 (傾偈) 2021年8月22號 (日) 18:04 (UTC)
        % @Cangjie6、H78c67c、Z423x5c6、Detective Akai、Pokman817、Deryck Chan、SC96:如果要講參考資料,都係有唔少當下嘅粵語圈用緊「漢字」:
        % 1(呢個係粵語網站嚟,左邊欄已經有「漢字部首索引」「漢字筆畫索引」;左上角都係有「輸入漢字」)、
        % 2(香港大學網站,第二段已經用緊「快速學漢字」,再加埋個段都不斷用「漢字」可以去睇行文)、
        % 3(香港教育大學語言學校材料網站,喺第一個小標題已經用緊「漢字」佢入面啲內容:「可查考漢字古今義」都係用漢字)、
        % 4(同資料1一樣又係粵語拼音嘅網站嚟,標題「漢字→廣東話/粵語拼音轉換工具」用緊漢字、入面嘅「漢字輸入欄位」到網站下面嘅「顯示漢字「廣東話」變更的示例」等等 都係用緊漢字)、
        % 5(文匯報網站,但係文章標題同內容一睇已經係主打粵語化嘅介紹 Longway22唔好乘機話文匯報係撐共而盲反話大一統 亂攻擊人;文章標題「【粵語講呢啲】日本漢字:発、売、駅、沢、丼」係用漢字 唔係唐字)、
        % 6(呢個到《粵典》創辦人嘅文章,去到文章中斷有個小標題「漢字以外嘅方案」都係用漢字,個段文章嘅「「用漢字寫粵語」呢個前設」佢嘅話亦都證明我地依家的確係用緊漢字嚟寫粵維,而唔係特登走去話自己用唐字寫粵維,再下面嘅小標題「漢羅並用」配合個段文章亦都係指「漢字」而唔係特登去叫「唐字」,再到文章後半段中間有個標題「【漢字可取,盡用漢字,如無漢字,考慮英數】」亦都係咁樣寫法)。
        % 以上係粵語粵文網站同粵文圈媒體資料,再加埋User:Cangjie6提倡嘅用字唔好生僻 要貼近現實嘅用字,都證明嗮「漢字」係一個更好嘅選擇,而唔係「唐字」。雖然有人會繼續死撐唔理主流意見。特克斯特 (傾偈) 2021年8月22號 (日) 19:53 (UTC)
        % @Cangjie6、Detective Akai、Z423x5c6:笑,我頭先先發現我喺上面畀嘅資料6原來係粵典創辦人阿擇 (Chaaak)寫嘅,佢原來入過粵維寫嘢(戶口:User:Chaaak),佢仲同過User:Detective Akai傾計添。特克斯特 (傾偈) 2021年9月13號 (一) 12:47 (UTC)
        % @Chaaak:邀請擇前輩入嚟講兩句XD——Z423x5c6 (傾偈) 2021年9月13號 (一) 13:06 (UTC)
        % 我都想睇下User:Chaaak前輩既意見Akai 博士 (傾偈) 2021年9月14號 (二) 11:28 (UTC)
        % 補充少少:
        
        % 睇返本人搵到嘅參考材料,其中有幾本係用嚟教人(番鬼佬)講廣東話,除非覺得本書作者係老作嘅啫,呢部份資料記低喺廣東人口中,教育未普及之前,「唐字」一辭極之有可能嘅用法;
        % 《唐字調音英語》莫文暢 in 1905(呢篇嘢有《唐字調音英語》嘅內頁); 本書嘅內容同「睇通勝學英文」好似。「唐字」係呢個環境可以定義成「廣東話發音嘅字」。
        % 《唐字音英語和二十世紀初香港粤方言的語音》 / 黄耀堃, 丁國偉著,香港 : 香港中文大學中國文化研究所呉多泰中國語文研究中心, 2009.10 ISBN:9789627330202 。
        % 呢篇嘢有三張圖,三本書嘅封面,都有用上「唐字」。
        % --WikiCantona (傾偈) 2021年8月25號 (三) 12:16 (UTC)
        % 唐字音學英文書
        
        % 請問WikiCantona你自曝其短夠未?你嘅所謂搜證,正正反映晒你啲論證有幾離譜牽強死撐。黄、丁二人篇論文,係直接硏究莫文暢《唐字音英語》呢本書,理所當然直接引用莫書嘅書名,無論中間斷咗幾多纜都一定會直接引個書名唔會改。噉唔代表黃、丁兩位教授佢哋認同「唐字」係個common term,唔代表佢哋認爲應該用「唐字」取代「漢字」!你bold起個2009年10月,想呃邊個傻仔食豆腐,誤導佢以爲喺2009年大家係普遍用緊「唐字」而唔係「漢字」,誤導佢以為由1905年到2009前大家普遍都係噉樣叫「唐字」仲一路傳承落去?你啲手段唔好再卑鄙無恥啲?!你搵嚟搵去都係啲孤例、個別單丁例子,始終無論咩時代,地球上都有咁多人,只要個數據庫夠大,就算未係好齊,你要搵出某啲特定嘅二字相連組合,都總會搵到。而家一路講緊嘅問題係,大家基本上都唔咁叫吖嘛,任何你所講嘅年代,喺同一時間,用「漢字」的文獻都多多聲,「漢字vs唐字」的比例係無限趨近 100:0 吖嘛!面對現實,唔好再夾硬屈啲假古董出嚟迫人跟得唔得?!
        % 唔好要我動眞火。繼續喺我面前擺明車馬噉造假,我唯有判定你係造假慣犯,以後講乜都唔可信。我先小人後君子,而家有言在先,最後一次警告你唔好再造假。-Cangjie6 (傾偈) 2021年8月28號 (六) 15:28 (UTC)
        
        
        % @Cangjie6:最驚你唔出聲,睇晒你啲偏見。都就你每一點講下。
        % 「擺明車馬噉造假」,「假古董」,張圖我整張圖出嚟?假圖?如果閣下面對同你主觀世界唔同嘅現實時,唔好老屈我!
        % 「黄、丁二人篇論文,係直接硏究莫文暢《唐字音英語》呢本書」講得冇錯,一直想知佢哋書裏頭,論文點睇「唐字」。最慘就係香港公共圖書館冇。你有冇睇過本書呀?有就 share 下。
        % 「噉唔代表...認同「唐字」係個common term」你有冇睇過本書呀?你嘅講法合情,不過,你有冇睇過本書呀?
        % 呃傻仔食豆腐(taam 鬼食豆腐),唐字(冇咗呢頁!之前 User:Z423x5c6 睇過)呢個大陸啲 AI,防火牆眞係噉犀利。佢都有用唐字,不過唔能夠唔承認,呢個站係唯一個有接受唐字呢個講法。
        % 你講來講去就係話,唔知由幾時開始「漢字」成為最普及、常見嘅用法。
        % 「你所講嘅年代,喺同一時間,用「漢字」的文獻都多多聲」,好呀,證據?!搵本 1874年出版嘅「漢字」表示 Chinese 嘅書嚟睇呀?笨。
        % 自己唔做功課,意見主導就以為天下無敵,你估你真係大學教授,國學大師咩?就算係,都要拎證據。
        % 你警告乜嘢?!話你知,自由百科全書,唔係噏得就噏,誣蔑指控!唔發火當我病貓。--WikiCantona (傾偈) 2021年8月28號 (六) 22:34 (UTC)
        % @WikiCantona:《唐字調音英語》,有掀過下睇過內容就知點解我哋話佢舊到唔可以用嚟支持「唐字」。隨手掀下啦,就睇「時令門」。
        % (凡有圈聲字音要大聲讀,如頭字讀偷,走字讀周,仍字讀英,之類是也)
        % 前兩日.Two day ago.(圈)吐爹時亞高。睇得出當時粵音「吐」係讀tu
        % 十二月.December.地三罷。「地」讀di6。
        % 用一本咁舊嘅粵語書去支持你用「唐字」嘅講法真係無乜說服力,麻煩自己睇一睇本書,做好功課先,唔好見到個書名有「唐字」兩個字就扯哂旗噉掉上嚟,唔該。——Z423x5c6 (傾偈) 2021年8月31號 (二) 02:55 (UTC)
        % 補充返,啱啱先見到原來呢個語言現象喺粵維係有文嘅,叫元音裂化(仲上過DYK添!)——Z423x5c6 (傾偈) 2021年8月31號 (二) 03:07 (UTC)
        % 今日食飽飯散步去圖書館,睇咗《唐字音英語和二十世紀初香港粤方言的語音》嘅序同第一章,似乎除咗引用標題《唐字調音英語》、《唐字音英語》之外無用過「唐字」呢個詞。——Z423x5c6 (傾偈) 2021年8月31號 (二) 11:39 (UTC)
        % 唔該晒Z423x5c6幫手查證,更加證明咗WikiCantona造假。
        % 而家講緊嘅,係要喺粵維度用「唐字」而唔用「漢字」,聲稱「唐字」係所謂「傳統」嘅粵語習慣用法,而且佢嘅資格大過「漢字」。淨係搵到幅有「唐字」嘅相,而幅相眞嘅,就等於證明咗『「唐字」係「傳統」嘅粵語習慣用法,而且佢嘅資格大過「漢字」』咩?等於唔係造假咩?
        % 要證明呢樣嘢,要公平咁搵證據,最最最少都要搵喺某個時空嘅共時比較,搵過嗰個時空嘅語料,有幾多係支持「唐字」,有幾多係支持「漢字」。而家WikiCantona唔只冇搵過,仲要靠惡嚟大人,鬧人『搵本 1874年出版嘅「漢字」表示 Chinese 嘅書嚟睇呀?笨。』,鬧我『自己唔做功課』,眞係荒謬到爆炸。我唔係提出或者支持『「唐字」係「傳統」嘅粵語習慣用法,而且佢嘅資格大過「漢字」』呢個說法嘅人,我嘅論點係「現實成個語言習慣,只要唔係有私心有偏見,大家都睇得一清二楚,冇理由捨近就遠,離地離到上太空,違反語言現實」。係噉,點解要由我嚟『做功課』,而唔係提倡者(即係WikiCantona)自己公平咁做嚟到說服人?
        % 而家WikiCantona嘅所謂搵證據,完全係發狂噉摷好大堆好大堆嘢,然後只要一摷到有乜嘢係用咗「唐字」呢兩隻字,唔理個餡點,就即刻攞出嚟曬,講到話「唐字」係資格大過「漢字」嘅。完全冇共時嘅使用對比。完全係「先有咗要導向嘅結論,然後夾硬砌啲『證據』出嚟」嘅反學術手法。呢種僞證手段,已經唔知有「偏見」咁簡單——而佢仲惡人先告狀,調番轉頭屈我有偏見。
        % 我警告乜嘢?警告好似WikiCantona噉造假(仲要當自己嘅造假係眞係有證據)法,只會玩殘成個討論兼玩殘粵維。我唔使睇WikiCantona你發唔發火,WikiCantona你自己嘅造假手段,已經show畀人睇你係三腳貓都不如。粵維再畀你控制,離地離出太陽系,就眞係歸西。我都冇咁多時間陪你癲。--Cangjie6 (傾偈) 2021年8月31號 (二) 21:51 (UTC)
        % 真係唔該WikiCantona唔好再攞以前個套嚟睇依家D野,唔好咁守舊啦;依家二十一世紀啦仲同你玩「前兩日.Two day ago.(圈)吐爹時亞高 粵音「吐」係讀tu」咩?又係個句,時代變,唔該WikiCantona尊重返呢個時代。我地二十一世紀都終有一日會畀二十二世紀取代,尊重啊。Akai 博士 (傾偈) 2021年9月1號 (三) 06:19 (UTC)
        % 同意Akai 博士。事實上,經過學術界嚴謹査證,已確定一、二百年前粵語好流行噉用嘅詞彙同語法有好多。例如嗰時啲人講「食嘵飯」而唔講「食咗飯」,嗰時啲人講「食飯唔曾」而唔講「食咗飯未」,嗰時啲人講「莫個食嘵飯之後咁遲正去尋個個朋友」而唔講「你咪食咗飯之後咁遲至去搵嗰個朋友」。點解呢啲眞正喺當時咁流行、咁普及嘅詞彙同語法,粵維又唔使跟?反而一個冇普及跡象嘅「唐字」,今日粵維就要老吹佢係「傳統」嘅粵語習慣用法,離地離到上太空都要用?點解咁雙重標準?-Cangjie6 (傾偈) 2021年9月3號 (五) 11:39 (UTC)
        % 「...粵維又唔使跟?... 今日粵維就要老吹...」User:Cangjie6 閣下俾我覺得,你嘅定位,總係有啲「超然」。努力批評粵維點點,但係又唔係好見你幫手寫下嘢,批評雖然係好事,維基百科係一個開放嘅平台,冇人寫嘢,就會停滯不前。Cangjie6 閣下之前勁批評本字係偽本字,我心諗,佢知咁多,點解佢又唔去寫「本字嘅批評」呢?佢肯寫我第一個就走去睇。佢知識廣博,唔係一味批評,肯貼地咁去寫、改,對粵文維基百科會係一件好事。P.S. 我仲寫緊對佢嘅回應。--WikiCantona (傾偈) 2021年9月4號 (六) 04:12 (UTC)
        % 對住一大堆假嘢已經作嘔,仲要想修正時唔畀人修正,焗我同流合污用假嘢,仲鬧我「唔係好見你幫手寫下嘢」?「本字嘅批評」,九座樓主兄有晒硏究,我當時已經貼咗出嚟。「肯貼地咁去寫、改,對粵文維基百科會係一件好事」——我而家咪係貼地噉去改囉!我冇咩熟,係漢字叫做熟少少,首先咪要改番做貼地嘅、現實嘅名先囉!結果一改,就畀WikiCantona閣下你用盡種種卑鄙污糟的手段去唔畀改啊!係你唔畀,然後你詰我唔做嘢?你仲有冇無恥啲?
        % P.S.你唔好再夾硬砌所謂「回應」,死都要砌到「要用『唐字』唔用『漢字』」呢個假「結論」得唔得?你接受下現實,唔好再玩死粵維,停止你嘅超離地操控粵維得唔得?-Cangjie6 (傾偈) 2021年9月4號 (六) 04:34 (UTC)
        % 唔知好嬲定好笑,你覺得自己有理嘅使乜咁緊張啫,講得多,講得快,唔係代表你啱嘅,畀我講埋先囉,又唔係唔畀你回應。--WikiCantona (傾偈) 2021年9月4號 (六) 11:08 (UTC)
        % 講多少少唔多餘嘢,你講晒出嚟?點解唔正正經經,喺「本字」文章之中,將個批評寫返出文裏頭,公諸同好,唔使咁超然喎,唔係畀喺呢度噏有意思得多咩? :-)--WikiCantona (傾偈) 2021年9月4號 (六) 11:52 (UTC)
        % 同User:Cangjie6講埋啲打稻草人嘅說話係冇意思,例如「唔知好嬲定好笑,你覺得自己有理嘅使乜咁緊張啫,講得多,講得快,唔係代表你啱嘅,畀我講埋先囉,又唔係唔畀你回應。」、「唔使咁超然喎,唔係畀喺呢度噏有意思得多咩? :-)」。刪走呢啲打稻草人,喺到釣魚式字句,猜測人哋諗緊咩嘅發言,仲要係零根據性嘅嘢,先得個一兩句係真正嘅討論。仲要依家共識未變到返去用「唐字」,共識仲係「漢字」。特克斯特 (傾偈) 2021年9月5號 (日) 04:30 (UTC)
        % User:特克斯特閣下,做嘢有步驟。寫完啲參考出嚟,畀人話係【造假】,定義就係:「製造假的來偽裝真的,《教育百科》。」我引外面嘅嘢係「製造出嚟」?圖係 P 出嚟嘅?!好,由得佢講。下面我整理咗搵到嘅資料,我哋意見一向好唔同,不過,你一向對外面嘅資料嘅態度相當認真,唔係盲反,亦請你睇睇,challenge,指出問題。「真正嘅討論」應該基於事實,唔係講得多,講得快,重複又重複嘅意見。--WikiCantona (傾偈) 2021年9月5號 (日) 22:02 (UTC)
        % 我上面已經講咗,而家講緊嘅,係要喺粵維度用「唐字」而唔用「漢字」,聲稱「唐字」係所謂「傳統」嘅粵語習慣用法,而且佢嘅資格大過「漢字」。要證明呢樣嘢,要公平咁搵證據,最最最少都要搵喺某個時空嘅共時比較,搵過嗰個時空嘅語料,有幾多係支持「唐字」,有幾多係支持「漢字」。而唔可以發狂噉摷好大堆好大堆嘢,然後只要一摷到有乜嘢係用咗「唐字」呢兩隻字,唔理個餡點,就即刻攞出嚟曬,完全冇共時嘅使用對比。否則就只係「先有咗要導向嘅結論,然後夾硬砌啲『證據』出嚟」嘅反學術手法,係僞證手段,係絕絕對對嘅造假。點解我要重複?係因爲WikiCantona你堅持無視、堅持造假、堅持宣稱自己嘅造假行爲唔係造假!唔係WikiCantona你「堅持」落去,就等於你唔造假。WikiCantona你一日繼續係噉做唔肯改,你一日都係造假!
        % 至於其他人身攻擊嘢,多謝特克斯特幫我講咗,我慳番啖口水。-Cangjie6 (傾偈) 2021年9月8號 (三) 07:23 (UTC)
        % 外部來源,續
        % 將所有我搵到嘅資料,完完整整咁樣整理咗一次,下面每一項,會首先列出書名、出版日期、作者、出版地點 / 機構。然後將「唐字」兩個字出現嘅地方 / 頁數 link 出去或寫低,照原文引,俾大家自己去睇,攷證。
        
        % 幾本 Handbook,Phase book,教學用書
        % 第一本,初學階 A HANDBOOK OF THE CANTON VERNACULAR OF CHINESE LANGUAGE,1874年,N.B. Dennys 寫嘅,喺英國倫敦出版。對象係講英文嘅人。
        % 其中第 33 頁,兩句:「19 Can you interpret Chinese characters 唐字你會翻譯唔會 20 I have interpreted Chinese characters 我有翻譯過唐字」。
        % 第二本,A CHINESE AND ENGLISH PHASE BOOK In the Canton Dialect; or DIALOGUES ON ORDINARY AND FAMILIAR SUBJECTS FOR THE USE OF THE Chinese resident in America, and of Americans desirous of learning Chinese Language; with the Pronunciation of each word indicated in Chinese and Roman Characters.,1888年,T. L. Stedman 同 K. P. Lee 作,喺紐約出版。本書寫畀住喺美國嘅唐人,想學中文/唐文嘅美國人。
        % 其中第 39 頁,有句噉寫「十八 你唔通係寫唐字啞。'ni ,t'ung hai' ‘sye .t'ong chi' 'a 十九 唔係我現時寫緊英國字。.... 」。
        % 第三本,ENGLISH AND CHINESE LESSONS,Rev Augustus Ward Loomis 作,1922年,AMERICAN TRACT SOCIETY,美國三藩市(舊金山)。目的係畀傳教士同埋教書先生用,教大人講英文同傳道用。
        % 喺個「序」有一句:「倘有番人欲學唐語者、宜用筆每唐字之下、加寫音韵...」
        % 第四本,《Cantonese Conversation--grammar, 第 2 篇》,1967年,Xiling Huang, University of Hong Kong. Institute of Oriental Studies, Hong Kong. 係作者。Education Dept Government Printer, under the auspices of the Institute of Oriental Studies, Hong Kong University 係出版商。
        % 第 142 頁:「B.係,書法即係要將漢字(唐字,中國字)的筆畫,寫得齊齊整整,睇起嚟似個字樣,至算合格。 ... you have to interpret it into Chinese for a Cantonese teacher.」
        % 第五本,First Year Cantonese,1966年,Thomas A. O'Melia 作,Catholic Truth Society 出版。
        % 第 49 頁寫有:「佢 而家 寫啲係 唐字 Those are Chinese characters he's now writing」
        
        
        % 「唐字」音外語嘅書
        % 第六本,English Made Easy《唐字音英語》,1905年,莫文暢。(註: 呢本喺內容,全部收入 2009 年出嘅《唐字音英語和二十世紀初香港粵方言的語音》 )
        % 第七項,係一張圖 ,分別係兩個網站出現:香港人母語學英文,【千金難買少年窮】睇通勝,學英文,類似嘅圖片,可以睇:網上圖片。佢哋係「五、六十年代出版的自學英語工具書」。
        % 第八本,《唐字音日語》,1972年,張秀英編著,香港志文出版。(註: 呢本係香港公共圖書館嘅藏書)
        
        
        % 近期嘅書
        % 第九本,《舊時風光──香港往事回味》,2006年,陳雲作,花千樹出版社。
        % 第 144 頁,陳雲咁寫: 「哪些是簡體字,有些人仍認為不是唐字。」。
        % 第十本,《不可不知的历史常识》,張雪芹作,飛翔時代出版。
        % 其中寫:「从此,海外人对中国的一切都以“唐”字加称,如称中国人为“唐人”,称中国的字为“唐字”,称中国为“唐山”」
        % 第十一本,《華僑日報》與香港華人社會 (1925-1995),2014年,丁潔作,三聯書店(香港)有限公司。
        % 第 34 頁:「如貴客有事要印落此紙內,務宜早一日走字通知未事孖剌,便妥印刷。唐字價錢如左:每次落唐字者,五十個字已【以】下收銀一員【圓】...」。同一段字亦出現喺第 89 頁,香港報業史稿,1841-1911,2005年,陳鳴作,華光報業有限公司出版。
        % 第十二本,《唐字音英語和二十世紀初香港粵方言的語音》,2009年,黃耀堃教授,丁國偉博士合著,香港中文大學中國文化研究所吳多泰中國語文研究中心出版。
        % 第 24 頁:『所謂「唐字」指漢字,而「唐字音」指「廣東話」,該書凡例稱「所有字音,用正廣東話讀」。』;
        % 第 91 頁:『至於「唐字音」和「唐字調音」是指甚麼?「唐字」是指漢字,「唐字音」指的是粵語方言的標準音。』;
        % 第 444 頁:「030.08 壹個唐字 One character 溫、詫力唾」;
        % 第 528 頁:『5 「song」缺唐字音』;
        % 第 564 頁:「2 最前疑脫唐字音」。
        % (註: 呢本係香港公共圖書館嘅藏書)
        % --WikiCantona (傾偈) 2021年9月5號 (日) 22:56 (UTC)
        
        % 評
        % 呢十幾本書/項,唔同年代,有唔同作者,不約而同講出「唐字」就係「Chinese Character」、「中國的字」、「中文字」、「漢字」。 我俾咗呢啲書一啲背景資料,其中俾我哋睇到「唐字」一辭嘅時代脈絡,之後我會再詳細講。至於題目嘅討論,亦會再進一步講。 --WikiCantona (傾偈) 2021年9月5號 (日) 22:56 (UTC) 呢本《唐字音英語和二十世紀初香港粵方言的語音》,主要兩個學者寫作,係用咗 1904 年同埋 1913 年(通行版)《唐字音英語》,裏頭嘅用廣東粵語字音配對英文字音,對「早期粵語色語音特色」嘅研究,從中討論「粵語音調」百年來嘅變化。--WikiCantona (傾偈) 2021年9月5號 (日) 23:03 (UTC)
        
        % 「「唐字」是指漢字」,明白,咁即係用「漢字」先係正統,多謝哂WikiCantona。我想講,之所以會出現「唐字」,係因為廣東話承接了左中古漢語(又叫唐代漢語),加上唐朝國力強盛,令到廣東呢邊啲人響19世紀出到去外國就以「唐人」「唐字」自居,依家都21世紀啦,仲同你玩呢套咩。另外WikiCantona所提供既書籍入面,大部份都係研究以前既廣東話,為求用字準確,理所當然係會用「唐字」(就好似我以前中學本英文書剩用「Mainland China」唔敢膽用「China」 目的都係為求用字準確 同埋驚畀人話搞港獨)。Akai 博士 (傾偈) 2021年9月7號 (二) 11:15 (UTC)
        % 我上面已經講得清清楚楚。而家講緊嘅,係要喺粵維度用「唐字」而唔用「漢字」,聲稱「唐字」係所謂「傳統」嘅粵語習慣用法,而且佢嘅資格大過「漢字」。要證明呢樣嘢,要公平咁搵證據,最最最少都要搵喺某個時空嘅共時比較,搵過嗰個時空嘅語料,有幾多係支持「唐字」,有幾多係支持「漢字」。而唔可以發狂噉摷好大堆好大堆嘢,然後只要一摷到有乜嘢係用咗「唐字」呢兩隻字,唔理個餡點,就即刻攞出嚟曬,完全冇共時嘅使用對比。否則就只係「先有咗要導向嘅結論,然後夾硬砌啲『證據』出嚟」嘅反學術手法,係僞證手段,係絕絕對對嘅造假。
        % 結果,WikiCantona只係將佢嘅造假發大嚟造,造得大型啲,繼續係發狂噉摷好大堆嘢,一摷到有「唐字」呢兩隻字就大曬特曬,繼續完全冇共時嘅使用對比。講明噉樣係造假,佢就繼續發大嚟造,然後死鹹魚拗返生噉夾硬話唔係造假!噉只係反映WikiCantona繼續肆無忌憚噉造假。WikiCantona堅持無視、堅持造假、堅持宣稱自己嘅造假行爲唔係造假,根本只係濫用拖字缺,假扮仲拗緊未討論完之類。根本已經完咗,講落去都嘥氣。成個共識就係普遍認同用「漢字」,只係幾個「歸西派」堅持造假,佢哋將會繼續造假落去,但唔會改變到個共識。完。-Cangjie6 (傾偈) 2021年9月8號 (三) 07:23 (UTC)
        % @Detective Akai:閣下講得啱!「漢字」係正統,「唐字」係異端;普通話係正統,廣東話只係方言;「書面語」為正統,「粵語」唔係攞來寫嘅;「我們」係正統,「我哋」就係地方俗語;「中文維基百科」係正統,「粵文維基百科」就係傍枝;.... --WikiCantona (傾偈) 2021年9月24號 (五) 23:40 (UTC)
        % {@Cangjie6:講「造假發大嚟造」,「發狂噉」,「肆無忌憚噉造假」,有乜意思呢?我全部嘅 source 都列晒出嚟,click 一 click 就睇到,你睇到係假嘅,指出嚟囉。如果鍾意撩交嗌,好多網上 forum 可以去嘅,good day。--WikiCantona (傾偈) 2021年9月24號 (五) 23:40 (UTC)
        % 睇返十九世紀尾嘅「漢字」文獻,有東印度公司出版嘅字典 A Dictionary of the Chinese Language: English and Chinese 作者:Robert Morrison,用嚟比生意人同滿清官員溝通,Ignorance of Chinese 不識漢字。眾所周知,漢字等於 Chinese,呢個用法係出現喺滿清官話之中。1880年喺上海出嘅語言自邇集,用嘅係北方中文拼音。
        
        % 相反睇法,「唐字」嘅用法就正正係外地唐人(賣豬仔過埠嘅廣東人)社群,同香港英治地時代嘅用法。對比起上來,你話邊個傳統啲啦?!「唐字」字嘅用法,係十九世紀末廿世紀初嘅香港,報紙上的確咁用。
        
        % 應該點樣,作為粵文維基百科,有粵語嘅用詞,即使過時,理應使用,喺內文度澄清。再唔係嘅話,可以考慮用「唐字 (漢字)」或「唐字/漢字」,講完。--WikiCantona (傾偈) 2021年9月24號 (五) 12:43 (UTC)
        
        % 本章節異論嘅收窄同對策嘅整理
        % 遺憾等待未見提論嘅紛擾可適當再梳理返新舊認知變遷間嘅大概,淨系睇到上邊訴諸人身嘅成份太多,可以續議嘅細節度話,如果做返跨時空論述連英皇道、差館或者其他啲唔受部分歡迎嘅傳統字眼都可以即時overthrow——顯然有啲朋友仲係未曾跳出時空限制嘅誤區,要依賴返維基百科繼續編輯、就必須秉持返一貫吸納百川、兼收並容嘅精神同普羅多元化內龍所真認可嘅共識,與其以新舊一時衝突去左右,倒不如將新舊認知嘅改變歷程提煉寫落文本睿饗粵維同各用家,將參考資料同討論意見一齊整理成適當草稿形成一個唔希望再為激蠻遲滯文案嘅改進。——Longway22 (傾偈) 2021年9月8號 (三) 10:43 (UTC)
        
        % 共識制度?同討論用字方面
        % @Longway22、WikiCantona、Cangjie6、特克斯特、Z423x5c6、Dr. Greywolf、Pokman817:各位應該要尊重街坊嘅共識,如果有大部份人都係傾向用選項A,咁就應該要尊重返呢個意見。目前可以見到,一共有十一位用戶參與呢次討論,其中:Cangjie6、特克斯特、Z423x5c6、Dr. Greywolf、Pokman817、UU、⼥⼉、N6EpBa7Q,再加埋本人,支持用「漢字」嘅總共有9人;而Longway22、WikiCantona,係反對用「漢字」,目前只有兩票。跟據國際慣例(議會制),一旦同意票超過三分之二(即百分比六十六又二分之三)嘅門檻,該法案即屬通過;而睇返依家形勢,支持嘅人佔討論總人數嘅十一分之九(即百分比八十一又九分之十一),已經超過三分之二同意票門檻。如果跟住落嚟嘅一個禮拜之內,反對人數係少於五個人嘅話,就應該尊重大部份嘅意見,用「漢字」。Akai 博士 (傾偈) 2021年8月22號 (日) 04:03 (UTC)
        % 唔好話我荒謬,就連號稱「最民主嘅國家」美國,佢地參議院嘅通過門檻係超過二分之一,即屬通過;而就算係眾議院,通過門檻係五分之三,目前支持用「漢字」都有十一分之九,都超過五分之三(即60%)。Akai 博士 (傾偈) 2021年8月22號 (日) 04:11 (UTC)
        % sor,呢度正正要點名akai同其他幾位朋友係有強行投票取代衡常研判辯論嘅做法,@Dr. Greywolf@Pokman817@WikiCantona@Deryck Chan同時關注有正接受提名嘅候選人Z423x5c6 都有參與呢個所謂投票。如果事態發展唔可以衡常遏制,本編認為係可能有必要發起緊急動議關閉候選案同其他一啲關聯議案,保護有關版面或限制部分編輯嘅權限等等,以阻止本地可能出現嘅集團軍類型事態 Longway22 (傾偈) 2021年8月22號 (日) 04:18 (UTC)
        % akai引用嘅唔係正常本地應該遵循嘅做法,反對以上嘅表述同衍生嘅任何強制力——上述表達係有強行以投票取代討論、甚至乎係可能有多數暴力/暴政嘅做法,同時係無視史料同事實嘅情況下、以單一時間點嘅集中谷票搞出所謂嘅共識,唔單止違背咗有關嘅現實記錄同習慣傳統,認為亦係嚴重衝擊咁本地一啲既有嘅基礎同普世少數嘅認可平衡。若果係有繼續強制用點人頭等嘅方法去達成本案嘅單一目的,本編會循例動議討論本地正出現嘅多數暴政等新問題,尋求更多唔同朋友留意返依家嘅部分所謂投票正罔顧粵圍衡常文明邏輯甚至有強化大一統風險嘅問題存在 Longway22 (傾偈) 2021年8月22號 (日) 04:12 (UTC)
        % @Longway22:荒謬,呢啲叫多數暴力?Longway22唔係好講求民主嘅咩?大家唔信去睇下好多民主國家,佢地邊一個議會唔係「少數服從多數」?唔該Longway22上網查下民主嘅定義,民主有個花名叫「多數的統治」,原來有人唔知架?梗係有人反對,有人支持,如果兩方爭持不下,唔用返「少數服從多數」,你叫法案點通過?Akai 博士 (傾偈) 2021年8月22號 (日) 04:25 (UTC)
        % @Longway22:依家啲人完全將民主個意思偏離。原先民主響希臘文嘅意思係「交畀多數人統治」,呢個正正符合依家嘅情況。請Longway22尊重多數人嘅意見!如果Longway22再唔承認我地九個人嘅意見,咁你係當我地九個人係死架?Akai 博士 (傾偈) 2021年8月22號 (日) 04:29 (UTC)
        % @Longway22:少數服從多數的嘅民主制,雖然我地唔可以強迫少數反對者支持多數贊成者嘅意見,但少數反對者要尊重返多數贊成者嘅意見,呢個先係民主!Akai 博士 (傾偈) 2021年8月22號 (日) 04:35 (UTC)
        % 我估冇人會聽嘅啦,即管講吓。呢道一直唔搞投票,一句講晒,「有心人」一定贏。「有心人」可以係組職票,「有心人」可以一人多戶。投票成本低,討論成本高。如果開先例,後果可以好嚴重。例如一篇文睇唔順眼,湊夠人就可以點都得。呢個噉細嘅維基,投票一定冇得玩㗎。今次你哋想點就點,我亦唔會再拗。只希望大家睇到,可能會有手尾長。最後就算搬去大家嘅叫法,千祈千祈唔好用「投票通過」做理由。--WikiCantona (傾偈) 2021年8月22號 (日) 05:31 (UTC)
        % @WikiCantona:你係咪對政治敏感?呢啲叫「少數服從多數」,乜鬼野「投票通過」?Akai 博士 (傾偈) 2021年8月22號 (日) 05:41 (UTC)
        % 「少數服從多數」、「投票通過」、「投票結果」都係一樣手尾長。--WikiCantona (傾偈) 2021年8月22號 (日) 05:49 (UTC)
        % @WikiCantona:「少數服從多數」唔係搵共識嘅最好方法之一咩?一係你話畀我知,有邊個方法比「少數服從多數」更加好搵共識啊?Akai 博士 (傾偈) 2021年8月22號 (日) 05:53 (UTC)
        % 而且畀人睇到嘅係,有少數偏執狂點都要用自己嘅主張做準,無論有幾多人唔同意,有幾多客觀證據證明佢哋嗰套唔得,幾位偏執狂都會砌出各種藉口,去「污名化」大家、「污名化」啲證據。Cangjie6 (傾偈) 2021年8月22號 (日) 09:03 (UTC)
        % 呢幾日大家關注番呢條條目嘅命名,係因爲我有個朋友見到蠶豆症度,有人自己作咗個定摷咗個「提子糖」嘅叫法出嚟,而唔用正常嘅「葡萄糖(/醣)」。於是User:特克斯特喺Talk:蠶豆症度提起,然後User:WikiCantona事隔1年又10個月後忽然回應呢一版,話「費事拗」,要繼續用「唐字」。我知道,有啲朋友可能好堅持要同中維唔同,只要中維用邊種叫法,粵維就死都要盡可能用唔同嘅叫法。我想講,呢種心態係唔健康嘅。粵語嘅普遍叫法同中維用嘅唔同,噉就當然用唔同叫法啦。但係,如果環顧成個粵語社群,普遍叫法真係同中維一樣,噉就冇理由無視語言嘅約定俗成,死都要標奇立異。用「提子糖」取代「葡萄糖」如是,用「唐字」取代「漢字」(或「中文字」)亦如是。堅係同官話白話文唔同嘅地方,就當然要唔同;但堅係一樣嘅地方,就唔應該夾硬要為咗唔同而扭曲語言。我最最最初見到粵維,就係畀各種標奇立異嘅叫法,寫埋晒啲火星文僞正字(甚至當時你唔跟住寫會畀人改),同埋所謂「國維行話」(同粵語組嘅好多位朋友傾閒偈傾開,大家都話連本身呢四個字都勁難明)有殺過冇放過嘅屠刀嚇窒嘅。我到而家都冇放棄(但亦唔算多產)係少數,我有好多friend都係畀呢兩大欄路虎趕走,係直頭覺得呢度黐黐哋線,冇興趣參與嗰隻。再係噉落去,堅持呢啲標奇立異嘅嘢,對粵維眞係有害㗎。-Cangjie6 (傾偈) 2021年8月22號 (日) 09:06 (UTC)
        % 請唔好以呢啲毫無邏輯同建樹嘅偽術去討論唐字等粵文傳統習慣,跟著就為大一統話術背書,呢點經已係嚴正表述過一旦實證到本地係唔會容忍嘅、同時呢家係明顯由本案度睇到係正正出現咁——轉移話題唔代表到本案家下強行投票谷數有任何嘅合理度,亦唔係反映到維基機制下嘅共識。 Longway22 (傾偈) 2021年8月22號 (日) 09:33 (UTC)
        % @Longway22:講邏輯講建樹?閣下喺呢一個頁面入面除咗對唔同嘅用戶攻擊之外,唯一嘅論點就係「考慮返對歷史文化嘅充分傳承同保育」,我真係唔知閣下嘅建樹喺邊到啦,咁鐘意保育啲舊時嘅講法,隔籬文言文維基歡迎你。唔送,拜拜~--Z423x5c6 (傾偈) 2021年8月22號 (日) 16:13 (UTC)
        % 首先,如果真係「粵文傳統習慣」,咁就唔應該只係得零丁證據,而係應該有可見嘅量。而家呈現嘅結果,就證明咗根本唔係傳統習慣,傳統上大家冇習慣咁叫,當然更加冇將呢個「習慣」傳承到今日。套用《三五成群》嘅「神仙B」對白:「乜×嘢傳統呀?冇人識你喎!而家冇人識你呀!人起朵你起朵!傳?邊×度傳呀?睇下!睇下!邊×度傳呀?××××!」根本喺現實上就冇人噉叫。要Longway22專重事實、專實現實,係咪真係咁難?
        % 至於老屈我「以呢啲毫無邏輯同建樹嘅偽術」、「為大一統話術背書」,哇哈哈,有啲人就係鍾意老屈當討論。-Cangjie6 (傾偈) 2021年8月22號 (日) 16:29 (UTC)
        % @Cangjie6、Deryck Chan:同意,上面參與討論嘅仲要有兩位用戶都係管理員嚟:User:Pokman817、User:Dr. Greywolf 居然都可以直接無視意見(仲被標籤做「有心人」)(上面參與嘅用戶已經係粵文維基嘅常客嚟 唔係中文維基用戶集體過嚟投票畀意見個款)。至於「國維行話」呢樣嘢我初頭一嚟(以外人身份)嚟睇都係唔明,同意用字上有改善嘅空間,叫「北方話」都可以(呢個字應該人人明),呢樣係關乎User:Shinjiman整嗰個過濾器嘅標籤問題(亦都可以叫User:H78c67c幫手睇睇)。我喺Google揾「國維行話」,第一個結果竟然係出咗User:Kowlooner嘅留言:User talk:AngeCI\#國維行話,而唔係乜嘢粵典 粵詞網站之類嘅結果。特克斯特 (傾偈) 2021年8月22號 (日) 09:34 (UTC)
        % 依家嘅問題並唔係針對所有參與投票人人身,就事務層面係要指定(利益衝突)人身嘅,係提名Z423x5c6做管理員嘅Akai 博士,管理候選人Z423x5c6同有支持票嘅特克斯特,可能有串謀或關聯利益同影響其他人嘅取態,基於User:WikiCantona有同以上一人或多人有利益矛盾,WikiCantona經有獨立表達咗支持Z423x5c6候選,但本案所見認為係未需要考慮呢個因素而係睇到wikicanton可能係面對咗利益衝突方集火圍攻嘅多次做法,利用部分強硬姿態去影響咗涉及wikicanton編輯嘅專案、進一步針對咗本地原有獨立嘅傳統規例基礎,相信係合乎粉紅化集團軍(製造大一統化所謂共識)嘅表現@Dr. Greywolf@Pokman817@Deryck Chan 其他管理同其他未提及嘅編輯人等,個人判斷係基本係未有曾明顯咁長時間同持續不斷咁針對WikiCantona經手專案同編輯做法嘅表現(至少本地層面度)所以唔會考慮係有進一步嘅其他可能——而同時間考慮有關編輯係有多次對本地既有機制、包括粵文基準規例等係採取無視兼當唔存在,加上本案度嘅集火做法經係有極大風險連累本地運作同維持機制陷入癱瘓,必須緊急應對依家嘅情況包括關閉本案同發出適當告誡等 Longway22 (傾偈) 2021年8月22號 (日) 09:50 (UTC)
        % @Longway22:身為提名人點解唔支持返自己提名嘅人? 我唔支持佢 都唔會提名佢出嚟啦Akai 博士 (傾偈) 2021年8月22號 (日) 09:55 (UTC)
        % 講起利益衝突,點解唔講埋有份投支持票嘅User:Dr. Greywolf都有份支持用「漢字」同支持User:Z423x5c6做管理員?仲要ping埋Greywolf個下話我、Akai、Z君喺兩個討論意見一樣個下先好嘢(即係大家都支持用「漢字」同支持佢做管理員)。係咪要我拆穿你先至等大家知道你依家純粹係因為同你意見唔一樣嘅就盲反?(仲要濫用諸多理由「污名化」參與者,即係User:Cangjie6咁講)。上次喺Talk:吳君如音樂作品已經係咁樣盲目做嘢。User:Dr. Greywolf喺管理員投票同今次「唐字/漢字」寫法問題,同你嘅意見夠係完全相反。特克斯特 (傾偈) 2021年8月22號 (日) 10:00 (UTC)
        % 今次我水洗都唔清。「有心人」的確可以用 User:特克斯特 同 User:Cangjie6 兩位嘅演譯嘅(「標籤」「污名化」其他 User)。不過,我從來冇咁諗,算啦,點講都冇用啦。重覆,當用投票開咗先例,粵維呢啲咁嘅細維基,畀人利用投票嚟到控制,易到極。「有心人」就係啲對粵維有心破壞,惡意同化嘅人。
        % 其實大家都知啦,共識係傾返嚟,唔係投返嚟。我以為 User:Cangjie6 肯傾。好失望。--WikiCantona (傾偈) 2021年8月22號 (日) 10:01 (UTC)
        % @WikiCantona、特克斯特:好啦,粵維開始左投票,證明粵維唔再係以前既小小維基。點解我莫名其妙覺得好感動...Akai 博士 (傾偈) 2021年8月22號 (日) 10:06 (UTC)
        % @Detective Akai:我都希望係好事,亦都希望平安無事,唔係咁樂觀,用主觀意願、冇理性討論基礎嘅「投票」呢瓣嘢我會掂囉。Good luck。--WikiCantona (傾偈) 2021年8月22號 (日) 10:20 (UTC)
        % 我何來唔肯傾?我只係要求傾嘅過程,必須要尊重事實同現實,噉係要理性去傾嘅基本條件。但係WikiCantona閣下同Longway22呢?只要個事實同你哋先入為主嘅立場唔同,你哋有尊重過咩?都莫講話尊重事實,見到人哋有唔同意見,老屈我「為大一統話術背書」,老屈Akai 博士、Z423x5c6、特克斯特等等有「利益衝突」、「相信係合乎粉紅化集團軍(製造大一統化所謂共識)嘅表現等等,請問係邊個冇得傾?係邊個要打爛仔交?-Cangjie6 (傾偈) 2021年8月22號 (日) 16:39 (UTC)
        % User:Cangjie6 閣下如果你就憑三個字「費事坳」,斷定係唔尊重呢個討論,喺本人之前作出過道歉,之後亦繼續攞返出嚟講。首先,閣係對我搵到嘅參攷完全唔理會,之後加以詆毁,而到最近勉強承認「唐字」舊時用過吓,而家已經「斷纜」。以閣下將一部份用家標籤做「為改以改」,「標其立二」,亦唔見有咩幫助。喺「為改而改」到「盲目跟從/反對」兩個極端中間存有好多觀點,本人嘅觀點係,喺合適/合理嘅情況之,粵文維基嘅文章命名,可以唔同;睇本人最近嘅開文: 硬水,岳史迪嘅名都同中維一樣。上次喺維基 logo 嘅討論,同閣下有過一輪討論,雖然我哋意見唔同,覺得閣下亦係一個明辨事理,唔係為反對而反對嘅用家,所以今次嘅表現令我有啲失望。更加可惜嘅係我同你多年前都覺得「中文字」可取,呢個共同位亦都冇埋。只有嘆氣。 :-(--WikiCantona (傾偈) 2021年8月23號 (一) 06:42 (UTC)
        % 我而家畀人老屈緊「以呢啲毫無邏輯同建樹嘅偽術」、「為大一統話術背書」,我就真係唔慌唔失望,唔慌唔嘆氣!我指出「標奇立異」(注意:唔係「標其立二」,你做乜將原本啲字改到咁標奇立異?),係因為件事堅係標奇立異,而唔係標籤用家。一個叫法,而家現實度幾乎完全冇人咁講,就算査歷史文本文獻都係得幾個僻例而且睇唔到有普及過、有發展過的痕跡,然後有幾個人就死都話佢先係正,迫其他人一定要跟,噉嘅情況,我話係「標奇立異」,自問已經係有禮貌嘅、婉轉嘅講法喇。喺合適/合理嘅情況,粵維文章名當然可以同中維唔同,但係而家「唐字」正正係個唔合理嘅,違反粵語自然語境嘅離地情況,如果要可惜,我真心覺得粵維畀偏執做法揸旗搞到嚇走人,先係最可惜。--Cangjie6 (傾偈) 2021年8月23號 (一) 10:03 (UTC)
        % 投票結果當然都係去展現共識嘅其中一個方法同形式嚟。特克斯特 (傾偈) 2021年8月22號 (日) 10:07 (UTC)
        % 咁係咪本地可以喺度投票支持引入國安法咁取代其他維基方法? Longway22 (傾偈) 2021年8月22號 (日) 10:17 (UTC)
        % 送俾你㗎:Wikipedia:唔啱維基百科嘅嘢\#維基百科唔係民主試驗場--WikiCantona (傾偈) 2021年8月22號 (日) 10:20 (UTC)
        % @Cangjie6:「呢幾日大家關注番呢條條目嘅命名」,小小 friendly 提示,呢個維基百科叫「文」或「文章」--WikiCantona (傾偈) 2021年8月22號 (日) 10:25 (UTC)
        % 國安法同呢次討論無關。「不過有時亦會通過投票去達成共識,但任何投票或者問卷調查實際上都係有導向性」,3個選擇已經好有導向性,文中冇一刀切禁止用投票形式嚟達成共識。特克斯特 (傾偈) 2021年8月22號 (日) 12:18 (UTC)
        % 唉,我有陣時真係覺得你哋好無聊。In sum,我覺得呢篇文描述緊嗰樣事物呢,我腦海入面浮現嘅第一個 term 係「漢字」。「唐字」呢個 term 喺嚟粵維之前我從來都未聽過。如果俾我改名嘅,我會主張嗌呢篇文做「漢字」。係噉啦,我而家將呢一頁由我個監視清單嗰度攞走,費事我係噉勁收 email 話俾我聽呢頁有人改。Dr. Greywolf (傾偈) 2021年8月22號 (日) 12:29 (UTC)
        % 理解博士同其他認真做文嘅朋友可能未有涉獵專攻、未必有認真扒返一啲可能少關注嘅資料情況,上邊部分依家嘅個人判斷僅係針對返經開咗名嘅用家依家所見啲表現而定。wikicanton本案度係經已提交咗唔少史料文獻支持返呢個page name,博士同其他有意了解嘅朋友得閒都可以睇返下呢啲參考 Longway22 (傾偈) 2021年8月22號 (日) 12:33 (UTC)
        % wikicantona提供嘅參考已經被我駁返哂,最新嗰啲參考資料都係50年前嘅舊嘢。如果未睇上面嘅話可以畀啲時間你睇返,再繼續進行有意思嘅討論,而唔係好似你依家噉周圍lur地。--Z423x5c6 (傾偈) 2021年8月22號 (日) 16:15 (UTC)
        % User:Z423x5c6 閣下,我好有禮貌咁樣唔同意你反駁咗。有好幾個層次可以討論下,不過,時機唔啱,等下先。 --WikiCantona (傾偈) 2021年8月23號 (一) 06:47 (UTC)
        % @WikiCantona:的確,解決咗啲嚴重啲嘅嘢先。呢啲學術嘢遲啲再斟。--Z423x5c6 (傾偈) 2021年8月23號 (一) 11:59 (UTC)
        % For the love of Christ,如果唔係有嘢需要問我,唔該唔好喺度 ping 我。Dr. Greywolf (傾偈) 2021年8月22號 (日) 12:51 (UTC)
        % @Detective Akai、特克斯特: 我必須要講,投票當共識呢樣嘢係非常之危險嘅。相關政策無規定共識形式方式,但用投票嚟取代討論唔係可行嘅方法。但亦都希望@Longway22:注意返,唔好亂咁指控人係夾埋集團式攻擊。H78c67c·傾偈 2021年8月22號 (日) 18:39 (UTC)
        % 我個人冇話投票一定可取或者一定唔可取,但係唔肯睇現實同埋老屈就一定唔可取。至少上面Longway22犯晒。-Cangjie6 (傾偈) 2021年8月22號 (日) 21:48 (UTC)
        % 尊重cangjie閣下喺本案之前嘅參與意見,但希望唔好因為同wikicanton閣下有過諸多唔愉快而就依家加入埋谷票種種,cangjie閣下可以獨立就文本資料提供意見,或論述返點解唔考慮用中文字、就係偏好漢字呢個就依家環境變動下明顯含有大一統意味嘅代詞?本編亦希望閣下如果有意繼續理解唔同知識內容嘅話、可以再獨立參詳返wikicanton閣下提供嘅唔少寶貴資料 Longway22 (傾偈) 2021年8月23號 (一) 00:28 (UTC)
        % 請Longway22閣下停止將所有反對wikicantona嘅論點歸因於「同wikicanton閣下有過諸多唔愉快」呢一種稻草人論證。Z423x5c6 (傾偈) 2021年8月23號 (一) 01:24 (UTC)
        % 似乎大家都唔支持投票咁我就算數,但Longway22的確係選擇性無視大部份人嘅意見;而我相信Z哥亦唔係啲記仇嘅小人,請Longway22唔好諗多。Akai 博士 (傾偈) 2021年8月23號 (一) 02:01 (UTC)
        % 我冇參與任何「谷標種種」,相反,Longway22你憑咩將同你相反嘅聲音老屈做係「谷票」?我亦冇因為「同wikicanton閣下有過諸多唔愉快」而唱反調,一直以嚟,除咗曾經逼迫我要用偽本字兼企圖喺粵維用「原創法蘭西」取代粵拼嘅「殘陽孤俠」,我冇喺粵維上針對過任何一個人,或者因為一個人喺某事上嘅發言就令我討論另一件事時因人廢言。但係叫「漢字」而唔叫「唐字」,係自自然然嘅粵語生境度嘅自然現象,絕對絕對絕對絕對絕對唔係咩「含有大一統意味」,唔該你即刻停止貼呢個嚴重極度失實嘅老屈標籤!而所謂「wikicanton閣下提供嘅唔少寶貴資料」根本只係證明咗就算摷歷史文獻同文本,「唐字」呢個叫法都僅有好少數僻例,毫無普及、流通、傳承等等嘅痕跡。Longway22閣下仲要因為你嘅先入為主立場,去篩選事實、選擇性失明到幾時呢?請問Longway22又知唔知咩係Wikipedia:唔啱維基百科嘅嘢\#維基百科唔係用嚟鬥氣,同埋知唔知咩係Wikipedia:假定善意?只要睇法同你唔同,就又老屈又誅人心,點討論?我一路都冇話特別撐你或者唔妥你,但係喺呢度一個討論,你就露晒餡。唔好咁偏執狂,唔好咁陰謀論居心論,放眼睇下現實情況唔好咁離地,好難咩?-Cangjie6 (傾偈) 2021年8月23號 (一) 10:07 (UTC)
        % @Cangjie6:本人同意閣下呢句:「我個人冇話投票一定可取或者一定唔可取」;亦要一齊指出,你本人今次有份投票。本人上面講,細維基投票先例一開相當危險,用上「有心人,一定贏曬」做開頭,寫得差,可能令各位誤會而 say sorry 先。提起「善意假設」, 閣係講:「我有好多friend都係畀呢兩大欄路虎趕走」 ,「偏偏特登無視約定俗成... 係『偏執狂』」,維基係自由參與嘅,你啲friend係咪俾人封咗一世?何來 「欄路虎」? 「偏偏特登」? 本人明白閣下以日常用語做根據嘅重要,覺得粵文維基行錯路,谷埋谷埋啲氣(同呢篇冇關嘅本字,粵拼咁),不過咁樣話其他用家,我唔見得有乜嘢好處。最後,都想閣下對「唐字」嘅幾點睇法回應,除非本人突然間畀㗎的士撞死(touch wood),一定會同閣下繼續討論「唐字」/「漢字」呢個題目。當中涉及「漢字」嘅範圍(知道閣下對「統一漢字」(Unihans) 有相當嘅知識),不過家陣睇嚟未係時候(要去處理其他無理指控)。 --WikiCantona (傾偈) 2021年8月24號 (二) 09:41 (UTC)
        % @Cangjie6@WikiCantona :搵返華語嘅定義爭議(嗰時主中維),由返好似地方語言嘅少數同大一統嘅多數比較,回顧返我哋粵文同中文文明圈單一嘅用字都所呈現嘅諸多唔同姿態,係唔係可以話求其voting框死一個指向就叫天眼開?本編亦接受依家呢度原本有所表述嘅一啲認為係潮流化下普羅或者有嘅認知偏差,但係咪為咗徹底推翻經已可能唔時髦嘅稱呼或其他嘅存在,係咪真有改善嘅編輯?
        % Cangjie6閣下同本編,仲有各位留意過相應議案過程嘅朋友,都清楚就喺中文圈度華語係本身就有多個唔同嘅含義,同時間係唔可以話單獨用少數或多數嘅簡單化方程式去到徹底咁抹殺某啲嘅相對少數——換下超時空嘅立足點去話,亦甚至係可以話當下嘅多數亦未必係可以簡單化將例如作古咗嘅、唔再為當下多數所認知嘅少數徹底咁抹殺:我哋文明到依家發展嘅其中一個共識,相信應該係珍重同重視每一個可能係少數嘅存在,就似粵文圈同粵文咁都係一個相對少數,對於依家嘅主流文明圈(如所謂漢)更加係有著諸多稱之為少數嘅元素,就似有保留啲古文生僻嘅文法,呢啲都係同唐字有一樣嘅背景脈絡嘅。
        % 希望各位可以由返呢度再思考返,我哋議案達成嘅目標係咪得一個變天?定係要繼續認知返唔同嘅資訊脈絡?多多思量 Longway22 (傾偈) 2021年8月24號 (二) 10:48 (UTC)
        % 「欄路虎」咪就係「欄路虎」,咪就係會欄住人、令人打退堂鼓放棄嘅嘢囉。可以自由參與一樣嘢,唔等於嗰樣嘢冇欄路虎。譬如一撻地方嘅選舉,可以畀選民自由去投票,不過投咗啲議員出嚟之後,有人唔鍾意就會DQ嘅,咁啲選民自然冇乜投票意慾,呢個DQ力量自然就係欄路虎。我明白大家唔係有心做欄路虎,大概係基於乜乜乜信念揀咗咁。不過,呢個基於乜乜乜信念個結果,出到嚟就的的確確係變成欄路虎,大家呢個基於乜乜乜信念嘅選擇,出到嚟就的的確確係變成特登咁揀,噉我唯有照直講。就算你哋唔係立心想噉,但以結果而言,你哋的的確確係偏偏特登搞到粵維充滿欄路虎。如果你覺得唔好聽,我抱歉,不過請睇清楚個事實出到嚟的確係噉。
        % 至於「華語」個詞義,係直到今時今日,都有好多人、有唔少社經地位都重要、顯著、有相當嘅社群或族群,都仲用緊佢嘅本義。佢嘅本義,係現在進行式嘅約定俗成義項之一;佢本義嘅使用人數,就算總數上少過用狹化義嘅人,都唔係真係人少(正如用漢語、日語、韓語嘅人少過用英語,但唔可以話用漢語、日語、韓語嘅人數少)。同「唐字」而家根本幾乎冇人會噉叫,摷番歷史都冇流通痕跡完全唔同,「唐字」唔係「少數」、唔係「唔時髦嘅稱呼」,而係「根本就幾乎完全冇人咁叫」。一事還一事,請分清楚。大家應該要擁有分辨唔同事情異同嘅能力。
        % 仲有,我冇話過要「徹底咁抹殺」。你喺條目入面,如實噉話有「唐字」呢種叫法出現過、有何證據、流通程度或跡象若何,絕對永遠冇話唔得(但要如實噉講,唔好吹大造假)。而家嘅問題,係你哋堅持要違反事實又力排眾議,攞嚟做名,兼且迫其他維基人接受,大家一係就心灰意冷唔寫粵維(起碼唔寫同漢字有關嘅文),一寫就迫住要用呢啲遠遠過唔到底線嘅叫法。呢啲離地離到上太空嘅標奇立異做法,點會唔係欄路虎?
        % 語文嘅保育,係要尊重佢嘅自然發展,而唔係自己造啲假象、假古董出嚟迫其他人認同。你哋攤開你哋嘅「證據」,都搵唔到「唐字」呢種叫法有咩「背景脈絡」喺度,噉唔係咩「保留啲古文生僻嘅文法」,而係保留你哋造出嚟嘅假古董,同嗰啲「偽正字」一樣㗎咋!醒吓啦!停止再製造假古董,唔好再迫人跟住假古董做嘢,好唔好?-Cangjie6 (傾偈) 2021年8月24號 (二) 12:45 (UTC)
        % @Cangjie6、Longway22:閣下話本字係假古董,因為搵個「本字」,「本字作者」首先攞啲死字、辟字、冇人用嘅字,再去樔吓啲戲曲、文學、舊書,再去啲古老嘅韻書廣韻、切韻之類去合理化/證明/支持佢自己嘅講法。所以閣下視此為「偽本字」?事關成個做法冇系統、唔科學、太隨意、推理 牽強,所以就 「偽」嘞? 定係閣下認為廣東話根本就有音冇字,所以「本字」都係假。如果閣下係前者,我哋之間都仲有啲偈傾。可能兩者都唔係,或者你可以講嚟聽吓?
        % 睇返本人搵到嘅參考材料,其中有幾本係用嚟教人(番鬼佬)講廣東話,除非閣下覺得本書作者係老作嘅啫,呢部份資料記低喺廣東人口中,教育未普及之前,「唐字」一辭極之有可能用法。之後點解斷咗攬?暫時唔作出推測,可能同另一個叫法「字」有關。另一本「English Made Easy 《唐字調音英語》 An English Textbook for Cantonese Written by Mo Wenchang 莫文暢 in 1905」,呢本書喺《唐字音英語和二十世紀初香港粤方言的語音》 / 黄耀堃, 丁國偉著,香港 : 香港中文大學中國文化研究所呉多泰中國語文研究中心, 2009.10,ISBN:9789627330202 提過。「唐字」唔應該係你口中嘅「假古董」。下一個目標係歷史檔案館。你講嘅「假古董出嚟迫其他人認同」亦值得討論(遲啲)。
        % 閣下「欄路虎」之說,畀人一個賴地硬嘅感覺,因為某啲人講咗一啲睇法、表達咗啲信念,你啲 friends 就俾佢嚇親,打「退堂鼓放棄」,佢哋眞係唔識得堅持(唔似你噉)。閣下 DQ 嘅例正正確定我嘅講法,DQ 人用程序上嘅權力去阻撓其他人自由參與,如個你啲 friends 從來冇畀人封戶,繼續可以自由參與,只係揀唔玩,何來「欄路虎」之有呢?!似係決定唔玩就求其俾啲理由。P.S. 唔知你啲 friends 係咪一入嚟粵文維基百科就用「的」、「和」、「條目」... 呢啲字呢?如果係嘅話,呢度嘅熟客都唔會好 like 囉。 --WikiCantona (傾偈) 2021年8月25號 (三) 10:56 (UTC)
        % 賴地硬嗰個係你啊!你何來只係「講咗一啲睇法、表達咗啲信念」?你係迫人跟,搞到大家一係唔寫粵維,寫就要用離地離到上太空嘅「唐字」,唔用得正常人話嘅「漢字」啊!DQ他人、DQ正常人話嘅,係你啊!我或者我啲朋友一入嚟就寫「的」、「和」、「條目」有咩問題?粵語冇「目的」、「的士」、「和氣」、「打和」、「條目」呢啲講法咩?而家呢度係要寫眞粵文,定係要寫「粵維某老餅®咗嘅粵文」啊?
        % 你所謂搵嘅資料,正正證明你根本就係搵假古董,因為你先入為主,先有結論,然後就造假砌證據,根本故意忽視同時期嘅大量另一邊事實。我已經最後警告咗你唔好再造假。
        % 我再畫公仔畫埋出腸,你(同Longway22)害死粵維嘅「信念」,就係「盲反中維」——「只要中維係東,粵維就一定要西,無論幾牽強幾離地幾違反事實都要西」。你哋咁玩法,結果咪歸西囉!大佬,粵語堅抽眞係唔同嘅,當然要唔同,唔可以被統一啦。但係粵語堅抽同官話、同通泛書面漢語一樣嘅,你就要尊重語言事實先得𠺝!如果唔係,呢度並唔係寫粵文,而只係寫緊一種參考粵文改造而成嘅人造語文咋!
        % 我冇咁多時間再拗,我已經重複又重複緊,要講嘅嘢講晒。你哋都仲係要沉迷「歸西人造語」嘅,夫復何言!-Cangjie6 (傾偈) 2021年8月28號 (六) 15:42 (UTC)
        % @Sun8908、Matttest、路克天行者:再睇下您地有咩意見,對於用「漢字」定「唐字」。特克斯特 (傾偈) 2021年8月23號 (一) 16:12 (UTC)
        % 冇咩意見,如果要轉成「中文字」就會反對,但係「漢字」同「唐字」之間就冇咩意見,唔覺「唐字」嘅粵語使用差好遠,跟上面嘅討論出嚟應該都算有共識搬去「漢字」。Sun8908(傾偈) 2021年8月23號 (一) 16:35 (UTC)
        % User:@Sun8908:閣下,點解「如果要轉成「中文字」就會反對」?可唔可講多啲呢?--WikiCantona (傾偈) 2021年8月24號 (二) 01:22 (UTC)
        % 「中文字」算係SoP,比較累贅,但粵文要簡潔,而且喺粵維「中文」都跳轉去「唐文」,冇理由寫成「中文」開頭,係都應該叫「唐文字」或者「唐話字」。Sun8908(傾偈) 2021年8月24號 (二) 05:21 (UTC)
        % 論述有合理嘅意見,多謝sun君嘅寶貴意見。不過呢樣命名法,會係同依家可尋獲嘅記錄文獻啲基礎有衝突,亦有(人多勢眾)潮流背景嘅認受度問題,sun君認為呢啲分歧可唔可以緩減返到? Longway22 (傾偈) 2021年8月24號 (二) 05:27 (UTC)
        % 我覺得冇咩所謂?路克天行者 (傾偈) 2021年9月2號 (四) 09:30 (UTC)
        % @Z423x5c6、SC96、Cangjie6、Detective Akai、Pokman817、H78c67c、Deryck Chan:睇返上面嘅討論,再加埋下面2位用戶嘅死撐,正正符合管理員User:SC96喺Talk:多佛講過嘅:「頂多都只可以話係「各自表述」(Special:diff/996178),一直嘅各自表述再加埋User:Z423x5c6同W君對話,都睇到佢哋2個根本反駁唔到,跟User:Sun8908話齋「跟上面嘅討論出嚟應該都算有共識搬去「漢字」,睇落討論共識都係用「漢字」。特克斯特 (傾偈) 2021年8月25號 (三) 22:28 (UTC)
        % User:Z423x5c6提出嘅「反駁」,只要話係啲文獻歷史悠久,同而家嘅用法冇關係。呢點佢只係進一步講咗而家嘅「常用名」係「漢字」(呢樣嘢冇人否認過),亦即係咁多位覺得嘅共識,佢冇解釋到「歷史悠久」嘅文獻,唔可以用嚟支持一個命名,畢竟,一個文章嘅命題,「常用名」唔應該係唯一嘅考慮。「常用名」同「粵語用法」衝撞嘅時候,應該點樣考慮?!呢個問題應該進一步探討。--WikiCantona (傾偈) 2021年8月27號 (五) 10:11 (UTC)
        % 呢個唔係「常用名」同「粵語用法」嘅衝撞,係「現時粵語常用用法」同「過時粵語常用用法」嘅衝撞。Z423x5c6 (傾偈) 2021年8月28號 (六) 11:48 (UTC)
        % Z423x5c6嘅講法比較正確,「漢字」既係常用名,亦都係正常粵語叫法。不過容許我再修正:呢個唔係「現時粵語常用用法」同「過時粵語常用用法」嘅衝撞,而係「現時同以前嘅粵語常用用法」同「夾硬死摷爛摷屈出嚟嘅假古董」嘅衝撞。-Cangjie6 (傾偈) 2021年8月28號 (六) 15:47 (UTC)
        % 插句嘴,我覺得而家叫緊「唐字」,某程度上係同「唐文」篇文嘅命名(中維用名:漢語)有關,寫「唐文」嘅字自然就係「唐字」,正如寫「漢語」嘅字自然就係「漢字」。喺考慮「唐字」係咪改名同點樣改名嘅時候,有需要一併考慮埋「唐文」係咪一樣要改名 —— 如果「唐字」改叫「漢字」,「唐文」可以同步改叫「漢文」(「漢文」呢個稱呼舊時香港都好常用,譬如有唔少「漢文學校」);如果「唐字」改叫「中文字」,「唐文」可以同步改叫「中文」。兩個方案我都覺得合理。相反,如果只係將「唐字」改叫「漢字」,但「唐文」依然叫「唐文」,就會變成『寫「唐文」嘅字係「漢字」』嘅古怪論述。--XRTIER (傾偈) 2021年8月27號 (五) 09:04 (UTC)
        
        % @XRTIER:多謝你嘅睇法,你講出咗「一至性」嘅問題。閣下覺得如果就咁叫「文」同「字」得唔得呢?而呢度眞係未有認眞睇過 ,「中文字」呢個選擇。--WikiCantona (傾偈) 2021年8月27號 (五) 09:59 (UTC)
        % 而家成個討論到呢度,好明顯除咗幾位當年有份違規嘅、有份無共識就夾硬改名,然後又玩程序玩規程賴死唔肯改番嘅人,喺冇合理道理、憑住自己偏見,夾硬砌假證據企圖死鹹魚拗番生噉樣話要用「唐字」外,其他好多好多唔好嘅編輯(呢啲編輯本身立場唔見得一樣)都話要用漢字。而且點解用漢字嘅理據,嗰幾位死鹹魚拗番生嘅少數人士一直都反駁唔到。成個討論共識,係持平嘅人都睇到。而家粵維點解唔使跟共識造嘢?係咪因為嗰少數死鹹魚拗番生人士有權,就可以操控粵維?-Cangjie6 (傾偈) 2021年9月4號 (六) 04:34 (UTC)
        
        % 「漢字」範圍
        % 漢字嘅範圍,日文維基嘅「漢字」噉講:
        
        % 漢字(かんじ)は、中国古代の黄河文明で発祥した表記文字。四大文明で使用された古代文字のうち、現用される唯一の文字体系[1][2]。また最も文字数が多い文字体系であり、その数は約10万字に上る。古代から周辺諸国家や地域に伝わり漢字文化圏を形成し、言語のみならず文化上に大きな影響を与えた。
        % 現代では中国語、日本語、韓国語(朝鮮語)、広西の東興市にいるジン族が使用のベトナム語の記述に使われる。現在、韓国語では殆ど使用されなくなっている。20世紀に入り、漢字文化圏内でも中国語と日本語以外は漢字表記を殆ど廃止したが、なお約15億人が使用し、約50億人が使うラテン文字についで、世界で2番目に使用者数が多い
        % 重點係「漢字」唔係純粹只寫中文嘅字,而係發源喺黃河流域,黃河文明所做出嚟嘅字,即現在中文字/唐字嘅基礎,同時亦包埋日文用嘅 Kanji。日文維基百科嘅定義近似 a subset of en:Han unification。所以,我哋應該搞清楚「漢字」係咪只係中文而家用緊嘅繁體、簡體字(同埋粵語字)呢?--WikiCantona (傾偈) 2021年8月27號 (五) 12:40 (UTC)
        
        % @WikiCantona:粵維都開哂「韓文漢字」(Hanji)同「日文漢字」(Kanji)兩版。所以,我認為只係響內文提及就得。Akai 博士 (傾偈) 2021年8月27號 (五) 12:48 (UTC)
        % 閣下嘅意思係「漢字」就係中文/粵文而家用緊嘅繁體、簡體字,同埋粵語字?即係話「華文字典」有嘅字? --WikiCantona (傾偈) 2021年8月27號 (五) 12:59 (UTC)
        % @WikiCantona:跟返目前漢字響邊啲國家常用;之後再開「歷史」一段提及,就可以。Akai 博士 (傾偈) 2021年8月28號 (六) 01:00 (UTC)
        % 都係睇東亞各地對整體東亞文化嘅(詮釋)立足點問題,注意返日本度係會認返有一個「唐音」指向舊中國大陸嘅發音、呢點係現代日文發音系統嘅組成之一。韓文現代化係徹底拋棄咗傳統中文字,同越南文嘅改寫類似,可能都係徹底斷絕同中國大陸文書嘅聯繫、以求唔受現代大一統因素再影響嘅現代做法,所以話可能都可以再考慮返兩地嘅獨立情況去審視本地係咪都可以再獨立判斷返詮釋嘅進路 Longway22 (傾偈) 2021年8月27號 (五) 13:00 (UTC)
        % 唔係好明,可唔可以再講多些少呢?--WikiCantona (傾偈) 2021年8月27號 (五) 13:03 (UTC)
        % 即係似拉丁文一系咁,到依家雖然常規理解嘅一般西文(歐美語文)都係出自拉丁文,但係咪就話有一統嘅以單一國族所主導著嘅書寫詮釋?部分程度度講漢嘅概念亦唔係固定嘅單一族體,就似係粵文化圈主要嘅成分(基底)並非係單一族群,某部分仲類似返拉丁文背景嘅社群一樣,係有多元化唔同姿態嘅一個群體,「漢」表面依家本地相對大範圍講潛在係會少多元化同包容度嘅進路動力。另外「唐」同「華」某程度就有啲類似,係指代咗中國大陸史上一個包含咗多樣而唔同社群成分嘅時代,好明顯係深刻入咗粵圍人嘅歷史記憶度,係有代表返粵文化圈歷史文明度一個相對多元同包容嘅認知基底,外埠舊粵移民都會係用返「唐山」去到形容自己嘅鄉下,可以話「唐」嘅意義係多過「漢」、同時亦有更多嘅文明基礎獨立支撐,應該盡可能咁等本地同粵圍有多啲空間同機會繼續返呢個獨特嘅脈絡 Longway22 (傾偈) 2021年8月27號 (五) 13:21 (UTC)
        % 噉喃字係唔係「漢字」呢?「女真字」呢?「西夏字」呢?「金文」、「鼎文」、「甲骨文」呢?想講嘅係,如果「漢字」只係形容現今成日用嘅中文字,咁點解唔叫佢做中文字或者唐字呢?--WikiCantona (傾偈) 2021年8月27號 (五) 13:07 (UTC)
        
        % 「漢字」外來字
        % User:HenryLi 閣下咁樣講,「漢字呢個名係日本呢啲地方叫,表示係外來字... 好似係近年先至流行叫漢字」佢係隨口噏當秘笈,定係合理嘅呢? User:Cangjie6閣下講到辭彙有佢發展嘅歷程,「漢字」喺粵語嘅語景裏頭已經變為一個常用嘅字,「唐字」呢個用法,已經斷咗攬,瓜咗,好應該退位讓賢。有樣嘢叫 Etymology,即係,一個字/詞彙幾時出現,佢嘅發展史,「漢字」呢個叫法,幾時出世呢?反之,我嘅資料已經顯示「唐字」喺一定嘅空間同埋一定時間用過/用緊。當然一方可以完全當我對空氣講嘢,對成個討論一啲都冇幫助,支持用「漢字」嘅街坊朋友,都可以幫幫手。--WikiCantona (傾偈) 2021年8月27號 (五) 23:11 (UTC)
        
        % 註:呢道原先係有其他用家留低段嘢,不過因為唔係正宗粵文兼有唔有禮嘅表述,俾人喺2021年9月8號 (三) 10:53 (UTC)移除咗。閣下對呢個操作若係要提出唔同嘅意見,可以去城市論壇度傾下睇睇係咪值得再睇過。
        % 整體異論嘅收窄同編輯進路嘅整理
        % 翹仔下邊係由於參與多數嘅壓力有一個「結案」先,之不過就依家顯現出嘅問題講,幾丁人的卻會係無力扶正著。回應返個人同長遠再審視嘅需要,本編係大概留底返呢個谷票夾圍攻異議所呈現出嘅一啲需要再思考嘅位:
        % 表面係普通大眾嘅言論認識可徹底否定咗一個事實可查驗證明到嘅事物,係咪就可以徹底抹殺?
        % 為咗一時谷大同激化嘅有針對人身同觀點嘅目的(包括上點嘅目的),得到嘅係咪一個可繼續發展嘅「共識」?
        % 一致唔認同以後對於上述類似嘅事物、觀點等等,係唔會有共存或共處嘅理解可能,係會同維基百科嘅構築之間存在幾多嘅衝突?
        % 呢度係先有上述幾點嘅參照思考,供各位自己整理自己。——Longway22 (傾偈) 2021年9月24號 (五) 13:20 (UTC)
        
        % 收尾
        % 翹仔嘅結案辭
        % 首先講結論:搬去《漢字》,然後喺內文寫返「漢字」、「唐字」、「中文字」三個用法嘅沿革。
        
        % 多謝大家熱烈討論咁耐。唔少人叫我結案,大家今輪想講嘅嘢而家亦好似講晒喇,噉就結咗佢啦。
        
        % 歸納返呢個討論,大概有兩個重點:
        
        % 「漢字」、「唐字」、「中文字」三個選項當中,最多編者接受到嘅文題係「漢字」。
        % 「唐字」喺早期粵洋交往嘅時代多啲人用,但係今時今日嘅粵語粵文社群主要用同中文、日文通用嘅「漢字」。
        % 粵語維基百科十幾年前開站嘅時候,粵文學術寫作嘅風氣仲未好盛行。當年嘅粵維亦想喺用詞方面突顯粵文嘅獨特性,所以喺唔少嘅題目道特登揀咗同中文唔同嘅用詞。既然廿一世紀嘅粵文學術書寫未成氣候,噉就睇下十九世紀嘅粵文用詞。但係隨住粵文近年越來越多文字記錄,開始睇到粵文其實通常用通用詞「漢字」,有啲編輯甚至話未聽過「唐字」。
        
        % 粵文維基百科嘅寫作風格一路都喺「編者心中理想嘅粵文規範」(未必反映普及用法)同埋「普羅大眾點寫」(冇規範化,或者根本未有人寫過相關題材)之間拉鋸。呢個係教育潮流尖端一定會遇到嘅現象。今次我哋親身見證粵文用法嘅變化,然後集中討論,相信亦唔會係最後一次。無論你係接受到呢個結果嘅大多數定係唔鍾意呢個結果嘅少數,都多謝你參與,同埋希望大家將呢個討論搵到嘅參考來源寫入文章入面。翹仔 (傾偈) 2021年9月21號 (二) 09:11 (UTC)
        
        % 歧異嘅「共識」
        % 呢度暫時係有上述嘅「結案」,經歷咗多番嘅論述同各種旁證、觀點嘅補正,認為依家事實嘅成果就係一個建基喺類似中維部分社群pattern嘅擦邊手法得到嘅結果、長遠講係一個極為有問題嘅階段「共識」形式——緊要點就似WikiCantona閣下提到嘅、「要求用漢字一方,只係講咗一個證據,『現時嘅用法』」,如果就嚴謹嘅編輯思路度話,呢一個谷大聲出嘅「多數表決」亦根本就唔係正路嘅「共識」,不過依家係由呢度活躍嘅社群參與人所「接受」而可以「通過」。由此本編作為少數堅持咗一啲科本化思路同其他唔受呢度活躍參與人所「接受」嘅傳統尺度嘅編輯討論人,稍微再另外歸結咗呢個議案度出嘅「結果」,暫時可以有以下嘅「醜話」,唔針對任何人、但係務必各位繼續有疑問:
        
        % 只要多數出聲嘅人可以堅持否定根據任何歷史上參考資料同來源為基礎嘅論述,就可以無視維基同本地本身一般嘅粵文圈利益考慮、收錄指引同常規編輯查證鏈,可以直接得出任何唔需要嚴謹學識論證甚至係脫離埋傳統常識嘅結論。
        % 以群體劃一咁針對某啲編輯或某啲編輯方式嘅可違背禮儀同其他規例嘅言論行為,並唔會係因為喺規例技術同實質可造成對編輯同社群環境嘅負面影響而有任何適當反應,只要得到劃一咁有針對群體自身呢度可以得到嘅利益同影響、就唔需要遵守任何需遵循嘅慣例規約等。
        % 本地嘅編輯唔係以建設一個涵蓋唔同知識資訊嘅百科為目標,而係要透過編輯平台達成其他同百科構築目標大唔同嘅目標,例如係做出同普通社交一樣嘅各種結黨行為進而影響百科構築嘅內容方向,或者係利用編輯同活動去干擾或打擊其他特定嘅編輯同內容等等。
        % 謹歸結以上,相信係依家可顯現出嘅幾樣「共識」,作為各少數同唔出聲朋友嘅一啲參照。——Longway22 (傾偈) 2021年9月24號 (五) 13:08 (UTC)
        
        % 我睇個討論然後結案嘅時候,重點的確係而家嘅粵語粵文通常用「漢字」。呢個唔只係多數表決咁簡單;特克斯特等等編者有俾唔少例文添。如果全文唔俾提「唐字」呢個詞就話唔理歷史啫,但係呢個唔係我個歸納,我都話不如喺篇文度解釋下「唐字」、「漢字」兩個詞喺粵文度嘅沿革。你都話編輯規範要可以查證,要有規有矩,維基百科慣例就係文章標題要用坊間最常用嘅講法(如果有),而呢個亦係今次定案,「漢字」方面多啲人支持嘅原因之一。 翹仔 (傾偈) 2021年9月24號 (五) 22:32 (UTC)
        % 「漢字」的確係現代用法。如果用呢個係唯一嘅標準,其實呢度好多都文都搬得。用粵辭做命名,代表對粵語傳統嘅尊重,同保護。而粵語傳統正消失之中,粵語不斷畀北方中文同化。--WikiCantona (傾偈) 2021年9月24號 (五) 23:13 (UTC)
        % 粵維就梗係「用粵辭」。「漢字」都係粵辭。只有喺「就算罕見都好,總之特登揀個同其他漢字文化圈語文唔同嘅詞語」先會俾「唐字」贏。其實廣大粵語語言學界已經笑咗呢種特登為唔同而唔同嘅「粵維式粵文」好耐[1],而「架撐」等等唔係最常用但都普及嘅用詞已經係粵維以外嘅粵文社群接受到嘅極限。驚俾人同化嘅話,唔會在於粵文用多幾個內容語共通詞,反而因為粵維用詞生僻搞到未睇過粵維嘅粵語母語者唔夠膽貢獻粵維,到時就真係拱手相讓喇。 翹仔 (傾偈) 2021年9月29號 (三) 11:47 (UTC)
        % User:HenryLi 之前嘅一段嘢,值得一睇:Wikipedia:城市論壇\#唐人\_vs\_華人。 --WikiCantona (傾偈) 2021年9月25號 (六) 01:26 (UTC)
        % @WikiCantona:你想開多版叫「唐字」[2]?我自己覺得除咗話「唐字」係「漢字」十八世紀至二十世紀初喺粵地興過嘅叫法之外,冇乜嘢好講... 翹仔 (傾偈) 2021年9月30號 (四) 08:48 (UTC)
        
        % 如果今次呢種情況係發生喺「漢人」、「唐人」(兩個名都係現代用法)嘅話,又有冇解決方案?SC96 (傾偈) 2021年10月9號 (六) 12:38 (UTC)
        
        % 我見響中維,「漢人」「唐人」「華人」係唔同一樣野。粵維都可以考慮下咁樣做,即係將三篇文章拆哂佢。但漢文既情況同漢人既情況係絕對唔可以相提並論。 Akai 博士 (傾偈) 2021年10月9號 (六) 12:48 (UTC)
        % 對外連結有變 (2020年3月)
        % 最新留言:5 年前
        % 1則留言
        % 1個人參與討論
        % 各位編輯仝人:
        
        % 我啱啱救返唐字上面嘅 1 個對外連結。麻煩檢查下我改嘅嘢。有咩查詢,或者想隻機械人唔理啲外連,或者想隻機械人成版唔好掂,請睇呢版簡明嘅問答頁。我改咗呢啲外連:
        
        % 加咗存檔 https://web.archive.org/web/20150625054518/http://unicode.org/charts/PDF/U20000.pdf 落 http://www.unicode.org/charts/PDF/U20000.pdf
        % 如果隻機械人有錯,請睇問答頁嘅指示。
        
        % 唔該晒!—InternetArchiveBot (報告軟件缺陷) 2020年3月10號 (二) 07:45 (UTC)
        
        % 對外連結有變 (2020年3月)
        % 最新留言:5 年前
        % 1則留言
        % 1個人參與討論
        % 各位編輯仝人:
        
        % 我啱啱救返唐字上面嘅 1 個對外連結。麻煩檢查下我改嘅嘢。有咩查詢,或者想隻機械人唔理啲外連,或者想隻機械人成版唔好掂,請睇呢版簡明嘅問答頁。我改咗呢啲外連:
        
        % 加咗存檔 https://web.archive.org/web/20150625054518/http://unicode.org/charts/PDF/U20000.pdf 落 http://www.unicode.org/charts/PDF/U20000.pdf
        % 如果隻機械人有錯,請睇問答頁嘅指示。
        
        % 唔該晒!—InternetArchiveBot (報告軟件缺陷) 2020年3月28號 (六) 16:31 (UTC)