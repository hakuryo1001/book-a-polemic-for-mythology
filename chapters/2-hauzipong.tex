\chapter{口字旁}

\section{粵文一日用口字旁,粵語就一日係方言}

「一啲」、「呢啲」、「嗰啲」、「邊啲」、「啲野」、「快啲」、「少啲」、「呢啲」、「遲啲」、「 鬆啲」、「郁啲」、「一啲啲」、「住好啲」、「食好啲」,同「講呢啲」。

「啲」呢個詞,根據《粵典》,有兩個用途。第一個係解「少少」,通常放喺形容詞後面用嚟做比較。而第二個,就係表達複數,用作為一個不可數嘅量詞。唔難睇得出,呢個詞係一個具有重要語法地位嘅詞。
「啲」呢個字嘅結構,由「的」同「口」合併組成。「的」係聲旁,負責表聲,而個口字旁,就係意符,負責暗示或者指導個意思出黎。所以,「啲」係個形聲字。再嚴謹少少黎講,個口字旁所負嘅責,就係\lr{扌}{󰖖}低話呢個詞係口語用詞,係廣東話,係口語。

呢種咁樣嘅「口語詞」喺粵文好多,但唔係粵文獨有,東北話書面語、大眾普通話、吳文都有唔少。但係天下間五花八門嘅「口語字」,喺佢哋所出現嘅文章,都有同一個隱性效應—— 就係佢無時無刻都竊竊私語緊喺度同讀者講:「呢篇文章所寫嘅語言,根本就係方言」。
「口語字」,佢自己嘅本質,就係要表達口語。佢嘅運作,就係攞唔係口語字嘅字,借黎標音。而為咗我哋可以輕易地睇得出佢淨係攞黎表音而唔要個字本身嘅含意,我哋就加個口字旁,加以強調,以資閱讀。姐係話,你咁樣做唔係用漢字嘅表意功能啦,你本身寫嘅嘢唔係正統雅言啦—— 雅言要寫出來,點可能要用埋呢啲咁嘅同低莊手段嘅啫?齋睇一篇成篇都係口字旁嘅廣東話文章,同一篇全部都係正宗有晒甲骨文祖先嘅白話文,就知道邊一個份量深厚嘅古老雅言,邊一個係要借錢渡日嘅小方言啦。一篇文章,睇佢用嘅字,就可以知道邊一個係可以擔起文化大旗嘅嚴肅文學,邊一個係冇料扮四條兼玩玩下嘅方言文章。你嘅嗰啲「口字旁字」,個個都係用傳統漢字加個口字旁上去,仲唔係方言?乜你哋班野唔係心知肚明呢個原因,先至呃鬼去話「尐」先至係「啲」嘅正字咩?
聽到呢啲說話,我哋當然會好嬲。但係往往啲回應都係啲呃細路、經唔起深究、而且策略上其實懵盛盛煮重自己米嘅回應:比如啲咩「廣東話係唐宋雅言論」同咩「『攰』喺《說文》度有」嘅論調,同埋嗰啲本字考。講到尾,你咪仲係喺度「以華度己」?你哋想扮係中華正統,「咁噉呢嚟嘢咩哋嘅啦喇囉喎呀嘩啱㗎」一大咋亂七八糟嘅口字旁,呃得到邊個?根本就係太監扮皇帝。你就連個「啲」字呀,都係用官話嘅「的」作本位加個口字旁㗎咋。「的」字喺廣東話根本就唔係讀「󰦦」,係你哋班嘢唔知頭唔知路,將中文裏面最為重要嘅介詞字不問自取,然之後加個口字旁就當自己乜乜七七,笑死人咩。「有音無字」,係方言嘅特徵。

思哲誠實嘅人唔會對呢番說話駁嘴,因為思哲誠實嘅人係會自己同自己講呢番說話—— 按照佢個範式黎思考,咁樣嘅結論就係必然。

而我哋要嘅,就係跳出呢個範式,要挑戰呢個範式,要拒絕呢個範式。

乜你忍受到「啲」呢個喺廣東話裏面有舉足輕重語法地位嘅詞以普通話嘅「的」加個口字旁嘅方式存在落去咩?乜你忍受到廣東話嘅書面語視覺上呈現住「我哋係普通話嘅變體」係信息咩?乜你忍受到漢字對廣東話喺書面上宣判為方言嘅呢種對待咩?乜你忍受到廣東話嘅「啲」以普通話嘅「的」加個口字旁就算?

如果我哋繼續口字旁落去,廣東話就永遠都只會係一隻文字上用漢字 B 隊嘅一種 B 系語言。佢永遠都會係排第二冇得當家作主,仲要永永遠遠以其他人去量度同定義自己,而唔可以自己定義自己。粵語喺呢一個咁樣嘅遊戲度,可能可以做到二房,至多可以做到正室。睇埋家下時局個樣,你做到二奶小三就應該・劏豬還神喇,再唔係連妹仔都冇得你做,等死啦。

但係,我哋要嘅,唔係廣東話做正室。我哋要嘅,係廣東話自己當家作主,唔駛寄人籬下,唔駛睇人面色聽人說話。我哋要嘅,係廣東話好似日文韓文咁自己可以堂堂正正話自己就係語言。我哋要嘅,唔係要說服人嘅依據或者論調。我哋要嘅,係擺晒喺你面前,你冇得否認嘅事實。我哋要嘅,係一隻可以畀到我哋可以好似日語韓語咁唔會有人覺得呢隻語言只不過係中文方言嘅文字。我哋要嘅,係一隻會令到所有嘗試咁樣諗嘅人腦短路嘅文字。我哋要嘅,係一隻可以畀到廣東話尊嚴嘅文字。

粵語配有自己嘅文字,因為粵語應予有尊嚴。


\section{口字旁之弊}
口字旁之弊,一目瞭然。

你可能而家仲未熟粵切字。

但係,你嘅子女仔乸會。

你嘅子女會覺得,點解你地可以頂得順呢個咁恐怖咁不成系統嘅系統—真係神奇。你地嘅腦袋係用咩做嘅呢?

你嘅子女,會覺得你可以用到咁樣嘅文字系統,係咪個個有自虐癖好。仲要學校冇教,個個無師自通,好犀利,但係佢地一啲都唔羨慕。佢哋用廣東話黎研究天體物理學之外就冇晒時間去度邊個詞用邊隻字,做埋晒呢啲咁米缸數米嘅野。
最重要嘅係,你嘅子女,會為你選擇咗畀佢哋一個更加有理則、可預測、民主、可延伸、有尊嚴嘅文字,而對你感恩。

世世代代講廣東話嘅人,都會歌頌你嘅選擇。
