個世界變晒樣。
神——死晒。
法術——冇咗。
魔力——收埋咗。

香港人,講廣東話嘅香港人,冇得再靠神、靠咒、靠幻想去搓自己嘅神話。
我哋困喺呢個框框入面,
仲有條氣未斷——
仲渴緊屬於自己嘅神話。
咁我哋仲可以點?

就喺人類大航海時代之前,但唔係航海,係航星、航行行去星球、行入宇宙嗰啲空無一物嘅地方——
我哋香港人,喺呢個大國林立嘅世界入面算唔起眼,
就更加要自己整返個屬於我哋嘅神話,包晒所有陸地同海洋,因為冇得揀。
任何故事、任何神話、任何傳說,跨越時間、跨越空間——我哋都要攞嚟自己用。

但點解要止步喺度?
點解唔攞埋所有諗得到、想像到嘅世界?
好似我哋攞現實世界咁攞埋嗰啲世界返嚟?
唔好淨係攞呢個世界,唔好淨係話屬於自己,
要攞晒啲世界,無論係咪真係存在。

我哋要󰩡草菇,󰖒大󰩡眼,
望穿世界嘅簾幕同假皮,
穿過膜、穿過層層菊花瓣,
將我哋見到嘅嘢,實實在在記低。



個世道,變晒樣。
神明——斷咗氣。
妖法——收埋咗。
仙氣——散走咗。

香港仔,靠住廣東話呢條脷根,唔使諗住再搓到啲神神鬼鬼嘅傳奇。
我哋困咗喺咁嘅窄巷入面,
仲仲有口氣,仲仲有條喉——
渴緊自己一套神話。
問下天,問下地:咁我哋仲有乜路行?

就好似當年大航海咁,但唔係出海,係出星,出行行去星辰萬里,行過茫茫虛空——
我哋香港仔,喺大國圍爐個世界,渣都唔算,
就更加要自己整番套屬於我哋嘅神話,
管你係陸定係海,都要我哋收編。
故事也好,傳說也好,神話也好——跨越古今,跨越天地——
我哋都要攞嚟變自己袋嘅嘢。

但點解要止步喺度?
點解唔乾脆攞埋所有諗得到、幻想到嘅世界?
好似我哋攞實世界咁攞埋嗰啲虛嘅世界返嚟?
唔好淨係攞依家呢個世界嚟話:屬於我哋,
要攞盡天上天下,無論真定假,都係我哋嘅。

我哋就要吞晒啲奇花異草,食晒啲神菇妖菌,
開天眼,睇穿世間嘅布幕同假面,
穿過層層薄膜同菊花瓣,
將見到嘅嘢,逐粒字、逐個音,紮實實咁寫低。