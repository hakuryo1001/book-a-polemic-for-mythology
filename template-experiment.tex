\documentclass[a5paper, 10pt, openany]{book} % A5 paper size

% \usepackage[paperwidth=148mm, paperheight=210mm, top=1.27cm, bottom=1.27cm, inner=1.9cm, outer=1.27cm, headsep=1.5cm, footskip=1.75cm]{geometry} % Custom dimensions and margins
% Adjusted margins to ensure space for page numbers
\usepackage[paperwidth=148mm, paperheight=210mm, 
         top=1.27cm, bottom=2.5cm, inner=1.9cm, outer=1.27cm, 
         headsep=1.5cm, footskip=1.5cm]{geometry}


\usepackage[utf8]{inputenc}
\usepackage{ctex}

\usepackage{fancyhdr}
\pagestyle{fancy}
\fancyhf{} % Clear all header and footer fields
% \fancyfoot[C]{\thepage} % Center the page number at the bottom
% Define left and right page numbering
\fancyfoot[LE]{\thepage} % Left side for even pages
\fancyfoot[RO]{\thepage} % Right side for odd pages
% Ensure the chapter pages (plain style) also have this layout
\makeatletter
\let\ps@plain\ps@fancy
\makeatother

% *-----------------------------------------------------------------------*
% | Fonts and typography                                                  |
% *-----------------------------------------------------------------------*

% Set CJK main font (for Chinese/Japanese/Korean characters)

\setmainfont{Times New Roman}
\setCJKmainfont{BabelStone Han}
% \setCJKmainfont{JyutcitziWithSourceHanSerifTCLight}



% You can also use \newfontfamily for custom non-CJK fonts if needed
% \setCJKmainfont{JyutcitziWithPMingLiURegular}[Path = ./, Extension = .ttf]
% \setCJKmainfont{JyutcitziWithSourceHanSerifTCRegular}[Path = ./, Extension = .ttf]

% \newfontfamily{\jcz}{JyutcitziWithPMingLiURegular}[Path = /Users/hongjan/Library/Fonts/, Extension = .ttf]
% \newfontfamily{\jcz}{JyutcitziWithSourceHanSansHCRegular}[Path = /Users/hongjan/Library/Fonts/, Extension = .ttf]
% This has the best rendition for latin characters 
\newfontfamily{\jcz}{JyutcitziWithSourceHanSerifTCLight}[Path = /Users/hongjan/Library/Fonts/, Extension = .ttf]



% *-----------------------------------------------------------------------*
% | Math & Equations     |
% *-----------------------------------------------------------------------*
\usepackage{amsmath} % For advanced math formatting
\usepackage{amssymb} % For mathematical symbols
% \usepackage{tikz} % For drawing logic decision trees



% *-----------------------------------------------------------------------*
% | Table Management                                                      |
% *-----------------------------------------------------------------------*


\usepackage{graphicx}
\usepackage{array}
\usepackage{tabularx}
\usepackage[table,xcdraw]{xcolor}
% 

% Load ruby package for furigana (Ruby text)
\usepackage{ruby}


% *-----------------------------------------------------------------------*
% | Chinese and Soochow Numerals for chapter management                   |
% *-----------------------------------------------------------------------*

% Define Chinese numerals for numbers 1-99
% 〇〡〢 〣 〤 〥 〦 〧 〨 〩 十 〹 〺 卅

\newcommand{\soochowNumeral}[1]{%
  \ifnum#1<10
    \ifcase#1 〇\or 〡\or 〢\or 〣\or 〤\or 〥\or 〦\or 〧\or 〨\or 〩\fi%
  \else
    \ifnum#1<20
      〸\soochowUnits{\numexpr#1-10\relax}%
    \else
      \ifnum#1<30
        〹\soochowUnits{\numexpr#1-20\relax}%
      \else
        \ifnum#1<40
          〺\soochowUnits{\numexpr#1-30\relax}%
        \else
          \ifnum#1<50
            卅\soochowUnits{\numexpr#1-40\relax}%
          \else
            \ifnum#1<60
              〥十\soochowUnits{\numexpr#1-50\relax}%
            \else
              \ifnum#1<70
                〦十\soochowUnits{\numexpr#1-60\relax}%
              \else
                \ifnum#1<80
                  〧十\soochowUnits{\numexpr#1-70\relax}%
                \else
                  \ifnum#1<90
                    〨十\soochowUnits{\numexpr#1-80\relax}%
                  \else
                    \ifnum#1<100
                      〩十\soochowUnits{\numexpr#1-90\relax}%
                    \fi
                  \fi
                \fi
              \fi
            \fi
          \fi
        \fi
      \fi
    \fi
  \fi
}

% Helper command for units (ones digit)
\newcommand{\soochowUnits}[1]{%
  \ifnum#1=0
  \else
    \ifnum#1<4
      \ifcase#1 \or 一\or 二\or 三\fi%
    \else
      \soochowNumeral{#1} % Use Suzhou numeral for numbers greater than 3
    \fi
  \fi
}

\newcommand{\chinesenumeral}[1]{%
  \ifnum#1<10
    \ifcase#1 〇\or 一\or 二\or 三\or 四\or 五\or 六\or 七\or 八\or 九\fi%
  \else
    \ifnum#1<20
      十\chinesenumeral{\numexpr#1-10\relax}%
    \else
      \ifnum#1<30
        二十\chinesenumeral{\numexpr#1-20\relax}%
      \else
        \ifnum#1<40
          三十\chinesenumeral{\numexpr#1-30\relax}%
        \else
          \ifnum#1<50
            四十\chinesenumeral{\numexpr#1-40\relax}%
          \else
            \ifnum#1<60
              五十\chinesenumeral{\numexpr#1-50\relax}%
            \else
              \ifnum#1<70
                六十\chinesenumeral{\numexpr#1-60\relax}%
              \else
                \ifnum#1<80
                  七十\chinesenumeral{\numexpr#1-70\relax}%
                \else
                  \ifnum#1<90
                    八十\chinesenumeral{\numexpr#1-80\relax}%
                  \else
                    \ifnum#1<100
                      九十\chinesenumeral{\numexpr#1-90\relax}%
                    \fi
                  \fi
                \fi
              \fi
            \fi
          \fi
        \fi
      \fi
    \fi
  \fi
}

% Custom chapter title formatting with Chinese numeral chapter numbers
\usepackage{titlesec}


% Custom chapter title formatting with Chinese numeral chapter numbers
\titleformat{\chapter}[block] % 'block' means the title appears on a new line
  {\Huge\bfseries} % Font size and bold formatting for the title
  {\soochowNumeral{\thechapter}} % Chinese character for chapter number
  {1em} % Space between the number and the title
  {\Huge} % Custom style for the chapter title itself (can modify)

% Remove the default LaTeX behavior of forcing new chapters to start on a new page
\makeatletter
\renewcommand\chapter{\if@openright\cleardoublepage\else\clearpage\fi
  \thispagestyle{plain}%
  \global\@topnum\z@
  \@afterindentfalse
  \secdef\@chapter\@schapter}
\makeatother



% % *-----------------------------------------------------------------------*
% % | Proof trees                                                              |
% % *-----------------------------------------------------------------------*
\usepackage[tableaux]{prooftrees}
\renewcommand*\linenumberstyle[1]{(#1)}
\RequirePackage{mdwtab,latexsym,amsmath,amsfonts,ifthen}

%Line height in proofs
\newlength{\fitchlineht}
\setlength{\fitchlineht}{1.5\baselineskip}
% Horizontal indent between proof levels
\newlength{\fitchindent}
\setlength{\fitchindent}{0.7em}
% Indent to comment
\newlength{\fitchcomind}
\setlength{\fitchcomind}{2em}
% Line number width
\newlength{\fitchnumwd}
\setlength{\fitchnumwd}{1em}

% Altered from mdwtab.sty: shorter vline, for start of subproof
\makeatletter
\newcommand\fvline[1][\arrayrulewidth]{\vrule\@height.5\fitchlineht\@width#1\relax}
\makeatother
% Ordinary vertical line
\newcommand{\fa}{\vline\hspace*{\fitchindent}}
% Vertical line, shorter: Use at start of (sub)proof
\newcommand{\fb}{\fvline\hspace*{\fitchindent}}
% Hypothesis
\newcommand{\fh}{\fvline%
  \makebox[0pt][l]{{%
      \raisebox{-1.4ex}[0pt][0pt]{\rule{1.5em}{\arrayrulewidth}}}}%
  \hspace*{\fitchindent}}
% Hypothesis, with longer vert line: for >1 hypothesis
\newcommand{\fj}{\vline%
  \makebox[0pt][l]{{%
      \raisebox{-1.4ex}[0pt][0pt]{\rule{1.5em}{\arrayrulewidth}}}}%
  \hspace*{\fitchindent}}
% Modal subproof: takes argument = operator
\newcommand{\fitchmodal}[1]{% 
  \makebox[0pt][r]{${}^{#1}$\,}\fvline\hspace*{\fitchindent}}
\newcommand{\fn}{\fitchmodal{\Box}}% Box subproof 
\newcommand{\fp}{\fitchmodal{\Diamond}}% Diamond subproof
% Modal subproof with hypothesis in first line (as in Fitch)
\newcommand{\fitchmodalh}[1]{% 
  \makebox[0pt][r]{${}^{#1}$\,}%
  \fvline%
  \makebox[0pt][l]{{%
      \raisebox{-1.4ex}[0pt][0pt]{\rule{1.5em}{\arrayrulewidth}}}}%
  \hspace*{\fitchindent}}
% Rule: formula introduction marker. \fr with line, \fs without line
\newcommand{\fr}{%
  \makebox[0pt][r]{${\rhd}$\,\,}\vline\hspace*{\fitchindent}}
\newcommand{\fs}{%
  \makebox[0pt][r]{${\rhd}$\,\,}}
% Box around argument, like new variable in ql
\newcommand{\fw}[1]{\fbox{\footnotesize $#1$}}

% 
\newcounter{fitchcounter}
\setcounter{fitchcounter}{0}
%To avoid starting from 1, \setboolean{resetfitchcounter}{false}
\newboolean{resetfitchcounter}
\setboolean{resetfitchcounter}{true}
%To avoid increasing numbers, \setboolean{increasefitchcounter}{false}
\newboolean{increasefitchcounter}
\setboolean{increasefitchcounter}{true}
%\formatfitchcounter can be altered if need be, though only once per proof
\newcommand{\formatfitchcounter}[1]{\scriptsize \arabic{#1}}
%Typeset the counter
\newcommand{\fitchcounter}{%
  \ifthenelse{\boolean{increasefitchcounter}}{\addtocounter{fitchcounter}{1}}{}
  \formatfitchcounter{fitchcounter}}

%A line with a special number -- a tag, e.g. \ftag{\vdots}{}
\newcommand{\ftag}[2]{\multicolumn{1}%
  {!{\makebox[\fitchnumwd][r]{#1}\hspace{\fitchindent}}Ml@{\hspace{\fitchcomind}}}%
  {#2}}

\newenvironment{fitchnum}%
{\ifthenelse{\boolean{resetfitchcounter}}{\setcounter{fitchcounter}{0}}{}
  \begin{tabular}{!{\makebox[\fitchnumwd][r]{\fitchcounter }\hspace{\fitchindent}}Ml@{\hspace{\fitchcomind}}l}}%
{\end{tabular}}

\newenvironment{fitchunum}%
{\begin{tabular}{!{\makebox[\fitchnumwd][r]{}\hspace{\fitchindent}}Ml@{\hspace{\fitchcomind}}l}}%
{\end{tabular}}

\newenvironment{fitch}{\renewcommand{\arraystretch}{1.5}
  \begin{fitchnum}}{\end{fitchnum}}
\newenvironment{fitch*}{\renewcommand{\arraystretch}{1.5}
  \begin{fitchunum}}{\end{fitchunum}}

% The following is useful for giving a numbered formula, then the proof.
\newenvironment{flem}[2]%
{\begin{eqnarray}
    &#1\label{#2}\\
    &\begin{fitch}}%
    {\end{fitch}\notag\end{eqnarray}}

%To write comment field for two consecutive lines, with brace
\newcommand{\ftwocom}[1]{%
  \parbox[t]{3cm}{
    \raisebox{-.6\baselineskip}[\baselineskip][0pt]{%
      $\left.
        \begin{aligned}
          \,\\ \,
        \end{aligned}
      \right\}$\quad #1}
  }}

\usepackage{amssymb,amsmath}
\usepackage{amsthm}
\setlength{\parindent}{0ex}
\newtheorem{theorem}{Theorem}[section]
\newtheorem{corollary}{Corollary}[theorem]
\newtheorem{lemma}{Lemma}
\newtheorem{definition}{Definition}
\newtheorem{example}{Example}
\usepackage{adjustbox}
% \setlength{\parskip}{0.5em}
\usepackage{multirow}
\usepackage{booktabs}

% *-----------------------------------------------------------------------*
\begin{document}
% to avoid overfull hbox
\sloppy
% % \jcz{} must be run so the document can process jyutcitzi 
\jcz{} 
% % \jczSourceHan{}

\tableofcontents



\chapter{粵切字 Sample |  }




󱑘󱑙󱑚󱑛󱑜󱑝


々々々々々々々々々々々々々々々
  󱍑  󰼐
  󱍑  󰼐
  󱍑  󰼐
  󱍑  󰼐
々々々
々々々々々々
\chapter{《自序》}

% Example of furigana over kanji
これは\ruby{漢字}{かんじ}の例です。 %漢字 will display "かんじ" as furigana above it

% You can also use furigana for names or specific terms
\section{漢字と\ruby{平仮名}{ひらがな}}

この文章は、\ruby{日本語}{にほんご}を練習するためのサンプルです。

The quick brown fox jumps over the lazy dog.The quick brown fox jumps over the lazy dog.

󱜩
The quick brown fox jumps over the lazy dog.The quick brown fox jumps over the lazy dog.

\ruby{}{而家}搞$^{'}$
  󱍑  󰼐
󱑡


\ 
朕惟フニ我カ皇祖皇宗國ヲ肇ムルコト宏遠ニ德ヲ樹ツルコト深厚ナリ我カ臣民克ク忠ニ克ク孝ニ億兆心ヲ一ニシテ世世厥ノ美ヲ濟セルハ此レ我カ國體ノ精華ニシテ敎育ノ淵源亦實ニ此ニ存ス爾臣民父母ニ孝ニ兄弟ニ友ニ夫婦相和シ朋友相信シ恭儉己レヲ持シ博愛衆ニ及ホシ學ヲ修メ業ヲ習ヒ以テ智能ヲ啓發シ德器ヲ成就シ進テ公益ヲ廣メ世務ヲ開キ常ニ國憲ヲ重シ國法ニ遵ヒ一旦緩急アレハ義勇公ニ奉シ以テ天壤無窮ノ皇運ヲ扶翼スヘシ是ノ如キハ獨リ朕カ忠良ノ臣民タルノミナラス又以テ爾祖先ノ遺風ヲ顯彰スルニ足ラン斯ノ道ハ實ニ我カ皇祖皇宗ノ遺訓ニシテ子孫臣民ノ俱ニ遵守スヘキ所之ヲ古今ニ通シテ謬ラス之ヲ中外ニ施シテ悖ラス朕爾臣民ト俱ニ拳々服膺シテ咸其德ヲ一ニセンコトヲ庶幾フ
以呂波耳本部止
千利奴流乎和加
餘多連曽津祢那
良牟有為能於久
耶万計不己衣天
阿佐伎喩女美之
恵比毛勢須


諸行無常\\
是生滅法\\
生滅滅已\\
寂滅為楽\\

々々


Shogyō mujō
Zeshō meppō
Shōmetsu metsui
Jakumetsu iraku

いろはにほへと	Iro fa nifofeto	色は匂えど	Iro wa nioedo	1–7	Even the blossoming flowers [Colors are fragrant, but they]
ちりぬるを	Tirinuru wo	散りぬるを	Chirinuru o	8–12	Will eventually scatter
わかよたれそ	Wa ka yo tare so	我が世誰ぞ	Wa ga yo tare zo	13–18	Who in our world
つねならむ	Tune naramu	常ならん	Tsune naran	19–23	Shall always be? (= つねなろう)
うゐのおくやま	Uwi no okuyama	有為の奥山	Ui no okuyama	24–30	The deep mountains of karma—
けふこえて	Kefu koyete	今日越えて	Kyō koete	31–35	We cross them today
あさきゆめみし	Asaki yume misi	浅き夢見じ	Asaki yume miji	36–42	And we shall not have superficial dreams
ゑひもせす	Wefi mo sesu	酔いもせず	Ei mo sezu¹
Yoi mo sezu	43–47	Nor be deluded.


\section{Background}
This is the first section in the chapter.

\subsection{History}
This is the subsection under "Background."

\subsubsection{Ancient History}
This is a subsubsection under "History."

\paragraph{Key Events}
This is a paragraph under "Ancient History."

\subparagraph{Event Details}
This is a subparagraph under "Key Events."


語云:知多世事胸襟濶,識透人情眼界寬。知識兩字,由於自己之想象而明,亦由聞人之談論而得也。嘗見街頭巷尾月下燈前,閒坐成群,未嘗無語,但所論多無緊要之事,未足以有補身心。或有談及因果報應,則有聽有不聽焉,且有抽身而去者矣。非言語不通,實事情未得趣也。惟講得有趣,方能入人耳、動人心,而留人餘步矣。善打鼓者,多打鼓邊;善講古者,須談別致。講得深奧,婦孺難知,惟以俗情俗語之說通之,而人皆易曉矣,且津津有味矣。誦讀之暇,採古事數則,有時說起,聽者忘疲。因付之梓人,以備世之好言趣致者。\\ 
語云:知多世事胸襟濶,識透人情眼界寬。知識兩字,由於自己之想象而明,亦由聞人之談論而得也。嘗見街頭巷尾月下燈前,閒坐成群,未嘗無語,但所論多無緊要之事,未足以有補身心。或有談及因果報應,則有聽有不聽焉,且有抽身而去者矣。非言語不通,實事情未得趣也。惟講得有趣,方能入人耳、動人心,而留人餘步矣。善打鼓者,多打鼓邊;善講古者,須談別致。講得深奧,婦孺難知,惟以俗情俗語之說通之,而人皆易曉矣,且津津有味矣。誦讀之暇,採古事數則,有時說起,聽者忘疲。因付之梓人,以備世之好言趣致者。\\ 








\begin{table}[htbp]
  \jcz{}
  \centering
  \renewcommand{\arraystretch}{1.5} % Adjust row height
  \setlength{\tabcolsep}{4pt} % Adjust column padding
  \resizebox{\textwidth}{!}{
  \begin{tabularx}{\textwidth}{|X|X|X|X|}
  \hline
  % \rowcolor[HTML]{D0D0D0} 
  \textbf{坊間漢羅混用} & \textbf{漢字已整理版本} & \textbf{漢字粵切字混用(未組裝)} & \textbf{漢字粵切字混用(已組裝)} \\
  \hline
  咁都係果D嘢嘎啦,廿鯪蚊個餐又湯又剩唔通有得你食天九翅咩?求求其其有D肉有D菜蛋白質澱粉質撈撈埋埋打個白汁茄汁黑椒汁咁撐得你懵口懵面咪Lui返去返工返學返廠返寫字樓囉。唔係你估真係搵餐晏仔咁簡單啊。咁跟飯定跟意粉啊? 
  & 咁都係果啲嘢㗎啦,廿鯪蚊個餐又湯又剩唔通有得你食天九翅咩?求求其其有啲肉有啲菜蛋白質澱粉質撈撈埋埋打個白汁茄汁黑椒汁咁撐得你懵口懵面咪纍返去返工返學返廠返寫字樓囉。唔係你估真係搵餐晏仔咁簡單啊。咁跟飯定跟意粉啊? 
  & 丩今´都係丩个´大子¯野丩乍`力乍`,廿力正⁼蚊個餐又湯又剩𠄡通有得你食天九翅文旡¯?求々其々有大子¯肉有大子¯菜蛋白質澱粉質撈々埋々打個白汁茄汁黑椒汁丩今´止生゙得你懵口懵面文兮`力句¯返去返工返學返廠返寫字樓力个¯。𠄡係你估真係搵餐晏仔丩今`簡單⺍乍⁼。丩今´跟飯定跟意粉⺍乍`?
  & 󱜩都係󱟡󰦠野󱛒󰿒,廿󰻃蚊個餐又湯又剩𠄡通有得你食天九翅󰗘?求々其々有󰦠肉有󰦠菜蛋白質澱粉質撈々埋々打個白汁茄汁黑椒汁󱜩󰿽得你懵口懵面󰖚󰾠返去返工返學返廠返寫字樓󰼠。𠄡係你估真係搵餐晏仔󱜪簡單󰀓。󱜩跟飯定跟意粉󰀒? \\
  \hline
  \end{tabularx}
  }
\end{table}
  


\chapter{Mathematical Formulae}
% Quadratic Formula
\section*{Quadratic Formula}
\[
x = \frac{-b \pm \sqrt{b^2 - 4ac}}{2a}
\]

% Geometric Summation
\section*{Geometric Summation}
\[
S_n = a \frac{1 - r^n}{1 - r} \quad \text{for } r \neq 1
\]

% Definition of e
\section*{Definition of e}
\[
e = \lim_{n \to \infty} \left(1 + \frac{1}{n}\right)^n
\]

% Taylor Series for sin(x) and cos(x)
\section*{Taylor Series for sin(x) and cos(x)}
\[
\sin(x) = x - \frac{x^3}{3!} + \frac{x^5}{5!} - \frac{x^7}{7!} + \cdots
\]
\[
\cos(x) = 1 - \frac{x^2}{2!} + \frac{x^4}{4!} - \frac{x^6}{6!} + \cdots
\]

% Green's Theorem
\section*{Green's Theorem}
\[
\oint_C \left( P \, dx + Q \, dy \right) = \iint_D \left( \frac{\partial Q}{\partial x} - \frac{\partial P}{\partial y} \right) \, dA
\]

% Maxwell's Equations
\section*{Maxwell's Equations}
\[
\nabla \cdot \mathbf{E} = \frac{\rho}{\epsilon_0} \quad \text{(Gauss's law for electricity)}
\]
\[
\nabla \cdot \mathbf{B} = 0 \quad \text{(Gauss's law for magnetism)}
\]
\[
\nabla \times \mathbf{E} = -\frac{\partial \mathbf{B}}{\partial t} \quad \text{(Faraday's law of induction)}
\]
\[
\nabla \times \mathbf{B} = \mu_0 \mathbf{J} + \mu_0 \epsilon_0 \frac{\partial \mathbf{E}}{\partial t} \quad \text{(Ampère's law with Maxwell's correction)}
\]

% General Theory of Relativity
\section*{General Theory of Relativity}
\[
R_{\mu\nu} - \frac{1}{2} g_{\mu\nu} R + g_{\mu\nu} \Lambda = \frac{8 \pi G}{c^4} T_{\mu\nu}
\]

% Gödel's Incompleteness Theorem
\section*{Gödel's Incompleteness Theorem}

Any consistent formal system that is expressive enough to encode arithmetic contains true but unprovable statements.


Sed ut perspiciatis, unde omnis iste natus error sit voluptatem accusantium doloremque laudantium, totam rem aperiam eaque ipsa, quae ab illo inventore veritatis et quasi architecto beatae vitae dicta sunt, explicabo. Nemo enim ipsam voluptatem, quia voluptas sit, aspernatur aut odit aut fugit, sed quia consequuntur magni dolores eos, qui ratione voluptatem sequi nesciunt, neque porro quisquam est, qui dolorem ipsum, quia dolor sit amet consectetur adipisci[ng] velit, sed quia non numquam [do] eius modi tempora inci[di]dunt, ut labore et dolore magnam aliquam quaerat voluptatem. Ut enim ad minima veniam, quis nostrum[d] exercitationem ullam corporis suscipit laboriosam, nisi ut aliquid ex ea commodi consequatur? [D]Quis autem vel eum i[r]ure reprehenderit, qui in ea voluptate velit esse, quam nihil molestiae consequatur, vel illum, qui dolorem eum fugiat, quo voluptas nulla pariatur?


\begin{tableau}
  {                       % begin tree preamble
      line no sep= 2cm,   % distance of tree from line numbers
      for tree={s sep=10mm}, %control horizontal spread of branches
  }
  [P  
      [P\rightarrow Q
          [ \neg Q
              [\neg P, close]
              [Q, close]
          ]
      ]
  ]
  \end{tableau}
  
  \begin{tableau}
  {
      line no sep= 1.5cm,
      just sep= 1.5cm,  % Set separation of justification
  }
  [(P\wedge Q)\rightarrow R), just={Premise}
      [\neg(P\rightarrow (Q\rightarrow R)), just={Negated conclusion}
          [P, just={from (2)}
              [Q, just={from (2)}
                  [\neg R, s sep=30mm, just={From (4)} %Note "s sep" to spread fork below
                      [\neg(P\wedge Q),  just={Alternatives from (1)}
                          [\neg P, close, just={Alternatives from (7)}
                          ]
                          [\neg Q, close
                          ]
                      ]
                      [R, close]
                  ]
              ]
          ]
      ]
  ]
  \end{tableau}
  
  
  
  \begin{tableau}
      {line no sep=1.5 cm, 
      just sep=1.5cm,
      vertical/.style={
      before drawing tree={not ignore edge, edge=draw},
      close with=$\times$
      },
      }
  [((P\wedge Q)\vee R), just={Premise}
      [\neg\neg(\neg P\vee\neg R, just={Negated conclusion}
          [(\neg P\vee\neg R), just={From 2}
              [P\wedge Q, just={Alternatives from 1}
                  [P, just={from 4}
                      [Q, just={From 4}
                          [\neg P, close={5}]
                          [\neg R, just={Alternatives from (3)}
                              [\uparrow
                              ]
              ]]]]
              [R
              [,vertical
                  [,vertical
                      [\neg P, 
                          [\uparrow
                          ]
                      ] %and now we have two
                      [\neg R, close] %brances added
  ]]]]]]
  \end{tableau}
  
  \begin{tableau}
      {line no sep=1.5 cm, 
      just sep=1.5cm,
      vertical/.style={
      before drawing tree={not ignore edge, edge=draw},
      close with=$\times$
      },
      }
  [\neg(P\wedge Q), just={Premise}
      [Q\wedge R, just={Premise}
          [\neg\neg P, just={Premise}
              [\neg P, close={3,4}]
              [\neg Q, just={From 1, $\neg(\Phi \wedge \Psi)$}
                      [Q, just={From 2, $\Phi \wedge \Psi$}
                          [R, close={4,5}]
  ]]]]]]
  \end{tableau}
  
  \begin{fitch}
      \fj  A \\
      \fa \fh B \\
      \fa \fa A \\
      \fa  B \rightarrow A \\
  A \rightarrow (B \rightarrow A) \\
  \end{fitch}


  \chapter{Recitables}

  I have of late, (but wherefore I know not) lost all my mirth, forgone all custom of exercises; and indeed, it goes so heavily with my disposition; that this goodly frame the earth, seems to me a sterile promontory; this most excellent canopy the air, look you, this brave o'er hanging firmament, this majestical roof, fretted with golden fire: why, it appeareth no other thing to me, than a foul and pestilent congregation of vapours. What a piece of work is a man, How noble in reason, how infinite in faculty, In form and moving how express and admirable, In action how like an Angel, In apprehension how like a god, The beauty of the world, The paragon of animals. And yet to me, what is this quintessence of dust? Man delights not me; no, nor Woman neither; though by your smiling you seem to say so.

\end{document}
