\documentclass[a5paper, 12pt, openany]{book} % A5 paper size

% \usepackage[paperwidth=148mm, paperheight=210mm, top=1.27cm, bottom=1.27cm, inner=1.9cm, outer=1.27cm, headsep=1.5cm, footskip=1.75cm]{geometry} % Custom dimensions and margins
% Adjusted margins to ensure space for page numbers
\usepackage[paperwidth=148mm, paperheight=210mm, 
         top=1.27cm, bottom=2.5cm, inner=2cm, outer=1.27cm, 
         headsep=1.5cm, footskip=1.5cm]{geometry}


\usepackage[utf8]{inputenc}
\usepackage{ctex}

\usepackage{fancyhdr}
\pagestyle{fancy}
\fancyhf{} % Clear all header and footer fields
% \fancyfoot[C]{\thepage} % Center the page number at the bottom
% Define left and right page numbering
\fancyfoot[LE]{\thepage} % Left side for even pages
\fancyfoot[RO]{\thepage} % Right side for odd pages
% Ensure the chapter pages (plain style) also have this layout
\makeatletter
\let\ps@plain\ps@fancy
\makeatother

% *-----------------------------------------------------------------------*
% | Fonts and typography                                                  |
% *-----------------------------------------------------------------------*

% Set CJK main font (for Chinese/Japanese/Korean characters)

% \setmainfont{Times New Roman}
\setCJKmainfont{BabelStone Han}
% doesn't work
% \setCJKmainfont{JyutcitziWithSourceHanSerifTCRegular}[
% Renderer=Basic,
% UprightFont = * ,
% FallbackFonts={BabelStone Han}
% ]



% You can also use \newfontfamily for custom non-CJK fonts if needed
% \setCJKmainfont{JyutcitziWithPMingLiURegular}[Path = ./, Extension = .ttf]
% \setCJKmainfont{JyutcitziWithSourceHanSerifTCRegular}[Path = ./, Extension = .ttf]



\newfontfamily{\jczPMingLiU}{JyutcitziWithPMingLiURegular}[Path = ./fonts/, Extension = .ttf]
% This has the best rendition for latin characters 
\newfontfamily{\jcz}{JyutcitziWithSourceHanSerifTCRegular}[Path = ./fonts/, Extension = .ttf]
\newfontfamily{\batang}{batang}[Path = ./fonts/, Extension = .ttf]
\newCJKfontfamily\koreanfont{Batang}[Path = ./fonts/, Extension = .ttf]




% *-----------------------------------------------------------------------*
% | Quotes and formatting     |
% *-----------------------------------------------------------------------*

\usepackage{epigraph} 

\renewcommand{\contentsname}{目錄} % Traditional Chinese characters for "Contents"
% Set global paragraph indentation and spacing
\setlength{\parindent}{2em} % Adjust this value for the desired indentation
\setlength{\parskip}{0pt}   % No space between paragraphs

\setcounter{secnumdepth}{0} % no numbering for sections 
% Increase chapter title size in TOC
\usepackage{tocloft} % For customizing table of contents
\renewcommand{\cftchapfont}{\Large\bfseries} % Large and bold chapter titles
\renewcommand{\cftchappagefont}{\Large\bfseries} % Large and bold page numbers

\renewcommand{\figurename}{圗}

\makeatletter
\renewcommand{\@makefntext}[1]{\jcz{\@thefnmark.} #1}
\makeatother
% to control itemise spacing
\usepackage{enumitem}
% create index - run \makeindex in the document
\usepackage{makeidx}
\makeindex



% *-----------------------------------------------------------------------*
% | Math & Equations     |
% *-----------------------------------------------------------------------*
\usepackage{amsmath} % For advanced math formatting
\usepackage{amssymb} % For mathematical symbols
% \usepackage{tikz} % For drawing logic decision trees



% *-----------------------------------------------------------------------*
% | Table Management                                                      |
% *-----------------------------------------------------------------------*


\usepackage{graphicx}
\usepackage{array}
\usepackage{tabularx}
\usepackage{tabularray}

\usepackage{float}      % Add the float package


\usepackage[table,xcdraw]{xcolor}
% 

% Load ruby package for furigana (Ruby text)
\usepackage{ruby}
% \renewcommand{\ruby}[2]{%
%   \ruby{\jcz{#1}}{\jcz{#2}}%
% }


% *-----------------------------------------------------------------------*
% | Chinese and Soochow Numerals for chapter management                   |
% *-----------------------------------------------------------------------*

% Define Chinese numerals for numbers 1-99
% 〇〡〢 〣 〤 〥 〦 〧 〨 〩 十 〹 〺 卅

\newcommand{\soochowNumeral}[1]{%
  \ifnum#1<10
    \ifcase#1 〇\or 〡\or 〢\or 〣\or 〤\or 〥\or 〦\or 〧\or 〨\or 〩\fi%
  \else
    \ifnum#1<20
      〸\soochowUnits{\numexpr#1-10\relax}%
    \else
      \ifnum#1<30
        〹\soochowUnits{\numexpr#1-20\relax}%
      \else
        \ifnum#1<40
          〺\soochowUnits{\numexpr#1-30\relax}%
        \else
          \ifnum#1<50
            卅\soochowUnits{\numexpr#1-40\relax}%
          \else
            \ifnum#1<60
              〥十\soochowUnits{\numexpr#1-50\relax}%
            \else
              \ifnum#1<70
                〦十\soochowUnits{\numexpr#1-60\relax}%
              \else
                \ifnum#1<80
                  〧十\soochowUnits{\numexpr#1-70\relax}%
                \else
                  \ifnum#1<90
                    〨十\soochowUnits{\numexpr#1-80\relax}%
                  \else
                    \ifnum#1<100
                      〩十\soochowUnits{\numexpr#1-90\relax}%
                    \fi
                  \fi
                \fi
              \fi
            \fi
          \fi
        \fi
      \fi
    \fi
  \fi
}

% Helper command for units (ones digit)
\newcommand{\soochowUnits}[1]{%
  \ifnum#1=0
  \else
    \ifnum#1<4
      \ifcase#1 \or 一\or 二\or 三\fi%
    \else
      \soochowNumeral{#1} % Use Suzhou numeral for numbers greater than 3
    \fi
  \fi
}

\newcommand{\chinesenumeral}[1]{%
  \ifnum#1<10
    \ifcase#1 〇\or 一\or 二\or 三\or 四\or 五\or 六\or 七\or 八\or 九\fi%
  \else
    \ifnum#1<20
      十\chinesenumeral{\numexpr#1-10\relax}%
    \else
      \ifnum#1<30
        二十\chinesenumeral{\numexpr#1-20\relax}%
      \else
        \ifnum#1<40
          三十\chinesenumeral{\numexpr#1-30\relax}%
        \else
          \ifnum#1<50
            四十\chinesenumeral{\numexpr#1-40\relax}%
          \else
            \ifnum#1<60
              五十\chinesenumeral{\numexpr#1-50\relax}%
            \else
              \ifnum#1<70
                六十\chinesenumeral{\numexpr#1-60\relax}%
              \else
                \ifnum#1<80
                  七十\chinesenumeral{\numexpr#1-70\relax}%
                \else
                  \ifnum#1<90
                    八十\chinesenumeral{\numexpr#1-80\relax}%
                  \else
                    \ifnum#1<100
                      九十\chinesenumeral{\numexpr#1-90\relax}%
                    \fi
                  \fi
                \fi
              \fi
            \fi
          \fi
        \fi
      \fi
    \fi
  \fi
}

% Custom chapter title formatting with Chinese numeral chapter numbers
\usepackage{titlesec}


% Custom chapter title formatting with Chinese numeral chapter numbers
\titleformat{\chapter}[block] % 'block' means the title appears on a new line
  {\Huge\bfseries} % Font size and bold formatting for the title
  {\soochowNumeral{\thechapter}} % Chinese character for chapter number
  {1em} % Space between the number and the title
  {\Huge} % Custom style for the chapter title itself (can modify)

% Remove the default LaTeX behavior of forcing new chapters to start on a new page
\makeatletter
\renewcommand\chapter{\if@openright\cleardoublepage\else\clearpage\fi
  \thispagestyle{plain}%
  \global\@topnum\z@
  \@afterindentfalse
  \secdef\@chapter\@schapter}
\makeatother




% *-----------------------------------------------------------------------*
\begin{document}
% to avoid overfull hbox
\sloppy
% % \jcz{} must be run so the document can process jyutcitzi 
\jcz{} 





% % \jczSourceHan{}

\tableofcontents



\chapter{粵切字 Sample |  }

聲母

\begin{table}[H]
  \centering
  \begin{tabular}{|>{\centering\arraybackslash}m{2cm}|>{\centering\arraybackslash}m{2cm}|>{\centering\arraybackslash}m{2cm}|>{\centering\arraybackslash}m{2cm}|} 
    \hline
    \begin{tabular}[c]{@{}c@{}}b 比\\ ⿱\end{tabular} & \begin{tabular}[c]{@{}c@{}}p 并\\ ⿰\end{tabular} & \begin{tabular}[c]{@{}c@{}}m 文\\ ⿱\end{tabular} & \begin{tabular}[c]{@{}c@{}}f 夫\\ ⿰\end{tabular}  \\ 
    \hline
    \begin{tabular}[c]{@{}c@{}}d 大\\ ⿱\end{tabular} & \begin{tabular}[c]{@{}c@{}}t 天\\ ⿱\end{tabular} & \begin{tabular}[c]{@{}c@{}}n 乃\\ ⿰\end{tabular} & \begin{tabular}[c]{@{}c@{}}l 力\\ ⿰\end{tabular}  \\ 
    \hline
    \begin{tabular}[c]{@{}c@{}}z 止\\ ⿰\end{tabular} & \begin{tabular}[c]{@{}c@{}}c 此\\ ⿱\end{tabular} & \begin{tabular}[c]{@{}c@{}}s 厶\\ ⿱\end{tabular} & \begin{tabular}[c]{@{}c@{}}j 央\\ ⿱\end{tabular}  \\ 
    \hline
    \begin{tabular}[c]{@{}c@{}}g 丩\\ ⿰\end{tabular} & \begin{tabular}[c]{@{}c@{}}k 臼\\ ⿱\end{tabular} & \begin{tabular}[c]{@{}c@{}}h 亾\\ ⿰\end{tabular} & \begin{tabular}[c]{@{}c@{}}ng \scalebox{0.5}[1.0]{乂}\scalebox{0.5}[1.0]{乂}\\ ⿱\end{tabular} \\ 
    \hline
    \begin{tabular}[c]{@{}c@{}}gw 古\\ ⿰\end{tabular} & \begin{tabular}[c]{@{}c@{}}kw 夸\\ ⿰\end{tabular} & \begin{tabular}[c]{@{}c@{}}w 禾\\ ⿱\end{tabular} & \begin{tabular}[c]{@{}c@{}}m/ng 𫝀\\ \ \end{tabular}  \\ 
    \hline
  \end{tabular}
\end{table}
% 韻母

韻母

\begin{table}[H]
  \centering
  \resizebox{\textwidth}{!}{ % Adjust the width to fit within the page
    \begin{tblr}{
      colspec={|X[c]|X[c]|X[c]|X[c]|X[c]|X[c]|X[c]|X[c]|X[c]|X[c]|}, % Column alignment
      hlines, % Horizontal lines
      vlines  % Vertical lines
    }
        & \empty   & -i  & -u  & -m  & -n  & -ng  & -p  & -t  & -k \\
    /aa/ & aa \linebreak 乍  & aai \linebreak 介 & aau \linebreak 丂 & aam \linebreak 彡 & aan \linebreak 万 & aang \linebreak 生 & aap \linebreak 甲 & aat \linebreak 压 & aak \linebreak 百 \\
    /a/  &      & ai \linebreak 兮 & au \linebreak 久 & am \linebreak 今 & an \linebreak 云 & ang \linebreak 亙 & ap \linebreak 十 & at \linebreak 乜 & ak \linebreak 仄 \\
    /e/  & e \linebreak 旡 & ei \linebreak 丌 & eu \linebreak 了 & em \linebreak 壬 & en \linebreak 円 & eng \linebreak 正 & ep \linebreak 夾 & et \linebreak 叐 & ek \linebreak 尺 \\
    /i/  & i \linebreak 子 &      & iu \linebreak 么 & im \linebreak 欠 & in \linebreak 千 & ing \linebreak 丁 & ip \linebreak 頁 & it \linebreak 必 & ik \linebreak 夕 \\
    /o/  & o \linebreak 个 & oi \linebreak 丐 & ou \linebreak 冇 &      & on \linebreak 干 & ong \linebreak 王 &      & ot \linebreak 匃 & ok \linebreak 乇 \\
    /u/  & u \linebreak 乎 & ui \linebreak 会 &      &      & un \linebreak 本 & ung \linebreak 工 &      & ut \linebreak 末 & uk \linebreak 玉 \\
    /oe/ & oe \linebreak 居 &      &      &      &      & oeng \linebreak 丈 &      &      & oek \linebreak 勺 \\
    /eo/ &      & eoi \linebreak 句 &      &      & eon \linebreak 卂 &      &      & eot \linebreak 𥘅$_{\text{朮}}$ &      \\
    /yu/ & yu \linebreak 仒 &      &      &      & yun \linebreak 元 &      &      & yut \linebreak 乙 &      \\
    \end{tblr}
  }
  \caption{韻母}
\end{table}

聲調

\begin{table}[H]
  \jcz{}
  \centering
    \begin{tblr}{
      colspec={|X[c]|X[c]|X[c]|X[c]|X[c]|X[c]|},  % Equal-width columns and centered text
      hlines,  % Draw horizontal lines
      vlines   % Draw vertical lines
    }
      1 & 2 & 3 & 4 & 5 & 6 \\ 
      󰘠、󰘦 & 󰘡 & 󰘢 & 󰘣、󰘧 & 󰘤 & 󰘥 \\ 
      󰝰、󰝶 & 󰝱 & 󰝲 & 󰝳、󰝷 & 󰝴 & 󰝵 \\
      分 & 粉 & 訓 & 墳 & 憤 & 份 \\
    \end{tblr}
  \caption{切字 聲調}
\end{table}

\begin{table}[htbp]
  \jcz{}
  \centering
  \renewcommand{\arraystretch}{1.5} % Adjust row height
  \setlength{\tabcolsep}{4pt} % Adjust column padding
  \resizebox{\textwidth}{!}{
  \begin{tabularx}{\textwidth}{|X|X|X|X|}
  \hline
  % \rowcolor[HTML]{D0D0D0} 
  \textbf{坊間漢羅混用} & \textbf{漢字已整理版本} & \textbf{漢字粵切字混用(未組裝)} & \textbf{漢字粵切字混用(已組裝)} \\
  \hline
  咁都係果D嘢嘎啦,廿鯪蚊個餐又湯又剩唔通有得你食天九翅咩?求求其其有D肉有D菜蛋白質澱粉質撈撈埋埋打個白汁茄汁黑椒汁咁撐得你懵口懵面咪Lui返去返工返學返廠返寫字樓囉。唔係你估真係搵餐晏仔咁簡單啊。咁跟飯定跟意粉啊? 
  & 咁都係果啲嘢㗎啦,廿鯪蚊個餐又湯又剩唔通有得你食天九翅咩?求求其其有啲肉有啲菜蛋白質澱粉質撈撈埋埋打個白汁茄汁黑椒汁咁撐得你懵口懵面咪纍返去返工返學返廠返寫字樓囉。唔係你估真係搵餐晏仔咁簡單啊。咁跟飯定跟意粉啊? 
  & 丩今´都係丩个´大子¯野丩乍`力乍`,廿力正⁼蚊個餐又湯又剩𠄡通有得你食天九翅文旡¯?求々其々有大子¯肉有大子¯菜蛋白質澱粉質撈々埋々打個白汁茄汁黑椒汁丩今´止生゙得你懵口懵面文兮`力句¯返去返工返學返廠返寫字樓力个¯。𠄡係你估真係搵餐晏仔丩今`簡單⺍乍⁼。丩今´跟飯定跟意粉⺍乍`?
  & 󱜩都係󱟡󰦠野󱛒󰿒,廿󰻃蚊個餐又湯又剩𠄡通有得你食天九翅󰗘?求々其々有󰦠肉有󰦠菜蛋白質澱粉質撈々埋々打個白汁茄汁黑椒汁󱜩󰿽得你懵口懵面󰖚󰾠返去返工返學返廠返寫字樓󰼠。𠄡係你估真係搵餐晏仔󱜪簡單󰀓。󱜩跟飯定跟意粉󰀒? \\
  \hline
  \end{tabularx}
  }
\end{table}
  


\chapter{\ruby{振}{}\ruby{り}{}仮名用例}

% Example of furigana over kanji
これは\ruby{漢字}{かんじ}の例です。 %漢字 will display "かんじ" as furigana above it

% You can also use furigana for names or specific terms
\section{漢字と\ruby{平仮名}{ひらがな}}


この文章は、\ruby{日本語}{にほんご}を練習するためのサンプルです。



\section{漢字用\ruby{粵切字}{}}
漢字用\ruby{粵切字}{}󱝚例子

\epigraph{All human things are subject to decay, and when fate summons, Monarchs must obey}{\textit{Mac Flecknoe \\ John Dryden}}




% \section{漢字用\ruby{粵}{\jcz{}}\ruby{切}{\jcz{}}字}


\ruby{}{而家}搞$^{'}$


 


󱅅	{\LARGE 俗}	ゾ
ク
󲅗	話	ワ
󱦆	傾	ケ
イ
󰪯	談	ダ
ン


 
光緒丙申年新鎸,邵紀棠先生輯,羊城太平新街以文堂藏板。

\chapter{自序}

	語云:知多世事胸襟濶,識透人情眼界寬。知識兩字,由於自己之想象而明,亦由聞人之談論而得也。嘗見街頭巷尾月下燈前,閒坐成群,未嘗無語,但所論多無緊要之事,未足以有補身心。或有談及因果報應,則有聽有不聽焉,且有抽身而去者矣。非言語不通,實事情未得趣也。惟講得有趣,方能入人耳、動人心,而留人餘步矣。善打鼓者,多打鼓邊;善講古者,須談別致。講得深奧,婦孺難知,惟以俗情俗語之說通之,而人皆易曉矣,且津津有味矣。誦讀之暇,採古事數則,有時說起,聽者忘疲。因付之梓人,以備世之好言趣致者。
 
\chapter{橫紋柴}

	康熙間,四川省重慶府有一個舉人,姓安名維程,為人和平,無甚過處。生二子,長名大成,次名二成。大成之性,生來孝友;二成之性,一片愚頑。兩兄弟同胞不同性。安維程年四十餘,一病身故,剩下二子。田園可以足用,不至飢寒。大成之母沈氏,稟性極偏,不循道理,隨意所發,以執拗為能。此等賤婦\index{潑婦},不是家庭之福。鄰里婦女多鄙薄之,加其號曰「橫紋柴」,其人可想矣。

	橫紋柴見大成年紀有二十歲,為之婚娶。其新婦姓鄭,名珊瑚,生得十分美貌,極有禮義,柔聲下氣,奉事家婆。每朝晨早,定必到家婆處問安,捧茶獻餅,少不免修飾顏容,威儀致敬。誰不知橫紋柴一向性情佻撻,見珊瑚美麗,自覺懷慚,遂大聲罵曰:「做新婦敬家婆,是平常事,你估好時興麼?何用支支整整、聲聲色色,辦得個樣嬌嬈,想來我處賣俏嗎?我當初做新婦時,重好色水過你十倍!唔估今日老得個樣醜態,減去三分。」

	家婆教新婦,理宜話:「亞嫂你都算有禮,但係仔乸上頭,駛乜咁拘束呢?粗衣麻布,到來問候,便是規模,不用太為着意。」如此說話,方是教道後生。你話佢賣俏,唔通做新婦,向家婆處賣俏麼?此等家婆就是惡得無理,而且講到自己做新婦時好色水,更不成個家教。

	珊瑚聽罷,低頭順受,不敢出聲。明早又奉茶餅問安,粧得雅淡潔淨,着件洗水藍衫,頭面不施脂粉。橫紋柴一見又發怒曰:「昨朝話一句,今朝敢就花唔戴、粉唔搽、新衫唔着,想來激惱我,你估我唔知你!估我唔知!」極似惡婆聲口。珊瑚又低頭無語,自怨不曉奉承。自後,踢着櫈仔,將珊瑚罵;鷄唔食米,將珊瑚罵。珊瑚去探外家,三日歸來,被罵了十日。大成見老母不悅,遂將珊瑚拷打,以順母心。打得冤枉呀!橫紋柴暫時安然,不久病氣復發,古怪離奇,無情無理。

	咒罵既慣,如鴉片烟引一樣,引起之時,唔咒罵唔做得。又如發冷症,三日一囘,或兩日一次。所以發冷有鬼,咒罵亦有鬼。發冷之鬼,至怕胡椒;咒罵之鬼,至怕口向火燒。

	一晚,不過因些小事不合意,便企在門口大罵一場。珊瑚捧張竹椅出來,請婆婆安坐。橫紋柴坐下,腰骨挨斜,手指天、脚拍地,罵不絕聲。珊瑚煲茶一碗,捧來請婆婆解渴。橫紋柴飲了,喉嚨既潤,氣更高、聲更响。罵到三更,聲漸低、力漸微、氣漸喘。就是狗吠得多氣力都倦。珊瑚跪下稟曰:「婆婆所教,媳婦盡得聽聞,今知改過咯。請婆婆囘牀安睡,免至在此受了生風,通夜呌\footnote{呌:「叫」󱝚異體字,睇同年代󱝚粵語文獻,󱜩樣寫都係常見普遍。}肚痛。」橫紋柴曰:「我要罵!我要罵!拚音伴之唔睡,罵到天光。」罵到豪興勃勃,人睡靜後又有鬼來聽。珊瑚從旁啼哭,鄰里共來勸止。珊瑚點燈來引,扶住歸房安歇。整好被鋪蚊帳,移正枕頭,囑咐婆婆安睡而去。

	明早,即到家婆處問候。看見家婆唔出得聲,睜開雙眼總冇神情,發亂頭搖,似死一樣。嚇得珊瑚魂不附體,奔告鄰里。老伯婆一齊來\footnote{「來」係𠄡係應該訓讀󱃲「嚟」(󰹚),值得思考。}到,一見光景,呵呵大笑,話珊瑚曰:「你唔在慌,佢不過昨晚劈大個口,出得氣多,撞了生風,蠱住個肚,以至血脉不通,精神\ruby{困}{󰝲?}倦。靜養三兩日,自然好咯。」珊瑚方明其故。即買防風、羌活\footnote{羌活,又稱作蠶羌、川羌、竹節羌、大頭羌。}、蘇梗、薄荷,以驅風邪。又買黨參、北耆,以補元氣。食了兩劑,僅能出得聲、食得飯。

	橫紋柴要買豬肉煲湯,以潤腸肚。珊瑚從命,照樣奉承。誰知肚內尚有風痰,未能疎發得透,食了豬肉,謂之傷風夾膩,啞了喉嚨,十餘日不能出得一語。請一個醫家先生來看脉,誰知此位先生,係初學手,唔識脉理,思疑風熱傳裏,悞用大黃、樸硝,大劑濃煎。橫紋柴飲了,\ruby{疴}{󰄦}\footnote{疴:通常寫「屙」}得眼核俱深,瀉到周身疲倦,不能起坐,面黃骨瘦,不似人形。更兼瀉壞元神,脾胃俱弱,以至飲食無味,日覺乾枯。

	橫紋柴一肚鬱勃不平之氣,憎厭無定之情,妙得兩味大黃、樸硝,瀉得乾乾淨淨,五腑六臟,忿恨皆消。此位先生精醫婦人惡毒,雖話初學工夫,其實可稱老手。

	及後,另請過一個醫家,幾番調治,僅可開言。如是者有數月餘,頗見安靜。珊瑚暗中歡喜,以為婆婆納福,此後可以安枕無憂。誰知聲音响亮起來,仍係照前怒罵。大成出館讀書,身中常帶微病。橫紋柴罵珊瑚:「辦得好樣,致我個仔昏迷,傷損元氣。我個仔若死,要你命填償。」又罵大成不知好醜,唔中用,不顧身,貪愛老婆,致老母遇時憂慮。大成本來知得珊瑚賢孝,無奈老母不合意,遂寫分書一紙,吩咐珊瑚曰:「我聞娶妻所以事母,今致老母時時激惱\footnote{惱:姐係今時今日嘅「嬲」。},要妻何用。我將分書與你,你可別尋好處,另嫁他人,不宜在我屋住也。」話完翻袖出門而去。

	珊瑚聞言,心神俱喪,將分書扯碎擲於火盤,歸房暗哭一夜。自知事不能挽,只得捲好袱包,擇三兩件緊用衣服,自行攜帶,其餘物件雖多,無心掛念也。拜別家堂香火,及沈氏婆婆,欲語不能成聲,濕洒兩行珠淚,垂頭喪氣,行步遲遲。出到門前,停足企住,想起當日出嫁之時,父兄叔伯戴纓帽、着長衫、點燈籠,一班隨護送我落轎,曾經囑咐教我孝順翁姑。今者被不孝之名,趕逐出來,有何面目歸家見父兄叔伯?不如一死便了。想完,即向袖裏拿出一張較剪仔,對正喉嚨,用力一剪。適值旁邊有一個婦人見他如此凶性,即用力擒住他手,盡勢推開,大喝一聲:「乜你咁勢凶\footnote{
  「兇」在今日粵語多獨用作形容詞,在《俗話傾談》經常與其他字組合成詞,意思除了本意「兇惡」外,還可指人「狠心」,略同今粵語詞「狼死」。
  \begin{itemize}[itemsep=0pt, parsep=0pt]
    \item 適值旁邊有一個婦人,見他如此兇性,即用力擒住他手,盡勢推開,大喝一聲,乜你咁勢兇呀?
    \item 各人見他咁兇勢,咁撒賴,難以用手相爭。 
    \item 有咁樣惡法,我个新婦既死,巳經傷心不了,重來毁我房屋,散我家私,將我老婆咁樣凌辱,有咁太過兇橫。佢恃拳頭在近,官府在遠麼?
  \end{itemize}
  }呀!」誰知\ruby{較剪}{󱛢󱂹}已到喉處,僅傷喉皮,血出不止。此婦人即扯落珊瑚包頭帶,快快札住,大喊救命。鄰里紛紛走來,各拈跌打丸散來敷,止住血流。珊瑚挨\ruby{凭}{󰊂?}門前,面如土色。各人看見,俱有可憐之意,或出嗟嘆之聲。

	橫紋柴大罵曰:「你故意裝傷,想來累我。你要死,去歸外家處死,勿惹得咁多人在我門前嘈鬧。」旁人看見尚且悲傷,做了家婆,無一毫憐憫;大凡惡婆,良心先死。族中有一個守寡婦人,係王氏,素知珊瑚係好人。今家婆不容他在家,又既受傷不能行走,遂扶珊瑚歸到自己屋,買藥調理。不滿十日,傷痕好了。橫紋柴又來大罵曰:「你個賤人,既被\footnote{或應訓讀為「󰉡」(畀)?}丈夫逐出,為何不歸父母家?在此作我\index{眼中釘},動我\index{心頭火}。」王氏答曰:「㗇㗇,你個橫紋柴,真正好笑咯!你個仔既寫分書,就如路人,那一個重係你新婦呀?走來罵人,問你醜唔醜?珊瑚係我親戚,我親戚來探,你都唔許佢住嗎?」罵得落花流水,無非代珊瑚出一肚悶氣。罵得橫紋柴無言可答,含羞忿忿,直走囘家。

	珊瑚對王氏曰:「此處原非久住之所,我今去矣。」捲包袱直往姨婆家。姨婆嫁姓駱,即橫紋柴之大姐,大成之姨母也。年老而無夫,有媳守寡,而孫尚幼,與大成相離甚遠。平日來探,見珊瑚孝義,十分愛惜。故珊瑚投到其家,將事情略說與聽。姨婆曰:「我盡知我妹稟質奇離,不近人性,我是以懶於行探,為此故也。總之難為你受此抑屈淒涼。」珊瑚曰:「不關婆婆之事,總係我唔曉孝順,致激惱婆婆,自知罪該萬死。」只是怨自己不是,不怪他人,所以好到絕頂。姨婆曰:「你不須如此說,我知你委曲咯。」

	住了幾日,珊瑚之母走來見女曰:「你母相隔得遠,一向唔知。今聞得女壻既寫分書我女,為何不囘母家,而在此攪擾姨婆,因乜緣故?」珊瑚曰:「女今無顏囘見父兄叔伯,就在此處繡花織布,粗茶淡飯,度日終身。」母曰:「女呀!睇你唔出做乜咁錯見?以你咁樣人材品貌,何憂冇好處。我要揀一個女壻,大多錢,好人品,又冇家婆拘束,然後嫁你。」珊瑚曰:「我聞忠臣不事二主,烈女不嫁二夫。女有一個家婆尚不能曉得奉事,更有何面目再入他家。母親如果要將女另嫁他人,女惟有投河吊頸,食藥自盡而已,斷不願偷生人世咯。」

	詩曰:淡淡春風氣力微,池塘一水綠漪漪。蓮根自種深泥裏,不逐楊花到處飛。

	話未完,喉頭哽咽,氣倒在地,哭不成聲。姨婆看見,眼中出淚,話其母曰:「你勿苦逼佢,由得佢咯。你逼佢太過,佢一時淺見,輕生個陣點算好呀!」其母亦拭淚而言曰:「唔知點樣解,天生得你個\index{壞鬼女},有好處你唔行,有好人你唔做,其母心盲,未分好醜。重來掛念個\ruby{的}{󰦦}\index{惡家婆},自怨唔\index{奉事}得佢透徹。你嫌佢羞磨得你少麽!制節得你少麽!提起個\index{昏婆},我就想咬佢兩啖,你重唔捨得佢,係你賤咯!老母做主張尋訪好頭路,你去要有得食,有得着,你唔肯去,甘願捱饑抵餓,問你賤唔賤!你餓死,勿怨我老母;你冷死,勿怨我老母。你唔遵我講,我此後割斷条腸,總之作生少你一個。個吓唔慌重來望吓你!」珊瑚只管哭,其母只管罵,姨婆只管兩便開解。其母見女意終難轉,遂抽身抽勢,發脚就行。留他食飯,忿忿不答,出到門口,囘頭以手指珊瑚曰:「自後我唔認你做女,你亦不用認我做老母。」話完,忙忙而去。寫得老母火氣句句如生。其母去後,珊瑚遂在姨婆之處守志安居。

	「忠孝節義」四字,為萬古綱常,頂天立地人物。此四個字,如大祠大廳之有四柱。祠廳之內,如簷前花板、板障花窗,可以粉飾浮誇,穿崩鬦湊;獨至四條大柱,須用堅石,須用實木,自頭到脚都要咁堅,都要咁實。外面雖然質樸,其中硬直不移,然後可以頂住棟樑,撐支大廈。天地之間須有忠孝節義等人,然後可以扶植綱常,轉移風俗。若使並無忠孝節義,個個俱是奸淫邪盜之人,吾恐日月無光,天翻地覆矣。忠孝節義,天上地下稱為四大名家。吾謂做忠臣難做,節婦更不易。少年之婦,曉得從一而終,立志不肯再嫁。無奈死者之骨肉未寒,而外家之親戚紛紛到門相勸,話有好頭路、好人家,早宜出脚。於是亞姑來勸者有之,亞姨來勸者有之,亞妗來勸者有之,而為之母者,更不知幾多甜言蜜語矣。媒人婆、竹筍䯻,又不知幾多花言巧語矣。若非鐵石心肝,未必不為其所動。今珊瑚之被逐出,夫雖未死,而恩情已斷矣。夫不以佢為妻,家婆不以佢為新婦矣,而猶情念故夫,心存孝道。老母幾番辱罵,百折不囘,節孝之心,可貫天日。吾願世之為婦道者,當繡其像,以香花奉之。

	橫紋柴自珊瑚出門之後,招集做媒人等來吩咐曰:「我有好仔,唔憂冇新婦。你等媒婆,務宜代我尋一個好女子,送年庚入來。婚姻事成,我自有厚謝。別人謝媒婆,送銅錢二百,我謝媒婆,微微薄薄都要封銀兩大員。」各媒人領命而去,四處尋訪。誰知橫紋柴之名通傳遠近,各家父母見了佢個後枕就怕了九分,誰肯將女嫁佢個仔呢?是以兩年之久,都無一紙年庚入屋。橫紋柴嘆曰:「㗇,㗇!真正古怪唔通。我間屋唔好住?我\ruby{的}{󰦦}飯唔好食?為何總無人共我做親家呢?實在難明其故咯。」人人都明,總係自己唔明。

	因見二成長大,不得不與他計策成婚。第二個新婦娶姓周,名呌臧姑。初歸入門,橫紋柴教之以孝順:「要低頭下氣,奉事家婆,千祈勿學我從前大新婦個\ruby{的}{󰦦}醜品。果然依你個句說話。你要好過佢為是。論起番來,你好我好。做家婆有乜唔愛新婦呢!總係做新婦唔明,家婆多\ruby{的}{󰦦}怒氣。有時家婆唔明,做新婦多\ruby{的}{󰦦}屈氣。你肯聽我教,我就心頭跌落脚踭筋咯。」

	誰知二成個老婆名臧姑,其實呌做\index{冇天裝},花號又呌做\index{霸巷鷄乸}\footnote{霸巷鷄乸:指經常發惡󱝚婦女}。花號亦新。家婆話佢一句唔中意,佢就頂嘴十幾句。朝朝睡到日高三丈,然後起身。要治家婆洗碗洗碟,煮菜煮飯;家婆唔肯做,就大聲喝罵:「幾十歲人,各樣工夫唔做得\ruby{的}{󰦦},唔通飯都唔煮得餐食吓?你估同我地後生,慢慢梳光頭,搽了粉,戴好花,又要扎周致个雙脚麽!」橫紋柴有時落得水多,落得水少,其飯煮得太軟太硬,臧姑就沉吟密咒,好似稟神咁樣稟,又罵\index{老龜婆},又罵\index{老狗乸}。被橫紋柴聽知,怒曰:「你來咒我媽?」臧姑凸起眼睛曰:「我就咒你,你點樣惡法呀!我唔怕你惡,共你打清,然後食飯都做得。」話完即捲起衫袖,扎緊包頭帶,抽身抽勢,裝模作樣,好似猛虎下山想人肉食。原來臧姑生得又高又大,又肥又壯,又兇又惡。橫紋柴見其兇氣滿面,當時怕了三分。及至臧姑發起威來,橫紋柴即走出門外,大聲呌苦救命,圩咁嘈,蝦咁跳,話:「唔知乜頭路,娶着個\ruby{的}{󰦦}\index{衰家狗},專門制治我!我一生純善,有鄰里所知,何嘗有你個\ruby{的}{󰦦}後生咁惡。豈有此理,新婦惡過家婆,你話難唔難呢!」臧姑聽聞,置之不理,皆掩口而笑。是晚家婆新婦企住門口,大鬧一場。橫紋柴咒至三更收功,臧姑偏咒至四更,然後收口。橫紋柴知自己鬦他不住,忍氣吞聲。

	詩曰:臧姑偏要治家婆,只為家婆惡得多。嫩草怕霜霜怕日,惡人自有惡人磨。

	一日,罵次子二成曰:「二成你個\index{乞食骨},你個\index{盲虫頭},你咁樣做仔嗎?你睇你老婆咁大膽,遇時咒駡,你做丈夫總唔喝佢一聲、打佢一棍,問你點解?」二成曰:「佢又冇得罪我,打佢做乜呀?」橫紋柴曰:「照你講來,唔駛拘管佢,由得佢刻薄老母嗎?」二成曰:「你原本亦係多氣。我前者大嫂,你話佢唔好;如今我老婆,你又話唔好,唔知那一個中你意呢!我老婆自己話好,我都話佢幾好。」世界之中,有等幫住老婆,所以共成忤逆。橫紋柴見二成如此,更加惱悶,染成病症。只有大成請醫調理,捧藥捧茶。二成兩公婆,九不知十不知,總不打理。

	大成話二成曰:「細佬,你知老母睡在床中,所為何事?皆由你夫妻激氣所致。你不能勸化其妻,連你都成不肖。老婆係外姓所出,你係老母所生。獨不思你幼時有病,老母長夜點燈不息,懷抱服事,眼水唔乾。僅到天光,頭唔梳、面唔洗,將你搭在背上,尋訪醫家,用藥調理,求神拜佛,額頭叩崩。你有病,老母苦切關心;老母有病,你總不着意。你將來亦望生子生孫,做人父母;照樣學你做法,有何用哉!細佬,須聽我言,明早到老母牀前,問候幾句:尚請醫家來看脉否?食粥或食飯?抑或想食甚麽物件?低聲和氣,以慰老母之心,方成子道。此段說話,非止勸二成,即謂勸天下之人子可也。咁多樣說話,你記得唔記得?」二成一肚局宿氣,答曰:「你估我好\index{蠢才}麽!你慌我唔記得!」話完就去。

	第二朝,晨早起來,臧姑喝曰:「你發顛麽!僅僅天光,就起身展開張被,冷着我膊頭。問你去何處?」二成曰:「我去老母處問安。」臧姑曰:「你勿整成個\ruby{的}{󰦦}假心事\footnote{心事:󱪙《俗話傾談》裏󰊺出現同「心事」有關󱝚詞語包括「假心事」、「有心事」、「好心事」,󱛖󱀱粵語「心事」用󰹚做名詞,一般係指負面󱝚諗法,但󱪙《俗話傾談》󱀥可組成形容詞短語,意思𠄡一定負面。}來戲弄我。假心事都勝過冇心事。我知你底子不是個樣人,不知你聽誰人所教。」二成曰:「係亞哥吩咐我。」臧姑曰:「你聽別人猶自可,好聽唔聽,聽你亞哥話。你亞哥係廢人,佢既明白,為何又冇老婆呀!大約你想唔要老婆,然後學佢。學佢你就該衰,終須有錯。你聽我話便有好人做。我不準你去!你若要去,我今晚早早閉埋門,不許你歸來睡。」二成曰:「要我不去,有何難哉!我就走上牀,睡囘我處。」臧姑笑曰:「咁樣方係好老公呀!」

	詩曰:忽聞枕畔喝聲高,膽碎魂驚嚇縮毛。自願叩頭裙底下,二成真是老婆奴。

	癡心男子、惡舌婦人,共一張牀,可稱蛇鼠同眠矣。大成一心以為細佬必來母處問候,誰知又是空望一場。自想母親\ruby{的}{󰦦}病,由屈氣而成,須得一人常時與他講話,解悶消憂。縐眉一想,喜曰:「有計,有計。我本來有一個大姨母,年老得閒,何不請他來與母相伴。姐妹之間,得來談論,可以開懷。」就定了此意。遇有人去姓駱處,順寄一聲,姨母竟然來了。由是橫紋柴頗不寂寞。夜靜更深,茶水亦便,情投意合,講話常多。大姨之媳婦,日日使人送食物來供奉:有時墨魚煲豬肉,或生魚煲羹,或柑橙桔蔗,或粉果糖糕。大姨所食不多,橫紋柴則亂吞亂嚼,大滿所欲。歡喜而言曰:「大姐乜你咁好福分,娶得個新婦,如此孝義。你來探親,尚且有物件送來;不知你在家食盡多少咯!」大姨曰:「曉做好家婆,便有好新婦。此句千真萬真,但世上亦有好家婆唔得好新婦者,有好新婦唔得好家婆者。總之各盡其道而已。世界事,隨隨便便,你識我識,多得\ruby{的}{󰦦}食。」橫紋柴曰:「我冇咁好新婦。你睇吓我個\index{冇天裝},都唔望佢買過我食,但願佢勿咁惡,勿激我咁多,我都願咯!」大姨曰:「前者珊瑚在家,情性亦好。你罵佢肯低頭,你打佢唔怨氣。總係你太醜頸,未免不情。」橫紋柴歎一聲曰:「我今者因第二新婦唔好,想起大新婦果然係好。如今悔恨難翻,未知他嫁了何處,天南地北,難再相逢。等我病好之時,去看吓你個新婦罷咯。」

	詩曰:無端淩逼少紅顏,追悔當年太恃蠻。常在眼前生厭賤,好人去後見真難。

	又遲廿日,病體好清,大姨既去。一日,橫紋柴往探,入門坐定,就問:「大姐,你個新婦咁好,往了哪處呀?」大姊曰:「我個新婦唔好,你個新婦算好。」橫紋柴曰:「我之新婦不知嫁了何方,好好我亦無份。」大姊曰:「你珊瑚尚住我處,織布度日。所買食物供奉,皆是佢積之錢。」橫紋柴聞言,心神震動,長聲歎曰:「可憐他!可憐他!做乜咁好新婦,我都唔知,真難為佢。既在你家,為何不見?」珊瑚由房中出來,跪在面前曰:「媳婦不孝,不能奉事婆婆,萬望婆婆恕罪。」橫紋柴雙手扶起,忙忙答曰:「十分孝,十分孝!孝到冇人有。自古及今,都算你第一。總係我\index{老\ruby{懜}{󰚱}懂}\footnote{「懵懂」。},唔中用,罵人不分輕重,你勿怪我。食飯後肯跟隨我回家,就是家門之福咯。」珊瑚曰:「若得婆婆收留媳婦,就算恩德如天。媳婦有不是處,還望婆婆教道。」橫紋柴曰:「不用教,不用教,照從前咁樣孝法便好過頭咯!」

	古人云:「書到用時方恨少,事非經過不知難。」凡人當富貴之時,氣勢豪雄,作自己唔知幾高、唔知幾大,諸般奉承,尚不能滿其意。一經貧窮患難之後,得少自足而不求多,逢人可交而不敢傲。凡事幾經磨挫,心氣易得和平。如珊瑚前後都是一人,何以橫紋柴初時見之咁憎,後來見之咁喜?想其日長月久,被\index{冇天裝}諸多拂戾,無地可消。回憶始做家婆,未免刻薄太過,有我罵人,無人罵我。方信順我者珊瑚,敬我者亦珊瑚也。悔恨方深,感懷倍切:裙釵影隔,誰來捧藥牀頭;環佩聲沉,不見提壺桌面。怨我生之不幸,嗟彼美之難為。種種傷心,莫補當年之錯;宵宵作夢,何時異地相逢。故一得見而氣已先伸,亦一得見而情不自禁者也。

	大姊殺鷄切肉,同席暢飲。珊瑚擇一件好鷄肉勸與家婆,橫紋柴就擇囘幾件勸與新婦。勸鷄頸與珊瑚曰:「你一生好喉頸。」勸鷄腸與珊瑚曰:「你後來日子長。」勸鷄尾與珊瑚曰:「你將來好尾運。」又勸珊瑚飲鷄酒,話:「後生飲過好兆頭。」個餐橫紋柴飲了幾十杯,醉得面紅紅,頸軟軟。食完飯後,振起精神,撥把亞婆扇,擺手擺臂,帶珊瑚歸家。歸到巷口,好多人問及,橫紋柴曰:「我個新婦未有嫁,佢話要歸來奉侍我,我亦唔捨得佢,是以帶佢歸來。你話好唔好呢?」衆人曰:「難得咯,難得咯!真正第一好新婦咯!」歸到家,丈夫愛老婆,家婆愛新婦,一團和氣,滿面春風。

	詩曰:新人原是舊時人,別後相逢倍覺親。夫亦愛妻婆愛媳,此時化作十分春。

	惟有二成夫妻,自見冇乜趣味。二成惱氣曰:「前者我個亞哥話唔要老婆,如今又收囘,點樣對得人住?我個老母更加\index{發戇},初時話大新婦唔好,如今作佢一個寶,點樣解法?唔合我心。我要分開家產,各有各食。」大成聞之,話二成曰:「細佬,你要分便分。」二成曰:「我要分。」於是請埋個\ruby{的}{󰦦}舅父、大姑丈、二叔公、三伯爺來分家。二成曰:「坑田我要多五六畝,沙洲地我要多七八畝,好果木我要多十条。」舅父曰:「老子剩下家財,兩兄弟一人一半,只見佢做長子嫡孫要多\ruby{的}{󰦦}為是,為何你重要多過亞哥呢?」二成曰:「亞哥讀了十幾年書,考了六七案試。亞哥娶老婆用兩副八音,我娶老婆不過一副六吹,所以要補\ruby{的}{󰦦}過我。」大成曰:「細佬,我唔爭,由你要剩,然後到我。」二成佔埋\ruby{的}{󰦦}好田好地,好物件東西,大成總不與他計較。二叔公曰:「唔話得咯!咁樣大佬,算世間第一人。我七十多歲人,一生共人分家,不計其數。有因爭田頭地角數尺之間,甚至打崩頭,打裂額,至結怨成仇而鬧官司者;有爭器用什物,大小不均,爭至眼紅面赤,相見而不相呌者。惟是你算至睇得破,特出離奇,高人一等。」大成曰:「父母家財,亦唔係定局。佢話要多\ruby{的}{󰦦},我作父母剩少\ruby{的}{󰦦}。假如生多幾個兄弟,唔通硬板要翻咁多麽?」二叔公拍掌喜曰:「不枉你老子教你讀書十幾年,算見得到,做得出。」

	大成出外教館,以養老母;珊瑚繡花織布,奉事家婆。一室同居,十分和樂。二成夫妻暗偷歡喜,可以無拘無束,自作自為。置一張鬼子枱,油了金漆;兩張竹椅,可以伸腰。象牙\index{\ruby{筷箸}{󰜚󱅽}}\footnote{筷箸:粵西四會、廣寧一帶皆稱「筷子」為「箸」。台山話讀「󰜚󱂤」},磁器碗碟,白\ruby{釉}{󱕡}茶壺,描花局盅等項,件件俱全,鮮明雅潔。居然鬧做亞瓜,老婆好似十萬銀身家都有咁鬧駕,餐餐要飲有色酒。有一朝飲到半處,呌老公趕往去斬叉燒、切鹵味,用蓮葉包住。被老母撞見,問:「乜樣東西?」二成曰:「你不用問我。我與你分開食,你唔管得我。個\ruby{的}{󰦦}就是龍肉,與你無干。」橫紋柴大怒曰:「你個\index{盲虫頭},可惡大膽,出言不順,得罪老娘。我不容你食!」伸手一拋,將二成蓮葉之包,盡撒在地上。剛剛有兩隻大狗在旁,發狂搶食。二成快低頭執拾,與狗相爭。狗開牙咬他,幾乎咬斷手指,咬得血淋淋、紅滴滴。拾回幾件燒肉,又染泥沙。旁有一班兒童,拍掌呵呵大笑。二成喃喃咒罵,忿忿而歸。臧姑問知其故,亦覺可惱,又覺可憐。

	兩公婆只怨老母不仁,沠老母不是。四時八節唔呌老母食一餐飯,唔請亞哥飲一杯酒。大舅來盡禮致敬,買魚買肉陪待;外母來歡天喜地,殺鷄殺鴨留餐。有一年八月十三,請外母來做生日。捉一隻大肥鷄,三斤四両重,用蓮米、風栗、紅棗、香信、正菜、薑片,會齊來燉。煲到火候到,香氣透過鄰家。二成生得兩個仔,臧姑遇時自己贊好命。其大仔有數歲,見燉鷄待外婆,問其父曰:「我去呌亞媽來食飯,好唔好呢?」二成曰:「問你老母方能做得主意。」臧姑曰:「你勿去。呌佢做乜呀!個\index{老狗乸}罵家婆做\index{老狗乸},誰知自己係嫩狗乸,終須輪到你做。好死唔死,畀狗食都唔好畀佢食!」臧姑呌其仔去買豉油,吩咐之曰:「亞媽見你買豉油,問你食乜樣,你話食生豆腐,唔好話食鷄。」

	後被橫紋柴聞之,惱氣話珊瑚曰:「天地間有\ruby{的}{󰦦}咁樣人,冇心肝冇到極處。外母來殺鷄陪待,兩公婆唔呌老母食一件。想起來養仔做乜用!娶新婦做乜用!」珊瑚笑曰:「唔通個個都學佢麽?有\ruby{的}{󰦦}人做醜,亦有人做好呀!個個學佢,唔成了世界。你去佢處食,食得幾多件呢?我明日去墟上捉一隻肥鷄、買一個豬肚,用豬肚笠鷄,任你食飽。」橫紋柴曰:「點樣笠法?我幾十歲唔曾食過咁好味道。」珊瑚第二日,竟然照樣製法,橫紋柴食得又飽又飫,掃吓個肚,伸吓条腰,十分滿願。逢人向說,話得個珊瑚真正好新婦矣!

	老年人想遂口腹之欲,未必明言說出「我想求飲求食也」。為子為婦者,默知其意,當盡情而供奉之。亦有人因時講及,不覺露出心情,尤當豐厚一餐,以暢其意。今者橫紋柴想食鷄肉一味,珊瑚加多豬肚,添多兩味,仍用香信紅棗,各樣同煲,自執酒壺,滿斟歡飲,同枱樂叙,大嚼無拘,擇其好者而敬奉之。橫紋柴當亦點頭稱讓,飲一大醉,食一爛餐,連汁撈理,連砵舐淨。想見橫紋柴之飽飫,大滿所懷,能無坦坐椅來,捧住個肚,呵呵大笑也哉?孝婦之心,曉遂老人心意,觀於此事,何等快活,何等神情。

	且說臧姑暴戾兇橫,日甚一日,任情自縱,孽滿生灾。一日,因些小事不合意,將婢亂打,一時錯手打破腦門,流血至死。婢之父怒曰:「我窮然後賣女,賣過你使喚,唔係賣過你打死呀!你買婢好出奇麽?我女將來做財主婆都唔定!你唔通照得命過,世世子孫都唔駛賣女嗎!你打死我個女,我與你誓不干休,要告官治你。」真真告到官。太爺即時出差來捉臧姑,鎖住頸拖去。太爺開堂審曰:「你個\index{賤婦人},心腸惡毒,將人性命作為兒戲。問你該當何罪!快快招來。」臧姑跪稟曰:「太爺明見。小婦人一生好善,初一十五都有拜佛燒香,何至有打死人之事。只因此婢好偷飯食,被我撞見,捶佢幾拳,不覺打破頭顱,佢就轆倒在地,敢就死了。小婦人拳頭有幾多力呢!都係此婢肚有風痰,運當命盡,借意身亡,又唔作得我打死佢呀!」太爺曰:「你養婢不飽,至饑餓難堪,所以要偷飯食。你不憐憫,重奮揮拳,此婢氣弱難當,無怪死於毒手。殺人依律,你有何言?」

	詩曰:打婢原來想氣消,任他無食餓終朝。肚饑難抵拳頭重,白白收人命一條。

	臧姑曰:「以刀斬人謂之殺,以手打人都謂之殺麽?小婦人心實不服。」太爺曰:「賤\index{潑婦},好逞刁蠻,將他打嘴巴一百。」差役發起威,打得臧姑牙肉腫浮,血流滴滴,兩便腮頰凸起,好似豬頭咁大。臧姑且哭且罵,以手指住太爺,話官恃强欺佢。太爺發怒,喝起差役,重打一百藤鞭。打得血肉交飛,仍然未肯招認。官呌差曰:「且將賤婦押住班房安置。」第二巡放告,婢父又來催紙。第二堂又審臧姑。臧姑恃牙尖齒利,辯論多端。官喝差曰:「拿夾棍來。」遂將臧姑夾起,夾得眼中水火齊來,十隻手指夾拆,抵痛不住,轆倒在地,氣絕幾回。用冷水噴醒,遂嗚呼大哭曰:「我認咯!係我打死佢咯!」官曰:「既招認了,將他押在監房。」二成見妻受苦,好似刀切心肝,即跑回家,向財主佬生借錢銀,作打救老婆之用。各稱不允。出於無奈,將田地作賤出典,得銀三百両之多。將一百補回婢父,作止淚銀;其餘二百,作衙門之費。臧姑在官門又嘔又瀉。押了兩月,然後放回。面目乾枯,形容似鬼,皮消肉削,黃瘦如柴,不似從前之神精氣爽矣。

	\index{冇天裝}忤逆家婆,積埋一身罪孽,何處消除。豈料意外生灾借端而發,因打死婢一事,捉去公門。官府開堂,尚敢花言巧語,任你逞刁恃潑,難當三尺嚴刑。毒打幾番,方信醜人難做;呼天呌苦,生平之惡氣皆消。惡人自有惡人磨,天倉滿係掘頭路。至於二成之計,爭佔家財,膽敢欺兄,自為享用。誰知一場冤孽,究竟成空,\index{負心人}終無好結果。可知皇天有眼,最憎不孝不弟之人。

	臧姑歸家,二成請跌打先生來醫傷痕,浸藥酒、埋補丸,朝朝問候。臧姑有時出入,二成扶住而行。鄰里或笑其愚,二成曰:「你唔在笑我,為夫之道應當如此。佢係我老婆呀!唔應份要愛佢麽?」知有夫道,不知有子道,所以謂之愚夫俗子。

	一夕,大成睡中,夢見其父喜色而來曰:「大成你果然好仔,更難得咁好新婦。你老母一生醜稟,我與佢做半世夫妻,豈有唔知?惟大新婦能容忍佢,能順受佢,能愛敬佢,可謂孝義賢良。你兩公婆個\ruby{的}{󰦦}孝心,灶君每月上奏;西天值日功曹,遇時奏聞玉帝。玉皇大帝十分歡喜,將來賜你兩子登科,現在賜你金銀滿甕。」大成曰:「兩子登科,後來之事;金銀滿甕,此銀何處而來?」父曰:「銀在後花園紫荊樹頭之下,小鬼移來。特報你知,你明日可往掘取。」父說完,含笑而去。

	大成驚覺,推醒其妻,告以父親所言之事。珊瑚曰:「我兩個唔係點樣孝法。平心而論,將來生仔學翻你,娶新婦學翻我,自己都心足咯。」大成曰:「順理行將去,隨天吩咐來。」珊瑚曰:「如果掘出銀,先捉一對豬仔來養,然後買幾隻牛仔與人看守。年中亦有牛租穀呀!前者二叔所典之田,其價極賤,不如贖囘此契,亦是相宜。所剩之銀,開一間當舖,或做糖房,捐個功名,起兩間書房大屋。你話好唔好呢?」大成笑曰:「你即時想做財主婆麽?」珊瑚曰:「唔通唔想?」夫妻通夜講做財主佬之事。

	講到天光,燒熱水洗了面。大成謂妻曰:「你去巷𡰪音篤亞美叔借一張熟鐵鋤頭,鄰巷亞德三伯爺借鋤頭一張。」大成脫了個件金線帽,蚨蝶頭鞋,洋布白襪,藍布長衫,抽高褲脚,捲起衫袖,手執鋤頭。珊瑚亦執一柄,精神爽利,得意洋洋。兩人到樹頭處,你一鋤我一鋤。珊瑚只曉繡花織布,鋤不上三四十吓,自呌手軟。大成笑曰:「如果冇力,容你歇吓手,坐片時,然後再鋤都做得。」大成亦係拈筆拈扇斯文之士,安能有幾多氣力呢?誰知鋤至七八十吓,氣嘈起來,又要伸吓腰,又話臂頭痛,話珊瑚曰:「你起身來鋤,又到我歇手來坐吓咯。」珊瑚笑曰:「你講乜本事,重話想棄文習武,去學彎弓。」大成亦大笑。鋤到大半朝,謂珊瑚曰:「你去歸煮飯,買\ruby{的}{󰦦}豬骨煲湯,炙幾両好酒,壯吓氣力,補吓手骨。另切過二両瘦豬肉,切爛蒸鷄蛋與老母食。」珊瑚曰:「記得咯。」臨食飯時,橫紋柴曰:「樹頭工夫不是你兩人鋤得,不如請人鋤起便罷。」大成曰:「柴數無多,除了工錢,所值有限。現無別事,即管作拾柴燒。」

	食完又鋤。鋤至午後,連根拔起,易見功程。再鋤幾吓,轟震一聲,似有白光飛出。捫泥細看,色白片片,圓面似杯口大者裝滿一大甕缸,知其銀也。夫妻神情起舞,欲笑不能成聲。二成忽來看見,忙忙指其兄曰:「亞哥你太不良。紫荊樹頭,乃係父親遺下,我着一份。你擅自鋤掘,而不與弟商量,是欲瞞騙我也。唔做得,唔做得!是必要對分一半。你想獨得,我與你鬧官司。」前者扑死婢,曾經問過?大成曰:「你不須憂,務宜兩兄弟照派。」二成曰:「一字咁淺,唔通重要請舅父來處置麽?我在此看守,呌大嫂去祠堂托秤。」珊瑚即去,臧姑亦得聞之,急將幾隻老糠籮倒轉在地,任由滿地老糠而不計矣。担籮跑到放好,秤架吊起秤杆。二成手執秤鉈,睇住秤星;臧姑扒銀入篸,倒轉於籮,每籮重一百斤。大成之銀,秤輕幾両;二成之銀,足重有加,因二成掌秤故也。秤完,兄弟各抬回屋內。

	二成拍掌而高跳曰:「做人至要有本心!我一世冇難為人,不過專工難為老母、難為亞哥而已。故此天唔虧負我。前者為官門事,破費數百,心實不甘。如今得回幾籮,添多幾十倍。財壯人膽,此後買多幾個婢女,就打死奈我乜何!」說到此句,何得話有本心。臧姑曰:「以錢頂住佢。」惡氣復發。二成曰:「個吓重唔係輪到我做財主佬?今晚可以飲得杯安樂咯。」即攜銀二員,出到市上,入京果燒臘舖,買好燒酒,糴白米頭,秤燒鵝一隻,切燒肉二斤。該價多少,拈銀出來秤。掌櫃先生曰:「二成哥,你兩個都係銅銀,為何向至相熟舖頭來混帳呢?」二成曰:「現在樹頭掘起,何得偽銀?必定古時所藏千百年間,銀色改變,不妨將錐試吓,方知我係好人。」掌櫃果用一錐,謂二成曰:「全係精光銅,總唔駛得,非比夾心尚有番\ruby{的}{󰦦}皮。」二成見無可奈何,求其賒住。掌櫃曰:「費事登簿,勿買為佳。」將米倒回籮,將酒倒回埕,燒鵝豬肉掛番起。

	二成失意而歸,殊無趣味。謂其妻曰:「初頭作勢,被佢當作銅銀,真正唔抵。快將鷄乸煮酒,飲過啖起過彩。」飲完,話妻曰:「明日快\ruby{的}{󰦦}共我漿洗衣服,我要去省城買貨。」臧姑問其故,二成曰:「鄉村間小墟場,舖戶應承做掌櫃,未曾學得半個月師。話好銀係銅,真正好笑。今日所掘之銀,係日久變色,拈到省城,銀師必能識得出。等我辦二百銀貨歸來,拭開佢雙眼,丟佢駕,勿使佢自認咁非凡。」是夜夫妻斟酌,俱是講買田置地,建造樓房,捐功名、做財主之事。通夜不睡,講完又笑,笑完又講,不覺天光。

	第二朝,臧姑出巷,所講說話,大有精神,高聲响亮,三句唔埋,便說:「我地個吓唔憂窮咯!」有\ruby{的}{󰦦}人想貪佢肥膩,走來佢屋,坐立講話,恭喜佢,奉承佢,褒獎佢,話佢好心,話佢好品,所以天有眼,賜福賜祿與佢。臧姑聽聞,十分歡喜。第三日,主意往省城,因開列貨單,採買什物。時值寒天,如大紅絨被、縐紗蚊帳、漆枕頭、佳紋蓆、金漆槓、長皮袍,諸般衣物。臧姑說:「我要金釵玉鈪,珠圈銀鈕,大紅裙,花衫袖,種種華麗衣裳俱備。其餘酸枝枱椅,及古玩東西,各樣都買。」兩張紙方能寫得完。落渡後,逢人便問:「省城至大綢緞舖,是那一間?買皮草,要去那一條街方有?」先坐頭艙既問,經過尾艙再問,後上蓬面又問。各人云:「你到省城便見,何必咁贅氣?」二成曰:「我買皮草呀,你估比同買草皮麽?聖人話,每事問就係是禮也,你相欺我唔識禮嗎?」滿船人皆大笑。二成唔見醜,重揚揚好得意。

	既到大城,尋着一間至大蘇杭綢緞舖,自己居然做一個辦貨大客,口講指畫,要某件貨物、某樣東西,逐一搬來,看過合式。二成說:「價錢總要老實。後來重有交易,非止一次便了。」掌櫃先生踢起算盤子:「共該銀幾多,煩貴客拈銀出來,上天平兌。」二成抽身抽勢,向兜肚內擒出一渣袋,約一百之多。掌櫃先生看過,變色怒曰:「盡是銅銀,此人定必光棍。」喝起伙計,埋手搜身。再搜出一百両,亦係銅色。通舖嘈鬧起來,不由二成分說,即用麻繩綑綁,以墨搽黑面,交與掌街巡丁,毒打一回。

	明日,搭渡歸家。臧姑知丈夫約於某日歸家,到此日近晚之時,請定四五個人,往渡頭肩挑枱椅衣物。等到渡船埋岸,一見二成扶住船篷出艙,垂頭喪氣。臧姑話:「人夫在此,可將所買什物,交他擔回。」二成搖頭搖手曰:「勿咁心急,待他起清貨,明早來擔未遲。」呌各人且歸家去。臧姑曰:「貨物放在艙底麽?」二成曰:「是也。」

	歸到家,臧姑曰:「看你個樣情形,似乎有病。定必到省城歡喜之極,在酒樓花艇食煎炒太多,發大熱氣都唔定咯。」二成抽起後衣,披開背脊,與看曰:「你試睇吓?」臧姑見腰背俱黑,驚曰:「做乜呌人刮痧刮得咁淒涼呀?」二成曰:「刮刮刮,刮你個条命!分明係被藤鞭所打,重話我刮痧?」臧姑曰:「你既做了財主,做乜重去做賊,被人捉住鞭撻麽?」二成曰:「唔係做賊。人家話我做光棍,用假銀買真貨,白白受打一場。」臧姑曰:「唔通都係銅銀?伯爺真正係唔好人咯。佢所用之銀,聞得俱是好\ruby{的}{󰦦},我所用係假\ruby{的}{󰦦},分明欺你愚蠢。你快快要佢換過。佢唔肯換,你唔怕共佢打,料得佢係教館先生,冇你咁好力。佢若不服,我走到佢屋內,睡倒地上詐死,怕佢唔換麽!」到底係女人見識高。二成曰:「着!着!着!今晚床上再斟酌。」臧姑急買紅花歸尾,及跌打丸散,又敷又搽。二成曰:「真正好心事,唔話得咯!算第一個婦人。蠻惡第一。」臧姑曰:「你亞哥、你老母,都唔來問候一句,枉費佢係同胞,枉費佢生得你出。如此無情,唔怪得兩公婆心淡。」二成曰:「不用講,不用講!個\ruby{的}{󰦦}都唔係人。」

	明早起身,走去大成書房問曰:「亞哥你真正冇本心!盡將銅銀分過我,你自己要了好銀。我被人捉住,搽黑面,辦做烏龜,毒打一身。真正唔抵咯。我唔要我個\ruby{的}{󰦦},我要你個\ruby{的}{󰦦},將銀換過方得。」大成曰:「分銀之時,你自己執秤,又係你老婆執篸,手扒手捧。我夫妻並無動手,何得有彼此之分?」二成曰:「我唔理得你咁多!總之要換過。」大成曰:「有乜緊要,你要換就換與你。」二成將銀幾籮抬來,籮換籮,盡行換過。是晚,二成歡喜不了,對妻曰:「此銀樣實在唔同,個吓唔慌有人丟我駕咯。省城唔利市,再去龍灣大埠辦過衣裝。」

	遲得兩日,又開單寫列採買什物,逐一覆記出來。問臧姑:「係咁樣嗎?」臧姑答曰:「我都唔記得。你從前所列之單,何不取回再鈔?」二成說:「個陣時被人綑綁,魂都冇了,尚敢取回單麽!」夫妻覆想幾回,方能寫得齊備。二成曰:「尚有一件至緊要未寫。」臧姑問那一件,二成曰:「要買一埕跌打藥酒,補吓背脊及周身骨節。」臧姑曰:「我都着飲。前者入宮門時,個\ruby{的}{󰦦}狗屎原差,唔顧人性命,昏咁打昏咁夾。至今皮肉似覺無傷,但遇寒風冷雨之時,骨節未免痛刺。」二成曰:「你唔好早\ruby{的}{󰦦}話。既然如此,順寫買北鹿筋五斤,虎骨膠十二両,大人參一枝,歸來補你。」臧姑欣欣然有喜色,囑咐曰:「你記得要買個\ruby{的}{󰦦}先。」二成曰:「你慌我冇記性麽!」不過唔記得老母。遂搭渡去。

	既到龍灣大埠,尋着大綢緞舖,手指貨架上說:「事頭公,我要這\ruby{的}{󰦦}貨,又要那\ruby{的}{󰦦}貨。」搬摙落來,擇其合意者買之。既講成價,二成擒一包銀五十両出來兌。事頭看過,驚曰:「豈有此理!前日有一個光棍,以三十両銅銀騙我,如今你又以五十両來騙我麽!」喝起伙計,埋手又向身內搜出尚有一百五十両之多,俱是銅色。又搽黑面,用麻繩綑綁,交與巡丁。

	詩曰:强換兄銀更不該,分明此物引衰頹。堪嗟緊被麻繩困,禍不單行又再來。

	一班巡丁來捉回館內,大聲罵曰:「你\ruby{的}{󰦦}\index{脚色},止許你食飯,唔許我地兄弟食飯嗎?我等看守此街,為何苦苦要來幫襯我呢?」二成哀告曰:「你等大哥自是明見。我本係耕田人物,忠厚至誠。我亞哥都係做教館先生,可保可結。此銀在後花園樹頭掘出,不是私鑄銅銀,千真萬真,並無虛假耶。」跪在衆巡丁處,叩頭乞免。不向老母處叩頭謝罪,所以要跪他人。巡丁曰:「不用多言,即剝下衣服,打之可也。」一脫了衫,見背脊俱現黑色,係被藤鞭打痕。巡丁曰:「你既係好人,為何被人打得個樣?實係做光棍無疑。」二成無言可答,但哀求:「唔好打咯。前日受苦痛氣未除,你估真正係牛皮鼓麽?」巡丁曰:「你唔願打,要用吊法。」二成未曾見人吊過,以為吊好過打。二成曰:「我願吊罷咯。」巡丁將他吊起,名為吊燒豬盤。吊了半夜,求生不得,求死不能,呌苦連天,喊到頸喉都破。巡丁放下,二成向各巡丁跪過,叩頭認罪。願認光棍,不肯認忤逆。

	詩曰:忤逆誰人告到官,百千罪過總能瞞。蒼天自有牢籠計,要你無端苦萬般。

	次日,在街遇着一個頗相識朋友,借得渡錢歸家。臧姑知到約于某日回家,又請工人往渡頭擔取物件。渡船埋岸,見二成在艙內行出,扶住一條竹棍,曲腰低頭,十分病色,慢慢行來。身上所着光鮮衣服,一切俱無,只剩一件汗衫,好似扯得穿崩爛破。心內大驚,料必\footnote{料必,當中󱝚「必」󱪙當年粵語實在係一個有生產力󱝚語綴,譬如󱪙「勢必」、「是必」、「想必」,都睇󰧱。}又係個一板豆腐咯。等待二成上岸,細聲問及,二成曰:「唔好講,唔好講,你扶我歸家罷。」先打發工人回去。臧姑拖住二成,二成以手扶住臧姑膊頭,一路行一路講:「該定冇財氣,唔係自己福。贃得辛苦,反為不美。我想將此銀交回亞哥便了。」臧姑曰:「唔似陣勢,都要交回,重怕衰起翻來,連命都死乾淨,個吓點算好呀!總之有彩數,唔駛怨咯。」是晚兩公婆再斟酌一夜,欲捨欲不捨。明早點香燭去拜神問菩薩,拋筶杯唔主張要;又求得簽,俱指示「此銀不可要,要之必有禍患」等語。遂決意交回,呌妻搬運送去。

	詩曰:存心行事惱天公,用盡好謀總是空。厚福本來當不得,依然幾次變成銅。

	對大成曰:「亞哥,個\ruby{的}{󰦦}銀唔利市,交回你罷咯。」大成想起,亦見奇趣,不覺微笑起來。二成曰:「亞哥,你唔在笑我,你終須要被人打過。」誰知大成所用之銀,人人話佢銀色極高,與平常銀爭得遠。每員重七錢二分,傾銀店願加多一分,每員作七錢三分計。大成亦不過取,只照平常而兌耳。

	詩曰:心也真時銀也真,皇天原賜孝心人。公平不作三分計,空笑貪婪冇一文。

	二成曰:「㗇㗇!真奇怪咯,唔通老子個穴山只發亞哥,總唔發我?到清明時拈一張鍬,拍吓老子山墳,拍鬆醒佢,呌他轉便,勿淨係發理一邊。」大成聞之,亦見好笑。大成見細佬遇時困手,未免可憐,時時以銀照顧於他。二成一執轉手,便變銅色,大成每要自己親手代佢結帳,然後算作好銀。二成話:「唔通亞哥個對手有寶?」大成亦不知其何以解法。不是手寶,為善以為寶。

	廣州省城城隍廟,掛一個大算盤,寫數句云:「人有千算,天只一算;陰謀暗算,終歸失算。」今二成可謂日算夜算矣,而總不就算,何哉?初分家時,田地爭多,為打婢告官一場賣去。後見大成掘出銀両,又要平分,可謂恃蠻霸佔。自喜多得天財,何以初用之而成銅,既換之而又銅?如果係銅,當與大成一樣;為何大成所用,稱為銀色極高?是二成之心,變詐百出,而銀両之色,亦變化不窮也。論二成所作,可以剩錢:一者,不用養父母;二者,做事冇人情;三者,不用顧本心;四者,可以講惡氣;五者,又得\index{冇天裝}內助之賢,做大幫手;理宜十年一運,世界翻新,何至東跌西崩,不見南和北合,窮途困手,酒米難賒?而且妻受官刑,夫遭吊打,天灾橫禍,意外紛來。方信大成孝心發達,土變黃金;而二成忤逆該衰,見財化水也。

	大成屢勸細佬孝敬老母,無奈二成總不依從,作老母如仇人一樣。一夜,夢見父親來怒罵曰:「二成,可惡!可惡!不孝子,賤\index{潑婦}。妻既不賢,夫亦不肖,可謂一床不出兩樣人。你兩公婆刻薄老母,你估我唔知麽!你做仔更加係一團夢,將老婆作如珠如寶,將老母作如泥如土。老母生你出來,唔係老婆生你出來呀!老母與你移乾就濕,唔係老婆與你移乾就濕呀!老母共你娶老婆,唔係老婆共你娶老婆呀!此等道理,可以壓倒泰山。為何知到愛老婆,唔知愛老母呢?你兩公婆忤逆之罪,灶君每月上奏於天,值日功曹遇時奏聞玉帝。玉皇大帝十分震怒,前日降下灾星,將你夫妻要受非刑吊打,報你不孝之罪。誰料不生悔心,依然忤逆,將來要你兒孫滅絕。你兩公婆不日要死在地下,打落酆都地獄,永無轉輪。」話完,其父忿忿而去。

	詩曰:任你公婆戾氣多,鬼神添注命如何。生前放肆無拘束,到了閻君細挫磨。

	二成驚醒,汗濕通身,推醒老婆。臧姑怒曰:「我睡得好好,你推醒我做乜事呀?」二成將父親怒罵之言,說與他知。臧姑曰:「你不過心躁而已,豈有為人父走入來被底講說話麽!況有新婦在旁,唔通總冇\ruby{的}{󰦦}禮體?別人做家公,都唔入新婦房間,何況來到新婦枕旁,共你談論?」二成曰:「話起亦有理。今晚我飲酒,食了一砵仔鹹蘿蔔,唔通真正係心躁發夢?」臧姑曰:「他話你不孝,我兩公婆點樣不孝法?你冇打老母,我又冇打家婆。不過我兩個唔好頸,冇幾何呌佢。本心之講,佢做老大,都唔呌我後生先;我做後生,呌佢老大先,我又冇咁吓作呀!」二成曰:「亦是道理。聽盡老婆咁多道理,豈有唔明白。睇你唔出,做女人咁伶俐呢!你個把嘴,真正係審死官咯!唔審得閻羅王死。」臧姑曰:「前者到衙門時,官都講我唔住。好聲價。總係佢恃蠻恃惡,原差多板子,便不由分說,打得我咁淒涼,所以輸了過佢。你老母算有名人等,做乜佢都要怕我呢?家婆要怕新婦,其新婦可知。」二成曰:「我都拜服你,果然你有本事。」

	是年十一月,天行時症,各家小兒紛紛出痘。二成大仔七歲,出黑痘死。次仔五歲,出黑痘又死。二成夫妻傷心到極,日夜悲啼。

	世上有一等人,買魚買肉,多讓與仔食,而不肯多讓以奉親。觀其心意,仔長大,將來可以有望,我望佢養老而待老者也。獨不思他時仔大,養我不養我,尚未可知;而父母則自幼養我至成人者也,未養我之仔,了不得關心。既養我之親,似不甚着意。亦如供會者,未執之會,其銀不待問而自己先交;既執之會,其銀既屢催,而猶不想出會。未執者,望日後之多收;會既執者,忘從前之領惠。誰不知生會或有爛之憂,熟會先入囊之飽,而世人喜供生會矣,不樂供熟會矣。猶之世人喜養其子矣,不樂養其親矣。獨是盡心養子,至長成而不肖者有之,將近長成而先我去世者又有之。愛子之心,付之流水矣;鞠育之情,徒勞無功矣。唯是以愛子之心愛父母,敬奉一日,報得一日之恩;敬奉一年,算盡一年之孝。就使吾父母明日死亦可,明年死亦可,在我為不虛生,在父母為不虛老。況自古及今,只有稱人之善養父母者,未有稱人之善養子女者;天地鬼神,只有庇佑人之能愛其親者,未有庇佑人之偏愛其子者。非謂子女不必愛,但恐知愛子女而不知愛父母耳。今二成夫妻愛子之心如此其誠,愛母之心如此其薄,無論兩子俱死,就使長大,亦未必佳,所謂忤逆還生忤逆也。論起大道理,我還我,仔還仔。我能孝順,無論子死,與並無所生,究竟我是天地間第一等人。生則無慚,死而無愧,若是我原不孝,即使兒孫滿眼,自己問心難去,究竟係忘恩負義之徒。

	二成怨氣不消,話:「我兩公婆一世無難為人,唔知點解個天難為我。一世冇虧負人,唔知個天點解虧負我。」日日怨天怨地,罵鬼罵神。族中有一個老太婆,素性剛直,不怕人憎。走來勸解曰:「二成,你話冇難為人,你專難為老母呀!你話冇虧負人,你偏有虧負老母呀!我唔怕你老婆刁,唔怕你老婆惡,我唔做閻羅王則可,若係我做,重要將你夫妻打落地獄,永無轉輪。」若得嫁閻羅王,可以收盡世上好多惡婦。話完拂袖而去。

	二成初聞此言,心中忿恨。再想一吓,此人與我父親之語,道理相同。唔通我兩公婆真正係忤逆,為天地所不容?料得人之所憎,必為鬼之所厭,大約菩薩怪責我都唔定咯。天光咯,將醒覺咯。臧姑眠在牀中啼哭,二成走入房曰:「你唔在哭,想起都係我兩人之錯。亞哥亞嫂十分孝順,所以又發財,又生子。我今人財兩失,必因罪重,厚福難當。若不回頭,孽深無底,地獄之苦,斷不能辭。不如立轉心腸,歸於孝義。或者天恩寬厚,赦我前非,未知賢妻你話可乎不可?」臧姑曰:「我昨晚通夜想過,將自己性情與伯娘比較,實係萬不及他一分。想起我固刁蠻,你亦懵懂。枕邊癡愛,總是昏迷,一事無成,到底如何結果。你真知悔,我願相從。」

	夫妻是晚,發心行孝。即剝花生,四更後起身煲粥,晨早捧獻與家婆食。二成買肉餅一包,來獻與老母。夫妻歡喜恭敬,甚覺有情。食粥一碗,又勸一碗;食餅一個,又勸一個。老母唔想食,苦苦勸佢食多\ruby{的}{󰦦},飽得老母個肚膨膨脹。二人去後,橫紋柴笑曰:「奇哉怪也,兩公婆一年唔呌一句老母,一年唔呌一句家婆,為何今早如此恭敬?好似亞崩養狗,轉了性都唔定咯。」臧姑歸家,即時燒水殺鷄,呌丈夫去買豬肉。個朝請老母來食飯,夫妻捧酒勸母,你敬一杯,我敬一杯,老母飲之不了。擇好鷄肉勸與老母,你敬一件,我敬一件。老母捧起碗飯食,鷄肉重高過鼻哥。老母話:「我唔食得咁多。」臧姑曰:「你作飯食呀,有幾何來到我處呢?不過十年一次。」是餐勸得老母又飽又醉。醉了難行,共扶入房安睡。臧姑往家婆處,想檢點牀鋪被席、衣物東西,或補或聯,或漿或洗。誰知蚊帳被褥,樣樣虔潔光鮮,方知珊瑚每日整理\index{周至}\footnote{
  此詞現今粵語未見,大槪意思是「仔細」、「妥當」:
  \begin{itemize}[itemsep=0pt, parsep=0pt]
  \item 你估同我地後生,慢慢梳光頭、搽了粉、戴好花,又要扎周致雙脚麼?
  \item 誰知蚊帳、被裖,樣樣虔潔光鮮,方知珊瑚每日整理周至。
  \item 父親臨病之時,見我服事得佢周至,話我孝心,父在牀頭,親筆寫云,七畝餘田,交與亞定永遠耕管。
\end{itemize}

}。臧姑歎曰:「我罪大矣,怪不得伯娘有好處也。」

	二成夫妻每日以孝順老母為心,而且敬奉兄嫂。誰知奉事一月之間,母以年老忽受風寒,染病而死。大成夫妻守喪盡孝。至於二成與臧姑,哭得似倒地葫蘆,橫轆直轆,眼胞腫起大似鷄\ruby{膥}{󱌮}\footnote{膥:會意字,讀「」,󱛖󱀱通常寫做「春」。}。

	詩曰:十年忤逆作平常,一旦回頭自主張。想奉高堂人不在,可憐哭得淚汪汪。

	鄰巷一伯婆問曰:「二成,你為何得咁悲切呀?」二成曰:「十年忤逆之罪,此罪難消。忤逆須用孝順補之,今者老母既死,不孝之罪何處消除。惟有遺恨終天,長嗟短歎而已。」

	俗語云:「得到知憂人又老,得到好眠天大光。」「明心寶鑑」云:「過後方知前事錯,老來方覺少時非。」「成語考」云:「樹欲靜而風不息,子欲養而親不在。」此等說話俱是傷心悔恨之詞。大約為人子者,於父母生前,人稱其孝,則謙讓曰:「斷不敢當。」及父母死而居喪,人問曰:「誰是大孝子者?」其子應之曰:「我是也。」不止曰孝,而且稱大孝。無論平日之忤逆父母、怒罵父母、刻薄父母者,皆得以大孝稱之。非特不肖之子,可稱為孝。即如刁蠻之新婦,惡毒之新婦,無情無義之新婦,皆可以孝字稱之。故喃魔先生高聲唱曰:「孝男、孝女、孝眷人等,行埋來奠酒呀。」聞唱一聲,此時做仔跪埋去奠幾杯,做新婦亦跪埋去奠幾杯。口水又來,鼻水又出,嗚嗚咁哭,其孝敬之情,可謂切矣。獨是父母既死,其魂影或落陰間,或即為轉世,亦未可知。就使靈魂尚在,依附神主牌,坐在高枱之上,而見一班男婦啼哭聲喧,在此者亦當眼淚交流,捧起酒杯,喉頭哽咽,而不能入口者矣。想到此時,咁樣敬法,點似得當父母在生之時,遇良辰佳節,及生日吉筵,為子者捧敬一杯,而父母喜矣,勝過死後哭奠靈前矣。況且生前敬酒,捧到唇邊,喉頭活活之聲,親見飲入肚內。乃於生前不肯敬獻,定必要等待父母死後,情願奠於地上,要父母曲腰低首,嘴向泥沙,而後方得飲此幾啖也,亦太無情矣。雖奠酒之禮自古不廢,而生前敬奉,亦人子之所當然。乃有等於父母生日之期,及正月初一之日不肯向父母跪下叩幾個頭者。問其何以不肯,則答曰:「我見醜,不能做得也。」情願於父母死後入殮之時,跪棺材做七之時,跪木主燒紙錢紙槓之時,跪屋角街頭;此時亦不見醜,亦作平常。可惜哭倒跪,不如父母生時,笑倒跪也。若向生時跪叩父母,必拖住你手,而歡喜曰:「唔在咯,唔在咯,總之中用便好咯。」其實父母心中必贊歎你有禮,必知到你感恩,父子之情,何等趣致。論起父母之恩,殺身難報,豈拜跪所能酬?而禮在則然,應當如此。生不能敬,死又何為詐哭哉!

	及時臧姑所生男女,共十餘胎,不能養得一個。或三五歲而死,或一兩月而亡,或三朝七日而絕氣,或初生落地而失聲。眼都哭乾,腸都痛斷。一晚對二成曰:「唔知得咁衰,見生唔見養。唔想佢來偏要來,既來又唔肯在此住,你話點解呢?」二成曰:「我明白咯!个\ruby{的}{󰦦}係冤孽鬼。別人家話前世唔修,我共你實係今世唔修。想起從前個\ruby{的}{󰦦}忤逆法,唔知重要點樣折墮。」臧姑曰:「我兩個曾經知錯,孝順過來。」二成曰:「可惜日子淺,開手做得遲。若係早得三五年,兩個仔或者唔駛死;抑或老母死遲三兩載,亦可消多\ruby{的}{󰦦}罪過。無奈咁撞板,想孝心老母就死,天不從人願,整定要該衰咯。」枕上夫妻又長嗟長歎。三更時,二成夢其父來告曰:「二成,你\ruby{的}{󰦦}罪孽,理宜兩子死後,夫妻即要雙亡,受地獄之苦。因你發怨悔心,改行孝義,奉母兩月,亦極算真誠,所以得留存至今日,知錯之力也。你命中應有五子七孫,因夫妻不孝,盡折去矣。其餘多生而不育者,無非個\ruby{的}{󰦦}挑生鬼,故意來惱悶你老婆也。你老婆一生之惡,戾氣難消,應受此報。」二成曰:「父親呀,小兒可免地獄否?」父曰:「免了咯!你算好彩數,幸母未死,發勇猛心盡孝一月。若非如此,刀山劍樹即是你結果之場。」二成曰:「小兒敢就絕了香煙?」父曰:「向你兄求一子傳後可也。但你毫無福澤流蔭後人,他日子孫零落不振,不似你兄,後代世世富貴榮華也。」話完父去。二成一驚而醒,以夢告其妻。臧姑曰:「苦惱之來,自知甘受無怨。但地獄之事,你止知問自己,不代我問及一言。你一生做事總冇益人咯。」

	珊瑚生得三子,兩子中進士。大成以細仔過繼二成。至今,大成子孫昌盛無比,而二成三代僅至數人,不過貧民而已。

\chapter{七畝肥田}

	雍正初年,潮州普陽縣來得一個新官,來做知縣,審事甚明白。普陽縣內村民,有一人姓陳名智,生下二子,長子陳亞明,次子陳亞定。幼年之時同讀書,長大之時同耕種,兩人相親相愛。及至各娶妻後,分開財產,別宅而居。其父陳智死後,剩有肥田七畝,本來係父在生之日,作口食之田。及父死後,兄弟相爭,親族不能解散,兩相結訟,告到縣官。

	官問其點樣原由。亞明曰:「此田當日父親應承交與我耕種。」遂呈分單簿出來,內寫字云:「老人百年之後,此田交與長孫收領。」亞定曰:「兄雖係有分單,我亦有執照。父親臨病之時,見我服侍得佢\index{周至},話我孝心,父在牀頭,親筆寫云:『七畝餘田,交與亞定永遠耕管。』」。亦將執照呈上。官曰:「照講起來,你兄弟俱着,總係你父親唔着。當取你父棺,破開問其何解,如此反覆,致你兄弟相爭。」亞明、亞定默然無語。官又曰:「田土,小事也;兄弟爭田,大惡也。我不能斷。你兩人各伸一隻脚來,兩脚合埋用夾棍夾之,能忍得住不言痛者,則田歸你咯。但不知你兩個左脚痛呢?右脚痛呢?左右惟你自家揀擇,我不能勉强。你兩人各伸一隻不痛之脚來。」亞明、亞定曰:「俱痛也。」官曰:「奇哉!兩脚真無不痛麽?你之身猶你父也,你身之看左脚,好似你父之看亞明也,你身之看右脚,好似你父之看亞定也。你兩脚尚不肯捨其一,你父生兩個仔,肯捨其一麽!此事須他日再審。」呌差役拿鐵鍊一條來,將亞明、亞定,各鎖住一隻脚,封其鎖口,不許私開。使他兩人同櫈而坐,同席而食,同牀而睡,同起而行,大便小便兩相同去,如此親密,片刻不能相離。更使人觀他兩個動靜詞色,每日來報。

	初之時,兩兄弟好似忿忿不平,總無言語。背面側坐,一個向東,一個向西。至第二日,則漸漸相向,對面而坐。第三日,則垂首低眉,兄歎一聲曰:「悔不聽房長之言。」弟歎一聲曰:「悔不聽舅父之勸。」第四日,兩兄弟相與講話矣。晚餐同席,兄弟勸飲勸食矣。差役將此情景報官,官知其有悔心也。

	第五日,呌差牽亞明、亞定上堂。官問:「你兩人有子否?」亞明曰:「我有二子,約十七八歲,有\ruby{的}{󰦦}十三四歲。」亞定曰:「我亦有二子,其年紀與兄之子亦相上下。」官呌差役捉其四子俱來。官呌亞明、亞定謂之曰:「你父不應生你兄弟兩人,是以今日至此。假使單生你一條身,田宅皆係己所獨得,何等快樂!今你亦不幸,兄弟各有兩子,他日長成相爭相奪、欲割欲殺,無有了時,深為你等憂之。今本縣代為思慮,預為之計,你兩人各留一子足矣。亞明居長,留長子,棄去次子可也。亞定居次,留次子,棄去長子可也。」命差役將亞明次子、亞定長子押去養濟院,交與乞食頭做親男,來取執照,收領存案。「彼乞食之人,無田可耕,有何爭法。獨留一人,他日得免於禍患,豈不省事便宜麽!」亞明、亞定聞此判斷,心慌起來,伏地叩頭,啼哭曰:「太爺!太爺!我不敢咯。」官曰:「你話不敢,何也?」亞明曰:「我知罪咯。願讓田與弟,至死不復爭。」亞定曰:「我不敢受,願讓田與兄,終身無反悔。」官曰:「你兩人未必真心,我不敢信。」兩人叩頭曰:「真咯,真咯!若係假心,天誅地滅。」官曰:「你兩人或者真心,你兩人之妻未必肯讓。你兄弟歸家與老婆斟酌過,遲三日再來定讓。」由是兄弟放回。

	是晚,亞明對妻說知。妻曰:「我至好係第二個仔,又精靈,又好相貌,我至中意佢。乜佢做官得咁新樣呀!將我個仔來分過乞食佬,我\ruby{的}{󰦦}仔有咁下賤?佢得咁曉判斷?我遲日去見佢,問佢做官點樣解法。」亞明曰:「太爺一一解過我知咯。我又想過咯,都係自己唔着。你遲日去見官,共二嬸上堂唔好講惡氣。你若恃嘴刁,唔肯輸服,但將你兩嬸姆,一人鎖住一隻脚,個陣要你兩個同牀同席、同坐同眠,往則同行,企則同立,了不得咁牽纏,了不得咁費事。此時你知怕咯。」妻曰:「我唔俾佢鎖。」亞明曰:「你唔肯鎖,官喝差打你。」妻曰:「佢\ruby{的}{󰦦}板子得咁便?」亞明曰:「你估板子便了嗎?藤鞭便,夾棍便,枷又便,鎖又便,隨你中意個樣有個樣。」妻曰:「我今年四十一歲,未曾見過官,我唔駛怕佢。」亞明曰:「唔怕官,總怕管呀。你唔怕,我怕咯。你兩個仔,如今押在差房,嚇得面青青,魂都冇了。」妻大驚曰:「點算呀?撞板咯!嚇死我兩個仔咯!」即流眼淚怨丈夫曰:「乜你先時唔話過我知呀!?」亞明曰:「你估衙門係花廳麽?重要話你知?唔怪得你淨曉快活。」妻曰:「我見你初去告官之時,講得咁豪氣,話呢場官司定必贏佢,七畝肥田拿手可得歸來。燒紙還神,請親族來飲過,個朝飲了兩壺燒酒,重更精神,得意揚揚,托睡鋪落艇。我以為你到衙門,原差佬要恭敬你、奉承你,請你飲、請你食,太爺要陪你坐。因你話告官,我估如仔女稟告父母,子姪投告父叔,無拘無束,企亦得,坐亦得,隨隨便便咁樣告法。見你又話去打官府,我估太爺唔遵你講,你就捉住官府來打。你又好力,官府怕你,就要依你,你就拿手得此肥田,所以我日日歡喜。誰不知官府打你,唔係你打官府,實在白白去到受苦。早知咁苦,何不忍讓三分。」亞明聽完,又見可惱,又見好笑,不覺拍枕罵曰:「你個\index{蠢婆},就係眼前之事一毫不知,要你何用!」妻曰:「官府衙門,眼所不見,婦人不曉,情有可原。家中兄弟,日在眼前,男子不明,亦屬欠解。你今為爭田之故,致我之仔分離。講甚麽肥田,我作佢係海外浮沙,高山巖石而已,有何用呀!明日即時要去,帶我仔歸來。」亞明曰:「我之與你商量正為此也。」

	又到亞定,是晚與妻講及,將官判斷說話。現今兩仔押住差房,聽我夫妻主意。妻曰:「我勸你勿去告官,你偏偏要去。好好聽叔伯排解,兄弟各得一半,豈不省事。無奈你兩個,兄既不從,弟亦不順,致今日公堂對審,失禮於人。為何你做男子總不見醜呢!我自己對人亦覺失愧。你只知利欲薰心,不顧倫理;誰不知你行前人指後,話你等豬兄狗弟,實在都唔係人。今鬧起官司,要將我大仔沠與乞兒,問你於心何忍?」亞定曰:「此事係太爺主意,非我心情。我今不願要此田,自願要仔。官恐你地女人心中不尤,要你親身同去,大衆言明。」妻曰:「我豈有愛田而不愛仔麽?我個大仔將近成人,可以幫得手。唔講話七畝肥田,就係千両黃金,當作廢鐵。明早即要到官門望吓我仔。伯娘唔去,我自己都要去咯!」

	第二朝,亞明妻郭氏、亞定妻林氏,請同族長陳德俊、陳朝義,到官門當堂求息。郭氏、林氏兩嬸姆,相扶攜跪案前,伏地涕泣,請自今以後永相和好,皆不受田。亞明、亞定亦泣曰:「我兄弟愚蠢,不知義理,有費太爺一番教訓。今如夢初醒,慚愧欲死,悔之無及。我兄弟皆不願受此田。」官曰:「不要此田,如何安置?」亞明、亞定曰:「願將此田送入寺門,作買香油敬佛。」官拍案罵曰:「可惡!可惡!此不孝之甚者也。講到送入寺門,便當用大板打死你。你父一生辛苦,勤儉艱難,然後得此肥田,為子孫之計。未明白之前,相爭相告;既明白後,則又送與和尚坐食安居。你父之心,在九泉下豈能閉目麽!為兄則當讓弟,為弟則當讓兄,弟兄不受,則當歸之於父。今以此田為你父嘗業,兄弟輪流收租,為每年春秋二祭之用,子孫世世永無爭端,豈不極妙?」於是族長及亞明兄弟夫妻皆叩頭稱善,歡喜而去。

	是晚兄弟歸家,殺鷄買肉,拜了家神父母祖先,一齊所請。然後一家暢飲,大樂團圓。第二日,再辦海味嘉菜殽,豐筵滿席。弟敬其兄,兄敬其弟,子姪奉勸叔伯,叔伯亦勸子姪。嬸姆亦共相勸飲,喜色融融,親愛百倍。由是鄉村之間,有言禮讓者矣。

	俗話傾談卷下,博陵紀棠氏評輯,番邑黃從善堂敬刊。

\chapter{邱瓊山}

	邱瓊山先生,係廣東瓊州府瓊山縣人。其祖呌做邱普,家有餘資,生平樂善,好救濟貧難。凡春耕之時,貧人無穀種者,或來乞借,即量與之。待至禾熟之日,收回穀本,不要利也。若有負心拖欠,亦不計焉。遇一歲大饑荒,邱普自捐米賑濟,煮粥以救鄉鄰。而遠近之病餓者,仍死亡滿野。邱普買幾處荒郊之地,設為義塚,請人執拾屍骸,埋藏安塟,免暴露焉。其義塚在縣內第一水橋等處,若亂塟墳也。每遇清明時節,多具紙錢酒飯,祭奠於義塚諸墳,生者含恩,死者得所矣。

	邱普生一子,名呌亞傳,娶妻後,少年早死。衆皆歎惜,怨皇天冇眼虧負好心人。邱普亦不甚悲傷,安於命運。嘗對人曰:「我少時遇一個名公先生,精於睇相,斷我之相富而不壽,無子無孫。後又遇一個批星盤先生,精通命理,我求其算命,他亦批我短命無兒,若問孫不必言矣。由是凡遇睇相算命者,無不求其判斷,所有批斷,亦是多同。後十餘年,總不再問。今既失子,而幸有孫,子雖亡而我尚在。唔通靈一半,唔靈一半也?抑或我不久要死,而孫又死也?近有算命者,話我八字依然一樣;而睇相者,話我骨格大不相同,將來福未可量。唔通半生修善,不報於其子,而報於其孫;屈抑在眼前,而優遊在後日。欲問諸天,而天極高,相離百千萬丈,雖問亦不聞聲。而《易經》云:為善降祥。禍福興衰,不如靜把寸心,問之自己而已。」

	邱普之子既死,剩得一孫,名呌亞濬,即係邱瓊山先生也。邱瓊山幼年喪父,其母李氏,苦志守寡,上則孝順翁姑,下則撫養孤兒。日夕勤勞,不敢有慢。更能體貼家翁之意,寬厚待人,亦為其子造福也。邱瓊山生得聰明,勝人百倍。經書一讀就熟,過目不忘。數歲初入學堂,時有歸田官,生得一子,年紀亦幼。遂會三五小童,請一個先生教專家館,封窓誦讀。

	一日間,亞官仔歸家食晏晝。天落大雨,瓦上有幾點細漏,滴落邱瓊山之書枱。邱瓊山遂將自己書席移去亞官仔個坐位之處,將亞官仔書席移來自己坐位之處。因近在枱边,易於相換也。此幾點漏,大雨時方有,非真大雨亦無也。及亞官仔回館,見自己枱面上有濕氣,又見不是舊時坐位,知係邱瓊山所移,遂要苦苦換回,不換不肯。邱瓊山曰:「你讀書,我亦讀書。雨滴落來,我在坐,你不在坐,唔通白白由得佢滴濕頭壳麽?你如今歸來,天又冇雨,駛乜換呢?」亞官仔曰:「你坐之處,原係我舊日書位呀。」邱瓊山曰:「你講舊日,點似得我講先時。先時移來,就係我坐在此。猶之乎我買你田,現在耕種,即是我田。唔通你講祖公耕過,重係你田麽?事以現在為真,又以舊時為假咯。」教學先生見他兩個幼童如此爭論,亦覺好笑。其時亞官仔年十二歲,邱瓊山年僅八歲。兩人當時學做對聯,亞官仔時時自稱本事。先生曰:「我出五個字,但能對得通者,我就幫佢為是。」亞官仔曰:「好呀,好呀!做得,做得!包要贏佢。」

	先生出對曰:「細雨肩頭滴。」邱瓊山即答曰:「青雲足下生。」先生贊賞曰:「果然好對。」亞官仔曰:「佢好得過我個比?」先生曰:「你點樣好法?」亞官仔曰:「等我想通透,然後話你知。」由是摩頭摩耳,眼望天,脚拍地,磨吓墨,又拈吓筆,走去小便個處企住。想一回,行埋書位,坐住椅,扭完手指,伏低枱頭,都唔想得出。先生曰:「你勿咁多事,算佢第一罷了。」亞官仔忽然歡喜曰:「有咯,有咯。」先生曰:「點樣對法?」亞官仔曰:「對頭係細雨肩頭滴,我用咁樣對法曰:流濕到衫襟。你話妙到極唔呢?」先生笑曰:「唔通,唔通。」亞官仔曰:「上下相生,文情貫串,何得話唔通?況且流濕因雨滴而來,衫襟與肩頭相近。佢個比由雨講翻到雲,未免倒亂。雲起山頭,空中來往,佢又不是神仙得道,安能足下生雲?照講起來,佢個比不通,我個比第一。」先生又笑,邱瓊山亦笑嘻嘻,書位總不肯換。

	亞官仔忿忿不服,哭去歸家,將委曲事情,如此如此投告父知。歸田官勃然大怒,曰:「佢咁可惡,就睇我唔上眼。佢點樣好對法?快呌佢來。個\index{龜蛋}唔對得好,收拾佢。」即使家僮到書館,呌邱瓊山來。先生知到歸田官發怒,定必生氣,又畏佢幾分,唔敢欄阻。邱瓊山聞之笑曰:「佢曉食人麽?佢冇咁大個口。」手執一把葵心扇,斯斯文文入到大廳內。見了歸田官,拱吓手曰:「老太爺有何見教?」話完,了不得咁雍容,了不得咁淡定。歸田官怒曰:「你移換我仔書枱,尚講咁多反蠻說話,實在大膽無禮,太過欺人。」邱瓊山笑曰:「膽自心生,福由心造,所言所做,自問一心。論起移換書枱,不過幼童情趣。老人家胸藏萬卷,量可包天,何必因些小事情發聲怒色?若以為欺人太甚,此句說話都要想吓為佳。」歸田官仍然怒氣未息,曰:「不用多言,且看你如何好對。」邱瓊山曰:「好話咯,不妨指示。」歸田官遂出七個字云:「誰謂犬能欺得虎。」邱瓊山即企起身答應曰:「焉知魚不化為龍。」歸田官一聞大驚,即拍手起身,拱手低頭曰:「拜服,拜服。老夫肉眼無珠,自知得罪。我仔係豚犬之兒,你個小孩子,將來係龍虎榜中人也。」邱瓊山曰:「蒙老人家過獎,小子豈敢當哉。」

	歸田官又呌個仔向邱瓊山拜謝。亞官仔曰:「你話我就唔好對麽?我駛服佢?」歸田官曰:「你唔服點樣對呢?」亞官仔抽身抽勢,走落天井,看過金魚缸,望吓各樣花,行埋來,點頭得意曰:「對頭係『誰謂犬能欺得虎』,我對曰:豈知虫可化為蚊。重唔勝過佢?」歸田官聽聞,亦覺可惱,又見好笑,遂罵曰:「你個\index{蠢才},勿氣死我罷咯!」亞官仔一肚局宿氣曰:「我與佢句法相同,又同了三個字,只爭四個字不同耳。況且佢講得荒唐又冇憑據,誰人得見魚化龍呢?就係父親你都唔曾見過呀!我講沙虫變蚊仔,人人共見。道理至愛真實,最忌虛浮。我句對文重實過鐵釘,落水都唔浸得爛,重話唔好過佢麽?」話完,引得邱瓊山掩口咁笑。歸田官搖頭歎氣曰:「愚而好自用,賤而好自尊,你之謂也。」又對邱瓊山曰:「亞濬,唔怪得你非凡。本來你亞公一生樂善,好事多為,所以出到你咁精靈秀氣,脫俗超羣。我自問生平冇乜好處,故此出到個\ruby{的}{󰦦}\index{脚色},無用\index{蠢才},悔之無及。」自後遂加意厚待,培護殷勤,而邱瓊山之聰明,震動遠近。

	明朝正統年間,甲子科中解元,甲戌科中進土,連點翰林。其祖邱普,老而康健,紅顏白發,親見榮封,始信天不虧人,心田變相。其後邱瓊山做官,陞到太子少保,兼武英殿大學士。死後稱為文莊公,入祀鄉賢,為廣東之名人也。曾撰「大學演義補」一書,係邱瓊山自己所作,亦可見其才學矣。邱公本名濬,係瓊山縣人,後人不敢直呼其名,而稱為邱瓊山,葢尊重之也。

\chapter{積福兒郎}

	明朝之時,浙江鄞縣有一人姓楊名忠諫,家貧,以教館為業。其教子弟讀書,先以動靜規模為緊要,再教之以孝弟,好講古事以發其心。故入其門者多曉禮義,而不至於澆漓。鄉里稱其善教,每年學生至二三十人,脩金亦有大半百。

	忠諫勤於教人,而儉於自奉,鹹魚青菜足以供餐。其待母也,必以酒肉。母之飲食雖少,而忠諫殷勤敬勸,歡喜奉承。故教館不欲遠離,若常得親近母也。生平最憐憫孤寡,凡寡婦被人欺佔,必多方扶護之。孤兒之貧者來讀書,則不計脩金,聽其自獻。

	楊忠諫,一童館先生耳。能教人以道,奉親以誠,憐孤寡以義。其立身處世,有此三大善,即為種福之根。

	楊姓之族分數房,惟忠諫之房最弱,財少丁稀,每為別房所侮。有二房人多財足,恃勢欺淩。而最强橫者,楊崇蘭也。崇蘭有二子,長子呌亞況,次子呌亞梯,生得聰明,習為奸惡,而崇蘭之勢如虎生翼矣。常理太祖數,吞騙蒸嘗,莫敢與他清算。忠諫自以立心正直,祖宗嘗業,不可糊塗。一日話崇蘭曰:「數目多年未曾清計,今欲於某日對簿,合族見個分明。」崇蘭曰:「你大膽!敢與我為仇,你將死矣。」

	嘗見各處祖宗數目,或各房分理,或各房輪理,或公舉賢良者而理之,或交有權勢者而理之。此祖宗之心,亦衆人之意也。乃有一等貪心,自懷私見,每事從中染指,藉此分肥。抑或借用虧空,未能還得,遮遮掩掩,混鬧糊塗,年推一年,月推一月,以至蒸嘗拖欠,數目難清,忍氣吞聲,衆心不服。你之敢為吞騙,自作把持,所恃者自己有權勢耳,自己居尊輩耳,自己兄弟多子孫衆耳。以為你想抽我後脚,無奈我何,誰敢與我抗也。獨不思數目者,太祖之蒸嘗也,凡做子孫皆有份焉。不過以你明白而經理之,非取你貪心而求你吞騙也。你能吞騙,則作自己為至精靈,而睇輕衆等子孫,皆為無用之人,為\index{蠢才},為\index{廢物}矣。此一錯也。無怪族衆心惱不平,而祖宗先靈且作你為對頭,為仇寇矣。先人得下幾多踴躍,而後積此蒸嘗。遇着一二貪心,東支西離,漸為消散,竟至人心冷淡,拜掃無情。祖宗之發出多人,又不如生少你一個也。此等人就是吞騙得財,子孫終無結果。如若不信,看吓各村吞嘗產者個\ruby{的}{󰦦}後人。

	楊崇蘭因忠諫之語,懷恨在心。遲日使二子楊況、楊梯,窺探忠諫出外,截在半途,故意撞膊而過。楊況詐跌在路旁,遂大罵忠諫曰:「我既閃避,為何你推倒我也?」發起兇性,兩兄弟你一拳我一脚,打得忠諫眠在地上。兩兄弟詐成,忿忿而去。忠諫既受傷,慢步歸家。各兄弟惱恨不服,欲去告官。忠諫止之曰:「不可,不可。告官決不能取勝,何也?其財雄,其力猛,其口刁,其心險。合用之可以製人,常用之足以造孽。彼將為天所棄矣,何必破財產而與他結訟哉?」各兄弟曰:「彼强我弱,受害終無了期。不如多請兇橫與他一戰。」忠諫曰:「虎與虎闘,麒鱗遠避其鋒;鷄與鷄爭,鳳凰不施其力。君子樂得為君子,小人枉自做小人。你怕衰微,急宜修善,為人盡道,定見福蔭兒孫。空忿不平,都是無益。」衆曰:「修福,吾不信其說。報應甚遠,能等得幾時親見呀。你信因果,你做多\ruby{的}{󰦦}好事,看你兒孫昌盛而已。我等無此意,與善無緣也。」楊忠諫曰:「肯做則有緣,不肯做則無緣。」各兄弟亦不能從其語。

	楊忠諫之忍氣也,大有見識矣。力能舉鼎,不與盲牛鬦工夫;快走如飛,不與顛狗鬦脚步。何也?佢盲我唔盲,佢顛我唔顛也。忍氣免目前禍患,修善望後日榮華。胸中有一個大主意,並能識出崇蘭父子家運當衰,出此妖孽,勸衆兄弟修福,以求興旺。無奈衆等善根淺薄,不肯相從,自表其心。惟有各行各路,各修各德而已。

	楊忠諫自老母死後,設館於市鎮墟場,門徒日衆,家道日豐,而濟人利物之心,功修日積。生得二子,大仔名自懲,第二仔名自創。兩子讀書長大成人,學習衙門事業。楊忠諫止之,要兩子教館便罷。誰知兩子決意不移,忠諫曰:「公門路上好修行,你能善心,亦積福之道。」

	自懲做縣衙門刑房書辦,自創做撫台衙門兵房書辦。自懲性樸實,心地慈祥,常勸人不宜結訟。自創性浮誇,心地奸詐,常勸人不妨爭訟。嘗對人曰:「吾之兄蠢人也,食衙門飯而有衙門田耕麽?既執此藝以藏身,即當索此財以養命。勸人唔好打官府,由得自餓死嗎?世事不平則鳴,人至告官,必有冤屈之處。訟不得伸,忿何以解?吾不曉兄之意,別具一副肝腸也。」

	自懲聞之歎曰:「父之德足蔭後人,弟之心其折盡矣。」因寄書勸之。自創笑其愚也。自懲做衙門,遇犯罪之人由遠來者,即呌家人煮粥以供食之,恐其遠行饑渴,轉生病也。後有一個姓蒙之官來做知縣,性兇殘,至憎賊。凡審犯,則怒氣不止,愈怒則鞭撻愈多,每有打至死者。楊自懲上堂跪稟官前曰:「上失其道,民散久矣。如得其情,則哀矜而勿喜。喜尚不可,何況怒麽?」官念其誠,從此減輕刑辱。

	其弟楊自創巧於謀算,護財至一二萬金。自懲慎於取財,只存二三百両。自創所交遊者,必以聲勢為尚;自懲所相與者,不以貧賤為嫌。自懲有四子,自創亦有四子。自創之子多習於偏,自懲之子盡歸於正。自創之子亦讀書,亦入學,亦中舉,亦發財;不滿三十年,而漸歸零落,衰敗無存。

	楊自創一生奸計,走入偏門,自己發財,仔又發達,一門富貴,榮耀一時。旁觀者必話自創之輕浮,勝於其兄之古董也。殊不知所享之福,俱由其父修善中來,正因自己不修,又做諸多折福。自己慌折不快,又呌數子幫手折之,無論科名草、吉祥花、子孫枝、平安竹,盡皆斬削,連福根都鋤起矣。

	楊自懲所生四子:守陳、守隅、守隋、守阯;其孫茂元、茂仁、茂義,或中進士,或點翰林。同朝七人俱為顯官,或為御史,或為中書,或為侍郎,或做給諫。而楊守陳之官,陞至東閣大學士。告老歸田,所居第宅,住在鄞縣城南鏡湖邊。有一個漁翁,吟一首詩獻與楊守陳云:「昔年曾向此中過,門巷幽深長薜蘿。令祖先生方秉鐸,賢孫學士未登科。將軍曹氏墳連隴,賣酒王婆店隔河。此日重經新第宅,輕舟緩棹聽弦歌。」守陳見詩,歎賞不置。謂漁翁曰:「你作此詩,可為吾家之寶也。當珍藏之,以示後人。」

	看到自懲個班子孫如此富貴,其榮華昌盛,又與自創之結果大不相同。楊忠諫一生為善,種落福根。自懲又發奮加修,栽培積厚,如山頭起屋,錦上添花,更高一層,更勝一着。究竟深山格木,古心古道,終為大用之材。而柳葉桃花,雖取豔一時,終非耐看。此所以同胞兄弟,作用各有不同。

	又說楊崇蘭之恃勢,欺人欺物,不知幾何。其後二子亞況、亞梯,販運於嶽州,經過洞庭湖,遇大風覆舟,沉水而死,家中人並不知也。後有鄰村一人,呌做胡永清,亦往嶽州,過洞庭。一夕灣船於湖邊,月影微茫,聞鬼哭之聲,終夜悲吟不絕。次早,見沙上有數行大字,寫成詩句云:「長鯨吹浪海天昏,兄弟同時吊屈原。千載不消魚腹恨,一家誰識雁行冤。紅粧少婦空臨鏡,白發慈親尚倚門。最是五更淒絕處,一輪明月照雙魂。」尾寫云:「楊況、楊梯兄弟同題。」方知即是楊崇蘭之子也。歸來報知,一家大哭。崇蘭自二子死後,門戶日衰,而世事諸多不就;楊忠諫之子孫,日新月盛。或遇忠諫,自己掩面而不敢見焉。

	種樹看結果,做戲看收場,做人要看到盡頭。出幾個兒孫點樣\index{脚色},一好一醜,兩下分明,方能定得勝負。猶之乎做戲,每日要掛完廿四個牌,然後收場。每見開枱之時,個班奸仔,耀武揚威,這文那武。有\ruby{的}{󰦦}做花花公子,有\ruby{的}{󰦦}自稱太保爺,佔人田園,霸人妻女,打人頭壳,傷人性命,百般醜態,敢膽為之。而奸仔之父兄,包庇心偏,容縱子姪,代為出計,指點行藏。而被累之好人,無端受苦矣。此時被累者,呌天而天詐作不知,呌地而地置之不理。地下咁多人企住,主棚咁多人坐住,你望我望,眼白白見奸仔害得人咁淒涼。氣唔平,心唔服,欲殺他而不得,欲救彼而不能,而無容你惱悶也。你莫行開,且睜眼看看,再一時間,掛一個牌,而好人走脫矣。又掛一個牌,而得人打救矣。又一時間,而奸仔敗露矣,好人出身矣,好人殺死奸仔矣,奸仔無處藏身矣。小武打死\ruby{的}{󰦦},二花面打死\ruby{的}{󰦦},公爺打死\ruby{的}{󰦦},而一班奸仔唔剩得幾多個矣。所以好人每受虧,先磨練一着。果能做出忠孝節義等事,必為天地所哀憐,鬼神所庇佑,終有出頭之時也。古人戲棚對云:「奸仔似虛花,盛極終須無結果;好人如夜月,缺時究竟有團圓。」是經歷世情、見得世果之語。

\chapter{閃山風}

	山西富商,多在江南金陵大城放官賬,結黨為之。有一放賬客,係姓關者,亦山西人。在金陵大鬧烟花,折去資本,所存白銀二百両。思得一計,專放私債,刻剝重利,九出十三歸,誠妙算也。

	蜂狂蝶亂賞花枝,骨軟魂消日日迷。散去錢財偏不惜,還從私債剝人皮。

	因平日交結衙門,差役係佢心腹,故能以聲勢嚇人。人皆畏懼,衆加其號曰「閃山風」,言無情之暴氣也。

	有一人姓朱,名大寬。家貧,以賣菜為生,而又好賭。向閃山風生揭錢壹千文,計及一年,共計利銀三両。不但指大於臂,而且發重於身矣。閃山風之放債於人也,必待其利有一倍,然後往問取之。又因朱大寬有病在身,屢積屢重。到此時,每日持刀往索,不許拖延。朱大寬約以十月十五日,賣了幼女,本利一并清還,斷無失信。

	到十六早,將僅天光,閃山風到了門前,持刀直入,問朱大寬曰:「本利交來。」大寬伏地叩頭乞曰:「事尚未成,容遲十日。」閃山風喝罵曰:「你無口齒,屢次推遲。你不知我威名,作我為兒戲嗎?有刀在此,你唔殺我,我要殺你,即刻了此事罷。」話完,即將刀柄向與朱大寬。其意以為大寬見我如此心急,定必剪頭賣發,都要即時清還咯。實在恐嚇朱大寬,令他心怕而已。大寬心內一想,見佢來得咁兇,均之一死,不若先下手為强,償了佢命。就接住刀柄,盡力刺去佢肚。閃山風呌痛一聲,用手掩住傷口,發脚走出。

	登門尋死罵聲多,罪滿難寬奈乜何。快\ruby{的}{󰦦}拈刀來殺我,即時要去見閻羅。

	走過橫街,有一間米舖,其東家呌做王其勉,一向父子與閃山風熟識。見其彷徨走過,呌一聲:「老關,請入來飲茶呀。」閃山風不能答。走入坐椅未定,忽然跌倒在地,血從肚內沖出,滿地紅水橫流。嚇得王其勉魂驚膽破,連聲大喊救命。左右鄰舖走來,望之氣已絕了。

	通傳街坊地保,衆將此事鳴官。山西放賬等商,又聯呈控告。官來驗看,鎖王其勉回衙,開堂審訊。先問街坊鄰舖,各對以不知原委,只聞呌救之聲。又問王其勉,答曰:「小\ruby{的}{󰦦}與佢係好朋友,佢走過,呌佢飲茶,不知其被何人所刺,就死在我舖。」太爺曰:「既被人所傷,何以街上並無血痕?顯然係你因仇殺死。」街衆稟曰:「王其勉實係與關客相好,非有夙仇。」太爺曰:「既無仇,何以死在佢舖?」街衆不能答。太爺喝曰:「不打不招。」呌差役重打壹百。王其勉不肯招認,連用夾棍夾了幾堂,打了幾次,打得皮開肉裂,夾到脚拆頭昏,迫於無奈,只得認自己殺他。

	冤枉難招要你招,兩旁夾棍一條條。幾回魂魄飛天外,但乞官恩片刻饒。

	照大清律例,要刀與傷口相符,方能結案。太爺問取其刀,王其勉說藏於這處,着差役搜尋不得;又說藏在那處,又搜尋不得。又打兩次,押在監房。王其勉之子呌做亞勤,見父受苦淒涼,心有不忍,直到官前,願代父死。太爺不準。

	亞勤見無法可救,遂將紅紙寫一長條,貼於當衆之處。其詞曰:「閃山風之死,必有仇人。吾父冤枉難招,實為淒楚。今父所存家產,約值千金。若是兇手之人,有憑有據,取出刀來,肯來實認,我願奉銀五百両。先交銀,後到案,決無反悔。三光在上,實鑒臨之。」

	朱大寬初刺閃山風也,見王其勉無端受累,本欲甘心償命,直認鳴官。退後一想,見老母有六十餘歲,恐生養死塟照顧無人,是以隱縮。今見長條所貼,有銀五百,可以安家,遂使人去問王亞勤:「真實是否?」亞勤曰:「你若不信,請埋通街老成人等,立了合同,先交銀為證。」

	朱大寬接銀回家,攜刀到官處,將始終原委,稟明事迹,如此如此,此刀係刺閃山風是實。官看此刀與傷口相符,論起殺人償命,理所應然。關客既如此惡勢,威逼貧人,自有取死之道。此事不知真假,未肯盡信你一言。即着一心腹家人,查訪定案。其家人回報曰:「訪得死者呌做閃山風,索債俱用持刀相向,逼人賣仔賣女,致人忿氣自盡者,屢次有之。」

	食人骨血破人家,未必黃金兩手拿。半世積埋冤孽帳,一場風起路飛沙。

	太爺曰:「閃山風該死有餘,勒索錢財,今竟何用!佢係為兇手,律例難寬。照事原情,當減一等,充軍為是。」發往烏龍江而去。去了十個月,遇乾隆元年,皇恩大赦,歸家養母。自後發奮,竟做好人。

	又說王其勉,本係安良守份做生理之人,何以橫禍飛灾,無端受屈?原來王其勉之兄呌做王其敏,其敏以販賣豬仔為生,養父母與弟,無不盡其誠意。弟長大,又出錢與之婚娶。是王其勉之受兄恩也,可謂深矣。

	當年養育得哥哥,又況恩深娶老婆。若使發財應報答,同分產業不為多。

	及後其勉做生理發財,兄因病後困手,其勉總不照顧一毫。兩兄弟輪養父母,一五一十,必要計清。未幾父死,兄亦死,只剩一老母,與大嫂輪流供養。嫂有二子讀書,歲底散館之時,尚欠脩金兩員。先生催得甚緊,嫂徬徨無計,向王其勉曰:「求二叔借銀二員,交與先生清數。」其勉曰:「要銀未嘗話冇,但係明年正月即交回我,方能做得。」嫂曰:「我到春來麥熟,自己唔食都清還於你。」話完,不覺暗淚滴下。其勉曰:「你勿怪我。數還數,路還路;亞叔還亞叔,大嫂還大嫂。你莫話我唔好,我唔借過你,重有得過你苦。」嫂拭淚曰:「我唔係話亞叔唔好,總之怨自己家窮耳。」

	太無情義太無良,嫂侄艱難實慘傷。不念一毫孤苦事,只知自己顧私囊。

	一夕由舖歸家,囬至村外社壇,壇上先有一人在坐。時近三更,松陰月影,涼氣風生。其勉以行得倉忙,身中出汗,不免登壇息步,一爽襟懷。與在坐者,畧相稱問。初未識為誰人,近細看之,乃胞兄其敏也。其勉知為鬼,大驚,但念兄弟至親,不須迴避。神魂稍定,問:「兄在此何為?」兄曰:「心中煩悶,並不能睡,故在此貪涼耳。」問:「弟近來生理好否?」其勉曰:「並無好處,不過平平而已。」兄勃然大怒曰:「細佬,點樣謂之平?你忘兄之恩而不顧其姪,不憐嫂之寡而薄待其親,世事至此不平甚矣。我最惱不平人,等你好久,今毒打你一場,而洩此不平之氣也。」話完,即揮拳亂毆。

	妻兒愁苦哭聲頻,有弟同胞不作親。雖在九泉難閉目,奮將拳打\index{負心人}。

	其勉伏地叩頭曰:「亞哥,唔好打咯!打咁多好咯!我怕你咯,我知錯咯,亞哥。」其兄曰:「打死你,打死你!」忽來得一個白髮公,手扶拐杖,行埋勸曰:「唔好打,唔好打,打乜樣呀!手足之情,豈可自相殘害麽!」即將拐杖欄住其兄。其兄曰:「個\ruby{的}{󰦦}手足,實在都唔係人。我能顧佢,佢不能顧我,講甚麽手足呀!我不是打細佬,打\index{負心人}耳。」白鬚公曰:「你細佬之事,我盡知之。佢前世修過善功,今世應有福享。總係今生變性,刻薄無情。灶君上奏於天,玉帝命飛天大神查訪\ruby{的}{󰦦}確,福根削去,灾禍臨頭。有人代你善打於他,無用你咁惱氣也。你有你好處,你雖貧困,有好兒孫。不信我言,試看後來光景。」

	前生修福今生受,得福而今又作殃。有等貧難修善果,將來發達有賢郎。

	其兄由是放手。其勉起來,不見了白鬚公矣。其兄忿忿下壇而去,其勉發脚走歸,睡倒牀中,神昏氣短。妻問:「因乜事幹,如此慌忙?」其勉曰:「我遇着鬼,被鬼打我。」妻驚曰:「乜樣鬼呀?你遇個\ruby{的}{󰦦}係大頭鬼?長舌鬼?抑或吊頸鬼?跳水鬼?男鬼?女鬼?竹篙鬼?瘟屍鬼呢?」其勉曰:「俱不是也,係亞哥鬼。」妻曰:「鬼做亞哥,唔通你就怕佢麽?」其勉曰:「此鬼非他,就係你伯爺作怪。」妻曰:「伯爺咁可惡!查出佢年庚八字,請喃魔先生大鑼大鼓駛\ruby{的}{󰦦}符法,收佢入禁罌。」其勉曰:「你勿高聲。白骨無情,佢聽知,連你都作怪。」妻曰:「我有名呌作惡婆,駛乜怕佢呀!你大嫂我都唔讓佢一分,都要治佢。生者不怕,要怕死者麽!」

	弧兒寡婦總之難,仰面求人幾個彎。為叔不來相照顧,嬸娘又是恃兇蠻。

	其勉受嚇,病了一月。然後回舖,不滿半月之久,又遇閃山風一案,破去家財大半。歸家又病一年。其子亞勤變性,賭蕩花消,閉埋個間舖。王其勉一貧如洗矣。兄之二子,長大發財,遵循守慎,孝義可稱。其勉倚賴兩姪,養老終身;亞勤無所歸着。

	哥哥兒子正當興,弟歎人財兩不成。天惱無情憐有義,到頭好醜自分明。

\chapter{九魔托世}

	浙江湖州歸安縣,有一個財主呌做王柱偉,先父遺下家財十萬。十八歲娶妻金氏,至三十二歲共生九個仔。可謂丁財兩盛,衣祿豐盈。半世未曾做一好事。時值大饑荒,餓死人無數。金氏謂夫曰:「我家錢財足用,何憂子女饑寒。現當兇年,人多餓死,安能見死不救,坐觀滿眼淒涼?我欲將銀數千,買米賑濟,未曉丈夫之意,以為好否?」

	欲與夫君細酌斟,濟饑救死發真心。婦人有此慈悲念,即是尼陀觀世音。

	柱偉曰:「賢妻所言,甚是道理。我聞古人為善,福蔭兒孫。況自己可以做得來,亦是代天行道。」於是搭兩個大施粥廠,男廠、女廠,各列東西。初發手買米,約銀六千。本欲賑濟鄉鄰,誰料各處聞之,愈來愈衆,僅半個月,米既成空,而一二百里之內,尚來不絕,攜男帶女,呌苦啼饑。老者扶杖而來,幼者手抱而到,紛紛似蟻,逐逐如雲,得飽一餐,願行百拜。柱偉不覺善心勃發,慷慨豪雄,任意揮金,呼羣助役,搬柴運水,奔走如雷,廠列星排,好似屯軍散餉。如此者,賑至五月中旬之後,田禾將熟,人各歸家。破費資財,共成七萬,柱偉毫不掛意,且自喜為獨出一時也。自行賑後,千里馳名,或出遊行,見之者莫不指為大慈悲生菩薩。衆謂竇燕山濟人利物,五子登科;王柱偉做此陰功,定必九子連登,開科發甲。柱偉夫妻暗喜,亦謂修善者得福,此後天官賜福,而且五福臨門矣。

	仰首呼天天不聞,天公難救衆人身。誰知遇着王財主,財主原來救得人。

	王柱偉自賑饑之後,其子或疴或嘔,或跌或傷,不滿兩年之間,九個仔都死乾净。人皆歎惜,話天眼無珠,虧負好人,無怪世間有\ruby{的}{󰦦}一毫不拔咯。亦有等說:「時歲饑荒,天意要將人餓死。王柱偉大施賑濟,逐一救生,分明抝天,與天作對。抝父母都有罪,何況抝天?」柱偉夫妻閉門日哭,哭得眼胞腫起,大過鷄膥。金氏怨曰:「我估修善有報,誰料無功。早知賑濟抝天,我自一錢不出。今既家財大破,子又俱亡,何以為生?不如一死。」夫妻想尋短見。

	究竟其中委曲,死亦難明。聞人話請得仙來,方能知得因果。於是掃潔廳堂,焚香淨几,燒符念咒,禮拜當空。來得一位觀音大士,童子扶乩。此童素不識字,乩字揮洒如飛,寫來明明白白。其文曰:「王柱偉,唔怪得你傷心,唔怪得怒氣。你知先父之夙世原因乎?」柱偉跪稟曰:「不知。」乩曰:「你父前生原盡孝弟,係做生理之人。有憐憫心,扶危救急,積成善果,今世應享豐財。無奈一轉今生,忘却本來面目,貪財重利,刻薄成家。怨氣上沖,天神震怒,分發九魔下降,托生你家。九魔者,天上之\index{掃把星},人間之\index{敗家精}也。你父所積者,好多產業,其實好多冤業。你所生者,望其為興家肖子,其實俱是亡家賊子。將來長大,賭蕩花消,姦淫邪盜,種種獻醜,玷辱門風,以報你父一生陰謀暗算之罪。豈料你夫妻發念,大結善緣,動地驚天,救人數萬。上帝將九魔收回天上,賜過五個好仔;另有兩個文星降世,顯你門庭,大享榮華,拭開人眼。你不須苦惱,且放心懷,因果原由,一言剖白。」話完,大士回去矣。

	濟饑只望大榮昌,豈料翻成一掃光。為祖不修殃後代,諸孫俱是大魔王。

	王柱偉聞言,方知明白。自後夫妻相勸,盡解愁懷。不及八年,復生五子。長大讀書,亦皆入學。第三仔所生兩孫,長孫呌做王以銜,次孫呌做王以鋙。教以讀書,少年入學,及至者等。遇一個學院大人,呌做竇東皋,來湖州考道試。在明倫堂講書,講大學首卷:「民之所好好之,民之所惡惡之,此之謂民之父母。」個一章書,講得極有精義。當時數百秀才在此共聽,亦作平常,惟王以銜兩兄弟聽到入心,以為至精至妙,勝過高頭講書解法百倍。二十餘歲,兩兄弟同科中舉,上京會試。是年係乾隆六十年乙卯科,又遇竇東皋做大總裁。會試頭場,首題出「民之所好好之,民之所惡惡之,此之謂民之父母」其三句。以銜兩兄弟,作得極好,意義精微。文章中試官,合了竇東皋之意。開榜看來,王以鋙中了第一名會元,王以銜中了第二名進土。

	當日聽書在學宮,會元題目在其中。作來喜合宗師眼,方信文通運亦通。

	當時各舉人,有不能中得者造起是非,話天下咁多人非凡不少,何以第一第二俱係佢兩兄弟中埋?況文字意思,與高頭講章微有不合,似不公道。各有浮言。當時和珅做奸宰相,素與竇東皋不睦,時時想陰謀害他。剛遇會試,各衆浮言,遂具本章,奏之皇上話:「竇東皋今科會試所取第一第二名進士,係同胞兄弟,文章不甚精工。此中必有徇情,應交禮部議處。」皇上准其所奏。禮部議竇東皋罰俸降級,第一名會元趕逐歸家,不准殿試。

	和珅有一個西賓,教其公子之先生也,亦中進土。去拜見和珅曰:「遲日殿試,未知作得好醜如何,惟望相公另眼相看,提高後手。荐拔之恩,同於天地矣。」和珅曰:「翰林三及第,我與聖上做主意。但名字彌封,不知誰是先生之卷,此處難以着力。須用淡墨寫卷,作為暗號,我自然有關照也。」既殿試後,和珅取卷來看,忽然執得一個淡墨卷,看過亦好策對。和珅喜曰:「此必西賓之卷也,我自有講法。」遂對聖上曰:「此卷文章極好,可以中得狀元,望我主准奏。」上曰:「文章雖佳,但嫌墨色太淡。」和珅曰:「正在墨淡能寫得好字,方稱老王。中佢第一,值得無疑。」上曰:「卿家話可中,則中之而已。」遂取為榜首。及開榜,唱名曰:「第一名狀元係王以銜。」

	狀元想中與西賓,淡墨為憑事有因。用盡巧言施盡計,誰知第一屬他人。

	聖上發怒,話和珅曰:「卿家,你話竇東皋唔識文章,中錯王以銜兄弟,何以你又取得佢中狀元呢?平地風波,多生議論,總係卿家糊塗之過。」罵得和珅滿面通紅,羞慚無地。和珅暗地歎曰:「㗇!㗇!乜咁古怪呢?本來一個淡墨卷,為何又多一個來?真真不可解也。」誰知王以銜殿試之日,想起細佬被逐歸家,大總裁因我降級,功名兩字,水淡心灰,就係點得翰林,不外如是。故此墨都懶磨,順筆寫去,遇着和珅以為西賓之卷,盡力吹噓,以至大魁天下。所謂人算不如天算也。聖上准竇東皋復回原職,着王以鋙第二科來京殿試。以鋙遲一科,亦點翰林。以銜官至尚書,以鋙亦官顯職。子孫昌盛,丁財壽貴全。

	在王柱偉之父,當日所為多不合衆,必有暗地笑之而罵之者。而彼則日盛月新,財源滾滾。未嘗不曰:「你笑即管笑,你罵即管罵,你不妨學吓佢咁樣本事,咁樣發財呀!」俗人唔明,有等又話:「真咯,學佢都唔錯。任你至忠直、至慈祥,好之又冇佢咁多錢,又冇佢咁大福。」買田買地,生子生孫,似乎天亦要順其心而就其計也。若謂陰謀暗算定必發財,何以世上好多周身八寶,計多過米,曉做光棍,曉謀害人,曰撈曰縮,到底依然貧困也?若話唔奸頑難贃得錢駛,何以世上好多愚愚直直、忠厚至誠,亦有人請佢打工,亦有人出本與佢做生意,而且不知不覺又發財矣?做個樣就個樣矣?今王公之財發十萬也,非因刻薄而得,實因修福而來也。刻薄要發財,忠厚亦要發財,非因忠厚發少\ruby{的}{󰦦},而刻薄發多\ruby{的}{󰦦}也。天以財十萬報你前生之善,而你好刻薄,又留後世之殃。所謂祖公個世唔修,留到子孫個世折墮矣。

	王柱偉年少而生九子,共以好命稱之,豈知其收債鬼也。及後大積陰功,救人無數,其仔即見快高長大,無病無灾;豈料風掃瓜棚,盡行傾跌,一個二個,倒地無存。無怪王柱偉之心傷,即旁人亦有不服矣。假使王柱偉對人曰:「我九個仔死乾净,將來生過幾個好仔,要孫中會元、狀元。」人必笑之而不信矣。總之前生後生,自己亦不能知而記;或凶或吉,鬼神亦未必顯而言。而以眼前順境,信前生定有修行;現在奸心,斷將來無好處而已。

\chapter{饑荒詩}

	明朝之時,景泰五年,陝西省大饑荒。皇帝使一個大官呌做周文襄,往陝西開倉賑濟。既到之後,回覆一道本章奏上,并吟詩兩首,送與朝臣一看。云其詩語語傷心,能使人滴出眼淚,算寫盡淒涼苦楚之景矣。

	其第一首曰:「蕭蕭行馬過長安,滿目饑民不可看。十里路埋千百塚,一家人哭兩三般。犬啣骸骨形將朽,鴉啄骷髏血未乾。寄語當朝諸宰輔,鐵人聞着也心酸。」

	又第二首云:「艱難百姓也堪悲,大小人民總受饑。五日不燒三日火,一家關閉九家籬。隻鵝只換三升穀,斗米能求八歲兒。更有兩般堪歎處,地無荒草樹無皮。」

	將此二詩常時吟詠,可以止驕奢,可以省浮費,可以養靜氣,可以息貪心。想到此饑荒難捱之時,安有心唔肯知足之理。

\chapter{瓜棚遇鬼}

	滄州河間縣,土名上河涯,有一人姓陳名四,年方二十二歲。家貧,未有娶妻,以賣瓜菜度活。一晚,往瓜園看守。時值五月初三四,月色微明,望見岡邊樹底似有四五人來往遊行,相聚而語。陳四思疑,此等\index{脚色}唔通想來偷瓜?雙手執住一條青蘭棍,藏身密葉之內,試觀其動靜。

	忽聞得一人曰:「我等且去瓜園一遊,行吓瓜地,聞吓瓜花,睇吓瓜仔,你話如何呢?」一人曰:「唔好去,唔好去。衰起番來,遇着陳四,被佢嚇死,重反為不美。」其人笑曰:「你既死了為鬼,重要再死一回麽?只見人怕鬼,有乜鬼怕人?你真正細膽咯。」彼鬼曰:「你咁大膽,唔駛怕人,又何以唔敢白日出現?」此鬼曰:「你真正尖利,一句頂住我。但我怕他人,不怕陳四。」彼鬼問其故,此鬼曰:「我於十日前,曾經入土地祠,見陰司勾魂票到,有陳亞四之名,不兩日要死。遲得幾晚,陳四與我等攜手遊行,怕佢甚麽!」又一鬼曰:「你只曉得講鬼話,知一不知二。陳四唔死得咯。」此鬼笑曰:「乜你咁長手脚!何解緣由?」答曰:「我昨日入土地祠,見案上有一角文書,係城隍發來,說陳四老母近日做一件陰功,添多十二年壽。」此鬼曰:「點樣陰功法?」

	答曰:「陳四鄰屋有一個財主婆,失了錢二千,思疑大婢偷去。日日鞭撻,話要認了便罷,若不肯認,要打死為止。若係自己仔女偷去,未必打得咁淒涼。婢之父聞之,怒曰:『如果我女做賊,要將他投於海中,不使生於人世。』此婢日夕悲啼,進退無路。陳四老母不覺傷心,代為憂慮,其偷與不偷,尚屬無憑,但有死無生,實為可憫。想得一計,將自己衣裳首餙盡行押去,得錢二千文,捧向財主婆處告曰:『我老身前數日入來你屋,并無人在此,見有錢百餘干堆在地上。忽起貪心,竊取兩吊,以為咁多錢數,未必記得分明。不料查察起來,疑婢所竊,將他毒打,心有難安。老身前世唔修,致今生窮苦,唔通重結此冤債,待來世酬還麽?今將錢數交還,望你寬容大量,赦我一時之錯,勿計前非。』財主婆曰:『原來如此,我又不知。老伯婆既是拈去,若係緊支,何妨借用。今既交回,事經明白,我不怪你,無用懷慚。』話完兩別。灶君將此事上奏於天,玉皇大帝將此事發落河間縣城隍注簿,查得陳亞四老母前世唔修,今世應要冇仔養老,孤零獨立,苦楚難當。其子陳亞四,壽該二十二歲,注於乾隆三十四年五月初六日死。今既有此件陰功,應將其子添壽一紀,長多十二年命,以養此婦終身。你都唔知頭尾,想陳亞四遲幾晚共你遊行,唔怪得你咁快活。」此鬼曰:「㗇!㗇!數日之間,又是一場變卦。方信閻王簿上有添有改,都無梗板寫法也。」

	陳四聽到此言,不覺咳嗽一聲,數鬼忽然散去。陳四聞言,又驚又喜,終夜思量,方知陰功可以補壽,藥物不能補也。陳四初時見老母托錢交囬於人,一肚怒氣。聽了一番鬼話,方知老母救人之故,怨氣皆消。又細想起來,自己命短,得母一善,能添一紀;十二年後又要死亡,有何長策?不如我自己立志,日日去修,到了十二年,其功不少,玉皇大帝又將我壽數加增。壽愈增,我善愈積,將來有福有壽,有子有孫,亦人生之大想像\footnote{「想像」係可以當動詞用,但係󱪙《俗話傾談》󰧵󱀥可當名詞用,係「計劃」、「打算」󱝚意思。}也。但家道貧難,難做救人之事。細思善莫大於孝,能盡孝道,莫大之功。於是歡喜奉承以待老母。其母又安享八年而死,陳四此時取妻生子矣。後修善行,晚年福壽而終。

	世界之間,有修善而見報者,有修善而不見報者。非無報也,報之而人不覺也。假使當時鄰里盡知陳四老母救婢一事,衆人必曰:「亞四老母咁好心,好之又唔見有好處。亞四並非發財,並非發貴,亦不過挑瓜賣菜,辛苦度日而已。何嘗有點樣榮華呢!」誰不知唔係做個點善心,想有個仔賣菜奉養老母而不可得。若非瓜棚遇鬼,點曉得前生今世,禍福原由。世界事許多難解之處,而鬼神消息有大算盤,不外添補扣除,統前後其計之也。

\chapter{鬼怕孝心人}

	晉陵城東門外,有一人姓顧,名呌亞成。生子,娶媳婦錢氏。其子遠出雇工,錢氏在家十分孝順。遇順治十三年,城之東便大起瘟疫證,轉相傳染。有一家死盡者,有一巷僅留數人者,親戚不敢過門探問。顧成亦染此病,一家八口病在牀中。未起症時,錢氏歸寧母家一月之久。一日,有婦人報到曰:「亞嬌,你翁姑個處時證大行,一家之人俱受重病,做乜你唔去歸睇吓呀?」錢氏聞言大驚,面變憂愁之色,歎曰:「相離甚遠,我點得而知。」即捲起袱包,辭別父母。老母留住曰:「女呀,你唔好去。個\ruby{的}{󰦦}唔係別樣病,係呌做有牙老虎。你偏回去,若撞板起來,連你都死乾净咯。」錢氏曰:「唔怪得老母憂,但男子娶妻,無非為翁姑生死之計。曉得大道理。今者有病不歸奉事,與禽獸何異?女今要去,就係死亦甘份。父母不用掛懷。」人話忠臣不怕死,我話孝婦不怕死。父曰:「照你講起番來,大條道理。況且生係佢人,死係佢鬼,在父母亦難强留。」父親甚明白。錢氏起行,老母送出村外,流淚囑咐曰:「女呀,你要去即管去,至緊要知避忌,須買\ruby{的}{󰦦}蒼术塞住鼻哥方好。」錢氏曰:「謹遵老母所言。」遂分手而去。

	錢氏望住路直走,想即時見了翁姑之面,方得心安。將歸到村邊大社壇,家中病者似見一鬼,自外走入來報信,形影徬徨,急喊各鬼曰:「我等快\ruby{的}{󰦦}走出去,不宜在此也。」衆鬼問其故,報信鬼曰:「今者孝婦歸家,諸吉神皆擁護而來,我等再留,有些不便。」各鬼慌忙失色,有\ruby{的}{󰦦}想縮入牀下底,有\ruby{的}{󰦦}想躱埋門角頭。報信鬼曰:「唔做得,唔做得,終須被佢睇出。你唔走,我去咯。」報信鬼即奔,各鬼跟隨而出。

	錢氏入門,病者俱能起坐。錢氏先到翁姑牀前問曰:「公公呀,婆婆呀,病得咁淒涼,新婦都唔知到,有失奉事,罪實難容。有請醫藥先生來調理否?」家婆曰:「此等病證,有誰人肯來探問呢?惟有自己辛苦待死而已。我斷唔估重得見你咯!」錢氏曰:「如今病體如何呀?」翁姑曰:「一連幾日辛苦,都唔話得過你知。頭又重,喉又乾,口又苦,心腹又飽脹,脚骨又困倦。欲轉側不能,欲起身不得,實在一世唔病過咁淒涼。如今忽然間頭見輕,喉見潤,口見涼,心腹見自在,脚骨見寬舒。可以起得身,可以移得步,你話奇唔奇呢!」瘟疫鬼去了。

	錢氏大喜曰:「公公婆婆,我扶你出去中庭坐吓。」家婆曰:「好呀!好呀!我睡倒床中,\index{迷迷懜懜},好久不知天地。出去看吓日頭在那處。」家公曰:「我都想出去。」錢氏遂扶兩老人出坐。家公歎曰:「枱櫈生塵,蛛絲掛滿簷前咯。」家婆曰:「你睇神樓上個\ruby{的}{󰦦}燈盞,被老鼠拖跌在地呢。」錢氏又扶衆等姑叔出來,一齊共坐。有\ruby{的}{󰦦}尚帶歎息聲,有\ruby{的}{󰦦}似帶歡喜色,有\ruby{的}{󰦦}挨住椅,有\ruby{的}{󰦦}扶住枱,有\ruby{的}{󰦦}問答懶出聲,有\ruby{的}{󰦦}挨斜伸開脚。錢氏曰:「公公,我去煲粥與你大衆食。」家婆曰:「好久唔聞米氣咯。今日食粥,明朝食飯,可以無妨。」各人曰:「前者唔肚餓,今見餓起來,唔知得咁古怪。」家公曰:「我亦係如此。」既食粥之後,出\ruby{的}{󰦦}微汗,個個精神,行動自然,聲音清爽,鄉里皆稱為奇事。翁姑遂將瘟疫鬼說話傳之於人,男婦聞之俱化為孝順,此處百餘年之久,瘟疫全無。錢氏所生之子,長大以征戰有功,官居武職,至今子孫猶昌盛焉。

\chapter{張閻王}

	乾隆間,浙江杭州有一秀才張繼興,素無品行,欺壓鄉鄰,醜事多為,人皆笑罵。一日去探一朋友,聞得某村有一婦人做鬼婆,能呼神召鬼,各婦女信而問者無數咁多人。張繼興與友亦去看其舉動。正值鬼婆焚香作法,說出鬼聲鬼氣,鬼模鬼樣,講鬼話,着鬼迷,衆人亦以為真鬼來也。各人拱立靜聽,惟恐不誠。張繼興一見,勃然大怒,走上前以掌打其嘴巴曰:「你妖言惑衆,欺騙人家錢財。若係我做閻羅王,必要扭斷你個頭。」各人睇見,掃興掃興,索然無味,俱散而去。紳𥘞來散場。遲得幾日,此鬼婆頸上生一大疽,變成斷頭瘡而死。人人驚異,遂稱張繼興為張閻王。

	又數年,張繼興得病,魂夢之中,見有兩人如官差一樣,素不相識,請繼興同行。走到一間宮殿,闊朗輝煌。左右兩神捲簾而坐,中間一神垂下竹簾,面不得見。張繼興問:「神帶我到來,有何吩咐?」神曰:「有一個鬼婆告你,因此召你而來。你怒罵鬼婆之事,道理甚公,原無冤枉。但你亦非正經人物,須自將生平作惡,共有多少,要一一自認出來。」呌左右與以粉牌,令寫其上。張繼興執筆直寫完兩個粉牌,尚覺未盡。神曰:「只此數條,罪有餘矣。照你自話,應得何罪?」張繼興想了好久,答曰:「應遭雷打。」神曰:「罰猶未足,當打三次。」捲起中座簾,呌繼興抬頭一望。看見中座神像儼然自己相貌,方醒悟前身即閻王也,因有過失,又罰轉世為人。

	一息間,兩差役又來送張繼興回里。忽然大驚,如夢初覺,汗流遍體。盡日思量,想起根底原深,只因肆無忌憚,以至罪大惡極,當受雷誅。枉費半世讀書,自稱明白,與聖賢道理大相反背,更有甚於庸俗之流,生受人憎,死遭鬼責。自思堂堂七尺有志男兒,豈甘為不善之徒,空生世上。就是從前既錯,悔亦難追,而今做過一日好人,猶得謂不甘於自棄。立定此意,囬頭是岸,決志不移。自後一洗前非,改惡為善。

	忽一日,雷電交作,將繼興震死於地,既而翻生。又數月,看戲於臺下,又雷聲至,繼興知打自己,呌衆人急避行開。話未完,果然震死;未幾回生,慌忙而歸。在鄉間教館,細心教導,苦志殷勤。又聞雷響之聲如大鼓震地,繼興恐怕第三次定必打死,斷難活矣。因走避入黑漆枱下。霹靂一聲,盡燒被鋪蚊帳,而繼興得生。張繼興心知劫數已過,仍復勤於修善,苦習文章。三年又中舉人,安享十年而死。張繼興常將自己之事,勸人肯直認不諱,話得久留人世者,改過之力也。

	陰間有十殿閻王,張繼興之前身,或十殿之一也。因有不謹,率意而行,判斷多差,受罰再生人世。假使繼興一向能不作惡,好事多為,其前程豈可限量。或做進士,或做翰林,亦未可知。至於打罵鬼婆,理之正者。而自己所行,諸多不正之處,誰敢向而罵之。繼興自己係秀才,只知罵人,不知罵自己矣。非但不知罵自己,並不知自己有過惡處也。然自己不知,而鬼神知之,而且記之。菩薩語你惡,似乎誣賴你;呌你自己寫出罪狀來,都算公道。兩個粉牌寫之不盡,生平之作事,勇於見惡必為,自認甘受雷誅。菩薩以為未足,要誅三次,方可抵其兇橫。

	嗟!嗟!人生在世幾十年間,好人唔做,偏做醜人,是何解也?殊不知,你舉拳頭以打人,雷公磨定斧頭以待你;你用毒心頭以謀人,雷公睜開眉頭以看你。任你做,任你暴,天地自然有分數。世事到頭終有報,天倉滿係掘頭路。觀張繼興之對兩神招認案也,此時無惡氣矣,而且低頭心息矣。若使既醒之後,依然不改,恃勢行兇,雷公必打死他,第二世要打,第三世又打,以滿三世雷誅之罰。可幸繼興能知既往之非,勇於為善,將功贖罪。菩薩亦鑒其心,初打一次死而復生,第二次又打不死,第三次打幸而免焉。非雷公怕漆器也,譬如父母打仔,其仔如果真知錯過,悔罪心誠,縮入牀底避之,父母亦有時忍住手而不打者。雷公能使山崩地裂,大樹破開,何況小小一張漆枱,斧頭不能用力麽?因見繼興有改過之心,知其誠切,故免其死。至於後來又能中舉,做過好世界,此是繼興從苦海跳出來尋上岸也。

	所以人要修行,修整爛船,修整爛命,肯修未嘗不好。如張繼興以閻王轉世,其命定必好過常人。無奈作得多惡,要受雷誅三次,其命可謂又爛到極矣。竟然不死,掩過時灾,以勇於為惡之心,變而為勇於為善,真算大英雄、大豪傑、大力量、大手段之人。比不同別人,既錯之後,將錯就錯,任由錯到底,拚作一鋪爛也。

\chapter{修整爛命}

	今人遇着抑鬱事、愁苦事,開口就怨自己唔好命,總之係前世唔修。此語亦是有理。人之富貴貧賤、妻財子祿、一好一醜,皆由前世帶來,命中注定。前世或善或惡,所以今世有吉有凶。既是前世唔修,今世要修;前世唔修,今世怨前世;今世唔修,後世又怨今世;世世唔修,世世贃得怨。不做一個變換法,怨氣終無了期。

	或曰:「今世去修都無益處咯!大約都係益及後世,無補於今生。」你如果真心實力,去修過半世、或十年八年毫無果報,然後話得無功。你並末曾修,而先先安定修善無益,亦自惰之見。或有做過\ruby{的}{󰦦}好事,而被別樣過失消除者,所以疑其無功也。因世人心意,只知計及生平之善,而於諸多錯處,每每忽畧而不覺者有之。又有等人,今日修行,明日即思報應;忽遇拂意之事,便自負我曾修善,而竟來意外浮灾。獨不思種禍種福,田漸而成。就如種荔枝,加以生泥,淋以肥水,要抉要植,加意栽培,勿使兒童扳其枝,勿使牛羊傷其蔃。待至根深蒂固,遲之久而發葉,遲之久而開枝。又四五年而出花,又一兩月而結子。一個二個荔枝仔掛滿樹枝,仍然未能食得。既結子,又要長核;既長核,又要長肉;既長肉,又要成熟。其外皮也,始而青,繼而黃,久而外皮紅。又要皮肉都紅,而荔枝好食矣,甜而香矣。

	假使既種荔枝之後,日日加泥,日日淋水,可謂着力栽培。而以欲速之心,望之太過,五月種荔枝,六月就想食荔枝;每日托一張竹椅,坐在樹脚下而嗟怨曰:「荔枝荔枝,我咁苦志加功來\index{作置}你,乜你都唔出幾個荔枝仔與我食吓呢?」殊不知時候未來,想求一個而不得。功果既熟,摘兩三籮而有餘。所謂果熟自甜,禾熟自實也。人有日日修善,而仍貧困者,而仍辛苦者,而仍衰微者,而仍諸多不合、所謀不遂者,何以解也?所謂荔枝種生,而未曾開枝發葉也。開枝發葉,而未曾開花結果也。果既結矣,滿一樹矣,只爭未熟未能食得到口也。故世有多積善功,而未見有好處者,福未來也,一來則世界翻新矣。有等不種荔枝之人,而笑種荔枝之人曰:「你將銀買荔枝種,點似得我將銀買酒殽?你請人挑泥培樹頭,點似得我請船去看戲?我贃得快活,你贃得勤勞。我見你種荔枝三年矣,而並無得食,真可笑也。」但有荔枝,勝過冇荔枝;現時冇得食,終須有得食。果熟之後,不止食一年,可食幾十年;不止自己有得食,連子孫都有得食呀。種果之事如此,種福之事可知。

	然種物亦有遲速之分,而種福豈無大小之別?有種福而得好衣祿,有種福而得好子孫,有種福而發一名秀才,有種福而發幾個進土,有種福而位極人臣,有種福而富堪敵國,有種福而子孫昌盛,有種福而世代綿長。善之量有深淺,福之報有厚薄,仍視其種之道為何如耳。如種莧菜,三十日收成;苦瓜,五十日收成;桃樹,兩年;欖樹,六年。為桁為桷者,十餘年之杉木也;為樑為柱者,數百年之格木也。故為善須有耐心,要有堅志。古來忠臣孝子,義夫節婦,做出頂天立地事業,扶持萬古綱常,其性何等咁真,其福何等咁大。姑無論別樣,即如廿四孝之中,王祥卧冰求鯉,官居太保,所出子孫發至九代公卿,東晉王氏皆其後也。董永賣身塟父,仙女成親,其親生子發至狀元宰相,漢董仲舒即其人也。所謂修福之事,不出戶庭而功圓行滿矣。然則話今世修善都無益者,未必然矣。

	前世今世,有兩樣解法:如未投胎之前謂之前世,既投胎之後謂之今世,此一說也;又有照太祖族譜傳來,派到我之本身為第九世,則我之父為第八世矣,此又以我之身為今世,以父之身為前世也。俗語云:前世唔修,今世折墮。故有時子孫冇福,由祖父唔修者亦有之。若子孫不修,後世更難為矣。亦有子孫不修而安享富貴者,祖父之厚福蔭之也,若子孫加修,而富貴不盡矣。總之,祖父之好醜,自己不能操其權;一身之行為,自己可能立其意。世人曉得修整爛船,修整爛屋,為何不肯修整爛命?修整時辰鐘,修整舊字畫,無關緊要之物,用心去修,至緊要個條命總不知修。勸佢修行,則答曰:「前世整定,駛乜修呀。」你既知到整定,可以不用求神拜佛;想望發財,不用改向修山營求發達。何以別事深信,而修善反多思疑?

	或有等曰:「我一世冇難為人,好之個天又唔庇佑我。」豈知天地生人,原為有用?生你出來利益人,唔係生你出來難為人。你難為人,人難為你,尚豈得成個世界。可知唔難為人是平常事,是本份事。若說冇難為人,就要天庇護你,則是與人同行,唔推人跌落水,就要人請你飲茶;入人舖戶,唔偷人貨物,就要人請你食飯;見人婦女,唔玷辱人身體,就要人請你上高樓。亦斷無此理。你定必要有恩惠於人,然後人家敬你愛你;亦如你必有善行於世,然後個天憐你念你。所以古來為忠為孝,長發其祥;積善積功,大昌厥後。若毫無行善,不過未有得罪於人,便怨天無報應,此亦愚之甚者也。同生斯世,你咁高我咁大,誰人容你得罪呀?獨不思牛狗有功於人,柴草有功於世,及至做人,竟有無益於人者有之,無益於鄉鄰者有之,無益於親戚者有之。更有不肯益及兄弟、益及父母者亦有之。至親骨肉,淡薄無情,又不如草木之益人為甚大,水火之益人為甚便也,非徒無益,而又害之。有難為父母,空養一場,費盡心血,而不見有孝順之事者;有難為兄弟,體貼多端,反作為仇,而敢做出相爭之事者;有難為鄉鄰梓里,以侵削為能,以欺凌為事者;有包庇子孫,容縱子姪,謀人害人,肥己潤家者。有時尚對人曰:「我一世在難為人,唔知難為多少矣。」此等之人,不修善而修惡,不積福而積禍;罪孽之氣滿身,凶灾之氣滿庭,尚寫無數咁多吉祥字句,五福臨門、天宮賜福、定福灶君。你知五福臨門,唔知有五禍臨門呀;你知天官賜福,唔知天官能賜禍呀;你知定福係灶君,唔知定禍亦係灶君呀。

	到了新年正月初一早,善人曰恭喜,惡人亦曰恭喜;孝順子曰事事如意,忤逆子亦曰事事如意。到了正月初二朝,你拈香燭上廟祈福,我亦拈香燭上廟祈福。廟堂之上,擠擁紛紛,逼身唔轉。你叩頭,我拜跪;你密稟,我高聲;你拋得陰杯,我跌得陽杯。總之不能合意,務以賜到勝杯為止。若是杯多反覆,許了這樣、又許那樣,認了不淨、又認不該,務須要菩薩應承庇佑,跌下勝杯,然後安心樂意燒元寶,發脚而行,以為一年好景矣。論起廟堂,住其鄉、祭其社,拜神之道,禮所應宜。至於降福消灾,非叩求所能得。葢人之禍福,由於善惡轉移。玉皇太帝主其權,而廟堂諸神,頒行佈令者也。譬如一舖之中掌櫃先生,出入錢銀皆經佢手,必要論錢交貨;即使或賒或借,必要填還。唔通我與掌櫃相熟,買貨總不用交錢?共佢講幾句知心,隨便可以拈貨出門,不須計數?斷唔得也。何也?重有東家做主意,要承東家命,而為東家所管也。廟堂菩薩至上重有玉帝操賞罰之權,看其為善者,降之百祥。若不修善,只向廟堂之內拜多幾吓,奠多幾杯,跪多幾叩,以為菩薩領我盛情盛意,定然有感皆通。唔通奉承掌櫃先生,即將貨物分送你嗎?可知福澤不是輕求,罪孽亦非輕赦。

	惟是世俗婦人,又有燒赦書之法。話半生所作,不論奸淫邪盜、忤逆貪婪、十惡等罪,若肯出十個錢,買得一張赦書,逢天赦日,具香燭酒果,去廟堂之內,請個喃魔先生喃唱幾句,稟過菩薩知之。抑或惜錢,不用喃魔,自己跪稟亦得。或在當天燒化,亦無不可。既燒之後,菩薩執此赦書上奏天庭,玉皇大帝見了赦文,即要凜遵,無敢延悞,去查簿上,將犯罪之人所有罪惡,一切勾消,即使孽重如山,無容計較。自後若有再犯,另注新簿作起個頭。年積一年,月積一月,自知罪將及滿,又擇一個天赦日,燒一張赦書,前罪皆消滅,其計如此耶。照此等滅罪之法,亦不甚難,而且甚便。人人依此而燒,地獄可以不設。吾恐此理未必然也。天律極嚴,天心極厚,如果真心悔悟,未必不許以方便之門。必須實力而行,將功贖罪,非僅徒燒一紙,便可消除。亦猶欠人債項,講句好話,認句唔該,許以陸續還償,不敢再多積欠。銀主念其心切,未嘗不寬恕三分。若一片虛浮,毫無實迹,只寫幾封信札,乞減求情,唔通就因你篇信,來將你欠數盡行勾免?是知赦書之燒,寄信求情也;改過修善,還銀結債也。銀債如此,冤債可知。

	然兩債雖同,亦有分別。銀債之欠,有錢則還,無錢則罷;窮到極處,銀主亦不問焉。講到冤債,不論你有錢無錢矣。銀債之欠,逼人之女,未必逼人之妻;冤債之欠,禍你之身,而并禍你之子。銀債之欠,住在此坊,居在此鎮,可以追問;若往了別州,逃之別省,難追問矣。至於冤債牽纏,任你去到九州十八省、番鬼國、天根脚,依然隨身而行,隨處而在。銀債之欠有時,問其父而不問其子,追其祖而不追其孫。至於冤債,父不結,而子必清還;祖欠多,而孫猶累足。銀債之欠,問於生前而不問之死後;至於冤債,做鬼難逃墮落,轉世尚要填償。是冤債之累人,甚於銀債之苦也。故必還銀債,而欠數可清;亦必消冤債,而福氣方聚。

	故修整爛命,又如醫治爛脚,必要謹慎行藏,忍痛調理。去腐者,改過之功也;生肌者,修福之漸也。外用湯水洗之,藥散敷之,內用好飲好食之物以養之,補氣補血之藥以生之;而毒氣清矣,瘡口平矣,皮紅而肉滿矣,行快而能走矣。曉醫病之法以醫命,依然好\index{脚色}矣。

\chapter{骨肉試真情}

	番邑黃從善堂敬刊

	香山縣有一人姓明,兩兄弟,兄名克德,弟名俊德。父母先亡,遺下家產值數千金。克德娶妻淩氏,識情達理,女中之君子也,上能敬夫,下能愛叔。俊德十七八歲,尚未成婚,在家管理耕種。

	克德相交兩個朋友,一個姓錢,一個姓趙。兩人不是正經人物,本係無賴之徒,到來一味奉承,想貪飲食。克德又唔明白,以姓錢為知心,以姓趙為知己。克德心盲,又遇瞳人反𣅜,所以唔望得真自己,又唔望得真人錢趙兩人得意,遇時\footnote{
  係「經常」、「日」:
  \begin{itemize}[itemsep=0pt, parsep=0pt]
    \item 二成生得兩個仔,臧姑遇時自己贊好命。
    \item 我前日買定一張單刀,放在床頭,遇時預備要用佢,若真來尋打,就先下手為強,免至受虧一著。
  \end{itemize}
  }講三都七國本事非凡。克德本來唔好性情,遇人得罪佢,就一肚火氣。錢趙不為潑水,反去添油,話:「駛乜怕佢呀!有咁\index{丟駕},就打佢奈乜何?就告佢亦易事!」姓錢話:「兵房師爺係我姐夫。」姓趙話:「三班縂頭係我老契。」克德拍掌喜曰:「有咁樣人事,隨便車天。」滿斟一杯勸姓錢曰:「好手足。」又斟一杯勸姓趙曰:「好兄弟。」三人暢飲,劈口高歌,或猜拳,或大笑。克德大聲曰:「喊我細佬來,快\ruby{的}{󰦦}趕注炙燒酒。殺雞唔得及,將廿只鴨蛋打破,濕半斤蝦米,切十両臘肉絲,發猛火,洗鍋仔,快\ruby{的}{󰦦}炒熟來!」

	誰不知俊德見個樣情形,聽此等說話,心內帶幾分唔中意;又惱錢趙二人常來攪擾,俱是無益之談,漸漸生出怒氣。有時錢趙二人來探,值克德不在家,俊德不甚招接,錢趙二人知其憎厭。一日與克德飲酒時,姓錢帶笑開言曰:「老明,你地出來處世,真第一等人。與朋友交,疏財大義,可謂慷慨英雄。」克德曰:「好話咯,不敢當。」姓趙曰:「在你無可彈,但係你令弟,與你性情爭得遠。佢待我亦唔醜,見佢待你太冷淡無情。論起番來,長哥當父,對亞哥唔恭敬,未免都不合理。」克德曰:「唔知點樣解?我又冇罵佢,又冇打佢,就見了我好似唔中意。個\index{龜蛋}想起來真可惡咯。」漸漸火起咯。姓錢曰:「睇佢心事,好似思疑你做亞哥瞞騙于佢。」克德曰:「有點瞞騙佢呢?不過有好朋友來,姓錢共姓趙飲多\ruby{的}{󰦦}食多\ruby{的}{󰦦},咁樣之嗎?」姓錢曰:「佢唔係思疑你個\ruby{的}{󰦦},必定思疑你吞騙錢財,慌你舂了落荷包,何樣是真。」姓錢咁伶俐克德曰:「我个心如青天白日,誰知墨咁黑朋友所知呀。」姓趙曰:「朋友尽知,縂係你令弟唔知。」克德曰:「難咯,難咯!有時話朋友好過兄弟,正為此也。」遲吓你就知。錢趙兩人勸曰:「老明你莫激氣。細佬唔明白,務宜忍住个肚,不可怒出外面。好勸諫講起來似乎離間你兄弟,都唔係似乎,分明便係真正不過蒙你過愛,即管講句,不是即管講,其實尽力講以知你委曲耳。」錢趙之心重更曲克德自從聽過兩人之言,心中漸漸不同,作細佬如仇人一樣,一語不合就罵,一事不合就打。

	一夕睡在床中,淩氏諫曰:「翁姑生你兩人,兄弟之親能有幾個呢?為何一見細佬,就憎得咁凄凉?唔通骨肉之情,不如朋友?你知厚待朋友,何以萡待同胞?是愛疏而不愛親,顧外而不顧內也。」此張枕頭狀原甚少見,又好呈詞,理應批準為是。明克德曰:「莫講、莫講,個\ruby{的}{󰦦}\index{脚色}不中用,唔做得料駛。」批出不準凌氏曰:「細佬唔中用,你\ruby{的}{󰦦}朋友好中用麼?」再入紙克德曰:「我\ruby{的}{󰦦}朋友,唔係嘻嘻。聽錢趙兩友講起來,可以落水舂牆,替生替死,與我細佬爭得遠咯!」凌氏曰:「替死之事,都要試過方知。以我心意,朋友要交,兄弟要愛。睇你\ruby{的}{󰦦}友,都係貪你飲貪你食,重怕拖你落水?都唔定也。」此婦人乜咁本事,能料得咁透。克德曰:「你女人家,曉得乜東西?只曉得買好油搽䯻。男人大丈夫,有乜聽老婆說話呢!唔听,你有錯我自有主見,你不得多言。」淩氏歎曰:「別个婦人向枕上造是非,故意想離人骨肉,人家做男子尚肯听從。惟我勸你愛自己細佬,你做老公,唔謊信我一句。嫁得你咁硬頸,有乜法子呢!」克德曰:「細佬無好處,我就唔愛佢。你共佢實久好麼?」淩氏歎氣一聲,默然無語。克德遂將細佬趕逐出門。俊德走往鄰村酒米舖,做火頭棲住。錢趙兩人,自後更無忌憚,三日來一輪,五日來一次,捉狗仔,切魚生,彈琵琶,吹鴉片,嫖賭飲蕩,練得周身引,好似大花筒。相與個\ruby{的}{󰦦}邪朋匪友,練做\index{敗家精}規模淩氏泣諫不從,付之長歎。

	一晚,克德在祠堂飲酒歸,形容半醉。淩氏在門邊等候,以手指之曰:「你止曉得日日醉。」克德曰:「唔醉有乜事呀?」淩氏曰:「你話有乜事,就有事過你理?」克德怒氣入房,橫眠床上。淩氏附耳細語低聲曰:「如今後花园殺死一人,棄屍在地,你尚睡得咁安樂麼?」克德聞言大驚,如冷水澆背,面色發青,即拍床起曰:「殺死誰人?」淩氏曰:「不知。」問誰人所殺,淩氏曰:「不知。」克德曰:「快引我去看。」跟隨淩氏跑入园中。時值点燈之候,夜色微晦。果有一人眠在地上,頭面難認,但見所着白褲,血色淋淋。克德一向胆少,惟飲酒量大一見赫得魂飛,搖頭歎曰:「該衰咯!該衰咯!不知那個想來攞我命咯?」淩氏曰:「唔知你與誰人結怨,故此移屍嫁禍,想來累你身家。」克德曰:「有乜折法呢?」凌氏曰:「趁今未有人知,快將屍骸埋沒,可保無事。」克德曰:「我去呌士工來。」凌氏曰:「士工未可輕信,將來恐有洩漏,借禍生端,受累不淺。此事惟有心腹人方可信托。」克德喜曰:「有計,有計。」即點爝灯籠,先到趙友處。

	趙友聞知,請入坐下。趙友笑曰:「咁夜到來有乜好意?」克德執住趙友手,出門外細聲說:「今晚因係咁樣如此之事,想求你幫一臂力,埋沒屍骸。」誰知趙友忽聞此言,心中暗想:「此事所關人命,後來有人告發,白白贃得打死。」你曾經話可以替死呀遂對克德曰:「老明,你待我都算好咯,唔講咁樣事,就係替死,弟輩可以做得來。怕未必但我一生至怕見死佬,就係講起來,聞之都怕。獨不怕狗肉魚生前者自己父母去世,都係請士工執拾,唔敢到棺材邊望一吓。好孝子你如今講過,重有好久可慌。老錢大胆,你去請老錢惟真。」老錢係真,唔通你尽假了?

	克德又去姓錢之處,急拍開門。錢友曰:「乜咁慌忙,有何貴幹?坐坐坐。」克德曰:「我唔得閒坐,共你斟酌一句。」錢友曰:「有乜好斟酌?必定係好頭路。」克德遂攜錢友在密處,以花园死佬之事說知。錢友聞言吐出舌曰:「那個咁陰毒,製單咁樣貨來累人?真正冇本心咯!」克德曰:「老趙不肯來,我想求你如此如此。」錢友想起:「人命關天,終須告發。老趙不肯做,我有咁\index{蠢才}?」遂對克德曰:「老明,我唔怕死佬,我作佢冬瓜咁轆都做得!但係撞板,今日發大熱氣,周身唔自在,都冇食飯!現在想呌老婆刮一身痧,点能替你做得呢?」克德曰:「求你委曲吓。」你慌佢將耒唔委曲你麽?錢友曰:「我共你有乜第二句呢?你從前呌我飲呌我食,我都冇乜推辭,何況舉手之勞,成乜說話呢?我都唔共你坐,要歸床睡,養吓精神罷咯。」克德遂心麻意亂,垂頭喪氣而帰。又被風吹息灯籠,踢崩脚趾,幾乎跌落深磡之下。險些執住個条樹蔃,扒進上來。

	歸到家,淩氏問曰:「兩個朋友來了麼?」克德惱氣曰:「豈有此理!一個話唔見得死佬,一個話發大熱氣,總之係一片虛。」淩氏曰:「去呌二叔歸來,或者可能幫手。」克德曰:「冇錯冇錯,果然高見不差。」個陣要信老婆說話咯。即用碎布札住脚趾,又点灯籠而去。拍開門入,東家曰:「夜深呌令弟,有乜緊事嗎?」克德曰:「佢大嫂肚痛,呌佢去执藥。」東家話:「要咯,唔係要兄弟做乜呢?」跟出門去,隨路隨問曰:「亞哥,現今大嫂痛得好凄凉麼?」克德曰:「唔係、唔係,因花园中有如此如此,要你帰家同了此事。」俊德曰:「應份要,應份要。」曉講應份兩字就曉得天倫回到屋,凌氏用蓑衣夾大蓆包,捲好\index{周至},兩兄弟用竹棍抬起,并攜一張鍬、一鉄鋤。不動聲色,轉過後岡,直到山脚幽僻之處、水邊濕地。發勢尽力掘了三尺深,將屍埋葬,用脚踏平。兄弟歸來而睡。

	克德睡在床上,心頭仍跳高跳低,不勝驚恐。凌氏曰:「夜靜更深,料得無人知覺,可以無妨。」克德曰:「千保萬保,無人知到。」凌氏曰:「你話錢趙两友可以替死,今竟何如?」反案咯克德曰:「不消提,悔之無及。」凌氏曰:「你話細佬唔做得料使?大約勝過他人。」克德曰:「患難見真情,此言不錯。古人云:打虎不離親兄弟,上陣不離父子兵,果然真事。」凌氏曰:「我地女人个隻䯻,值得好油搽否?」克德不覺笑起來,答曰:「不止搽油,戴枝金釵都值。就係繡条大紅裙、聯件花衫袖過你着,你都無愧咯。但係世上婦人,只曉插花搽粉,裝整風情,縂想外人睇佢,話佢好樣,話佢光鮮。点似得你曉得天倫,勸人骨肉和好呢。一向我唔知你咁明白、曉睇相,識出我两個朋友唔中用,算你非凡。」凌氏曰:「朋友相交,未嘗不設飲食,亦唔係專以飲食為題。當飲食時,講得了不得咁知心,唔通冇飯食就水咁淡?觀佢形容,整聲色、講惡氣,如敗水亞瓜、新出匪類,此等將來斷無好結果。實在我慌佢引壞你,負累你,害到你不成人,所以憂到今時,無一日安樂。你試想吓,你自從共佢两個相與,便粗飲大食,不計錢財。遇有\ruby{的}{󰦦}景致,两個就來。這個話請定船,那個話灣定艇;你就神情跳扎,催捲睡鋪行李,好似要即刻開行。或五日不囘,或十日不返。就係睇過快活,又点樣生肉呢?更有時昏咁嫖,昏咁賭,不知所以,大鬧烟花。你試想吓,近兩年間,混混鬧鬧,去了多少錢財?唔通你都冇想吓,你藉先人之福,當日翁姑唔知幾多辛苦,費尽幾多心血,一生勤儉,然後積此資財,望你兄弟守成,為子孫長久之用。今者無端破散,豈能對父母於九泉?并不能對得細佬住呀!你從前頗知謹慎,縂係自相與。此两個\index{攪屎棍}撥馬尾,致到你顛倒得咁凄涼。」凌氏講完,克德搖頭歎氣曰:「唔駛講咯,縂係錯咯!如今明白咯!個吓唔作興佢咯。」

	睡到天光起身後,見並無生事,凌氏殺雞買肉,向家內香火酬神,兄弟、叔嫂、夫妻三人同飲暢敘。明克德謂弟曰:「天灾橫禍,意外生端,可幸無人知覺,消除大難。藉先公先祖之靈,從今以後,賢弟不用出外僱工,只可帰來耕種。愚兄尽知從前錯處,賢弟不用執怪,另敦友愛之情可也。」俊德答曰:「弟自不賢,非兄之過。至囘家耕種,弟當盡力而為。」俊德推辞東家歸來,如金似玉,一飲一食,兄弟同歡。弟敬其兄,兄爱其弟,凌氏開顏含笑,尽解愁懷。

	又說趙友一日到來,笑容請曰:「老明,近來好世界呀?」克德無心答之,曰「坐呀,飲茶呀,食烟呀」,縂不起身迎接。一息間,吩咐趙友曰:「你坐住,我要去淋菜。」趙友見冇趣味,抽身而去。遲數日錢友亦來,克德亦無心應接,識破唔值一個爛桔錢友亦去。

	一日趙友往市上,剛遇錢友,先以手招之曰:「來來,同去茶店飲茶。」入店坐下,趙友声細声告曰:「老明個人,真正唔過相與。我前日去探佢,冷冷淡淡,因從前熱過頭,今要冷;從前鹹過頭,今要淡。無情無義,冷水都唔打牙,前者咬得多咯!食豬脚雞骨,牙都崩咬到痛咯!極之冇引咯!」姓錢曰:「我前日去探佢亦係如此。大早知此人,淺才萡行,反骨無情。實係罵自己但念一向相好而來,唔通就反面麼?大約因個晚之事,嫌我两個唔去幫手,故此埋怨。本心之講,事關人命,連累非輕,非比同狗肉魚生,就幫吓手,都贃得\ruby{的}{󰦦}食呀!個死佬見過都衰,有乜咁\index{蠢才},捉虱上頭壳養呢?」姓趙曰:「我亦為此之故,所以即刻推辭。佢尚唔知利害,實在佢有條人命案在我两個手來。我两個若容忍他,佢便有碗安樂飯食;若係唔顧舊相與,我要佢鹹豆都唔食得一粒。」錢友曰:「到是真咯。遲數日兩個去探過佢,若係恭恭敬敬,有\ruby{的}{󰦦}禮貌便了;若仍然冷淡,要整佢色水\footnote{󱛖󱀱通常都係指貴價金屬或玉器󱝚色澤。󱪙《俗話傾談》裏󰊺「色水」似乎有兩個意思,較常用󱝚意思󱪙指女性\ruby{的}{󰦦}外表樣貌,例如:
  \begin{itemize}[itemsep=0pt, parsep=0pt]
  \item 我當初做新婦時,重好色水過你十倍,唔估今日老得個樣醜態,減去三分。
  \item 唔通六七十歲老大婆重整作咁好色水麼?
  \end{itemize}
  另一個意思󰳞,係指贓害別人󱝚把戲,或類近今日粵語「畀顏色佢睇」󱝚「顏色」,󱛖󱩑粵語唯一相關󱝚用法「整色整水」:
  \begin{itemize}[itemsep=0pt, parsep=0pt]
  \item   遲數日兩個去探過佢,若係恭恭敬敬,有\ruby{的}{󰦦}禮貌便了,若仍然冷淡,要整佢色水開井水過人食。
  \end{itemize}
  },開井水過人食都係好。」就立定這樣主意。

	遲數日,錢趙二人又來探咯。克德隨隨便便,不甚着意,呌聲坐,呌聲飲茶,呌聲食烟,仍用手指打算盤,拈筆抄數簿。两人亦見無味,辞別而去。出到村外,錢謂趙曰:「人之無良,一至於此。豈有此理!好友到來,點樣好法呢?總不加意。我聞人之將衰,其心先乱,又係罵自己,冇本心人偏曉講好道理。明克德其將衰咯!不告此人,無以洩其忿。但係告人斃命,先要尋着屍骸,方為有據。」趙友曰:「確有主見,唔怪得三家村請你做師爺咯。」錢友曰:「你唔駛笑我。我雖然係矮細,一肚計隨便駛。老明衰夾滯,不久有好戲過佢睇。」有一本反骨戲做出來。

	約於第三日,兩人戴了白草帽,拈一張熟鉄鍬,隨岡尋訪。舊墳不必看,即有新高凸起,亦不必疑。何也?以夜靜不暇加泥也。遇新墳太短少者,知其不是。何也?料得係死仔窟也。一連尋了三四日,不見真迹。思起來此處原無河海,安能放去漂流?再尋一日,尋至山脚幽僻之近水濕地,見一幅新痕,平漫無堆,心疑此中有物。訪問掌牛仔曰:「此處新痕,何時方有?」掌牛仔曰:「一向俱無,近於某日初見。」問係誰家所葬,掌牛仔曰:「此卑濕地,誰人肯葬此呀?並不加泥,又不掛紙,如平地一樣,實在古怪離奇。」再問郊野之人,並無一人知其消息,皆笑曰:「鬼葬此麼?你咁\index{廢物}。」两人曰:「係咯,斷無差咯!」遂用鍬探到三尺,果見蓆包等物,內軟如綿,知到\footnote{
  係「真󱝚」、「真係」、「堅」、「堅係」,同而家粵語󱝚「真係」。「真正」󱪙今粵語已叫罕用,但󱪙 60 年代粵語電影󱄵聽󰧱,󱪙同年代󱝚《嬉笑集》裏邊都見到,譬如《真正惡做》,可知󰳞詞消失係近年󱝚事。󱪙《俗話傾談》裏󰊺「真正」不時同「係」連用,似乎係「真正」過渡至今日「真係」󱝚中間階段。
}係個單貨。錢友拍手喜曰:「得食咯!有八寶出咯!个吓重唔收拾你!」两人欣欣然。

	又一番斟酌,尋得一個乞兒,年十七八。錢友曰:「細佬哥,恭喜呀!」乞兒曰:「遇時抵肚餓,至到乞食,有乜喜處?」姓錢曰:「睇你個相,光氣滿顏,財氣到矣。遇光棍來,晦氣到是真。我有一条發財門路,想舉薦你,你肯從我唔從呢?」乞兒笑,喜曰:「點樣發財呀?敢望攜帶吓。」至好咯姓錢曰:「現有一個財主佬,謀死一個客商,現今想去告佢,但無人做苦主。你肯認失了亞叔,我两人與你做證。佢怕償命,要與你講和,必以銀賠補你。你个陣劈大个口,唔怕話要多;打開個蓆綹裝銀,不是裝飯。細佬哥,个陣拋了個隻砵頭,買\ruby{的}{󰦦}好衣裳,裝得周身輝,去歸買屋,娶老婆做財主,都係哩条門路咯!」乞兒又笑曰:「你算想得耒,講得有紋路。好係好,但係我有亞叔做死佬。」姓趙曰:「\index{蠢才}!乜你咁愚直呀!唔駛要有,白認便得咯。況且有我两人當頭,天大事情自有担帶,個\ruby{的}{󰦦}唔駛你憂。你整便兜肚裝銀,都做得咯。我唔係騙你,我两個都係撈世界,想錢入荷包,但無你不成,無我不就。我今與你非比他人,猶如拍手伙計而已。」乞兒信以為然,竟從其意。

	姓錢代乞兒做狀辞一張,告明克德挾仇殺其叔,錢趙两人做證。官發票出差,捉了明克德。克德魂飛天外,胆戰心驚。被好友拖了落水。香山知縣親來驗屍,要開棺看過。縣官來到山脚,坐住馬鞍,審問山鄰人等,俱說不知。淩氏走到官前,跪住叩頭稟曰:「小婦人之丈夫係明克德,一向在家耕種,守分安良,並無殺人之事,求太爺釋放,免受含冤。」官曰:「現有苦主在旁,證人在側,新墳可據,何得糊塗?」淩氏曰:「我家不過殺死一隻大\index{狗牯}\footnote{狗牯:係「公狗」。「狗牯」󱪙粵西、粵北都比較普遍多講。《俗話傾談》作者邵彬儒󱝚家鄉係四會,由此可見顯示四會話󱪙滲透󱃡入。},抬去埋葬,埋狗亦古人之事。若話假局,開棺自見分明。」官即命仵作撿起屍來,竟然一隻大狗。大雲鼎堡做頭壳,身穿一件綿衲,着一条白布褲,又加無數青磚,同包蓆裏。官曰:「既是狗死,為何這樣裝傷?」凌氏曰:「太爺有所不知。所因丈夫與錢趙二人為友,此二人係茶朋酒友,無賴之徒,引我丈夫賭蕩花消,離間我丈夫骨肉。小婦人遇時向丈夫勸諫,無奈丈夫不信,作两人如泰山可倚,可以同苦同甘。厭棄細佬,如路人一樣,趕逐出門。小婦無計可施,遂將大狗殺死,办作人形,值丈夫半醉歸來,朦朧夜當近黑,引丈夫去後園一看。丈夫胆小,一見就以為真,疑移屍嫁禍所為,必要將屍埋沒。素稱心腹,莫如錢趙两人。丈夫走去請他,脚迹不到。夫轉呌弟,我叔叔即走囘來,同心做事。丈夫識此两人係假局,信弟真情。此两人見似生踈,借端告發。望太爺治其好惡,勿使做漏網之魚。」

	官問錢趙曰:「你兩人說與明克德為友,素稱知己,為何反面操戈?」錢趙曰:「我兩人與他唔係点樣深交,不過因事相逢,也有半面之識。」克德指之曰:「我與你豈止相識!你來探我,魚鱗約有一籮,雞毛不止一担,飲尽多少,借去錢財,尚話不是深交,真真豈有此理!」官曰:「明克德不作你是心腹,未必呌你夜深共事。可知平日親密一定無疑。既不肯患難幫扶,為何將他控告?此中奸計,必有一段原由,若不肯講出來,即將乱棍打死。」两人仍不肯認。官喝差役曰:「拿夾棍來!」两人嚇得一額汗,姓錢推姓趙先講,姓趙推姓錢開聲。官喝曰:「打!」差役想動手,两人伏地乱叩頭。姓錢曰:「小\ruby{的}{󰦦}願講咯!」遂稟曰:「我两人近日往探克德,因他冷淡,是以挾仇,生端誣告。自知不是,望太爺大赦從寬。」官冷笑曰:「小人心術,古怪無情。有飲食而親,無飲食而怨,只知顧口,不顧良心。律有如虛反坐之条,理應將你两人重办。即管格外開恩,留你生路。」喝差役將他两人每个打二百大板,二百小板,二百藤鞭,打得两人皮開肉裂,血汗交流,呌苦不絕聲,手乱搖,脚乱振。打完,橫轆直轆,尚難起得身。官吩咐曰:「將錢趙两人發出頭門,枷號五個月釋放。」

	官又審乞兒曰:「呢个乞食仔,你話失了亞叔;个隻大\index{狗牯},就係你亞叔呢?」乞兒曰:「我本來冇亞叔,佢两个教我認有亞叔,又呌我到公堂要詐啼哭。」官曰:「佢呌你死,唔通你都去死麼?你都係唔好人,要重責!姑念你年輕,被人串弄,即管減刑一半,打一百大板,一百小板,一百藤鞭。」乞兒叩頭曰:「太爺呀,唔好打咁多,些少好咯!」官曰:「不用多言,照數打去!」打得乞兒魂不附體,哭到失聲。打完,又發出頭門枷號五個月。審完,官贊歎凌氏曰:「你呢個婦人,算你十分賢德,能出妙計化服丈夫,和好兄弟,是天地間第一个奇人。本縣今日賞銀二十員與你歸家買酒肉,與親戚鄉鄰多杯暢飲。以勸世間之為婦道者,學你咁賢良也。」話完,明克德夫妻叩頭領謝而去。

	又說乞兒在頭門怨錢趙曰:「你两個真正好舉薦,好發財門路。製个板豆腐,打得我死過翻生,真唔抵咯!」姓錢曰:「你唔抵,我两個實久抵麼?你做苦主,我两個做証人,我两個重打得多過你,講乜難為呢!」乞兒臼:「你今被打,從前贃得飽贃得醉呀!惟我認苦主,白白受苦一場便了。」姓趙曰:「老錢應承做師爺,你怨佢便冇錯。」姓錢曰:「㗇㗇,真正想不到咯!此婦人有咁深沉好計智,出我意外,幾乎条命喪佢手來,不死萬幸咯!」

	錢趙两人滿罪之後,人人都憎佢厭佢,忌佢怕佢,無一个人共佢相與,無一人請佢飲食。未幾两人大病,之後妻子死完,乞食十年,两人同餓死。明克德自此事之後,深服妻有見識,每事與他斟酌,言聽計從。淩氏所生子孫,俱成大富。道光初年,其子有在廣州十三行開洋貨舖者,發十餘萬金,皆淩氏之福也。

\chapter{\index{潑婦}}

	乾隆間浙江溫州府,有一農家,姓齊名仲良,衣食飽暖。生二子,長名思賢,次名思德。其大子思賢也,生得聰明伶俐,出外做生理。娶妻慎氏,頗有姿色,思賢愛之。慎氏百計逢迎,妖容媚態,狐狸精作怪加以三寸之舌,說話尖新,思賢作為掌上珍珠,言無不聽。每次歸家,將所帶錢財,交一半與妻,交一半與父母,妻大歡喜。

	一夕枕邊談及,對思賢曰:「自己算好命,嫁得好老公,自己亦十分心足。我冇乜好慌,至慌你死。你若死了,我都唔嫁,斷斷唔輕易尋翻個咁好老公咯!」你駛慌有麽?思賢笑曰:「到是真咯!唔講你唔嫁,就係你死我都唔娶。好義氣夫妻。不憂無老婆,難得你咁好心事呀!」慎氏曰:「我不嫁則易,你不娶則難。有翁姑在堂,不由你做主也。」思賢曰:「你若死了,我縂不歸家,父母亦難相强。」慎氏曰:「你唔肯帰家可以做得,怕你係講假話呢!」思賢曰:「我作你乜樣人呀!對父母亦有講假話,唔通對你都有講假話麼?本心之講,幼時要父母,長大要老婆。如今父母隨隨便便,可有可無;若係老婆,一日不可少矣。歸來不見你面,食飯唔安。」慎氏曰:「我亦話夫妻親過父母。」思賢曰:「你見得透,我亦不差。」自是夫妻之情如膠似漆。

	孟子云:「人少則慕父母,知好色則慕少艾。」今齊思賢之愛妻,愛其有色也。慎氏之愛丈夫,愛其有錢也。夫妻不明大道理,以父母為厭棄之物,兩個都是忘恩負義之人。所謂「你不嫁」「我不娶」,只是癡情習成昏性,非真義夫節婦,扶植綱常。假使慎氏忽然鼻上生瘡,柑橙咁大;眼睛凸出,腫似田螺,貌之好者變而丑焉,吾恐思賢必憎之厭之,斷不與以錢財,而欲其速死者有矣。假使思賢忽然跌折脚而不能行,跌折手而不能動,囊之豐者變而空焉,吾恐慎氏必萬怨千嗟,斷不事以小心,而自惱嫁錯者有矣。可知:愛丈夫,當在貧難而易見;愛老婆,不嫌醜貌而後真也。慎氏見夫如此作愛,遂恃起來。所得錢財,置衣裳打首飾。今日請人去拜神,明日探親去看景。肆無忌憚,自作自為。翁姑雖有勸諫之言,慎氏縂置之不聽。一次齊思賢歸家,其母告之曰:「父母家貧,望你照顧。家中人情世事,柴米油塩,日用支需,皆為切要。你有餘銀,何不交與父親,代你買田置地。何必多與你妻浪費,習慣奢華。」思賢縂不答聲,無言而去。

	歸對妻曰:「老母呌我唔好交銀與你,話你粗駛大用,不知你点樣撒潑呢?」慎氏聞言,就罵幾十聲:「\index{老狗乸},多言多語,造是造非!」通夜詐哭含愁。思賢幾番勸止,安慰之曰:「我唔係信老母說話,不過照樣學過你知,何在咁怪我呢?」慎氏曰:「你估我用個\ruby{的}{󰦦}錢文,真正冇想像麼?狗醜主人羞,唔打办吓光輝,人話齊思賢老婆衣衫襤樓,失禮到你呀!所以遇時拜神拜佛,無非見自己命鄙,歸到你門两年,未有所出,都係想菩薩庇佑,早日生個花仔。得到三十七八歲時,娶個新婦,學翻你咁好你做家公,我做家婆,有仔有孫,慢慢享福。不可先折福人家就話你好命咯!唔通等到五六十歲,生仔扒向棺材頭麼?你做男人,曉得發財,唔慌有个\ruby{的}{󰦦}想像吓咯!」思賢笑曰:「睇你唔出咁深沉,咁好計算呢!唔怪得人家呌你做伶俐三姑,果然不錯。」

	夫亦錯,妻亦錯,兩個都錯。老婆裝錯,老公睇錯,何也?婦人之意,只想丈夫專愛自己,又恐丈夫聽父母話而有分心,於是枕上桃言,輕試丈夫心事。如果丈夫以父母為重,不容說話多端,个張枕頭狀不行,不得不要依從丈夫而順翁姑之意。若是丈夫以老婆為重,話一句就信一句,連丈夫都派父母不是,知其入信之深;再催紙幾張,又蒙批準,而枕頭之案定矣。此後心中有胆,做事無拘,翁姑向丈夫雖有投詞,而我之密稟先一着矣。作翁姑如閒人亦可,作翁姑如仇人亦無不可。何也?丈夫深信到底而不疑也,此所謂裝錯也。何謂認錯?身為男子,豈不知生我養我,父母恩德如天。而自老婆歸來,言笑之間,服事之際,嬌容媚態,細語低声,其情趣與父母大不相同,其心意與父母又爭得遠。我所欲者,而妻能順之;我所悶者,而妻能解之。若父母不合意,只曉得怒我罵我,直直白白冇\ruby{的}{󰦦}隱藏,對人前去我駕,話我唔中用。又不如老婆之委曲殷勤,為真愛我切也,此所謂睇錯也。裝錯一道,婦人入手工夫,必用此法。認錯一道,男子順妻逆母,必係此心。然有等婦人,初愛丈夫、順丈夫、敬丈夫,後至治丈夫、罵丈夫、而惡過丈夫者,何也?皆由容縱日久,不知婦道;為男子者又夫綱不振,自失其權,被老婆睇透你唔中用,唔起得乜飛脚,唔奈得佢乜何也。又有等妻,非美貌,又欠精靈,不過平平常常,並無好處,而男子極怕此老婆,而不怕父母者,何也?所謂陽明之氣不生,而陰濁之氣太盛也。此等說話,不過為下一等者言之,世上無數咁多賢婦人、奇男子不在此內。

	齊思賢既囬舖,慎氏又自恃非凡,看翁姑不在眼內。一日,其叔齊思德來勸諫慎氏,先呌一声:「大嫂,我亞哥在外做生意,好辛苦然後贃得个錢,你咁樣驕奢,未免過分。況且我父母一生勤儉,你好閒遊,豈成婦這?都要謹守閨門方好。」慎氏曰。「你話我唔謹守,我晚晚打開門睡麼?你父母自取勤儉,誰一個唔許佢閒坐?誰一個唔許佢粗駛呀?你亞哥辛苦,好之帰來唔見佢講一句。我用自己錢,關你乜事?我嫁得好老公,享用係我之福。你唔識意趣,理女人閒事,問你醜唔醜?」思德曰:「亞哥係我父母所生,非你所出也。養兒待老,我父母未能享福,你就鬧咁排場。」慎氏曰:「你父母好出奇麼?你家中得銀來用,不過因我益到你。你亞哥話過咯,我若死了,你亞哥永遠不帰,要你一家都無倚望。」思德曰:「你莫講咁聲色,唔通你死了,我亞哥咁就縂冇老婆嗎?」慎氏勃然大怒,曰:「你話唔信,我就死過你睇吓!」思德曰:「我唔係逼你死,我以好言勸你,亦是平常。你丈夫親過我,你唔掛念丈夫,你死即管死,關我乜事呀?」話完即出。是晚慎氏帰房,唔思想自己錯處,只話我死了,便可以悞佢一家。半夜之間,懸樑自尽。

	論起慎氏,大不宜死。有丈夫寵愛你、\index{作置}你,如果遵循規矩、勤儉持家,翁姑必歡喜你,一家都贊歎你,做人何等快活!乃不能修婦道,一味撒潑,一味刁蠻,此等行為,又要應死。死之之法,莫慘於殺頭,其次問絞。今慎氏忤逆到極,誰敢打佢一棍,捶佢一拳?既無所施,則惡婦之罪,既漏天誅,又逃王法,惟有自刑之計,自家勒自家勒到死為止,不許偷生。懸樑一道,論番鬼之刑謂之問吊,論王法之刑作為問絞。嗟乎,人之一身,無論男女,父母許多心血鞠育而來,然後得長大成人也。所以肚餓思食,身病思醫,被嚇則驚,臨危則懼,未肯輕棄此身,作為\index{廢物}。豈可以微嫌細故,口角相爭,便甘心而為鬼物乎?大抵男子不孝,漸變而為奸淫邪盜,顛倒衰頹,致犯凶灾刑戮;婦人不孝,漸變而為逞刁撒撥,怨怒咒罵,致犯服毒懸樑。

	次早,使人投告慎氏父母家。其父母飛奔而至,大聲罵曰:「我女因乜事致死?必有委曲之處。壻不在家,惟你两老人是問。快\ruby{的}{󰦦}講出來!若不肯講,斷唔做得。」齊仲良曰:「親家,此事本無大故,不過因你個女粗駛大用,懶做工夫,我個細仔諫佢幾句,逆佢之心,佢就生氣起來,自尋短見,非有別樣冤情也。」媳之父曰:「照你講來,都是幫住細仔說話。定必佢做亞叔,調戲大嫂,致我女含羞自尽。此等大冤大屈,忝辱天倫,我要去告官,斷唔了得!」話完,抽身抽勢,發脚就走,話去請狀師,入稟呈告。

	齊仲良見如此誣賴,就係會打官司也要錢,何況官字两個口,佢口大,我口小,我話假,佢話真,終須受累不淺,不如忍氣吞声,使人留挽住他。請埋兄弟講說話,仍然不肯罷止,要補田三十畝方肯干休。仲良無奈依從,寫田契交他而去。將慎氏殯塟既畢,其子思賢帰來,理宜在父母面前,講幾句說話:「這賤人莫不是前世與佢有冤,故此今生到來累我?惟父母不用掛念。縂之另尋一個好品性女子,再娶帰來,奉事父母便是咯。」咁樣慰父母之心,方為合理。乃不如此講法,曉得日哭夜哭,飯都懶食,只知可惜死了咁好老婆。齊仲良不覺嗟歎曰:「我一生耕田,飽暖安樂,未嘗有意外之憂。唔估到今日,新婦死了,田產消磨;子不念父母之心,又來激惱,雖生何用?不如一死為佳。」半夜往村前大塘,跳落水死。

	次早,其妻問曰:「老太公今朝咁早起身,去了何處呢?」各人答以不知,是日不見形影。未免思疑,呌人訪查,尋之不見。第二日,屍浮水面,方知赴水而亡。其妻直走去媳之父母家,大聲罵曰:「你女之死,非有人拷打佢,非有人逼勒佢。但愛尋短見,自賤輕生,無關緊要。你架起大題故來嚇我,致我丈夫補去田地,實不甘心,今忿恨身亡,為你之故。我今與你誓不俱生,同歸一路便罷。」話完即撲身埋去,扭住媳之父胸前,執住佢把鬚死手不放,好似拉狗咁拉,声声話要共佢落塘跳水死。拉得個親家面青青,氣嘈嘈,口不能言,魂不附體。各人見他咁兇勢,咁撒賴,難以用手相爭,只得勸曰:「親家媽呀,你唔在咁發怒咯!死者不能復生,縂之將此田交還與你便罷。」仲良之妻曰:「咁樣交還,豈足遂我心嗎?我唔要,硬要共佢死!」又勸以厚買棺材,做齋超度,亦不肯從。媳之父母,見無拆法,願交前田之外,另將自己田,再補三十畝。仲良妻要寫契執據,請叔伯來看,方肯歸家。

	仲良之妻,去嘈鬧親家,要補囬田畝,似不為過。獨怪女親家,身為父母,由女之放肆忤逆而縂不知,是縱其惡也。幼時教訓,嫁後肯稽查,未必如是之太過也。即或女生外向,父母難拘,則當女死之時,細心追究根由,可以知其醜處。乃不由分說,只借女死誣賴於人,想錢入荷包,作含血噴人之計:其女不賢,其父亦醜類矣。誰不知你曉累人,人亦曉累你,冤冤相報,劫劫相纏。女親家之為人,即謂之拖屍鬼可也。但不知此公多少女耳?若生得十个女,一女自盡,三十畝田;一女輕生,三百畝矣。个\ruby{的}{󰦦}世界咁好撈頭,何必去掘金山,然後可稱發財也哉?所做之事,理不通行,人人學你所為,不成世界。取此不義之物,便可不憂貧也麼?吾恐餓不死時先飽死你矣。

	齊仲良之妻歸來,殯葬其夫既畢。又到女親家大忿氣曰:「我一世唔曾被人棍騙,今遭此\index{潑婦},勒去我田三十畝,實在不甘。想去告官,係我訛詐在先;若啞口吞声,實在唔抵。」對其妻曰:「我想去女家婆個\index{老狗乸}處,吊死佢門前;你即時去稟官,可以累得佢七零八落。」其妻曰:「乜你咁錯見呀!你先做不仁,人後做不義,亦是平常之事。你移屍嫁禍,未免失礼於人,為人所笑。人生在世,性命為重,錢財係倘來之物,就作破財攩灾,無容計較咯!豈可將條老命,去負累人麼?」其夫默然不答,其妻時時提防出入。

	一晚,親戚請去飲酒,半夜不見歸來。其妻使人去問之,親戚曰:「此老翁飲了幾杯,話肚痛而去矣。」其妻使人走往女處,誰知吊在親家門上,好似風吹臘鴨,搖搖擺擺咯。其妻明早即去告官,官約於某日到來驗屍。姓齊姓慎两村父老,齊集議曰:「論起此件事,女親家因女死而來訛詐於人,男親家因訛詐而自尋一死。一死一訛,一訛一死,訛無盡而两家性命已帰泉土矣。我等身為里老,應當排難解紛,豈可住其忿鬧官司,白受官差魚肉。」依公直斷,着男親家處將慎氏之田三十畝獻出交還,着女親家處將死者殯葬山頭,不得多生枝節。向官遞囘和息紙,萬事皆休,各依公了事。

	女親家婆所諫丈夫說話,亦極通情,亦極合理,可惜不諫於女死累人之時,而諫於夫想尋死之日。亦非不好,未免先錯一囘矣。两姓父老,勸解息訟,其功不少;但能於女親家公來誣賴之時彈壓其兇,及男親家婆來追補之時和解其忿,不至生出两条人命,多了一重冤結也。

	齊思賢不思己過,不悔前非,回舖後两年不歸家,只知掛念老婆死得可惜。一夕坐在床前,解衣欲睡,忽起一陣陰風慘淡,灯變綠色無光。有陰司差二人,一個手執銅鞭,一個手執鉄叉,以鉄鍊鎖住慎氏頸,披頭散髮而來,面肉乾枯,身上血痕点点。向夫大哭曰:「我以丈夫憐愛之故,自賤輕生。誰料禍劫牽纏,累到两家父母。陰司將我打落酆都地獄,要受苦二十年,變過两次畜生,方成人類。如今每月初一十五,受打一百鉄鞭,萬錯千差,悔之無及。丈夫聽信妻言之故,不顧高堂。以丈夫前生修善,今世應生三個好仔,發數千銀財;今因此事,福祿減去大半,三子將來無好處矣。丈夫他時死後,劍樹刀山之苦,斷不能辞,君其思之。」齊思賢曰:「賢妻呀,你咁樣受苦,等我請幾個和尚念經拜佛,與你超生。」慎氏聞此語,踢地悲啼曰:「君之一言,又使妾增罪咯!君不念老父之死,偏憐妾之冤;妾有何冤?自取罪耳。君速回家,尋一個女子,要好性情,識礼義,曉得尊卑上下,方可為人。勿惜多金,縂求賢配,夫妻誠敬,奉事高堂,以孝順贖忤逆之愆,補君之過,并減妾之罪也。」話完,苦哭而去。思賢自見驚疑,嚇得週身冷汗,終日難安。明日覆想,疑自己神魂散乱,未必真是鬼來。

	第二晚,妻又來責罵,且云:「你不信我,任你千般恩愛,付之東流。我在陰間仍咬恨你,看你將來有帰結否?」又哭而去。齊思賢大加醒悟,方怨從前之錯,即時計辦銀両囬家,請幾個真修和尚,誦經十日,超度父之靈魂。先同細佬完婚,自己擇一個好女子娶歸。同心孝順,作老母如佛如仙,買新衣買鮮果,時時酒肉奉事,極其誠敬。老母亦覺心歡。帶細佬往舖學習生理,更兼發心修善,又印廿四孝二千卷分送於人,以補己過。如是孝順約有十年,鄉里尽皆稱贊。一晚,其妻來托夢曰:「自君改行孝義,新婦又極純良,敬奉真心,夫妻如一,將功贖罪。陰司減妾十年地獄,免畜生一道,準我轉世為人。丈夫之身,亦補回衣祿。加修勿惰,莫誤前程可也。」語畢而去。齊思賢每將此事告與人知,聞者亦多感化。後竟發財數千,三子皆稱中用,自以為改過之報云。

	畏妻太過者,不成夫綱;愛妻太過者,亦釀成家變。如慎氏本非驕侈,其夫有以縱之;其叔本非逼勒,而嫂有以挾之:此婦之輕生,實其夫致之死也。乃女父村愚,以死命作生財之計。破家喪媳,做翁能不傷心?為子者當仰慰高堂,多方勸解。乃不念生身之愛,偏深結髮之情,自失靈明,癡心極矣。一波未平,一波又起,媳死而翁隨之,女死而父隨之,財與命相連,冤冤相結。人謂財可通神,豈知因財變鬼也。家本相安無事,因一婦人不肖,累及家散人亡,罪大難容,死當墮落。幽魂受苦,方識前非,幸能以夢告夫,使之補過。不然者,夜臺凄惨,何時得與超生;人子昏愚,一世甘為折福。

\chapter{生魂遊地獄}

	福建漳州進士丁蘭吉,別號夢靈。其為童生時,年二十四歲,值九月重九,乘興登高。攜酒一瓶,遊山四望,但見松聲萬樹,落木蕭蕭。坐在山頭,自斟自飲。忽起風來,吹捲地中黃葉,團團滾滾,極似有情。蘭吉曰:「此風如此趣致,莫不是有鬼神經過嗎?」即奠酒三杯澆地上,風葉旋轉而去。

	一息間,蘭吉似醉而睡,似見一人身着青衣,向前揖曰:「丁先生好人物,多蒙賜酒。」蘭吉問:「尊駕為誰?何出此話?」青衣人曰:「我非人,乃陰司差也。因帶文書往某處城隍,路經過此。生平有酒引,忽聞酒香,情不能禁,故在此盤桓。又蒙過愛情深,使我酒喉添潤。此鬼得酒解渴,與路上行人得茶解渴,均銘感不淺。如此美意何以為酬?」蘭吉拱手曰:「尊駕是地府貴差,尽知陰間情景。我聞得陰間有十八層地獄,未知真假如何,常時想去游觀,茫茫無路,今逢尊駕,可能帶我一行,做得唔呢?」青衣人曰:「個件事重易過執豆,執豆尚要顧低頭。」蘭吉曰:「你引我去,要帶我囘來。」青衣人曰:「自不然呀!唔通帶你去死麼?」由是相引同行。

	忽到一處,日色帶\ruby{的}{󰦦}陰沉,睇見往來人甚眾。行至一大宮殿,企在門前,青衣人曰:「你在此處,等我囘覆王爺,然後帶你遊玩。但我入內,或者事務多,未能出來。你不須憂,我有分數。」青衣人入殿裏,蘭吉在外。便見門前樹一聯大鉄板對,寫十個字曰:「萬惡淫為首,百行孝為先。」看見好多人,有\ruby{的}{󰦦}坐轎,有\ruby{的}{󰦦}騎馬,有\ruby{的}{󰦦}坐車,有\ruby{的}{󰦦}坐囚籠,有\ruby{的}{󰦦}披枷帶鎖。有擺手擺臂而來,有垂頭喪氣而至。看見殿內出者,有\ruby{的}{󰦦}歡天喜地,有\ruby{的}{󰦦}苦泣悲啼。有着大袍大褂而去,有着爛衫爛褲而行。有披牛皮馬皮者,有披狗皮羊皮者。世上所有之物,即陰間所有之形。一隊而來,一隊而去,刀山劍樹,苦海血池,遠望之而竟然在目也。

	約半時間,青衣人出曰:「我知你等我久矣。因有別事,是以延遲。」丁蘭吉曰:「世上竟有陰間一事。枉我讀書咁久,尚一肚狐疑。」青衣人曰:「世上不滿百年,為善得福,安樂亦有限;為惡得禍,苦惱亦有限。故造化議其善之大者,使他享福居於天堂,千百年不尽也。如文昌關帝,你話佢應在天堂唔應呢?惡之大者,使他受苦,坐於地獄,千百年不尽也。如曹操秦檜,你話佢應落地獄唔應呢?其餘尚有許多仙山佛國,在塵世之外者,逍遙自在,你所知也。此等快活,你話從修行得來抑或從罪孽得來呢?世有等大善,即有等大惡;大善要使他享極快活,可知大惡要使他受極苦惱矣。至於中善中惡,莫不有一個擺佈他、安置他,而使他各受其報也。讀書人於仙佛古典亦常用之,何以於蓬萊公之為地下修文郎、唐鍾馗之為南山進土,則又疑而不信?無乃以眼所不見,話其荒唐。」丁蘭吉曰:「正為此也。」青衣人曰:「若以眼所得見為真,眼不得見為假,則是鳳凰麒鱗都是假物,伏羲盤古都是假人。」丁蘭吉曰:「有書為據。」青衣人曰:「彼故有書,地獄之說豈無書麼?」丁蘭吉曰:「听尊駕所言,高談雄辯,是有才學之人,為何做這等\index{脚色}?」青衣人曰:「我生前亦係讀書人,專工筆墨,無他過處,只因不信果報,聞人談及必笑斥之,阻人為善之基,錯悞非少。生前已經受罰,\ruby{蹇滯}{󱞹󱀝}無成\footnote{蹇滯,󱞹󱀝,運氣𠄡好};死後又罰為差,勞勞奔走。我與丁先生相好,有夙世之緣,故乍面相投,如逢知己,不覺將胸中吐露。先生為我傳之,以補前生之過可也。」蘭吉曰:「得聞尊論,茅塞俱開。地獄十八層,煩為引我去看。」

	青衣人帶至一所大地方,陰氣慘淡,令人毛發悚然。有看守之人喝蘭吉曰:「你來做乜事幹?」青衣人曰:「佢係我好朋友,帶佢到此一遊。」守者曰:「係老哥\ruby{的}{󰦦}知己嗎?隨便進去。」入了第一層,見牛頭馬面,兇惡如狼,將罪鬼拷打:用蔴繩吊起,手執鉄棍仔數枝,如烟筒竹一樣,長\ruby{的}{󰦦}四尺,自頭打到脚,打完放落,再將第二個罪鬼吊起,照前打法。鬼哭呌不絕声,話:「我怕咯!唔好打咁多咯!望你輕\ruby{的}{󰦦}手,饒我罷咯!」牛頭獄卒曰:「你打得人多咯!到我打吓,唔係你唔知人辛苦。」所打之犯鬼亦是惡毒婦人,刻薄婢妾者居多。其餘差役兇徒,勒索人財者亦不少。更有一等做工藝師傅殘虐徒弟,教學師長耽誤門徒。無慈惠之心,任暴戾之氣,冤冤相報,事有輪流。

	忽然牽得一个犯來,頭帶頂,脚着靴,頸掛朝珠,身穿袍褂,昂昂而來,縂無畏懼。獄卒剝其衣服,脫帽脫靴,此犯尚以大脚踢其獄卒。獄卒驚曰:「乜㘃事幹,你想發顛麼?」此犯曰:「你正發顛。你都唔識人,咁大胆將我剝脫,你想打脚骨嗎?」各獄卒掩口大笑。此犯曰:「你作我乜樣人呀?我曾經出身做過縣官治百姓,係大爺身份,你比同做賊佬麼?」獄卒曰:「你做官人,又呌犯人。」此官曰:「我所犯何罪?」獄卒曰:「你先時王爺處就既審過,話你刻剝百姓,重關係過做賊。你重想來詐戅麼?」一獄卒曰:「你勿共佢講咁多,我都嫌費氣。王爺吩咐要打佢八百,就照數打之,何用多言。做官唔好,重要打重\ruby{的}{󰦦},捉佢吊起。」誰知此官又肥又白,肉多骨少,打了幾棍就呌苦連天,大聲喊曰:「我唔認做官咯!我認做賊罷咯!」做官唔好,原來係賊一班獄卒俱笑起來,引得旁邊所吊之婦人,亦不覺笑。一間濶大寮廠,此處有吊起,彼處有吊起,相離不滿五尺。又有一个吊起,彼吊者嗚嗚咁哭,執棍者紛紛咁打。有打三百,有打五百,多者一千,至少二百。有男有女,有老有少,一班既去,一班又來。有一个官在此點簿,打完牽去稟知,然後照閻王之簽發放。或變畜類,或轉為人,或留押禁,再受刑威。官坐之處,旁寫一聯,粉板墨字,其對文曰:「勸眾人切莫為非,恐死後要受苦刑,你又不信;向小卒乞從寬責,似陽間混埋公案,我實難饒。」

	丁蘭吉問:「為何有咁多人犯罪?」青衣人曰:「天地之大,四海之眾,九州十八省,你話幾多人呢?有\ruby{的}{󰦦}地方好風俗,有\ruby{的}{󰦦}地方醜風俗。然好之中亦有醜,醜之中亦有好,陽間官府安能逐一分別?擇其醜者而治之,為問一縣之中,治罪者有幾人?而民間不孝不弟、不仁不義之徒,又何止千何止百也?況且官府治罪,止論人身所行,不論人心之所想;惟陰間治罪,計其事并及其心。凡貪心、淫心、刻心、毒心、忤逆心、妬忌心,種種醜心不可對人之處,外雖無惡迹,此心已為鬼神所不赦之條。故虎在深山,未有食人,見者指之為惡獸;虎口雖無人肉,虎心欲食尽人身也。」丁蘭吉曰:「果然好講法!誅心之說,吾得聞矣。」

	又引去游第二層地獄,見橫床數百鋪。或堆滿簕在床,而背脊睡其上。或身眠在上,用大石壓其胸,綁住手足,欲起不能,欲脫不得,滿身痛苦,日夕咿唔。有一人一床者,有兩人一床者,有男與男同床,女與女同床者,有一男一女同床者,有一男而與數女同床者,有一女而與數男同床者,有七八人一床或十數人、數十人一床者。床之大小不齊,人之老少不等,形枯似炭,骨瘦如柴。丁蘭吉曰:「罪有數端,非言一例。世人惡事,由於惡心;消息之機,由於想像。大約日中行走,事務紛紜,有時唔想得咁透徹。惟睡在床上凝神閉目,想到人不及覺之處,人不及料之情。古怪離奇,變詐百出,其計多於床上得之。何況明謀暗騙者,安享而睡;行姦賣俏者,淫樂而眠。樂於床上得,苦亦於床上受也。一男一女同床者,夫妻枕畔擺弄挑唆,不孝父母由此生,不和兄弟由此起。或姦夫姦婦,密約私情,所以男女一床,取其同甘同苦也。或一男而姦數婦者,或一婦姦數男者,所以各有不同也。其餘各有毒心,各有毒計,所以一人一床也。至於事之同類、罪之同情,不論多少,共為一床矣。」

	話完,又引去看第三層。問何以有勾舌根、割口唇者,答曰:「此挑弄是非,毒口罵人之罪也。」問何以有挖眼睛、流眼血者,答曰:「此不識尊卑,目中無人之罪也。」問何以有斬手臂、切手指者,答曰:「此私竊財物,或誣賴指人之罪也。」問何以有截脚批踭者,答曰:「此拐帶人口,或引行邪徑之罪也。」問何以有割乳開胸者,答曰:「此裝腔作勢,霸佔欺凌之罪也。」問何以剮心抽腸者,答曰:「此做光棍,用奸計之罪也。」問何以有用秤勾背,以刀削面者,答曰:「此做事冇腰骨,不顧面皮之罪也。」問何以有銅汁灌其口,以尿穢潑其身者,答曰:「此貪不義之錢,不顧臭名之罪也。」丁蘭吉曰:「觀此形狀,亦覺可憐。」青衣人曰:「你以為可憐,閻王以為可惡。」丁蘭吉曰:「可惡莫如盜賊,謀人財,害人命,累人苦楚難堪,其幽魂落何處地獄?」青衣人曰:「賊有數等人,不以一概而論。其力或强或弱,所行或明或暗,其性或兇或怯,所犯多或少;所以名為賊也,其罪有重有輕。賊之類多在第九層地獄,劍樹刀山。其餘各地獄,亦有安置。人生所犯之罪,或以王法消之,或以殘疾消之,或以田園敗尽消之,或以妻子死亡消之,或以子孫不肖消之,種種亦有。若本人罪重,未有消除,或消之不尽,所以有地獄一途也。但家道不寧,世事不順,亦有關於前生修福未到,不尽關今世所行也。」丁蘭吉曰:「講得圓通,算你明白。」

	話完,又到去看第四層地獄,見有落\scalebox{0.5}[1.0]{木}\scalebox{0.5}[1.0]{磨}磨而水血淋漓,有落碓舂而縻肉飛起,蘭吉問:「何罪受此慘刑?」青衣人曰:「此不顧父母之無情人,激惱父母之忤逆子也。」蘭吉問:「不孝之條,何重若此?」青衣人曰:「百行孝為先,可知百無行者,必以不孝為先矣。受父母之深恩而置之度外,是忘恩也。不順其心,而敢忤逆,是欺其親也。欺君有可斬之罪,欺親無可殺之條麼?君之待臣,賜以功名,而不必出其心血。若親之待子,自幼孩至成童以後,費尽幾多心血,用尽幾多錢銀?養隻狗都曉搖頭擺尾,養隻牛都肯低頭拖犁,独至養大个仔,竟無中用,對父母冷淡無情,或作父母如路人,或作父母如仇敵。論天地間負義忘恩,當以不孝之人為首。」又行數十步,問何以有袈裟堆棄於旁,青衣人曰:「此犯姦之僧尼也。佛門破戒,罪加常人三等。以其借清修之名,恣淫邪之樂也。」蘭吉曰:「僧尼中亦有好人品者。」青衣人曰:「其好者或上昇天堂,或托生善地。其不好者,或為餓鬼,或作畜生者亦有之。」

	再深入一重,轉過一個曲處,見無數婦女,赤身露體,只有一小幅橫布僅僅遮羞,其餘裙釵衣履,堆置一處。牛頭獄卒執住女人个把頭髮,拖入\scalebox{0.5}[1.0]{木}\scalebox{0.5}[1.0]{磨}心。\scalebox{0.5}[1.0]{木}\scalebox{0.5}[1.0]{磨}口大約尺五六寸之間,可容一个人身落肉。婦人悲啼苦哭,大喊救命呀,苦苦扳緊\scalebox{0.5}[1.0]{木}\scalebox{0.5}[1.0]{磨}脚,唔肯上\scalebox{0.5}[1.0]{木}\scalebox{0.5}[1.0]{磨}盤。獄卒尽力一抽,將婦人頭放落磨內,兩脚向天,兩獄卒乱推乱轉,悽慘之形目不忍見。又捉婦人落碓砍內,碓口約有四尺之餘。婦人大哭,亦不肯落,轆倒在地,呌苦聲嘶。兩个獄卒一人抽頭,一人抽脚,抬落碓砍之內,只有五寸之布橫束腰下遮羞,亦係赤身露足。大碓舂落,舂一声呌苦數声,手乱搖脚乱動,而血肉花飛。蘭吉向轉面而行,便問:「何以婦人要受此苦?本來婦人情性溫柔,不奸不惡,並無為非作歹、恃勢行兇,何故受此極刑?有不可解。」青衣人曰:「世間婦女,其賢良者,好處皆知。其不善者,罪有不覺:有憎嫌丈夫娶妾,而願絕香烟;惱恨男子養親,而偏為刻薄;減翁姑之衣食,薄叔伯之親情,親反成踈,恩將仇報,助丈夫之罪孽,累後代之衰微。此等婦女,王法所不及誅,家法所不能治,惟地獄一道,可以勾消。又有串引為姦,專行拐騙者,其罪更當何等也!」

	又引至第五層,見數十大灶,見猛火烘烘,油湯滾滾,熱氣騰騰。近而視之,無數人形,隨湯起倒,或嗟或泣,或沉或浮,骨肉將霉爛。問犯此者何等人物,青衣答曰:「多是世上之土豪土棍也。」問何以能作淒楚声,能知痛苦也?答曰:「世上以肉身為至親至真,所以有補氣補血、補皮補肉,而不肯補魂氣之清靈。人之能曉飲曉食、曉行曉走者,魂也;能穿天入地、受苦受樂者,魂也。若失其魂,則肉身不能飲食矣,不能行走矣。無論骨化形消,終歸無用。即全屍具在,有口不能言,有耳不能聞,有手不能動,有足不能行,問之不知,打之不痛。是生前知痛者,魂在身也;既死不知痛者,魂離身也。到此時,肉身不能行走,魂影能任其去來;肉身不能食飯,魂影能鑒香烟;肉身不曉出聲,而夜靜曾聞鬼呌;死肉不知痛,而靈魂能知痛。今者靈魂既落陰間矣,是煎者煎其魂,煮者煮其魂,鞭其魂,打其魂,其魂既靈。靈者醒也,所以有謂之死肉,未有謂之死魂;有謂之爛肉,未有謂之爛魂。議論風生,句句透徹,此鬼三寸舌吐出連花不能死則常生,不能爛則常存。所以肉身雖死,而魂又托生別處矣。煮之不爛,而魂依然知痛矣。你不觀之古人麼?古有殺身成仁者。既謂之殺,則身一處,頭一處矣。世但知有無頭之鬼,而不知有無頭之神。忠臣孝子,義夫節婦,每有不避患難,白刃當前而赴死者,既被殺矣,豈做了菩薩,尚係有身而無頭者麼?可知肉身之頭可斷,而魂影之頭不可斷也。肉身之身,斷而不能續;魂影之身,離而可復合也。如抽刀割烟,如牽絲界水。譬喻十分精當,清楚玲瓏如利刀削藕若非如此,則地獄中有抽腸割舌之案;受苦既滿,將靈魂發他轉世,而遂舌不知味,腹不知飽麼?」丁蘭吉跳起拍掌曰:「好議論!好道理!無怪尊駕係前世讀書來也。既爽我心胸,大開我眼界,所謂與君半日話,勝讀十年書。我歸去咯!」青衣人曰:「十八層地獄,你未有看得一半,駛乜咁快囬家呀?我帶你去看第六層。」蘭吉不願行,青衣人苦苦牽手而去。

	既到第六層咯,睇見一班男女,或企在地,或坐在橙,或睡在床,俱是釘頭釘脚,釘手釘身,又另一个花樣光景。行轉一个曲,忽然看見自己个一位大嫂,坐在平石之上,有一條鉄鍊鎖住脚,有一管長鉄釘釘在左便乳頭。大發一驚,滿額流汗,曰:「㗇!㗇!奇怪,奇怪!我記得今早出門時,大嫂尚睡在床中呌苦呌痛,唔通一時死了?」淚即交流滿面。青衣人曰:「此是你個位令嫂麼?」蘭吉曰:「是也。」守獄卒曰:「你大嫂未死,此是生魂耳。」蘭吉問:「幾時勾來?」獄卒曰:「勾到三年咯。」蘭吉曰:「怪不得我大嫂生一乳瘡,三年不好,醫尽千般百計,種種無功,拜鬼拜神,都成混鬧。点估到陰司釘住佢,劫數難逃。究竟我大嫂所犯何罪,要咁樣受苦呢?」獄卒曰:「你大嫂所犯陰毒。因你亞哥無子,立一个妾,生得一子。你大嫂恐怕个妾母憑子貴,恃寵生驕,三朝後入妾房中,窺探無人,將繡花針刺入肚臍之內,小孩子呱呱咁哭。妾帰來,以為剪傷臍帶,引動臍風,又為風痰湧結,不肯食乳,哭不絕聲,尽一日夜而死。其妾只怨自己命運之衰,生兒難養,豈知別樣所為麼?灶君將此事奏聞玉帝,轉發落陰間。誰知佢以繡花針刺个仔肚臍,閻羅王亦以長鉄釘佢个只乳,你話有報應冇呢?」蘭吉曰:「好呀,好呀!乜知佢咁陰毒,唔怪得佢要個樣病法,真有天眼咯!但死者不可復生,我大嫂既受三年苦,亦可以準其罪過。求你一个方便法,將我大嫂乳上拔起一条釘,你可做得唔呢?」獄卒曰:「斷斷不能,要等王爺主意。」蘭吉曰:「重有乜方法?」獄卒曰:「除是勸佢修心,或可免罪。」蘭吉曰:「亦是道理。但如今近晚,我唔睇咁多咯,我速歸家便了。」青衣人曰:「我帶你囘去。」一路行一路講,一陣間歸到山頭,青衣人曰:「請別請別,後會有期。」丁蘭吉曰:「多煩大哥,有勞相送。」

	山鳥一声,即時驚醒,酒瓶倒地,酒亦成空。日色半落西山,發脚便走。歸至家,聽聞大嫂姚氏罵其妾曰:「食屈米,藥都唔曉煲,水又少,煲到乾,想來食死我,你做大婆咯?个\ruby{的}{󰦦}陰毒法,你估我唔知?」蘭吉曰:「亞嫂唔好咁怒氣,養靜吓罷咯。」姚氏曰:「我辛苦,佢又來激我,点能抵得呀!」蘭吉曰:「亞嫂你本來冇辛苦,你自己愛尋\ruby{的}{󰦦}辛苦來。」姚氏曰:「我去那處來呀?你亞哥唔作我係人,妾氏唔作為意,連你做亞叔都唔作我係亞嫂。我知咯,一家都宜得我死了咯!」蘭吉曰:「亞嫂,你唔死都作死一樣。」姚氏曰:「因乜事幹作我死了呢?」蘭吉曰:「你魂魄被勾落陰間,已經三年受苦。」姚氏大声曰:「你見了鬼麼?」蘭吉曰:「冇錯冇錯,我真真見了鬼。」姚氏曰:「你点樣見法呀?」蘭吉曰:「我因遊山如此如此落到陰間,見你被鉄釘釘住。」姚氏曰:「我所犯何罪,佢來釘我?」蘭吉曰:「你陰毒。」姚氏話:「我陰毒?我食你麼?我咬你麼?」蘭吉曰:「你唔係食我咬我,縂係將我个侄來害死,天就唔容得你。」姚氏拍床大喊曰:「天冤地枉呀!你个侄三朝七日死,人人皆知;今者發起顛來,話我害佢,我有咁樣心腸麼?我為個仔偷流眼淚,眼水唔乾,提起仔个字我就心刺。你重來話我不仁,我問你有乜憑據?你講出來就罷,若冤枉我,保佑先死了你。」蘭吉呵呵笑曰:「亞嫂你果然好心。前者我細嫂生得個好仔,你妬忌起來,三朝後入房抱起佢話:『亞蘇亞蘇,乖乖乖。』就將繡花針刺入佢肚臍,哭到死為止,你話陰毒唔陰毒呢?」姚氏聞此語大驚,面青青而呌曰:「你唔好冤枉我,睇雷公打你!」蘭吉曰:「雷公唔打我,閻羅王要釘你。你得做唔得做,你自己心知。我一向唔知,今日方知。若係我亞哥大早知到你咁樣心腸,包管打理你咯!我怕你痛死都唔醫你。」姚氏聽到此話,知係真情,个陣口軟聲低,細聲問曰:「亞叔,真正嗎?」蘭吉曰:「話係咯,唔通嚇你麼?」姚氏垂頭氣短曰:「你唔係嚇我,聽你講起來,我心都怕,大約都係冤孽咯。若話唔信,何以外科先生請得多,縂不見應效?其喃魔先生、盲公鬼婆都拜過,縂唔見功呢?二叔呀,乜你見个管鉄釘,都唔共我拔出呀?」蘭吉曰:「我想拔出,但是守獄卒唔肯呀。」姚氏曰:「唔通由得我痛死?我病了三年,痛到魂都冇了咯!咁樣重有乜方法呢?」蘭吉曰:「除是轉心腸,自後唔好咁惡毒,或者可以好得都未可定。」話完拂袖出門而去。

	姚氏在床左思右想,此事實自己之錯。論起世間至有情者婦人,聞人報到亞姨生仔,亞妗生仔,亞姑生仔,就歡喜不了。又買豬肉捉雞,送去做滿月,及賀開燈。何故自己之妾生兒,作為仇敵?況且个仔長大,將來發財奉養我,娶新婦服事我,就係做官先封贈我,百年之後忌辰拜我。世人認個契仔尚且親之愛之,何況妾氏之兒,與我着三年服也。如果當時唔害死佢,如今有三四歲,可以扶住床邊,行來問病。就係病死,亦有個仔捧我神主牌,拈枝幡竿柄風飄飄吓,身披孝服,曲背低頭哭我為娘,呼我為媽呀!此婦算深沉,真想得透想到此處,忍淚不住,以手掩口,哽咽低声曰:「孩兒呀,我知你死得苦咯!我知難為你老母咯!我如今知悔恨咯!你在九泉之下,勿怪責我咯!」話完又暗哭不止。停一息間,抹乾眼淚,呌婢買寶燭囘來,在天井中点爝,要婢扶出到簷前,跪住叩頭,密稟不知甚麼說話,以頭乱叩地上,叩得一頭沙泥,額上肉都凸起。拜完扶回床上,大歎一声,出一身冷汗。即將心腸改變,化作仁慈。人話江山易改,稟性難移,个句說話亦假。由是待妾如姐妹一般,親同骨肉,有不合處細心教道,不出高声,妾亦歡心奉事。姚氏自知罪過,不肯請醫調理,不過以香爐灰敷之。誰知十日之間,乳瘡生肌埋口,似有神助。姚氏自後更發心為善,有益人者方便為之。三年後,妻妾各生一子,長大讀書,皆稱俊秀。

	人話省城天子馬頭係殺人地,誰知閨房之內都有殺人地也。人話男子做殺手,不知女人亦有做殺手也:如家婆治死新婦,主人婆治死婢女,妻逼死妾,婦謀死夫,世界之間,亦時所有。今姚氏不害其妾而害其子,不明發於聲而暗施其毒,外貌施脂粉,細語嬌聲,欲得丈夫憐愛;誰不知溫柔手段有殺人刀,欲斬先人之血脈,覆轉香爐,黑火鳥燈,甘為餓鬼。為丈夫者不知其意,因妻有病數載調醫,豈知同枕而不同心,顧前而不顧後。姚氏能欺人不見,不能瞞得灶神。上奏於天,原情定罪,三年大病,苦楚纏綿,枕席難安,即是生前地獄。若非其叔說破,何時悔過收心?及至自怨悲嗟,方知前錯,一轉念間改頭換面,洗過心腸,臟腑之毒氣皆清,惡大婆變而慈悲菩薩,一團和氣,滿面春風,天降麟兒,吉祥歡喜。然後信前此者,孽由自作;後此者,福自已求也。

\chapter{借火食烟}

	嘉慶初年,福建廈門鎮地方,有一人姓龔名承恩。家資三十餘萬,捐到吏部郎中,歸來勢壓一方,看鄉人不在眼內。建造高樓大屋,又起一所大花園,泥水木匠石工三行人等共成百數,日做工夫。龔承恩移出一鋪大炕床,擺列一副鴉片烟燈,金漆烟盤,象牙烟鎗,在此坐立,督理做工人役,氣勢薰天。

	一日午後,有一个泥水師傅,赤身露體,腰下束一条捫巾,氣喘喘汗淋淋,手拈一枝短烟筒,長不滿六寸,走埋烟灯處向火吸烟。龔承恩一見不平,勃發罵曰:「你是何等樣人,乜樣\index{脚色},一身臭汗,走埋來借火吹烟,你都唔識意趣,唔知避忌,快\ruby{的}{󰦦}走開,不得再來混鬧!」其人滿面羞慚,氣忿忿而去。

	誰知此人心懷不服,素稱暴戾兇橫。窺見承恩左右無人,即向木匠處借利大斧一張,木匠以為別樣用法。時天氣炎熱,龔承恩脫衣避暑,體白如雪,肉滿如膏,橫睡床中,向吹鴉片。此人從後行來,出其不意,舉利斧尽勢劈落,腰脊破開。承恩大呌一声,眾人走來,兇手乘勢再砍一下,痛絕死矣。死得慘人多圍住,兇手欲走不能,當堂被捉,捆綁送去廈門同知。

	其官姓呂名有才,初上任三日,即接得龔家人命案。論此案工人殺死東家,青天白日,人所共見,應將兇手收押。是晚,此官吩咐爺們到兇手處,如此如此問話。爺們去見兇手,曰:「你為何殺死東家?」兇手曰:「佢咁樣毒口罵我,我忿恨不甘,持斧殺佢。殺人償命,更有何言?」爺們曰:「你真愚哉!你肯信我,我能救你。」兇手曰:「如果救得,真正係承恩似海,荷德如山。」話完即叩一个頭。爺們曰:「我話你知,明早太爺審你,你話我係持刀,皆由主人之妾呌我去殺。照此講法,罪減一等,不過充軍。」兇手不勝歡喜,又叩頭曰:「多蒙指示,無限沾恩。」及至太爺開堂審訊,兇手照爺們所教,一一而言。官即出差去鎖其妾。主人之妾生得二子,合家知其冤枉,安肯佢到官?若到官門,定必要受苦刑,逼佢招認,若然招認,定要凌遲。合家大小,尽日商量,此事並無拆法,惟有將銀頂住,或可推延。斟酌未定,誰知第二班差又來,即要捉人,一刻不能延緩。妾不願去,合家亦不肯放去,即將銀二萬抬送入官。官得了銀,遂免追究。官又呌爺們到兇手處如此如此。爺們又話兇手曰:「其妾不來,你有何計?」兇手曰:「有死而已。」爺們曰:「你乜咁爛命呀!我重有妙策。明早太爺審你,你對答曰:『說話雖由妾教,其主意實出於其妻。』此計更高一着。」兇手又拜又跪,叩謝爺們。第二堂又開堂審問,兇手又照爺們所說。官即出票發差,拿鎖其妻。合家齊集聚議,妾不肯去,妻安肯從?又抬銀二萬送官。官大滿所願,即勾消其票。第三堂又審兇手,官大聲喝罵曰:「本官細查此案,皆係你一人兇暴,縂與主人妻妾無干,何得亂說牽連!該當處斬。」遂將兇手正法,而呂同知之食囊飽滿矣。

	再說龔承恩一生做事,縂冇益人。鄉里貧難,一毫不拔,只好交官交宦,以勢欺人。豈知福尽有時,禍來不測,斧頭劈破,惨過天誅。其後兩子長大,無人拘束,習於淫蕩,因訟傾家。屋舍田園,為人所得,傳至孫有做乞食者。

	今人門口,每寫五福臨門。其五福之道,出在「書經」:一曰壽,二曰富,三曰康寧,四曰攸好德,五曰考終。今是則五福以長命為第一,有錢為第二,平安為第三,好善為第四,好死為第五,而功名貴格不在內焉。今者龔承恩,有四十萬家財,其福之厚可知。如果能通人情,識天理,以和平之道處己,以謙厚之道待人,則人亦愛之敬之,何至有憎之厭之也?孔子曰:「富而無驕,富而好禮,所以常守富也。」或能如竇燕山之濟人利物,蘇眉山之救苦憐貧,福蔭兒孫,富貴無尽矣。財主佬對貧窮人,肯向他稱呼幾句,益及三分,窮人了不得咁歡喜,話某某財翁真正好相與,好心福,好礼貌,好人情,托起你天咁高,且作你為活神仙,生菩薩矣。人話財主佬難做,我話財主佬容易做也;人話財主佬得人憎,我話財主佬得人敬也。世情都係想去相識財主佬,有誰想去相識貧窮?是則想識財翁、敬重財翁,無非望其照顧一二耳。若不能照顧,而反去睇輕人,霸佔人,謀算人,欺壓人,則人不獨憎之,而且欲殺之矣。龔承恩富有多金,而一生無好處,忽被喝罵泥匠一事致身亡家破,零落衰微,令人一歎惜矣!

	想其生於富家,自幼寶如金玉,父母憐愛辜息,作為掌上之珠,有誰拘束他、責罵他而勸化於他?你雖嚴教佢,而佢不受也。即見有順他、從他、饒他、怕他,而奉承他、褒獎他、孝敬於他,養成驕縱之性,不復知天高地厚,物理人情,只知自己係財主仔,一身錢,一肚氣,遇人得罪便忿不能平,些小不合意亦不能忍,罵人不知輕重,待人不識尊卑。於是嚴師益友,不敢勸諫其非;賤類小人,只知順承其過。自高自滿,無束無拘,隨其口之所言,手之所指,不顧人之體面,不順人之心情,以為我富且貴,你無奈我何。即不合理,你要受我氣也。誰不知你有氣,人亦有氣;你不能受人氣,人豈能受你氣麼?遇着能忍氣能下氣者,而亦受之;遇着暴氣戾氣之人,即生氣矣。今執利斧者,一泥水匠耳,發出惡氣,能使龔承恩即時絕氣。豈怕你錢多?豈怕你勢猛?後來即將兇手斬為萬段,亦無補於你之死也。嗟嗟,身居財主,頸掛朝珠,前生修下好多福來,而後有此富貴也。有福唔曉享,積惡以遺殃,橫禍之來,不過借端而發耳。

	朝廷刑戮,至於問絞問殺,可謂重矣。今龔承恩之死,要破脊開XXX,用畢生平積孽,何罪足以當之!話龔承恩之吝惜錢財,何以交結官府?話龔承恩之疎財大義,何以不拔一毫?善緣難化,冤枉甘心,到底成空,付之一歎。又短命,又破財,又不平安,又不修善,又不好死,所謂五福臨門者,而今一福都無矣。龔承恩一身豪氣,其實一身晦氣也。

\chapter{好秀才}

	番邑黃從善堂敬刊

	昆陽縣附城地方,有一人姓曾名恭禹,家資數千。結髮之妻顏氏,生一子名呌亞成。養至七八歲,值明朝天啟之時,地方盜起,不幸遭乱,妻子被賊捉去。乱定之後,續娶一个填房孔氏,又娶妾楊氏。妻生三子,妾又生三子。論起層次,長子亞孝派\footnote{}第一,亞忠派第三,亞信派第四;此三个仔,俱係正妻所生。亞悌派第二,亞仁派第五,亞義派第六;此三个仔,俱係妾氏所生。六个仔,名為孝弟忠信仁義,六个字俱是好字眼,似乎一家都是好人矣。六个仔其父時時呌,六个字之好其父未必時時講也。可惜可惜!六个仔之中,惟亞悌係秀才,果然好人品,依道理而行。其餘五子,俱是惹是招非,而性情暴戾者也。

	世有改其子之名呌做亞善,未有呌做亞惡者。有呌做亞良,未有呌做亞匪者。猶之乎改个堂名,有\ruby{的}{󰦦}呌做積善堂,有\ruby{的}{󰦦}呌做種福堂,諸如此類,不可勝計也。既稱積善,自問一年積得幾多呢?既稱種福,自問一世種得幾多呢?若非積善而自認積善,並無種福而自認種福,則是欺人騙人,而並欲以自欺自騙也。有時對人曰:「我一世唔好講大話。」如此重唔係講大話麼?或有寫積善堂,其實好積惡;寫積福堂,其實好種禍,即係掛家用招牌而專好賣假貨也。其後,曾恭禹因病而死,眾子相聚守喪。將入棺時,死者眼中淚如湧出。眾人看見,个个皆驚,以為奇怪。亞悌秀才曰:「父入棺而出淚,必有不祥。父親知我兄弟平日好鬦,將來必有禍患,故雖死不安,而流淚告我。眾兄弟務宜一團和氣,忍事為佳,免父在九泉猶難閉目。」各兄弟笑曰:「你勿講得咁廢,唔關个\ruby{的}{󰦦}事。縂係喃魔先生,擇時辰唔得乾淨耳。」

	殯葬既畢,兄弟分產異居。亞孝自高自傲,以亞悌、亞仁、亞義係庶母所生,不以骨肉相待,作佢為低一格而卑賤之。結埋亞忠、亞信作為一黨,話:「我三兄弟係大婆仔,佢三个係妾氏仔,就欺佢打佢,都唔奈得我乜何。」果然好亞哥、好帶頭、好倡率,所謂「一隻牛唔好攪壞一欄」亞忠、亞信亦以為然,好似狐假虎威,狼跟豺尾。有時客來探到,開筵飲酒,亞仁、亞義經過堂下,不呌一言。仁、義忿告亞悌曰:「豈有此理,咁無情份!唔通兄弟不如外人?朋友尚且交杯,而細佬行過,竟然不恤。佢不以我為弟,我亦不以佢為兄。不如我三兄弟亦聯埋結為一黨,共佢相抗。況且我二哥係做秀才,斷唔輸得過佢。」亞悌勸曰:「細佬,唔係咁講。佢做亞哥唔明,我忍讓吓佢,世界事情有乜緊要呢?路上相逢,尚且讓人三步,何況自己兄弟,講乜冤仇呀!細佬之言,我不從你。」真正好秀才,曉得大道理,心內有主張,不愧讀書人本領。亞仁、亞義年紀尚輕,因亞悌之言其意亦止。

	又說亞孝,有个女嫁縣城外姓周。亞孝誣賴親家,話唔醫理佢女,以至於死。喝起兄弟子侄及\index{潑婦}等,去捉親家婆,要打過以消此恨。又話亞悌曰:「你做个秀才,份外有\ruby{的}{󰦦}胆色,你都要去,唔好延遲。」亞悌諫曰:「佢做家婆,豈有唔愛新婦之理?請醫不效,難以挽囘。今糾率多人捉他凌辱,你做得出,難對鄉鄰。呌我同行,我斷不去。」唔係怕事,縂係怕羞。亞孝曰:「細佬你勿去咯。我估你做秀才幫得吓手,幫你欺人麼?誰知唔做得料駛,枉你三分貴,一片講執滯,我話你係廢。」

	亞悌个\ruby{的}{󰦦}廢法,正是超羣脫俗,高出庸眾之流。豈同\index{砧板蟻}、\index{溝渠鴨}、\index{臘豬頭}、\index{烏龍尾},遇人有\ruby{的}{󰦦}小事便想插身入內,挑三撥四,作浪生風,講周身本領,兜錢入荷包麼?

	由是不聽亞悌之言,呌齊忠、信、仁、義與子侄等,及族中無賴之徒,去捉周氏親家婆,拳打脚踢。有\ruby{的}{󰦦}去打爛水釭,有\ruby{的}{󰦦}去打穿米塔,有\ruby{的}{󰦦}去打崩飯鑊,有\ruby{的}{󰦦}拈斧頭砍破大門,有\ruby{的}{󰦦}執竹篙攏掃屋瓦,打得穿崩破爛,好處無存。眾等歸來,尽情投告,亞孝拍掌跳起曰:「好呀!好呀!將佢家私什物散清,都係爽呀!」

	將彼家私尽挫磨,不知爽法又如何?貪涼愛食生蘿蔔,只怕他時肚痛多。

	亞悌聞之,緊縐雙眉,搖頭歎曰:「你係爽咯,難為人苦得淒涼呀!」 鄉村間,或遇婦女投河吊頸,服毒身亡,其外家係好風俗、識情理者,可安然無事。若遇恃蠻恃惡之村,一聞此事,便紏率多人,呌齊个\ruby{的}{󰦦}强橫後生、撒撥婦人,疎者認為至親,遠者認為至近,有男有女,有老有少,如黃蜂出洞、猛虎下山,擦掌磨拳,呼天震地,大聲呌曰:「各人整定身勢,今日去擺人命呀!」東莞呌做食臘鴨飯有肉食,有錢駛,不論三七廿一,真假虛實,縂之要蠻可以做得。其中又有一兩个\index{攪屎棍}、\index{風爐扇}\footnote{風爐扇:話人無事生非},曉作幾句狀詞,識得幾个差役,自認有胆有識,村中稱佢做師爺,遂做主謀,從中撥弄,而一隊烏鴉黃雀飛去尋食矣。去到死者之家,如雀鳥歸巢、鵝鴨到埠,圩咁嘈蝦咁跳,話逼死佢个女、逼死佢个妹、逼死佢亞姨,詐哭得嗚嗚,含悲似切切,擠擠擁擁,風起塵飛,要捉死者之家婆抱屍,要捉死者之丈夫毆打。有\ruby{的}{󰦦}想牽牛,有\ruby{的}{󰦦}去捉豬,連雞仔雞春都煮熟食。又嫌豉油鹹,又嫌燒酒淡,又嫌豬肉肥。食完之後,各派封包,有\ruby{的}{󰦦}嫌輕,有\ruby{的}{󰦦}嫌少,認到至親至切,好多眼淚都無。一言不合,一事不周,即拋棄家私,毀破物件,要旁人講許多好話,要苦主認許多不是,要自己兜許多錢銀,尚詐作忿忿不服,其實欣然想去矣。腸肚飽矣,荷包重矣,隨路行,隨路講,隨路笑矣。平日與彼處眾相熟者,到此時亦不知醜焉;平日各稱為好相與者,到此時亦作反蠻焉。

	㗇㗇,真奇怪也!婦女未死之先,或饑寒或愁苦,為何無人來照顧?或死亡或孤寡,未必咁多人哀憐。一聞自尽輕生,你代不平,我更不服,虎威而至,蜂擁而來,如官差之來办大案,似盜賊之搶劫民房,無法無天,成何世界!獨不思自尽輕生,就架起大題,話翁姑逼死、丈夫治死。在翁姑豈有唔愛新婦?丈夫豈有唔愛老婆?不過因家庭細故,口角相爭,衣食之需,勤懶碎事,遂至你言我語,各負不平,怨怒憎嫌,私懷己見。為女子者,曉得身為婦道,應當孝順翁姑;內助之賢,必要無違夫子。就是諸多屈抑,還須自解愁懷;極地艱難,都望後來好處。何必一時忿氣,斷送終身?試思父母生你以來,費尽多少心血,用尽多少錢財,而後長大成人,嫁你作安身之計。早知你如此忘恩負義,不記父母劬勞,何不於你初生之時,投之河海,省了許多辛苦,免得今日眼水長流也。你話屈氣難當,怨翁姑刻薄你,怨丈夫難為你;似也,亦不過有時罵之,有時打之而已,安知自己尽合乎道理麼?其打罵也亦一時暴氣耳,過後可相忘,非真有用繩勒你頸,拖你推落塘,捧毒灌你口,如此逼法也。若非如此,不得謂之逼你之死也。非逼死也,自尋死耳,自賤而已。既自己想死愛死,又豈可以死累人麼?翁姑之娶媳,男子之娶妻,原望歸來孝順,掌理家庭,生子生孫,百年之計。是以一場慶鬧,不惜錢財。若早知你如此撒潑,爛命瘟屍,你即貼送大床,貼來花轎,人家亦不願要你矣!你一死易,執拾你難:要棺材,要殯葬,一家啼泣,失礼於人。你外家不知失教之羞,借女死作生財之計,逞威作勢,豈得為人?你之死也,生為\index{潑婦}之流,死作累人之鬼,九泉之下,罪實難容,而父母家為你添一重罪案矣。

	此風盛,大滅倫常。獨不思你有女嫁去人門,人亦有女嫁入你屋;你有女輕生,人女亦曉自尽;你去累人,人亦累你。冤冤相報,照樣而行,世界必至大壞。或有為之解曰:「所以累人者,無非要為女報仇,代女出氣也。」誰不知婦人水性,頭戴膏油,不識不知,原無遠慮。見慣外家惡氣,害得人多,有時因些小之事,忿恨不平,就生起死心,尋着死路。心內算曰:「我拚之一死,外家到來,要累你家散人亡,七零八落。」而真真死矣。是則女子可不死,而有外家累人之策,壯起个胆,割斷条腸,遂作催命符、勾魂票矣。照計起來,似非夫家逼婦死,而實母家催女死也。女想累人而死,外父母家又為女壻之對頭矣。此一說也,做女壻者,起人馬去妻之外家攞人命,要佢補囘一个老婆亦無不可。

	人平不語,水平不流,恃女死以累人,不平甚矣。若論平情之道,凡婦女有大冤大屈之事,難寬難解之情,則宜投告外家,稟公論處。在夫家之族,亦有老成明白之人,未嘗不可以調停,未嘗不可以排解。至於微嫌私怨,為父母者亦須教女勸女,而消散之。如若女性偏橫,竟尋短見,為外家者只可着三五親人,帶\ruby{的}{󰦦}寶燭往去吊香,尽哭泣之情,不許多端生事。此例一成,各鄉依樣而做,吾恐\index{潑婦}聞之,亦退縮曰:「我冇咁賤,就係死了,外家都唔共我出得氣,又唔累得乜出樣,我唔死咯。」你唔死,我唔死,一年畧計,天下救生一萬八千婦人。

	亞孝縱子弟去姓周捉親家婆打後,自謂爽神。親家公遠處歸來,見如此光景,勃然大怒,曰:「有咁樣惡法!我个新婦既死,已經傷心不了,重來毀我房屋,散我家私,將我老婆咁樣凌辱,有咁太過兇橫!佢恃拳頭在近,官府在遠麼?我就駛官府來收拾佢。」即時請人做一張狀,立刻告官。官即發票,出差三班縂頭,一齊到屋,重重圍住,捉了亞孝个班\index{脚色}。个个用鐵鍊鎖住頸喉,好似拉狗咁拉,拉得亞孝面青青一額汗。口想喊亞悌細佬來救,佢唔做得料駛,你不用呌佢。誰知差眾人多,呼聲震地,不由分說,亂打而行。到了官門,開堂審訊,周親家即來對證,所告無差。亞孝勉强支離,胡言乱說,話「親家自己打爛屋宇,來誣賴我,實在冤枉難招。」官大怒,發起威來,將各人每个重打一百。亞孝係喝令倡率,打二百板,更掌多二百嘴巴。審完,尽押入監房,後再定案。

	爽神何似在公堂,打得皮開嘴又長。鎖住頸喉拖你去,一羣羊犬入監房。

	官怒亞悌身居秀才,唔彈壓兄弟,任其放肆,恃惡欺人,欲將他詳革功名,將作文書,想詳上臺督撫。悌聞得,心內驚慌,親身去到官門求情乞免。縣官訪查其品度,果係品行端方,容情賞面。亞悌歸來,去拜候親家,千認不該,萬認不是。周親家體貼亞悌情面,是以不為催紙。此案丟開,縣官遂釋放亞孝等囘家矣。亞孝不知怨悔,惡氣猶存,對人曰:「奈得我乜何?好之又唔辦得我乜出樣,又要放我歸來。」

	人能知錯福非輕,亞孝而今禍未清。不肯囘頭思忍讓,一家從此起刀兵。

	亞悌聞之歎曰:「禍未了也,尚有甚焉,此後更難勸矣。」

	未幾而亞悌之母死,亞孝約亞忠、亞信唔來守喪,唔來着服。及送棺出葬,亞孝欄住,不許庶母葬於先父之旁。罵亞悌曰:「你老母係何等樣人呀?而敢葬在我父墳旁之右?唔做得!唔做得!快\ruby{的}{󰦦}搬遷,不許葬此!」

	嫡母死,為庶母之子者,着三年服;庶母死,為嫡母之子者,應着一年服,此通行禮也。今亞孝不為庶母守喪,是無禮矣。詩經曰:「人而無禮,不死何為?」亞孝又以庶母卑賤,不能葬父之旁。何以你父生時,能與庶母同床共枕也?亞孝不識人,非止眼盲,而且心盲矣。

	亞悌另尋一處地方,埋葬結塚。又一年,而亞孝之妻死,亞悌招亞仁、亞義同去尽禮。仁、義曰:「我前者老母死,佢都唔來着服;今佢老婆死,我要共佢守喪?我有咁\index{蠢才}咁下作麼?」亞悌再勸之,兩人不答而去。亞悌見細佬不從自己,到喪家堂俯伏而哭,哭到極哀。不是哭大嫂之死,實係哭兄弟之不賢也。亞仁、亞義在隔牆飲酒吹蕭,亦未免太過亞孝聞之,怒曰:「大嫂死,為叔不來守孝,已不成人;又飲酒吹蕭,整成咁快活!」即喝起亞忠、亞信,各執棍去打他。

	老婆死去淚交流,庶母因何作對頭。只曉罵人唔罵己,弟兄原是一羣牛。

	亞悌先行,亞孝等跟隨而去。亞悌入仁義之家,以眼角斜丟一下,露出个意。亞仁醒覺得快,急從橫門走出。亞義走不及,想跳過牆頭,亞孝在背後以棍打其腰。亞義翻跳落地,亞忠、亞信拳棍交加,好似亂捶大鼓。亞悌以身遮掩,欄住亞孝等,曰:「亞哥,唔好打咯,打咁多好咯!」亞孝喝罵曰:「亞悌,你幫住細佬嗎?」亞悌曰:「我不掩弟之過,亦不助兄之暴。吹蕭飲酒於禮不宜,然罪不至死;輕輕薄責,足以做戒前非。若以細佬作肉上之砧,我心實見不忍。若要再打一番,我情願將身抵罪。」亞孝曰:「就打你奈乜何?」遂向亞悌乱捶乱打,好似彈花。亞悌斂手低頭,由他洩恨,驚動左隣右里來勸,紛紛各自散去。亞悌扶住条棍到亞哥處請罪,亞孝曰:「你\ruby{的}{󰦦}都係唔好\index{脚色},同个一流人。勿來混賬,快\ruby{的}{󰦦}走去,不許在此居喪。」亞悌歸家,垂頭而歎。

	好人難做好人難,難處之中忍一番。要做神仙先受劫,幾經磨練脫塵凡。

	亞義既受重傷,不能飲食,眠在床上,呌痛難當。亞仁代稟告官,又告其不為庶母着服。官即出差去捉亞孝兄弟,又要亞悌到案秉公。亞孝等慌起來,避藏密處,縮在房間閣上,隱伏缸中。

	恃惡何須密隱藏,只因曾打在公堂。雖然口硬心猶軟,不若藏身在甕缸。

	亞悌因被毆之故,頭面損傷,眼痕腫黑,難以到官門對答。因作一張狀詞,稟覆太爺,哀求止息,免受吊審牽連。官順其情,遂消此案。亞孝等出來村前,又揚揚得意矣。亞悌埋\ruby{的}{󰦦}跌打丸、散瘀藥、木耳煮酒,送與亞義飲之、食之、搽之、敷之。十日之間,傷痕好了。因此一告也,亞孝因之與仁、義仇恨更深。

	仁義皆幼弱,常時要受亞孝兄弟欺凌,遭其毒打,仁義怨亞悌曰:「人皆有兄弟,我獨無!」葢嫌其唔來幫打也。亞悌曰:「此兩句說話,在我身份極合,非細佬之言也。」因力勸仁義要低頭順受,而仁義不從;勸亞孝等要平心為好,而亞孝不聽。亞悌自知難以勸化,遂關埋門、帶銀錢、攜妻子往別處安居。遷去一處地方,呌做義堂,離家有五十餘里,免得日見打鬧,而多添煩惱也。

	帶妻攜子往他方,別作生涯自主張。兄弟是非難到目,清風明月一炉香。

	亞悌在家,雖然唔幫助仁義,亞孝兄弟依然畏忌三分。見亞悌遷居,自後些少不平,兄弟登門打架,拳頭奮起,就將仁義毒打一場。仁義兩个,自知年紀尚輕,唔係佢敵手。欲喊胞兄而亞悌相離得遠,大呼天地而鬼神詐作唔知。左想右想,料得終難與抗,於是無事之時閉門抱膝,似避黃蜂之刺,如妨顛狗之追。出則懷刃在身,提防不測。若使他來打我,便當刀向而前;絕路窮途,豈肯甘為罷手。

	今人稱父之契仔者,呌為蘭兄蘭弟,意氣頗相親愛,恩情似勝交遊。以父所契者尚作為親,何況我父所生者,豈可作為仇敵?世人心意,日望生兒,生得一子,珍之寶之,而猶有慮曰:「可惜獨得一个。若生多三兩个,就係有人欺佢,佢有幾兄弟抬手幫扶,唔駛被人作咁熟肉。」今者曾恭禹生仔一兩个矣,再生至三四个矣,又生至五六个矣,唔慌人來欺你个班仔矣。何以人唔欺你,乜你自家欺自家?是當日生多幾个兄弟,實係生多幾個對頭也。生多幾對手足,實係生多幾對刀鎗也。執刀鎗以殺賊,不聞執刀鎗自斬手足也。家養幾隻狗仔,尚見其同眠共食,情趣依依;即使分賣鄰家,東一隻西一隻,未必東之狗仔,登門尋西之狗仔來打也。今亞孝兄弟,與仁義為仇,不但登門要打他,即路上相逢亦打他。就係席上飲酒講起亞仁亞義,火忿起來,想放落酒杯即時去打他。至於睡在床上,想起亞仁亞義心懷不服,就拍起枕頭,終須要打他。要打到佢眼腫,打到佢頭穿,打到佢血流,打到佢骨軟,要佢喊救命,要佢怕亞哥,要佢伏眠在地,要佢唔出得門。而我氣平矣,而我神爽矣,而周身安樂矣。

	嗟嗟!孔懷兄弟,不是他人。囘想父母生仔,提攜保護,寶如金玉,豈作泥沙?見仔跌倒在地,忙忙抱起,摩弄一番,與笑與言,憂其驚嚇。有時見仔不合,激惱於心,咒罵喃喃,未肯即執棍打。就打幾吓,尚且從輕,仔之肉未有傷痕,而父母之心痛不了矣。何也?仔之身,父母血肉分來也。今亞孝之毒打仁義,非打細佬而實打父母也。仁義之懷刃於身,非斬亞哥而實慚父母也。既不念父母之心,大傷父母之體,問你清明拜祭,上到墳頭,整成恭敬奉承,奠酒三杯,禮行九叩,猶且自贊歎曰:「祖宗有福,發出咁多人。」誰不知家運該衰,然後出得你个班無用子也。此等兄弟,豬狗不如。

	又說曾恭禹結髮原婚所生之子,名呌亞成,在賊中逃出,帶一个老婆歸來。亞孝兄弟以家產同分,聚謀三日竟無安置之方。亞成無所倚賴,仁義兩个就留在家酒肉供奉。亞仁往去投告亞悌知之,亞悌不勝歡喜,即走歸來。相見深深一拜,曰:「大哥歸來麼?好咯,好咯!這位就是大嫂嗎?」又拱手一揖,即問:「母親現在如何呢?」亞成答曰:「老母死已久矣。」亞悌聞言,不覺低頭欲淚,歎息幾聲。亞成又曰:「賊中搶得婦女,我認一个為妻,今帶歸來還居故里。又不料失我之後,父母再娶,生得兄弟多人,算萬幸咯!」亞悌是晚出錢捉雞,一室同歡。去請亞孝兄弟來飲,各推不到。飲後共坐傾談,將數十年世事講及一番。

	第二日,亞悌對亞成曰:「大哥,你不須憂。弟今遠在他方,其屋舍就送與兄嫂安居,無庸另擇。至於田地,我亦不過每歲收租而已。我今在外,幾好撈頭,衣食飽暖,唔志在此。我將田地送與亞哥,永遠耕管,不用交還。」亞成曰:「我有應得之田,無用你自捐出。亞孝想學蠻梗,作我做外人麼?我就告佢何難?打佢亦易。」亞悌苦勸曰:「大哥、大哥,千祈不可。萬事不過求其安置,今弟以田宅相奉,出於至誠,并非虛話。大哥如果不從所請,此後亦無相見矣。」亞仁亞義曰:「我亦願出田地幫助大哥,大哥都要順吓細佬為是。」亞成曰:「你三兄弟既此真情,我就忍住啖氣罷了。」个啖氣終須要出。亞成由是有田耕、有屋住咯,亞悌亦囘了義堂。

	亞孝兄弟到仁義門口罵之曰:「亞成哥係眾兄弟大哥,不是你自己大哥呀!事要慢慢斟酌,自有方圓。三日冇主意,唔知慢到幾時呢?駛乜你咁居功,另為幫助?你又幫助\ruby{的}{󰦦}呀。唔通淨係你做好細佬,我就唔好細佬嗎?」仁義默然不答。亞成聞之,走出來曰:「㗇㗇,又新樣呀?豈有此理!我身為長子,做一个大哥,數十年相別,今始歸來。你三兄弟唔請我食一餐、留一宿,佢見你帰來,慌你爭佔田地,佢重請你食飯麼?佢想你死了更好。感得三个細佬,與田我耕,與屋我住。你等尚唔知醜,走來怒罵,你想趕逐我嗎?抑或想打過我呀?」話完,火氣沖天,手捧一件大石,向亞孝打去,打中亞孝个身。亞孝轆倒在地,大聲喊:「救命呀!」亞成舉拳頭亂捶其背,曰:「打死你!打死你!」

	既知自己無情義,何必登門再逞刀。激起大哥唔抵得,拳頭相打不相饒。

	亞忠亞信看唔同勢色,即時扎起䯻氏\ruby{的}{󰦦},捲起衫袖,合手合脚來打亞成。亞成發起威來,手招脚跳,演出工夫仔,井井有條。亞仁亞義一聞𨳒聲,亦執棍齊出。幾兄弟打得落花流水,大戰一場各兄弟老母若係在生,見此光景定必哭破喉嚨。原來亞忠亞信練過十年武藝,拜過師傅,食過夜粥,打過沙袋,埋過生樁,手段高强,素稱無敵。唔怪得亞孝咁恃勢。誰不知亞成自幼充入賊營,殺人不知多少,生得又高又大,其兇暴之氣百倍於人,數十年能征慣戰,胆力俱高,亞忠亞信点能抵當得住?戰了數十回合,亞成用一道毒蛇捲尾之法,轉身用脚一勾,亞忠跌倒在地;又用一道魁星踢斗之法,出一脚打上胸前,亞信跌離丈遠。忠信哭呌曰:「大哥,饒手咯!算我怕你咯!算你贏我咯!」師傅教工夫,大哥來踢盤。所謂勸君莫逞强梁性,恐怕强中更有强。亞成向每人再打幾拳,鄰里來紛紛勸住。

	哥哥暴戾弟兇橫,骨肉俱從父所生。料想曾公輸教訓,只知生仔買田耕。

	亞成先往告官,訴明自己原委之處:今逃走歸,亞孝等唔肯分田地與我。官曰:「你既有細佬做秀才,自應呌佢到來秉公理處。」官即使人去請亞悌。此時亞悌聞得鬧出大事,即走囘家,與官差同去。既到公堂之上,淚流滿面,不出一言。官曰:「家庭之事,你尽知之,究竟你如何主意?」亞悌低頭拱手曰:「小生員不能調處骨肉,枉讀詩書,自愧庸才,毫無中用。縂求老父台公斷便是。」官曰:「此亦易事,就將你父所遺財產,七份分開,有何爭執呢?」官既判完,亞成與亞悌共路歸家,將田宅分得清清楚楚,亞悌回義堂去。自此,仁義與亞成倍相親愛。

	一日,講起從前母死之事,亞孝兄弟

	咁樣刻薄無情。亞成大怒曰:「如此不仁,是禽獸也!」亞成雖暴,尚曉得道理。要擇吉期即為改葬。亞仁走告亞悌,亞悌歸欲勸止之,亞成不聽。呌亞孝兄弟來,吩咐曰:「你太可惡!前者庶母之死,你不着服居喪,又不容庶母葬於先父之側,是何道理?」亞孝等不敢出聲,只顧低頭,似龜咁縮。亞成曰:「既往不追,來者可諫。今擇某日啟土,移棺遷葬於父旁,你各人要着孝服相送。」話完,以刀削樹曰:「如有不遵吾教者,與樹一般!看你頸硬,抑或我个張刀利!」亞孝曰:「自不然呀!應份要送。」亞成曰:「去送了麼?要着孝服。」亞孝曰:「我知到咯,着个件白麻衫。」到了遷葬之期,男婦大小相送。亞孝故意曲\ruby{的}{󰦦}腰顧低頭,慌亞成怒佢冇孝心,拭\ruby{的}{󰦦}口水做眼淚,惹得路旁人都笑。既葬之後,自此兄弟相安。

	但亞成之性太過剛烈,各細佬有不着處即動手打,而於亞孝更打多\ruby{的}{󰦦},葢憎其無情無義也。最敬重亞悌,當盛怒時,見

	亞孝所做事務,每多不合亞成之意,所以亞孝不滿。十日去探亞悌一回,有時靜對亞悌咒罵其兄,話:「亞成哥好死唔死,又走歸來。遇時將我凌辱,話我暴戾,佢重醜過我十分。」亞悌婉轉諫之曰:「究竟都係佢做亞哥呀,亞哥火氣大亦要忍讓下。佢有時自己都有唔着之處,豈可尽怨他人麼?」亞孝曰:「佢做亞哥好出奇嗎?大約我重先做過佢,佢\ruby{的}{󰦦}死剩種,罵得咁毒實係好彩得歸來,重來講惡氣,你話服佢唔服呢?我雖然惡,何嘗有將亞忠亞信日日來打呀?不過專打亞仁亞義而已。我打細佬都有,仍然依住道理去。無理認有理,豈有此理。獨至亞成哥,唔係人咁稟,恃自己高大,動不動講拳頭,你話有乜法呢?」共佢打過呀。亞悌曰:「我有一法,惟和平恭敬,日久可感其心。你話大哥兇橫,何以又唔打我?」亞孝曰:「你離得遠,而且咁斯文,唔通將紳衿來打麼?」亞悌勸了幾番,亞孝都唔肯聽。遲得幾日,亞忠、亞信來投告。又遲幾日,亞仁亞義亦來投告。

	更廿日間,亞成自己來探,曰:「細佬,我想唔做大哥咯。唔做亦極之難,个班細佬更加放肆。我有時火起,縂之用拳頭做家法。至於亞孝更可惡,我冇肯容過佢。」亞悌曰:「大哥不宜怒氣,个\ruby{的}{󰦦}細佬点能學得你咁明白呢?明白得凄凉。細佬唔明,慢慢教道。大哥拳頭重,自己唔知,恐一時打傷,骨肉之情,心有不忍。就是父母在九泉之下,亦有難安。」能體則親心,必能愛到兄弟。話完,不覺眼淚滴下。亞成歎曰:「細佬个个唔學得你呀!」兩兄弟講話一番,陪待飲食而去。不數日,又有兄弟來投告。一月數次勸諫,亦不依從。亞悌見無奈之何,不如三十六着,又以走為上着。即將家眷搬遷去三泊,離家百有餘里。路途遠隔,是非不聞,自尋安靜而已。

	善言俱作耳邊風,我亦從今詐耳聾。拍手又攜家眷去,買園三畝種通葱。

	眾兄弟等見亞悌秀才遠避,雖有委曲之處,難以分憂。論起亞成做事頗公道,縂係帶躁暴,唔函養得到,所以个班細佬多怨怒。今亞悌既往了三泊,家中所有大小事務俱以亞成大哥為主,不得不要怕他依他,而順承他。習久相安,亦少爭競矣。

	又說亞孝之年,有四十六歲,結髮妻生二子,妾氏生二子,隨又收起一个婢做妾,生一子,共生五子。長子繼業派第一,繼德派第三,此兩个係結發所生;繼功派第二,繼績派第四,此兩个係妾所生;繼祖派第五,此一个係婢所生。五子皆有家室,添得幾孫,村中有人稱亞孝做多仔公,又為好命公矣。

	亞孝一生做出咁多德業麼?咁多功績麼?若係生一个仔,難以承繼得完;妙在仔多,分開一人繼\ruby{的}{󰦦}。誰不知个班仔,性情暴戾,了不可當,个个俱能繼父之志。只有第五仔改名繼祖,不肯繼父而繼亞公,其餘皆學足亞孝規矩。所以古人有詩云:兄弟同居忍便安,莫因毫末起爭端。眼前生子又兄弟,留與兒孫作樣看。所謂有樣睇樣,學翻个形像也。

	一日,繼業話繼德曰:「細佬,我兩兄弟係大婆仔,佢三兄弟係細婆仔。本心之講,我着硬邊呀!恐怕骨多骾喉。就係欺佢打佢,佢有乜出尺呢?」繼德曰:「着咯,着咯!唔駛畀情面佢。佢呌我做亞哥,都唔好應佢。」你咁樣無情,恐怕當之不起。繼績聞之,亦話繼功曰:「亞哥,今者繼業兩兄弟會埋想來欺負我,唔駛怕佢!佢有細佬,我亦有亞哥;佢有兩對手,我亦有四隻;佢拈銅鞭,我執鉄尺,你慌駛輸過佢麼?睇來頭湊,唔似陣勢,一齊動手。」好似戲棚个\ruby{的}{󰦦}花花公子一樣。繼功曰:「自不然呀!我大早有此意,未有話你知。今講起來,不可不慮。你實在未有憂慮,就係殺死兄弟,可能了得事麼?我前日買定一張單刀放在床頭,遇時預備要用。佢若真來尋打,就先下手為强,免至受虧一着。」於是大婆仔結為一黨,細婆仔又結為一黨矣。家運衰到个樣子。

	獨至繼祖,係婢所生,並無同胞兄弟。母又早死,自己年輕,四个亞哥每欺凌佢。亞孝見幾个仔遇時嘈鬧,彼此不和,因罵之曰:「你兄弟点解得咁暴戾呀?兄不愛弟,弟不讓兄,你聚為一啚,我結為一黨,相憎相厭,似殺父之仇,成何規矩!你兄弟不尽同母而生,亦皆同父而出。曉得連枝同氣,當念手足之情,為何情義俱無,只想尋仇作對?你等將來亦有子孫生養,照樣學你,豈得呌做為人?」極好道理,實將自己大罵一場。个班仔答曰:「我非拜他人做師傅,原來學你之所為。父道而兼師道,喃魔先生教仔,尽符尽法。好之你會埋三叔四叔,專去欺五叔六叔,你想吓自己點樣好法呀?只曉得罵人,唔罵自己。」徒弟惡過師傅咯。亞孝聽聞幾句說話,即垂頭無語,長歎一聲而去。

	從前只管欺兄弟,子亦而今有弟兄。相打相爭如一陳,拜師學足我無情。孟子云:「身不行道,不行於妻子;使人不以道,不能行於妻子。」亞孝之謂也。

	又亞孝第五子,名繼祖。其外父外母家附近三泊地方,繼祖一次去探外父,順便拜候亞悌二叔。亞悌生得三個仔,大仔係秀才,名呌繼善,餘二子尚幼,亦讀書。

	亞悌一生好處,見善必為,又欲其子繼之。改為繼善,善愈添而福愈厚矣。若亞孝之諸子兇橫,改之為繼惡可也。

	繼祖來探,見二叔之三子,兄弟怡怡,相親相愛,父慈子孝,兄友弟恭,瑞氣一門,家庭歡樂。

	詩書男子婦桑麻,瑞氣融融聚一家。門外半生歡喜草,階前多種吉祥花。

	繼祖住了幾日,不願歸來。亞悌催他囘家,繼祖求寄居在此。亞悌曰:「你慌我冇飯過你食,冇屋過你住麼?因你父唔知,於理不合。你歸家稟明父母,然後來此未遲。」因亞孝正室雖死,又續娶囘一繼室也。繼祖由是回家。到了十月外父拜壽,繼祖勸妻曰:「我前者到二叔處,見其父子兄弟,和氣一團,十分快樂。今者岳丈壽旦,我與你恭祝之後,往二叔處住,永不歸來。未知你意如何,以為好否?」其妻答曰:「我見幾个伯爺如此拂戾,縂不同人。無論男子不情,即婦女亦不順,一家暴氣,何日能消?將來必有凶灾,爭在幾時發作。論起翁姑,生平薄德,而伯爺幾輩更甚兇橫。俗語云:『積善之家慶有餘。』吾恐君之家,五禍臨門矣。見機而作,不可延遲,吾恨無翼以高飛,斷不願久居此土也。」五个新婦算至明白係繼祖老婆,一家之中除亞悌,亦以此婦為第一。繼祖遂稟知其父曰:「兒無同胞骨肉,每為兄輩欺凌。今與妻往外父處祝壽,順探二叔,不歸來矣。」亞孝曰:「我與你二叔前有微嫌,恐難久住。」繼祖曰:「二叔非他,係聖賢人物也,豈記從前小怨麼?」亞孝曰:「細仔呀,我知你屈氣咯。个\ruby{的}{󰦦}\index{龜蛋}唔中用,我來教佢,佢一句頂住我喉嚨,好似橫吞欖核。生鵝喉都唔定。話佢唔聽,打佢唔贏,鬱抑憂愁,何處可寬懷一二。你既得棲身之所,還須要奮志做人,學二叔之所為,勿學你父,老來方悔也。」話完泣下,父子洒淚而別。

	舍愁難解倍心酸,戾氣遙知禍滿門白鶴高飛雲外去,任他雞鬦與鵝喧。

	遂帶老婆去祝壽,往探二叔,亞悌不勝歡喜。掃屋與居,使他從長子繼善讀書,學習文章詩賦。繼祖極聰明伶俐,苦志專功。讀了數年,文思大進。與善人交,如入芝蘭之室。亞悌見他有用,代佢捐一个監生,以勵其志。

	又說自繼祖遷居三泊之後,而家中兄弟怨罵尤多。亞孝詐作兩耳塞聾,低頭悶坐。聾早二十年真正好咯。繼功之母,庶妾也。一日,與繼業之妻爭論油塩碎項,繼業聞之,忿忿不平,接口罵曰:「你做家婆,駛乜認得咁正呀?我老婆話剩都未到你話。唔通工夫你老婆做剩,然後到佢做麼?你咁就整成裝模作樣嗎?你好声色咯!我劝你唔好講咁多,講得多你有錯!」你\ruby{的}{󰦦}說話就先錯了。罵得庶母兩淚交流。繼功忽然來到,聽聞如此怒罵,勃然變色曰:「大約我老母个\ruby{的}{󰦦}說話都是平常,冇得罪你老婆呀!照事講事,駛乜講声色唔声色呢?我老母唔声色,唔通你好声色麼?」繼業曰:「細佬,你大約想打過嗎?」都有幾分意。繼功曰:「想打唔打,要我自己知。對人之子而派人老母不是,實在唔服。」繼業曰:「你唔服点樣呢?」繼功曰:「要罵你!」繼業曰:「唔許你罵点樣呢?」繼功曰:「唔許我罵都要罵,唔通攞得翻?」講到个句說話,誰不知繼業粧定身勢,扎起䯻氏\ruby{的}{󰦦}。繼功亦抽高褲脚,捲實衫袖。繼業撒手曰:「不必講、不必講,打過分道理。」繼功曰:「就話打,怕你麼?」

	性如蟋蟀近中秋,亂呌聲聲惡氣福。今日相逢難罷手,拍身抽勢就埋頭。

	繼功劄定子午馬,繼業劄定四平馬。繼業一拳打向頭來,繼功用左手招開,右拳打囘繼業乳旁之側。繼業轉馬側身進前一挨,用手撥開,順拳撘上,繼功正額眼中水火都標。打交工夫學過幾年,孝弟工夫一毫未學。繼功自料力不能當,閃身就走,跑囘自己屋內,摸着床頭个張單刀。繼業知繼功囘取利器,自己亦發脚走囘家,尋着一雙鉄鐧。誰知繼業執鐧出門,繼功來到門口等定。見繼業出來,尽勢一刀攏去,此刀算利,亦算好駛。肚內流腸,滿地鮮血,大呌一聲而死。此時唔打得咯,唔好睇咯。

	是日適值圩期,男婦多去投圩,連繼德繼績亦不在屋。兄弟相打之時,婦人呌喊,而鄰里左右見他兄弟遇時打慣,當作平常。工夫純熟之至。豈料出起刀來,救之不及。宗族聚議,即將繼功捉住,捆綁鳴官。此時理應出工夫仔,要用拆法。眾口一詞,不能不認,重打數百,押入監房。單刀放在何處呢?照律殺兄之候,應議凌遲定罪;不料繼功染病,又因重受官刑,元氣大傷,忿悶而絕。監牢身喪,戾氣消沉,嗚呼哀哉,同歸一尽!兩兄弟唔耐打。

	又說繼業之妻馮氏,繼功之妻曹氏,兩人不同居也。馮氏每日到曹氏門前咒罵。一日罵入屋內,曹氏惱不能堪,出聲答曰:「㗇?㗇?你家男子死,我家男子生麼?你冇丈夫,我亦守寡,大眾都同一苦,你何為來罵我呀?」馮氏曰:「你唔好老公,斬死我老公,我要問你攞翻个老公!」

	一句老公,兩句老公,句句都係老公。你既愛老公惜老公,何不勸諫吓老公,開解吓老公?床上睡時,細心化導老公;門前罵時,尽力攔阻老公。呌老公忍氣,呌老公平心,呌老公保重自己,呌老公饒讓他人,然後老公不至鬧事,老公不至傷身。常得見老公,唔憂冇老公。若平日唆擺老公是非,當時任由老公打鬧,過後悲切老公唔在,許你點樣痛老公、念老公,都係呌做唔要老公。

	曹氏曰:「你講咁蠻咁惡,唔通想打過麼?」馮氏曰:「就講打都唔怕你。」話完即抽身抽勢,扎緊隻䯻,一拳打向曹氏面上。曹氏雙手推開馮氏,又尽勢撲埋來,推跌曹氏在地,頭披䯻散,覆面橫眠。馮氏快騎上背脊,伏低乱捶乱撼,以手扭佢耳朶,用口咬佢膊頭。寫得女人打法,情景極生。曹氏伏在地,氣嘈嘈眼白白,頭搖䯻乱,詐啞不出聲。原來馮氏生得高大,身駕重\ruby{的}{󰦦},所以責住曹氏唔轉動得。曹氏咬牙抵住痛,停一息間,覺馮氏氣帶嘈力帶倦。曹氏努起勢來,尽力反起身,望見枱面有一張菜刀,順手執來,照面削去。馮氏閃避不及,頭壳破開,鮮血滿身,登時倒地而死。曹氏知事不能了,即走去井邊,向頭落井而死。慌死唔得快。

	亞孝死了兩个仔,又死了兩个新婦,哭到傷心,愁懷滿腹,低頭無語。自怨前非,無片善之及人,積餘殃之累後。所謂福無重至,禍不單行也。尚未行得尽。

	一生惡氣難消受,留與兒孫作抵當。死得傷心如此慘,本來肚內有刀鎗。

	誰知一波未平一波又起且說繼業之外父,呌做馮大立,痛恨女之死亡,而發怒曰:「我女壻既受刀亡,又將我女殺死。唔通佢做家婆,縂冇家教,只曉得飲醋而已?」呌各子侄到來,吩咐各執銅鞭鉄尺懷藏身,曰:「去捉親家婆,打佢一身,罵醒佢心,拭開佢眼,丟過佢駕,然後心甘。」

	你個女既死,人之子亦亡,付之大數便了,可以無事。偏要去生事、滋事、惹事,鬧至累出大事,呌做一番招累。本無累也,而去招之,究竟有何所謂?眾子侄跟尾而去,一个二个,陸續而來。曾亞孝之家亦不知來尋打也。出其不意,捉住亞孝老婆,即時脫衣乱打,大聲喊「救命」。亞成走出來怒曰:「我家死人如麻,你重來找我晦氣?」喝起子侄,各執家伙而出。或持刀,或駛棍,蜂擁蚊喧,打得馮氏各人失魂而走。自取其灾,謂之解衣包火。亞成捉住馮大立,割去双耳。大立之子走囘護救,被繼績一鉄棍掃來,打折一脚。馮氏一班子侄各有所傷。問你爽唔爽呢?馮大立掩住双耳,血淋淋面青青,好似鬼追咁跑。甘心唔甘呢?剩下个仔,被打折脚,眠在路旁。此時定必大声哭呌:「亞爹呀!」亞成使人用大睡板抬囘馮氏村邊,放下急走囘矣。

	此件事,馮大立大有不該,有自取之罪。在亞成等,屢經打鬧,人命傷殘,亦當饒讓三分,忍頸就命。就係將亞孝老婆打了幾吓,未免受眼前虧,都係唔抵咯。然有咁多子弟可以欄阻得住,未必真正点樣凄凉。既不與講情理,喝出家伙打之,而馮氏飛跑而奔亦可以罷手,為何又切去耳、打折脚,剩\ruby{的}{󰦦}手尾來跟呢?縂之暴氣未消,必要大經折挫一場,方肯囘頭心息也。

	亞成呌繼績先到縣,將此事情稟上。惡人先告狀。話馮大立登門尋打架,自己裝傷。而馮大立之狀詞亦到,話帶子侄去吊香并問原委,誰知佢發起怒來,將我父子打傷,如此如此。官大怒,既發三班差頭,去捉亞孝全家。五更早來,四面圍住,此時亞成要喝起子弟出家伙為是。所有男人尽行捆綁拖去,只有亞忠走脫出來。亞成個班\index{脚色},捉入官門,打得昏天黑地。

	任你拳頭勝鋼堅,官炉有火不須烟。鑄鎔你\ruby{的}{󰦦}兇蠻氣,鉄骨銅筋軟似棉。

	打了一堂又一堂,受了幾番痛苦,押入監內。衙門罪犯,凡入去坐監者,必要買通監口,進奉錢銀,然後掌監及老犯之徒唔難為你。若無銀孝敬佢,就捉住你非刑吊打,打到你願出銀為止。如果打過十次八次都有錢銀,不用打咯。亞成等人監中,並無人來打点,打交乜得咁多人呢。錢銀冇得應用,所以打到險死還生。一日,掌監禁子喝起老監賊,將亞孝父子、兄弟、叔侄,一个二个用繩吊起,似廟內灯籠一樣。个班老監賊,你又打,我又打,有\ruby{的}{󰦦}打頭,有\ruby{的}{󰦦}打脚,打得這个喊「苦呀!」那个喊「苦呀!」父哭嗚嗚,子悲切切,叔呼罷手,侄乞求饒,而禁子愈打愈多,哭聲愈呼愈慘。兄不能救弟,弟不能救兄,骨肉之間,惟有你眼望我眼而已。

	監中打到各魂消,哭尽千声不肯饒。叔侄弟兄空眼望,臘腸吊起一条条。

	亞忠直走去三泊,求亞悌二哥來打救。將近到門猶不敢入,畏其憎惡己也。剛剛遇着亞悌,同其子繼善、其侄繼祖,三人入秋闈滿三場而歸。亞成等剛剛遇秋審,打了三堂,尚未得歸。望見亞忠心神尽喪,亞悌驚曰:「細佬,你由何處而來?」亞忠即跪在地,亞悌更加大驚,執手入廳堂之內。亞忠細談端\ruby{的}{󰦦},尽將原委告知。亞悌嚇得一頭汗曰:「如此奈何呀!一門暴戾,早知其禍久矣。無奈好多兄弟唔知。若非因此,我駛乜來此遠避呢?但我離家既久,與縣官無声氣之通,如今走去求情,贃得羞辱。但得馮親家重傷而不至於死,我三人或有一个中舉,此件案可以易得維持。如或不然,真費手矣。」乃留亞忠在此,晝與同餐,夜與同寢。亞忠感其恩惠,覺有悔心。又住十餘日,見其父子兄弟,和藹春風,一堂雍睦,不覺悽然下淚曰:「吾今而知前者之非人也。」亞悌喜其悔悟,樂教導之。

	及至九月初十,省城開榜。報到亞悌父子同科,繼祖亦中副榜,不勝之喜。

	新春門口對云:「安居之宅春常在,積善之家慶有餘。」亞悌之慶有餘,兩父子中舉中到剩,繼祖跟尾,執而拾之,尚得个副榜。可知與善人同行都有益也。

	生平忍讓受虧多,父子榮登共一科。天眼既開人眼見,兒童拍手笑呵呵。

	明朝科甲極重,凡登科者合邑生光,官府為之敬禮。亞悌與子侄入拜縣官,縣官大加賞面。亞悌即向縣官求情,稟曰:「治下个處,自己之賤兄弟一時暴氣,鬦毆傷人,原情定罪,律不能寬。但骨肉相関,安能坐視?求老父台處大開法網,賜以仁慈,不追既往之非,許以自新之路。某等不勝惶恐,無限沾恩。」官曰:「此亦易事,放他何難?但兄弟歸家,須宜約束,不可依然放肆,再犯前非。」亞悌歸家,復往馮親家處求情,自認不該,望為勿怪。又贈金銀藥物,作補醫理之資。大立心雖不甘,而見其貴勢炎炎,難與相抗,况又求情尽禮,事許從寬。而亞成等一班\index{脚色},俱放歸來矣。

	亞悌一見亞成,即走上前叩頭見禮。亞成大聲曰:「細佬,恭喜咯!皇天有眼咯,唔虧負你咯!你一生好相與,肯受虧,念骨肉之情,尽中和之道,唔怪得天庇你。自己中舉,仔又做舉人,連到個侄去你處住,教佢讀書,都中了副榜,你个点善氣了得咁大麼!大贊一番,識出亞悌好處。亞孝一世冇人緣冇情份,至薄倖做了,至反骨做齊,个\ruby{的}{󰦦}罪孽積埋,累到兩个仔、兩个新婦如此死法,連累到我一班兄弟子侄,重受官刑。大罵一番,詂出亞孝醜處。你話為善好呢,作惡好呢?打亦打得多,鬧亦鬧得多,惱亦惱得多,苦亦苦得多,究竟想來,都由自取,連自己都罵。以至人財兩失,雞犬不寧,為鄉里所憎,為親朋所笑。反不如細佬,隨隨便便,安靜無事,重快活過神仙,唔知幾得意也。你都知道麼。細佬你勿去三泊住咯,快\ruby{的}{󰦦}搬家眷歸來,兄弟叔侄有時坐埋,講吓道理,免至淨曉得一身蠻氣,被他人笑作馬牛也。」

	兄弟閒居聚一羣,不談惡氣講斯文。而今願曉人間事,禍福因由点樣分。

	亞悌曰:「大哥,我歸來亦易,但恐兄弟唔聽我勸,終何用呀?」亞成曰:「細佬歸來,各人以你為主,你話打便打,你話唔打就唔打,務宜要依你。誰一个敢不遵從呢,我斷唔肯佢。你若不信,各人要在當天盟誓,以表真誠。」亞悌曰:「如果兄弟同心,家門之福咯!」亞悌由是帶家眷囘來,燕飲幾日。亞成呌齊一家男婦,齋戒沐浴,焚香告天:從今以後,願改前非,所有嫌疑,冰消瓦解;家內一團和氣,彼此相安;好事多為,以求福蔭。稟完之後,紛紛叩頭,同坐大廳,分開男婦,各行尊卑拜跪之礼,喜色融融。晚晚在廳堂,男婦齊集,聽亞悌講家庭世事及古來忠孝賢良,抵掌而談,生氣勃勃。講到悲歡離合之處,令各聽者眼淚都來,方知天地鬼神,其禍福消息之機,原來如此。又聽到古今來有咁多好人物,想起從前大小,原是不成人也。講了半月,男婦之心變了八九,其惡氣消了八九,於是出見外人自覺羞顏矣,不覺低頭矣,久之而生和氣矣。又久之而有喜色矣,幼知敬長,而父知教子矣。有\ruby{的}{󰦦}稱亞悌做家先生,而且作生菩薩矣。

	及後,亞悌之長子繼善出仕做官,而幼子繼福又中鄉科一榜。一門之內幾代功名,天之愛善人,厚待如此。

	亞悌共七兄弟,手足如此其多,而心腹並無一个。假使眾兄弟尽如亞悌之意,其家興發不知如何。假使亦如眾兄弟所為,人物死亡,不知何底。想當日曾恭禹而生七子,自稱好命,人亦贊其好命焉;只知贊好命,未有贊其好仔也。其仔不好,命亦不好矣。且多仔不如少仔矣,有仔不如無仔矣。何也?一者費心血、破錢財,二者添煩惱、惹羞辱也。何幸生得个亞悌,係秀才而做芫茜葱、做香头也。假使亞悌自恃秀才,練成狀棍串弄衙門,而亞孝之身家家破矣,亞仁亞義个\ruby{的}{󰦦}惡氣,如虎生翼矣。亞孝之女死,馮大立之女死,兩个親家告發起來,有一場官府仔鬧吓矣。兄弟之蠻惡,加以紳衿之把持,生出無限風波,害人害己,而曾恭禹之祖德宗功,孫枝奕葉,一掃光矣。誰不知亞悌之做秀才,學聖賢之秀才也,講情理之秀才也,積福澤之秀才也。以倫理為真,以心田為主,任兄弟之鴉爭鵲噪,自己鶴立雞羣,亞婆心,赤子性,含情不怨,菩薩低眉,行委曲以圖存,真秀才中之表表者也。究之興者自興,敗者自敗,天亦難容惡業,惟伯善人。到底兄弟都以亞悌為好人,想去想來,縂以學他為好。假使亞孝早知錯過,前十年之上悔罪心誠,又何至家散人亡,一番招累?大抵肚中濕熱,積結多時,非真大瀉一場,未肯從新謹慎。亦如行姦要待事穿,做賊要待被捉,然後手忙脚亂,胆碎魂驚,方識前非,囘頭怨錯,亦已遲矣。故君子舉動,未見禍而預早修心;小人昏迷,禍臨頭而方知怨氣。一个先一着,一个遲一步也。

	此段事,又呌做眾虎一麒麟,以亞悌作麒麟而一班兄弟作老虎也。獸之猛者莫如虎矣:曉食羊,曉食豬,曉食狗,而且食人矣。老虎雖惡,有人敢裝老虎,捉老虎,剝老虎皮,食老虎肉,抽老虎腸,攞老虎胆,切老虎口,敲老虎牙,而且將虎皮送與菩薩坐,破虎骨來燉虎骨膠。虎嘯風生,何以个陣時無一毫猛氣也?麒麟為至善之獸,兒童見之不驚,男婦見之不懼,而能化煞消凶,亦頗有驗。每見人家屋內,寫麒麟在此而不寫老虎在此,有舞麒麟而不舞老虎,何也?取其善氣吉祥也。

	書曰:「柔勝剛,弱勝强。」此之謂也。三千斤大炮打向戰船,打向賊艇,能打折舵,能打折桅,推斷尾棚,推倒全隻,其氣勢之大,可謂壯哉。若將X網掛在船傍,炮蛋飛來,只僕一聲而自跌落水,何也?X網不受其力也。又曰:「舌柔常在口,齒折只為剛。」舌在口中,自初生時,以至臨死,露開个口而舌尚存。其牙出世得遲,而破敗得早,故有四十歲而脫落三兩隻者,五十歲而脫落六七隻者,六十歲而脫落十餘隻者;有\ruby{的}{󰦦}到老臨死時,所剩無幾隻矣。論口內之物,其硬莫如牙,其柔莫如舌。牙每先折而舌常留,有時牙不服曰:「亞舌哥,乜你撈世界得咁長久,而我一班兄弟好多墮落而不見了,何也?」亞舌答曰:「你壞在一个「恃」字,恃有上牙下牙、大牙板牙,上下有拍手,內外有照應。惡在一把牙恃兄弟多,恃氣力猛,遇食豬脚骨要咬到碎,食雞脚趾要咬到爛,誰不知硬鬦硬兩家散,你傷人,人傷你矣。你重有一件至可惡事,有時咬口唇咬舌尖,自家骨肉自取傷殘,所以門外多人憎,門內有人受也。你做人實在唔中用,只顧口头肥,不理心腹壞。一次\index{\ruby{食尿甕雞}{󱑝󰴭󰅲󱜞}}\footnote{食尿甕雞:話人講野難聽,用詞核突肉酸粗口々。},一次\index{食死顛狗}\footnote{食死顛狗:「食尿甕雞」同意思},臭口而不知,毒心而不覺。又不知分量,又不識細微,至大者牛而敢咬之,至小者虱而亦咬焉,是你之無所揀擇也。又有度量,又有隱藏,遇人不合自己意,就咬牙切齒,想去吞人,个\ruby{的}{󰦦}就是你之壞處。你一世所咬者多矣,而可以累你苦楚者,惟有流牙血,生牙虫,風火牙痛,牙肉腫浮,而你不知悔也,必至折磨,必至搖落,而後已焉。」

	亞牙曰:「你數我咁多碟脚,咁多牌底,句句亦真,我唔怪你。但我等做牙,亦有許多好人物,矜貴淡定,取細而食,擇潔而餐,不尽橫吞大嚼也。」亞舌曰:「別家別戶,得涵養之法,安享和平者,我不得而知。惟我與你同居,時時相見,今你自嗟零落,不覺直言得罪,望作戲言可矣。」亞牙曰:「我知你笑我咯。究竟你之安穩,在何所長?」亞舌曰:「我睇勢色來湊。好食之來,煩以應接,而不傷損於他,量其可吞者吞之,不可吞者吐之而已。唔似你兄弟咁縱橫,左咬來右咬去,咬到連渣都無也。我雖一人,可以長久獨立,你雖多眾,零落衰微矣。」亞牙曰:「人話我牙尖齒利,乜知你重舌鋒藏劍也。」兩人大笑而罷。此雖戲弄之談,可為恃强者作一笑柄。羅洪大仙有詩云:「為人不必逞英雄,萬事無過一理通。虎豹常愁逢獬豸,蛟龍又怕遇蜈蚣。小人行險終須險,君子固窮未必窮。千斛洋船沉海底,只因駛尽一帆風。」

\chapter{\index{砒霜砵}}

	江南金陵大城南門外,有一人姓鄔名家治。父子出外做生理,家中有老母年近七十,雙目久盲。妻梁氏,氣負兇橫,常以毒口咒人,人加其號為「\index{\ruby{砒霜砵}{󰐦󱓖󰌽}}\footnote{砒霜砵:話女人狠毒}」,事家婆尤為忤逆。娶媳韓氏,性頗柔順,心不服\index{砒霜砵}所為,亦無奈何也。

	一日,\index{砒霜砵}罵盲家婆曰:「你个\index{老狗乸},好死唔死,在此食屈米,偷生人世,要你何用呀!」盲家婆曰:「我食我子孫\ruby{\ruby{的}{󰦦}}{󰦦}米,又不是你在外家帶歸來,何用你咁眼緊哩?你一世都係欺負我。唔通个仔都唔知?我如今又盲又老,冇幾久世界,你自己都要顧吓本心,恐怕雷公打你。」\index{砒霜砵}發怒起來,蝦咁跳大声曰:「你个\index{老狗乸},乜知咁心毒麼?想請雷公來打我。我又冇得罪雷公,因乜事雷公來打我呢?我唔怕雷公,只怕老公。但係我好命,嫁得好老公,一世唔曾罵我一言,打我一棍。分明縱妻之惡。唔比同你個\index{老狗乸}咁心毒,日日要罵人,方得安樂。你話我欺負你,点樣欺負法?你逐一要講出來,若講不出,要扭歪你個嘴!」惡生个樣子。盲家婆曰:「且勿論前之事,即如近兩月間,我仔付囘臘鴨八隻,臘肉十斤。你將臘鴨送與亞姨,送與契友,東一隻,西一隻,我何曾食得幾多件呢?將臘鴨晚晚煲五更飯,今晚一煲,明晚一煲,我何曾食得幾多件呢?」今世人出外亦寄食物歸家,但老婆主權,父母所食有限。\index{砒霜砵}曰:「你時時怨冇牙,唔食得硬物件,个\ruby{的}{󰦦}臘鴨咁乾,你唔着食咯。你近來腸肚弱,食\ruby{的}{󰦦}肥膩就疴就瀉,个\ruby{的}{󰦦}臘鴨肉,你唔着食咯。惡婆亦有道理。况且信皮寫云:付回家下收入。丈夫稱我為『家下』,你呌做『家上』,照講來與你無干,做乜你咁累贅呀?做得大狀棍,無理議出有理來。盲家婆曰:「我冇得食,應要有得着。為何你着綾羅綢緞,我縂係粗衣麻布呢?」丈夫肯作置老婆,做仔唔肯打理老母。\index{砒霜砵}曰:「个\ruby{的}{󰦦}係老公打办我光輝,我修得到係我之福。你一世唔修,所以冇福。專門講折福說話。唔通六七十歲老太婆,重整成咁好色水麼?」盲家婆曰:「我唔要好色水,都要補破遮寒呀!為何我\ruby{的}{󰦦}衣服穿爛,有時鈕耳崩、衫袖裂,你為婦道,何解縂唔打理呢?」\index{砒霜砵}曰:「我有我事幹,点樣得閒打理呀?」娶新婦何用。盲家婆曰:「你唔得閒,我有孫新婦得閒。為何我呌佢漿洗,你定必呌佢去東去西,致我衣裳浸爛有之,發霉者有之,分明故意收什我。」\index{砒霜砵}曰:「我个新婦係我娶歸來,不是你\index{作置}\footnote{作置:解「謀」、「」、「偷」、「強取」、「敲詐」,「作置佢副身家」,「作置人地老婆」,「作置何公變何太」、「作置人一筆」}\ruby{的}{󰦦}。問你一世有何本事,既做人家婆,已經享福太過,又想做人太婆,你實在唔知足。」盲家婆曰:「我唔講得你贏。你个把嘴終須要折墮死,落閻羅王要勾你舌根。」\index{砒霜砵}以手指向盲家婆曰:「勾、勾、勾,勾你個盲鬼!冇人有你咁心毒,開口就呌我做\index{砒霜砵}。你試想吓,煮熟飯何嘗唔許你食?煲滾茶何嘗唔許你飲?你自己問心呀!有天知地知,人知鬼知,睇過係我欺負你唔係呀!睇過話你好,抑或話我好呀!」盲家婆曰:「你有錯,你有錯,終須个天饒你唔過。」孫婦韓氏多方勸解,兩人歸房而罷。

	六七日後,\index{砒霜砵}心猶火忿。一日心生一計,看見盲家婆在房中抽扯櫃箱,搬取物件,新婦又往鄰家磨榖,即解下縐紗包頭帶,打一個神仙索,輕輕移步人房。向盲家婆頸上一箍,出尽生平氣力,勒到盲家婆手乱扒、脚乱跳,欲喊不能出声,欲活不能通氣,雙膝跪在地上,頭搖髮乱,腰背擺左擺右、或高或低,眼中水火齊來。\index{砒霜砵}仍不肯放手,勒至死為止,嗚呼哀哉而氣絕矣。\index{砒霜砵}解脫縐帶,尚恐其生,用手掩住口鼻局了一囘,然後放手;又側耳向他口鼻處細聽,不聞声息,做事極点\ruby{的}{󰦦}幼細。知其真死無疑。尽勢拖挽,放在床頭上。

	一息間,新婦歸來。\index{砒霜砵}細声曰:「亞嫂,我有一件緊要事吩咐你知。只可你知,不能傳說於人。」韓氏見其面色慌忙,青筋起現,知其必有古怪之處,遂低頭答曰:「婆婆有何吩咐?」\index{砒霜砵}曰:「你个盲太婆,我一世共佢有緣份,個條命縂唔合得佢。佢係我\index{眼中釘},係我\index{心頭火}。我先時將他勒死,鄰里來吊香,我自有講法,你不用多言。就係我老公及我仔歸來,你亦不得洩露機関,講其來歷。你若踈言,我定必要死;我亦斷唔容得你,要先將你勒死,拚之同你一鑊熟。」韓氏聞言,大嚇一驚,只得對曰:「謹照婆婆所教,不敢多言。」\index{砒霜砵}曰:「隨得你,你唔怕死即管講。」韓氏心內呌苦,不覺淚流滿面。\index{砒霜砵}曰:「我都有眼淚,你眼淚得咁多?你好可憐佢麼?你个賤人份外多事。」

	遂將盲家婆檢点\index{周至},忽然哭起來。好傷心。鄰里走來問曰:「又與家婆打罵麼?」\index{砒霜砵}曰:「唔係打罵,我家婆如今死了咯!」大叔婆驚曰:「我先時見佢在門口呌雞,為何死得咁快?」\index{砒霜砵}曰:「唔講你唔知,因今朝煮多契女飯,契女唔來食。家婆呌肚痛,睡床唔起身。到了午後,喊肚餓,我話煲\ruby{的}{󰦦}白米頭,局\ruby{的}{󰦦}好臘鴨。佢唔願食,話要炒飯,我就切\ruby{的}{󰦦}臘肉粒、雞蛋絲、葱花正菜,炒得又香又爽。誰知佢食了一碗又一碗,食了四大碗。老人腸肚窄,点能受得幾多呢?勸佢唔好食咁多,似乎話我\index{砒霜砵}制折佢,又係唔好人。乜知炒飯性太焦,味太鹹,食完見喉渴,飲了一大煲茶,敢就飽得眼凸凸,兩脚都伸直。你話点算好呀?人家唔知,估我共佢不和,似乎毒死佢。但係能瞞得四鄰,不能瞞得佢孫婦,現有佢孫婦可據。我雖然丑稟,實係貌惡心慈,自家贊自家。唔比同人佛口蛇心,陰柔害物。我見佢飽得咁辛苦,實覺可憐;初時唔估咁撞板,若早知道,斷唔炒飯過佢食咯!」大叔婆曰:「我知佢一世愛食炒飯,但唔該食咁多呀!」又一鄰婦曰:「飽死好過餓死,勝過餓鬼,年年要等七月十四。」講完,\index{砒霜砵}放聲大哭,備買棺物等項,親手自己殯殮。恐怕被人看出。遮遮掩掩,有誰看到頸處?其夫及子歸來,殯葬已罷。遲廿日間,其夫及子又遠囘舖矣。計盲家婆死之日,其時係道光十六年十二月初旬也。

	\index{砒霜砵}見家婆死後,並無人知覺,新婦又不敢言,自以為安枕無憂,逍遙自在。每餐飲幾両好酒,局一串風腸,有時飲得醉霏霏,自言自笑。快活咯。忽然一日,身中大病,寒熱交作,一陣如冰凍,一陣似火燒。睡中反覆,滾滾團團,神魂飄蕩。見一官差,將鉄鍊鎖住頸上,拖狗咁拖,苦拖同去。\index{砒霜砵}曰:「你鎖我做乜呀?我又冇得罪你,不過得罪家婆。你恃惡麼?」官差笑曰:「你重\index{詐懜}\footnote{詐懜:詐懵,「詐傻扮懵」。}?你去就知到。」\index{砒霜砵}尚估人告發,差役來拿,心中僕僕咁跳。行至一處地方,陰氣慘淡,日色微茫,見無數披枷帶鎖、散發披頭,亦有\ruby{的}{󰦦}騎馬坐車、手舞足蹈。或人類畜類,滿眼紛紛;或含笑悲啼,情形種種。想起人話陰間光景,此處料必無疑。問官差:「此是何方?」官差答曰:「此是你結局之處。」真妙語。\index{砒霜砵}愈見愈傷心,方知牽我者原是鬼差。哭唔願行,坐在地上放側眠,逞蠻撒潑。鬼差喝曰:「你起唔起?」\index{砒霜砵}曰:「我願死不願行。」鬼差笑曰:「你尚估係生人麼?你真正係唔行?」\index{砒霜砵}垂頭閉目,縂不答聲,鬼差遂抽住一隻脚,隨拖隨定。拖得\index{砒霜砵}手脚撒開,\index{頭披䯻散},大声乱喊:「我願行咯!唔好拖得我咁凄凉咯!」情景極有趣。鬼差不由分說,苦苦尽力拖起勢走,只管拖,只管罵,話:「唔怕你\index{撒潑},唔怕你才蠻,你到來惡得過我?你話唔怕雷公,乜要怕我呢!」\index{砒霜砵}一生唔曾被人丟過駕,今到此處尽地丟清,好似惡人到官,有了一毫氣勢。\index{砒霜砵}曰:「差老爺,我有犯你呀?為何將我咁作賤呢?」好之你又作賤家婆?鬼差曰:「重有得過你賤,你估咁樣就罷了麼?你都唔知利害。」引動得來往之鬼都笑,連一班牛頭馬面鬼卒亦笑起來,笑其拖得咁趣也。有一相識之鬼來講情,方歇手唔拖,任其起身行走。

	去到一間大宮殿,企在門外。聞知殿內呼喝声,官差擠擁,忽牽入內。有一个判官唱名:「不孝婦鄔門梁氏到案。」\index{砒霜砵}即跪在地上。閻王曰:「你係金陵大城南門外鄔家治之妻,係你嗎?」\index{砒霜砵}曰:「正是不差。」王曰:「有人告你。」\index{砒霜砵}囘頭,望見盲家婆跪在後旁。王曰:「你勒死家婆,係你一人,抑或有別人幫手呢?」\index{砒霜砵}想:「此事難推過新婦,況且家婆在後,不能誣賴於人。」遂直認曰:「小婦人一時淺見,將家婆勒死,係我一人,並無幫手。有時勒死隻狗都要兩人,勒死家婆獨力可能做得,都算本事。今知罪過,悔恨難追,縂係望王爺格外施恩,大開方便,勿執勿怪。」記錯拜神之時,拋杯唔轉。王拍案大罵曰:「你个\index{賤婦人},好生大胆!將家婆勒死,不知罪大通天!在陽間律例,應當碎剮凌遲;在陰間律例,要打落酆都地獄,受苦五百年,變過豬狗畜生,方成人類。但係陰間受苦,陽世唔知。我今發你還陽,將此事轉傳於人,世上多一人知,免你地下多受一日之苦。你丈夫鄔家治平日夫綱不振,容縱其妻,任由老婆刻薄老母,世間每有此等人咁\index{蠢才}不中用。生前既不能發覺,死後又不能代老母報仇,在陰間罪案應當處斬。」\index{砒霜砵}曰:「小婦人不幸,未曾入過學堂。男子學堂亦有入過忤逆父母。頭戴膏油,不知不識,何以曉得罵家婆做\index{老狗乸}呢?原望丈夫教導。因丈夫毫無管束,是以犯此天條,係丈夫呌你勒家婆嗎?望王爺準我投胎轉世,另行孝義,以補前非。」王曰:「今生事做不了,何論來生!真爽快直捷。你一生壞在个把嘴,牙尖齒利,造是生非。如今在我面前,尚敢支離辯駁,况在陽間咒罵,重了得麼?」命小鬼將亞婦掌一百嘴巴。\index{砒霜砵}大哭嗚嗚,打得个嘴歪左歪右,\index{砒霜砵}想扭歪家婆嘴,誰知自己个嘴重歪得多。口唇都長多一寸,唔敢出声。判官看見,以袖掩口,側面亦忍笑不住,笑其想賣口乖而受打也。\index{砒霜砵}拭乾眼淚,又想開声向王爺求情。王曰:「不用多言。」着小卒帶他還陽而去。

	鬼卒又帶他一路行,一路走。\index{砒霜砵}曰:「差老爺,我如今魂飛胆碎,嚇破心肝,你本來有心肝,因何被他嚇得破?精神困極,腳骨軟了,家婆条頸先軟了。容我一坐,做得唔呢?」鬼差曰:「你慌冇得過你坐麼?五百年地獄任你慢慢坐,到厭都做得咯。个隻鬼講說話,乜得咁失利呢?重關係過\index{砒霜砵}。你願行即行,你唔肯行,又照先時咁樣拖你隻脚。」\index{砒霜砵}曰:「唔好咯!我怕你咯!我情願快\ruby{的}{󰦦}走咯。」一陣間,歸到屋內,被鬼差一推而醒。大嚇一驚,周身冷汗出來,床中被褥濕透。自怨歎曰:「該死、該死!就係一死都未能了局呀!婆婆呀,乜你唔翻生等我奉事吓呀!」你奉事得多,佢心亦足咯。

	一夜,暗中流淚,以手自己打頭,縂之怨錯。天光後新婦入房來呌洗面,唔願起身,新婦問其何故。\index{砒霜砵}曰:「我牙痛,牙骨刺,牙肉腫,大約有牙虫都唔定咯。」新婦曰:「我試睇吓。」\index{砒霜砵}搖手曰:「駛乜睇呀!我尚吓痛到死咯!」新婦走埋床,展開被一望,果見腮頰兩便,皮肉浮高,面似豬頭咁大。唇又長,眼又深,口旁之處俱現瘀黑色,好似打痕。新婦暗驚奇怪,遂問曰:「今朝另外煲過白米頭,局\ruby{的}{󰦦}好臘鴨,與你食,着唔着呀?」\index{砒霜砵}曰:「唔食得咯!粥水都唔輕易飲得啖咯!」竟然眠在床上,餓了三日。家婆飽死,佢怕餓死。忽然身中生得無數瘡仔,上生至頭,下生至腳,連到手指腳趾、頸喉耳鼻,處處皆然。周身黃濃白泡,藥散敷之,連肉都缷落地,醫家無計可施。惟背後一瘡更大,漸爛漸濶,穿了一个大孔,似巖洞之深,望見肚內,心肝脾肺俱現藍黑色,其心更黑幾倍。名醫家不能識其症。醫家曰:「書有載講惡毒大瘡,唔有見過毒得咁凄凉。」此醫家看外科書,不過曉得一半。知佢毒瘡,唔知佢惡呀。\index{砒霜砵}曰:「我一世好心,更兼好口。唔知点解生得个咁樣病,縂之係前世唔修咯!」今世是真。新婦向側面,掩口暗笑,知道係勒死家婆症也。醫家無法,只以大油紙鋪住,好似繃鼓一樣,免受生風。唔似得縐紗帶束住可更好。鄰里來問病,不敢望其背,因有一婦見之,被嚇一驚,歸家成病。醫家告退,\index{砒霜砵}呌苦連天。

	痛了十幾日,肉但似火,骨節似刀切,喉極乾,頸極腫,家婆死時有咁腫。如坐火坑,如睡簕床。即是生前大地獄。想拜天,手唔拜得;想\scalebox{0.5}[1.0]{足}\scalebox{0.5}[1.0]{見}地,膝唔跪得。重咁神心麼?一日痛到極處,呌新婦到床前,細氣低声曰:罵家婆个陣時得咁大声?「亞嫂呀,我一生唔好頸,唔肯饒讓人。你唔饒讓人,鬼神唔饒讓你。因被你太婆罵了一番,就懷恨在心,將他害死。我以為人唔知鬼唔見,可以安然無事,点估到地下真有閻王呀!被灶君奏天,婆婆又告發。前者勾我魂落陰間,與你太婆對審一堂,會經招認了案。閻王說要我坐五百年地獄。你家公因聽妻言之過,都要斬首遒刑。我今死去,地獄之罪斷不能辞,未知你家公將來如何結果?都係酸果若果,唔係甜果咯。我死之後,不妨傳與人知,或者減我罪過一二。」遂將閻王所判斷說話,逐一講與新婦知之。新婦聽聞,吐出舌驚曰:「真有陰司,怪不得婆婆咁樣病咯。」\index{砒霜砵}大呌幾吉:家婆死唔出得声,\index{砒霜砵}死可能出得声,而且大声。「我苦呀!我苦呀!」遂氣絕而死。其子歸來葬畢。

	一月後,鄔家治枕骨後生一大瘡,歸家調理,漸生漸濶。生了兩三个月,通条頸俱爛完。一日坐床,口頭低頭,不覺大响一声,頭跌落地。其声與大芋头在閣上跌落地下相似。新婦方知閻王話要處斬,即斷頭瘡也。其子又殯葬畢。

	約半年之後,一日有鄰里二五婦人,來到鄔家治之屋,與其新婦韓氏共坐閒談。一婦人講起\index{砒霜砵}一世忤逆家婆,毒心毒口,唔怪得咁樣死法,亦理所當然。獨至其夫鄔家治,一生柔順,順老婆。並無得罪於人,何以咁樣死?唔通天眼半明半暗,隻開隻閉,講得好新樣。亦未可知。計起番來,做醜人不宜,做好人亦無益也。韓氏曰:「我話天眼明過鏡,縂係人唔知。」眾問何故,韓氏曰:「我太婆唔係飽死,係我\index{惡家婆}將他勒死。」眾大驚曰:「此犯天条大惡,為何不出声?」韓氏曰:「極之難講。家婆吩咐,話我出声,先將我害死,所以不敢呀。其後佢魂落陰間,閻王審判,要佢落地獄,我家公要斬頭,所以咁樣古怪。此等說話,係我家婆痛到將死時講與我知,故此知其端\ruby{\ruby{的}{󰦦}}{󰦦}。」婦曰:「唔怪得咯,死都唔好可惜佢咯。連你家公都係\index{蠢才},一世陰陰濕濕,冇\ruby{\ruby{的}{󰦦}}{󰦦}丈夫男子氣。我有一次入來你屋,見\index{砒霜砵}咒罵盲家婆,你个家公只曉得坐住竹椅拈烟筒食烟,縂不出一言、喝一句。所以容縱\index{砒霜砵},惡得咁淒涼呀!至到盲老母,六七十歲人,遇時受苦。應承做仔,有咁冇本心;話曉發財,又話去幾遠地方;一間屋內,好似倒麻藍紗咁乱,講乜本事呢?呌做鄔家治,都唔治得一个老婆,重想治一家?个\ruby{\ruby{的}{󰦦}}{󰦦}都唔係呌做男子佬,實係呌做老婆奴。」又一婦笑曰:「你老公唔聽你說話麼?」其婦答曰:「我老公有咁\index{蠢才}?話着佢老母唔好,就好似打崩佢頭咁樣痛咯。有\ruby{\ruby{的}{󰦦}}{󰦦}好食物,要先敬佢老母,然後中佢意。天地間另生一等奇男子出來,顯與眾管。我雖然係醜稟,都唔敢得罪佢老母一句。你話我老公奇唔奇呢?你估比同鄔家家治咁衰麼?歸來伏在老婆裙頭下,要聽老婆声氣,自己唔做得主意,个\ruby{的}{󰦦}重係呌做人?」又有一婦答曰:「我地冇命水,嫁得个老公縂唔聽我說話。」前婦曰:「聽你話,實首好麼?即家治聽老婆話,好之衰生个樣。」有一老婦曰:「看如何聽法。勸唔好嫖,唔好賭,唔好吹鴉片;要顧身,要顧家,个\ruby{的}{󰦦}說話俱要聽。若只曉得派翁姑不是,叔伯不是,做男子就唔着聽咯。」眾婦曰:「究竟二叔婆講來有理,唔怪得二叔公一世都聽你說話。」各人大笑而散。自此,\index{砒霜砵}之事漸傳出來,遠播於眾。

\chapter{茅寮訓子}

	清朝滿州之官,並無姓氏,只以名為姓焉。康熙年間,滿州有一人,呌做同貞,為官做到宮詹之職。同貞有結髮之妻,生了三子。不幸中年妻死,續娶填房一个汪氏,十分美貌聰明,係旗下人家女也。汪氏歸來,持家極有禮法。厚待丈夫三子,意極仁慈,作如自己所生,無分別也。同貞性氣剛直,遇事不合,便忿忿不平。後因一件案情辦得太烈,致朝臣執奏,削職抄家,產物一空,漸成貧困。汪氏極力撐持,幫助其夫用度。同貞不以失官為意,貧淡順其自然。未幾同貞死,汪是哭絕,痛不欲生,水漿不肯入口,決意同亡。既而覆想一吓:「敢死易,養仔難。連自己死埋,个班仔向誰倚賴?況且先夫臨死,曾經吩咐床前,要我撫養諸兒,不可置之度外。若使自尋短見,夫在九泉之下,依然緊縐雙眉。」左想右思,死去亦難,不死亦苦。人生天地,不怕做辛苦事,還期苦尽甘來。於是立硬心腸,咬牙抵住,勉强起立,打点殯葬事宜。受痛含悲,難向諸兒解說。三子只知啼餓,誰憐寡母。腸斷魂離,哭淚難乾,惟有呌夫知道而已。

	其時,汪氏守寡,年僅廿二歲也。家既貧,無人照顧一二,備極艱辛。惟望三子學問能成,方有生路。勉强請一个先生來教三子,將所住之屋截出一半做書館。典當衣服首飾,備買紙筆,與及經書。先生修金,其價亦廉,而飲食供奉之情,極尽誠敬。捱了一年,而貧更甚,漸不能當。想呌三子出外從師,難供費用,於是自己教訓。手勤紡績,口授經書,三子企立一旁,眼觀耳聽。有時天寒冰凍,灯光如豆,火不成紅,而冷雨淒風破窓乱打,猶執諸兒之手,指向卷上,字句分明,而哽咽一声,不禁淚流滿面者矣。諸子旁侍亦泣,於是掩卷收灯,囘床而睡。枕孤被爛,破蓆零星,猶囑諸兒各於床上念書,沉吟覆記。僅到五更,呌諸兒復起誦讀,而汪氏已離床開卷矣。及後并無錢賃屋,無處棲身,因賃一空地,篷結茅寮,母子居住。或早朝無米煮,近晚食粥一餐。教三子奮志讀書,要做好人,以承祖父之志。三子若有懶惰,散步遊行,汪氏則啼哭呼天,自怨自責。三子恐懼,即時跪在母前,認了不是,願自後遵從母教,不敢荒疎。汪氏然後收淚止啼,方肯飲食。三個仔兄弟相勸,你勸我、我勸你,務要發奮做起人來,以慰老母之德。由是真正用功,苦心習練。每朝清晨到老母面前,拜了三拜,然後虛心下氣,企在於旁,以聽老母吩咐,若無別話,各去攻書。

	及至康熙癸丑科,大仔呌做逢泰,細仔呌做滿保,兩个中了舉人。申戌科,逢泰中進士,点翰林。庚辰科,滿保中進土,点翰林。丙戌科,第二仔呌做元旦,亦中了舉人。三子皆登科甲。康熙三十六年冬月,第三仔滿保陞去福建做撫臺。康熙四十年,滿保又陞福建浙江做兩省縂督。此時老母汪氏做了太夫人矣,隨任在衙門享福。凡地方有關於大利大害者,時時問及其子,滿保亦虛心稟告,與太夫人斟酌,而力行之。康熙五十六年,大仔逢泰出身去陝西,做欽差學院大人。太夫人教以「公明」兩字,逢泰謹遵母教。康熙六十年五月,太夫人身中染病,滿保小心奉事。五更早起,即往床前問安,藥湯茶飯,定必自己親手捧向母前,勸其飲食。從旁企住,等候太夫人飲完食完,再問可否,然後告退。時值福建台灣朱一貴招聚匪徒作乱,至數十萬賊攻破城池。滿保奉旨征打台灣,起程既去,過了重洋。太夫人修書寄滿保云:「兒乃尽力出征,不必以老母為念。你母親今好了,飲得食得,你不須憂,務宜一戰功成,以報朝廷之望。」其實太夫人身猶有病也。及六月,台灣征平文書報到,太夫人喜動顏色,焚香稟告天地,叩謝神恩。謂家人曰:「台灣平,地方寧,社稷無疆之慶。兒能了此事,我安樂矣。」閏六月十三日卒,死時光氣滿容,清風拂拂,雖大暑時候,而一室生涼,若有冰霜之象。見者皆稱爽朗,共以為奇。

	考太夫人汪氏之品格也,其貌美而正,其氣清而靜,其心切而平,其志堅而苦。當年少也,不施脂粉,至憎賣弄風情。及隨任也,不看戲景,至惱遊行散蕩。教息婦習礼,待婢女極慈。嘗謂新婦曰:「婦女讀書識字,原是有用之人至為好事。若不習禮義,不重名節,就讀千萬卷,終何用哉?只知學吟詩,學作對,要人稱做才女,便自滿足,而於大道理不曉一分,居家庭亦無好處。所謂枉讀詩書,亦無謂也。更有等婦女,生來庸俗,以正經書卷唔看得入眼,正經道理唔動得人心,專愛看邪書小說,歌曲淫詞,自號風流,以為瀟洒,誰不知滿紙邪氣,滿眼淫情,日夕流連,心神變動,日久不覺流於下賤,悞入迷途者有矣。故好插花搽粉者,惹人邪意也,好行遊看戲者,自起浮情也。故為婦女,無論聰明愚拙,富貴貧難,縂要存一片真心,一点正氣,然後生居世上,不枉為人,天必祐之,而鬼神亦敬之矣。」其教媳婦之道如此,子孫傳為家訓,故其家多正靜焉。太夫人享年七十二,眾稱其福祿壽全。

	汪氏守寡之時,年廿二歲,生得聰明秀麗,何憂無別處棲身?況前頭仔三个又非自己親生,苦楚難堪,在他人多有不安於其室矣。汪氏之心,無分彼此,三子非他,係丈夫之子也。愛丈夫而不愛其子,丈夫豈能安乎?惟看得丈夫真,然後愛得三子切。一班幼小,只曉得嚶嚶啼餓,何知母氏傷懷?吾想此時媒人婆、竹笋髻,紛紛來到,勸其改嫁者不少矣。汪氏以安於受苦,抵之鉄石心肝,終難轉動。独是一貧如洗,無米難炊,忍餓抵饑,凄凉多少。汪氏立定主意,只思教子成名,苦讀寒窓。知嚴師原是慈母,茅寮斗大,有玉堂金馬之人。辛苦十年,一生富貴,子官縂督,自己封一品太夫人,所謂苦尽甘來,竟如所望。世間亦有青年而守寡者,其困苦亦有相同;布教子之心,未必有如是之真、如是之切矣。何況非自己所出,原係前頭仔者哉。即自己所生,亦不過寶之愛之,如掌上之珠,作心頭之血,只憂他唔養得大,唔高得快。有\ruby{的}{󰦦}好食讓他食之,有\ruby{的}{󰦦}好着讓他着之,斷不肯打一棍、罵一言,如雞之護春,牛之引仔,只恐相離相失,而不知有嚴束之道焉。又安肯治其子用苦功,捱苦境,苦心習練,苦心琢磨也哉?所以寡婦之子,每多學壞,至不成人,其母有以縱之也。又有守寡之婦,飽衣足食度日,寬容正直,矢志堅貞,起居清淨,修善修德,愛己愛人,將來德蔭兒孫,魂歸樂國,堪稱賢婦,謂之能人。而乃有浮蕩之氣不收,懶情之情日縱,待人無礼,治己無方,以賭博為奇,以遊行為樂,不和於眾,不合於家,或太驕奢,或太吝惜。雖稱守節之名,而不知所謂守者,謹守規模也;所謂節者,行為節度也。失其真實,所以受人彈、受人笑者亦有之。若汪氏太夫人,可為守節中之表表特出者矣。

 

 
 

	\printindex % Print the index









%   \chapter{Recitables}


% I have of late, (but wherefore I know not) lost all my mirth, forgone all custom of exercises; and indeed, it goes so heavily with my disposition; that this goodly frame the earth, seems to me a sterile promontory; this most excellent canopy the air, look you, this brave o'er hanging firmament, this majestical roof, fretted with golden fire: why, it appeareth no other thing to me, than a foul and pestilent congregation of vapours. What a piece of work is a man, How noble in reason, how infinite in faculty, In form and moving how express and admirable, In action how like an Angel, In apprehension how like a god, The beauty of the world, The paragon of animals. And yet to me, what is this quintessence of dust? Man delights not me; no, nor Woman neither; though by your smiling you seem to say so.
% This is an English paragraph. It should start with indentation.

% 這是一個中文段落,應該有縮進。

% これは日本語の段落で、インデントがあります。

% {\koreanfont 이것은 한국어 단락입니다. 들여쓰기가 있어야 합니다。}

\end{document}
